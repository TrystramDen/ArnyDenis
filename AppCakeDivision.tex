
\chapter{Cake Division}
\label{Appendix:CakeDivision}

\section{Presentation of the problem}

We investigate here some strategies to divide a cake
(or any kind of resources). We are interested in dividing these resources \textit{fairly}, 
which means informally that all recipients believe that they have received a fair amount of resources among the whole cake. 
Each recipient has a different \textit{measure} of the value of the pieces of the resources. 

%There are two cases to consider that involve various techniques depending if the cake is homogeneous or not. 
Let us start by discussing briefly the first variant of the problem where the cake is homogeneous. 
The solving methods are based only on geometrical arguments. 
\begin{itemize}
\item If the cake is a circular tart, there exists a simple method based on geometry (cosines).
\item Consider now a rectangular cake with chocolate icing on all its faces as shown if Fig.~\ref{Fig:cakeHomogeneous1}.
\begin{figure}[htb]
\begin{center}
        \includegraphics[scale=0.4]{FiguresMaths/CakeHomogeneous1}
        \caption{Theoretical representation of a cake with chocolate icing on all its 6 sides.}
        \label{Fig:cakeHomogeneous1}
\end{center}
\end{figure}
There is also rather simple geometrical solution to divide it. 
Notice that the cake is no more fully homogeneous, but still \textit{regular}. 
\begin{figure}[htb]
\begin{center}
        \includegraphics[scale=0.4]{FiguresMaths/CakeHomogeneous2}
        \caption{Cut vertically the cake over its diagonal into two parts (1) and (2) ---left--- and move a part backward ---right---.}
        \label{Fig:cakeHomogeneous2}
\end{center}
\end{figure}
The idea is to cut the cake along a diagonal and move one part in order to make coincide the icing
(see Fig.~\ref{Fig:cakeHomogeneous3}).
\begin{figure}[htb]
\begin{center}
        \includegraphics[scale=0.4]{FiguresMaths/CakeHomogeneous3}
        \caption{Cut the two-parts cake in $n$ equal vertical slice.}
        \label{Fig:cakeHomogeneous3}
\end{center}
\end{figure}
Then, cut it into n pieces orthogonally to the icing sides (see Fig.~\ref{Fig:cakeHomogeneous3}). 
This way, each person obtains two parts. 
The division is fair because everyone gets exactly the same amount of cake and icing. 
\end{itemize}

In the general cake division problem, the cake is not homogeneous, 
one recipient may like marzipan while another one would rather prefer chocolate or cheries
that are put on the top.
Let us concentrate now on this case.
We detail first the mathematical model of the problem and discuss fairness issues. 

\section{Definitions}

Let us consider a cake to be shared between $n$ \textit{agents} (sometimes called players). 
Formally, the cake is represented by the interval $[0,1]$ of real numbers.
A \textit{piece} of the cake is a finite union of disjoint subintervals of $[0,1]$.

We assume that each agent $i$ has his/her own valuation function (denoted by $v_i$).
This function is a \textit{measure}, $v_i(A)$ represents how much agent $i$ likes piece $A$. 

As the size of the cake has been normalized, the value of the whole cake is equal to $1$
and the value of an empty piece is $0$.

\subsection{Properties}

The assumptions about the valuation of the pieces of cake are:
\begin{itemize}
\item Additivity: $v(A \cup B) = v(A) + v(B)$ for any non-overlapping pieces $A$ and $B$
(pieces are sub-intervals). 
\item 
Continuity: a small increase of a piece leads to a small increase of its value. 
\item The measure of a player is not known by the others.
\item The resources can be divided into parts of arbitrarily small values.
\end{itemize}

\subsection{Fairness}

The meaning of fair may simply mean a proportional sharing of the resources. 
However, there are some variants:

\begin{itemize}
%\item A cake-cutting protocol is said to be proportionally fair, if each of the $n$ agents can ensure he/she gets an amount of at least $\frac{1}{n}$.
\item The proportional fair division guarantees each recipient (agent) obtains a fair share. 
For instance, if three people divide up a cake each gets at least a third by their own valuation. 
Formally, this corresponds to $v_i(x_i) \geq \frac{1}{n}$ $\forall i$ for $n$ agents.
\item An envy-free division guarantees no one will prefer somebody else's piece of cake more than their own. More formally, 
a cake-cutting protocol is called envy-free, if every agent can ensure that he/she will receive a subjectively largest piece.
 $v_i(x_i) \geq v_i(x_j) \forall i$ and $j$.
\item Equitable division means that every person feels exactly the same happiness, 
i.e. the proportion of the cake a player receives by their own valuation is the same for every agent. 
This is a difficult aim since the players need not be truthful if asked their valuation:
 $v_i(x_i) = v_j(x_j) \forall i$ and $j$.
 \end{itemize}

%Another requirement for all proportionally fair protocols we will discuss, is that agents can guarantee their fair share by answering all questions truthfully.

\bigskip

Observe that for n = 2 agents, we have the following property:

\begin{prop}
envy-freeness is equivalent to proportional fairness.
\end{prop}

Let us prove that proportional fairness implies envy-freeness (the other side of the equivalence is proved below in the case $n=3$).

For both agents, proportional fairness means: $v_1(A_1) \geq 1/2$ and $v_2(A_2) \geq 1/2$.
Adding both expression, we obtain $v_1(A_1)+v_2(A_2) \geq 1$.

Since $v_2(A_1) + v_2(A_2) = 1$, we get 
$v_1(A_1) \geq v_2(A_2)$ and similarly for agent 2. 
\bigskip

But for more agents, this is no longer true, we only have envy-freeness $\Rightarrow$ proportional fairness.
\begin{figure}[htb]
\begin{center}
        \includegraphics[scale=0.6]{FiguresMaths/CakeEnvyFree1}
        \caption{xx.}
        \label{Fig:cakeEnvyFree1}
\end{center}
\end{figure}
\begin{figure}[htb]
\begin{center}
        \includegraphics[scale=0.6]{FiguresMaths/CakeEnvyFree2}
        \caption{xx.}
        \label{Fig:cakeEnvyFree2}
\end{center}
\end{figure}
\begin{figure}[htb]
\begin{center}
        \includegraphics[scale=0.6]{FiguresMaths/CakeEnvyFree3}
        \caption{xx.}
        \label{Fig:cakeEnvyFree3}
\end{center}
\end{figure}

Formally, Consider the case of $3$ agents who obtained each a part $A_i$.
The measure of any agent $i$ verifies: $v_i(A_1)+v_i(A_2)+v_i(A_3) = 1$.
Now, as the protocol is envy-free, we also have: $v_1(A_1) \geq v_1(A_2)$ and $\geq v_1(A_2)$.
Thus, $3.v_1(A_1) \geq 1$, thus $v_1(A_1) \geq 1/3$ which means that the protocol is proportionally fair. 

\section{Description of classical protocols}

A fair division protocol lists the actions to be performed by the agents in terms of the visible data and their valuations. 
A valid procedure is one that guarantees a fair division for every player who acts rationally according to their valuation. 
Where an action depends on an agent's valuation the procedure is describing the strategy a rational player will follow. 
An agent may act as if a piece had a different value but must be consistent. 
For instance if a procedure says the first agent cuts the cake in two equal parts then the second player chooses a piece, 
then the first agent cannot claim that agent 2 got more.
What the agents do is:
\begin{itemize}
\item Agree on their criteria for a fair division
\item Select a valid procedure and strictly follow its rules
\end{itemize}
%Finally, we assume that the objective of each player is to maximize the minimum amount they might obtain.

\subsection{Cut-and-choose}

For two agents, there is a simple solution which is commonly employed. 
This is the so-called \textit{cut-and-choose} method. 
Agent 1 divides the resource into what he/she believes are equal halves, and the other one chooses the "half" he/she prefers. 

Clearly, the person making the division has an incentive to divide as fairly as possible. 

This strategy provides an envy-free division.
However, this solution is not equitable since the non-cutter usually gets more than expected.

\subsection{Steinhaus Protocol}

Let us brieflyy present an old protocol, which somehow generalizes the previous one.

(1) Agent 1 cuts the cake into three pieces (which she/he values equally).

(2) Agent 2 “passes” (if she/he thinks at least two of the pieces are $\geq 1/3$) or labels those two as “bad”. 
If agent 2 passed, then agents 3, 2, 1 each choose a piece (in this order) and we are done.

(3) If agent 2 did not pass, then agent 3 can also choose between passing and labelling. 
If agent 3 passed, then agents 2, 3, 1 each choose a piece (in this order) and we are done.

(4) If neither agent 2 or agent 3 passed, then agent 1 has to take (one of) the piece(s) labelled as “bad” by both 2 and 3.
The solution is obtained by playing cut-and-choose between 2 and 3.


\section{Moving knives}

\subsection{The basic strategy}

The moving-knife procedure that is described below for $n$ agents gives an exact division for two agents~\cite{Dublins}. 

First, let us assume that there exists an external referee who is managing the knife. 

%There exists a moment when the division is fair.

(i) The referee moves a knife slowly across the cake, from left to right. 
Any agent may shout “stop” at any time. Whoever does so receives the piece to the left of the knife.

(ii) When a piece has been cut off, we continue with the remaining agents, until just one agent is left (who takes the rest).
\bigskip 

Now, let us study what happens if we remove the external referee. 
For two agents, a way to do this is as follows:

\begin{enumerate}
\item 
Agent 1 places two knives over the cake such that one knife is at the left side of the cake and one is further right; 
half of the cake lies between the knives. 
He/she then moves the knives right, always ensuring there is half the cake – by his valuation – between the knives. 
If he/she reaches the right side of the cake, the leftmost knife must be where the rightmost knife started off. 
\item Agent 2 stops when he/she thinks there is half the cake between the knives. 
{\Denis Give the argument why there is always a point at which this happens?}
\end{enumerate}


\subsection{Extension to 3 agents}

The moving knife strategy can be extended to guarantee envy-freeness for three agents~\cite{Stromquist}
with a protocol using $4$ knives.

(1) A referee slowly moves a knife across the cake, from left to right
(supposed to eventually cut somewhere around 1/3).

(2) At the same time, each agent is moving his/her own knife so that it would cut the righthand piece in half 
(with regard to their own valuations).

(3) The first agent to shout “stop” receives the piece to the left of the referee’s knife. 
The righthand part is cut by the middle one of the three agent knifes. 
If neither of the other two agents hold the middle knife, they each obtain the piece at which their knife is pointing. 
If one of them does hold the middle knife, then the other one gets the piece at which his/her knife is pointing.
\bigskip

It is rather simple to prove the proportional fairness.
\bigskip

A nice exercise is to prove that this protocol is envy-free. 


\subsection{Evaluation of the complexity}

Coming back to the basic moving knife strategy, each agent has to evaluate the measure as the knife moves over (for all the real numbers in the interval). 
For each continuous position, the agent has to evaluate the piece at the left of the knife.

Model:
 A reasonable protocol should be implementable in terms of two types of queries for agent $i$:
 
 \begin{itemize}
 \item $cut_i(\alpha,x) \rightarrow y$ -- agent $i$ cuts the part of value $\alpha$ in the interval from $x$ to $y$. 
 \item $eval_i(x,y) \rightarrow \alpha$ -- is the evaluation of the measure of agent $i$  between $x$ and $y$.
 \end{itemize}
 
 According to the number of queries, we are able to compare the complexity of protocols.

We can simulate the continuous moving knife protocol as follows:

(1) Ask each agent to mark the cake where she would shout “stop”.
Then cut the cake at the leftmost mark and give the resulting piece to the agent who made that mark.

(2) When a piece has been cut off, we continue with the remaining agents, until just one agent is left (who takes the rest).

At each round, each participating agent makes one mark. The number of participating agents goes down from n to 2 
for a total cost of $\Delta_n -1$.

\bigskip

Even and Paz introduced in 1984 the following divide-and-conquer protocol,
which improves the complexity:

(1) Each agent puts a mark on the cake at half of his/her own measure. 

(2) Cut the cake at the $\lfloor \frac{n}{2} \rfloor$-th mark (counting from the left). 
Associate the agents who made the leftmost $\lfloor \frac{n}{2} \rfloor$ marks with the 
lefthand part, and the remaining agents with the righthand part. 

(3) Repeat for each group in all the intervals, until only one agent is left.

It is easy to show that this division is also proportionally fair.

The complexity is $log_2(n)$ steps involving each $n$ operations. 
\bigskip

Note that there are more sophisticated protocols...


\ignore{
\section{A more sophisticated method}

let us now present an envy-free solution for $3$ agents.

(1) Agent 1 cuts the cake in three pieces (she considers equal).

(2) Agent 2 either “passes” (if she thinks at least two pieces are tied for largest) or cuts one piece (to get two tied for largest pieces).
If she passed, then let agents 3, 2 pick in this order, and then, 1.

(3) If agent 2 did cut, then let 3, 2 pick (in this order), but require 2 to take the cuted piece (unless 3 did) and then, 1. 
Keep the cuts unallocated for now (notice that the partial allocation is envy-free).

(4) Now divide the cuts. 
Whoever of 2 and 3 received the uncuted piece does the cutting. 
Let agents choose in this order: non-cutter, 1, cutter.

\bigskip

Evaluate the protocol in term of number of cuts.
}


\begin{thebibliography}{1}

\bibitem{Barbanel}
J. Barbanel and S. Brams.
\newblock {C}ake division with minimal cuts: envy-free procedures for 3 persons, 4 persons and beyond.
\newblock NY university, 2004.

\bibitem{Dublins}
L. Dubins and E.H. Spanier. 
\newblock How to Cut a Cake Fairly. 
\newblock American Mathematical Monthly, 68(1):1–17, 1961.

\bibitem{Endriss}
U. Endriss. 
\newblock Lecture Notes on Fair Division. 
\newblock ILLC, University of Amsterdam, 2009.

\bibitem{Robertson}
J. Robertson and W. Webb.
\newblock {C}ake-{C}utting Algorithms: Be Fair If You Can.
\newblock Natick, Massachusetts: A. Peters, 1998.

\bibitem{Stromquist}
W. Stromquist.
\newblock How to Cut a Cake Fairly. 
\newblock American Mathematical Monthly, 87(8):640–644, 1980.

 
\end{thebibliography}{}
