
\chapter{COUNTING}

\section{Elementary counting}

\subsection{Binary Strings and Power Sets}

\begin{prop}
\label{thm:b-ary strings}
For every integer $b > 1$, there are $b^n$ $b$-ary strings of length
$n$.
\end{prop}

\begin{proof}
The asserted numeration follows most simply by noting that there are
always $b$ times as many $b$-ary strings of length $n$ as there are of
length $n-1$.  This is because we can form the set of $b$-ary strings
of length $n$ as follows.  Take the set $A_{n-1}$ of $b$-ary strings
of length $n-1$, and make $b$ copies of it, call them $A^{(0)}_{n-1},
A^{(1)}_{n-1}, \ldots, A^{(b-1)}_{n-1}$.  Now, append $0$ to every
string in $A^{(0)}_{n-1}$, append $1$ to every string in
$A^{(1)}_{n-1}$, \ldots, append $\bar{b} = b-1$ to every string in
$A^{(b-1)}_{n-1}$.  The thus-amended sets $A^{(i)}_{n-1}$ are mutually
disjoint (because of the terminal letters of their respective
strings), and they collectively contain all $b$-ary strings of length
$n$.  \qed
\end{proof}

\medskip

\addcontentsline{toc}{paragraph}{-- A fun result: $n$-element sets
  have $2^n$ subsets}

\begin{prop}
\label{thm:power-sets}
The power set $\p(S)$ of a finite set $S$ contain $2^{|S|}$ elements.
\end{prop}

\begin{proof}
Let us begin by taking an arbitrary finite set $S$---say of $n$
elements---and laying its elements out in a line.  We thereby
establish a correspondence between $S$'s elements and positive
integers: there is the first element, which we associate with the
integer $1$, the second element, which we associate with the integer
$2$, and so on, until the last element along the line gets associated
with the integer $n$.

Next, let's note that we can specify any subset $S'$ of $S$ by
specifying a length-$n$ {\em binary (i.e., base-$2$) string}, i.e., a
string of $0$'s and $1$'s.  The translation is as follows.  If an
element $s$ of $S$ appears in the subset $S'$, then we look at the
integer we have associated with $s$ (via our linearization of $S$),
and we set the corresponding bit-position of our binary string to $1$;
otherwise, we set this bit-position to $0$.  In this way, we get a
distinct subset of $S$ for each distinct binary string, and a distinct
binary string for each distinct subset of $S$.

Let us pause to illustrate our correspondence between sets and strings
by focussing on the set $S = \{a,b,c\}$.  Just to make life more
interesting, let us lay $S$'s elements out in the order $b,a,c$, so
that $b$ has associated integer $1$, $a$ has associated integer $2$,
and $c$ has associated integer $3$.  We depict the elements of $\p(S)$
and the corresponding binary strings in the following table.
\begin{center}
\fbox{
\begin{tabular}{c|c|c}
Binary string & Set of integers & Subset of $S$ \\
\hline
$000$ & $\emptyset$ & $\emptyset$ \\
$001$ & $\{3\}$     & $\{c\}$ \\
$010$ & $\{2\}$     & $\{a\}$ \\
$011$ & $\{2,3\}$   & $\{a,c\}$ \\
$100$ & $\{1\}$     & $\{b\}$ \\
$101$ & $\{1,3\}$   & $\{b,c\}$ \\
$110$ & $\{1,2\}$   & $\{a,b\}$ \\
$111$ & $\{1,2,3\}$ & $\{a,b,c\} =S$
\end{tabular}
}
\end{center}

Back to the Proposition: We have verified the following: {\em The
  number of length-$n$ binary strings is the same as the number of
  elements in the power set of $S$!}  The desired numeration thus
follows by the ($b=2$) instance of Proposition~\ref{thm:b-ary
  strings}.  \qed
\end{proof}

\begin{quote}
The binary string that we have constructed to represent each set of
integers $N \subseteq \{0, 1, \ldots, n-1\}$ is called the {\it
(length-$n$) characteristic vector}\index{characteristic vector}
{\it of the set} $N$.  Of course, the finite set $N$ has
characteristic vectors of all finite lengths.  Generalizing this idea,
{\em every} set of integers $N \subseteq \N$, whether finite or
infinite, has an {\em infinite} characteristic vector, which is formed
in precisely the same way as are finite characteristic vectors, but
now using the set $\N$ as the base set.
\end{quote}






 within discrete frameworks, including introducing
discrete probability/likelihood as a ratio:
\[ 
\frac{\mbox{number of targeted events}}{\mbox{number of possible events}}
\]
