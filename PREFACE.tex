%version of 04-16-19 

\chapter*{PREFACE}

The goal of this book is to endow the reader with an operational
understanding of an activity that we, hopefully evocatively, call
``{\em doing}'' mathematics.  We want the reader to be able to
recognize situations---especially relating to the world of
computing---wherein mathematical reasoning and analysis can make a
positive difference.  Of course, a large component of the process of
enabling this activity resides in the imparting of mathematical
knowledge of the fundamental concepts and tools and intellectual
artifacts that our mathematical forebears have given us access to.
Another part, though, requires transmitting the understanding of how
to use the knowledge effectively and creatively.

Our goal in writing this book is for each chapter to provide the early
student of mathematics with at least one new tool for progressing
toward {\em operational} level of conceptual and methodological
understanding of the discrete mathematics that is useful for studying
and understanding both the activity of computing and the systems that
enable that activity.

Our approach toward achieving our goal is to present discrete
mathematics to the reader in a somewhat novel way.
\begin{itemize}
\item
We often present several fundamentally different proofs for the same
result.  With our tongues only mildly impinging on our cheeks, we
encapsulate the philosophy underlying this practice in the following
``self-evident truth'' (or, ``axiom'').  \index{The conceptual axiom}
\begin{quote}
{\bf The conceptual axiom}.
\index{conceptual axiom}
{\em
One's ability to think deeply about a complicated concept is always
enhanced by having more than one way to think about the concept.}
\end{quote}

An extreme example of this stratagem resides in our multiple
derivations of the sum of the first $n$ positive integers.  The reader
will encounter (in Chapter~\ref{ch:Summation}) derivations that adopt
the following worldviews:
  \begin{enumerate}
  \item
View each positive integer $k$ as a height-$k$ unit-width rectangle
  \item
View each positive integer $k$ as a collection of $k$ tokens (pebbles)
that can be aggregated in various ways
  \item
View the problem ``textually'', by writing out the summation
symbolically and manipulating the resulting string
  \item
View the problem geometrically, by drawing a figure and calculating
its area
  \item
View the integers via their familiar names (say, in decimal notation),
and calculate their sum by replicating, rearranging, and counting the
number of occurrences of different numbers
  \end{enumerate}
While the number of derivations of this result is greater than our
norm, the {\em fact} of exploiting multiple viewpoints is a hallmark
of the text.

\item
While this book is unquestionably a text on {\em mathematics}, we
never overlook insights that mathematics can provide as one strives to
understand {\em computing}-related concepts, tools, and systems.
Traces of this focus are visible, for instance, in
  \begin{enumerate}
  \item
Chapter~\ref{Ch:Graphs-Trees}, as we discuss aspects of graphs and
networks that are particularly relevant to topics such as:
     \begin{itemize}
     \item
social networks
     \item
the interconnection networks of parallel computer architectures
     \item
the design of integrated electronic circuits
     \end{itemize}
  \item
Chapter~\ref{ch:Recurrences}, where we discuss in great detail
material about recurrences that is particularly important in the
design and analysis of algorithms
  \item
Chapters~\ref{ch:numbers-numerals},~\ref{ch:numbers-advanced},
and~\ref{ch:numerals}, where we develop material concerning numbers
and their representations that underlies much of our world of
computing:
     \begin{itemize}
     \item
how to design an adder that does not expend time ``rippling carries''
     \item
how to efficiently encode complicated structures as integers

There are applications to security lurking in this subject.
     \item
how numbers and numerals can model {\em self-referentiality} in
languages

Familiar {\it paradoxes} arise from self-referentiality---but so also
does unavoidable {\em computational complexity}.
     \end{itemize}
  \end{enumerate}
\item
As we remarked at the end of our {\it Manifesto}, mathematics is
done---created, augmented, analyzed---by humans.  It is both inspiring
and illuminating to realize the variety of backgrounds, goals, and
motivations that our mathematical forebears had.  We sprinkle
digressions and stories throughout our text in the hope of inspiring
the reader, as well as to stimulate thinking and understanding.  The
digressions will also provide momentary diversions to help the reader
approach the ensuing material reinvigorated.

\item
Finally, we present a number of rather short and informal introductory
essays on material that goes beyond the level of mathematical
achievement that we expect many of our readers to start out with.  By
providing citations to further reading in these essays, we hope to
facilitate further exploration into these topics.
\end{itemize}

\bigskip

\hfill Falmouth, Massachusetts

\hfill Grenoble, France



