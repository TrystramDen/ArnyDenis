%version of 05-14-19 

\chapter*{PREFACE}

We live in a world of increasing professional and social diversity.
Computing curricula have responded to these changes by dramatically
changing focus and scope.

Early computation-oriented academic programs focused tightly on the
principles and practices related to designing and using digital
computers.  Discrete mathematics was ``automatically'' prepended to
such programs, in recognition of the fact that every aspect of the
design and use of computers built upon mathematical concepts and
tools.  Today, these traditional programs have been joined by a broad
range of curricula which focus on the computational concerns of {\em
  consumers} of the craft of computing: scientists or engineers;
scholars in the humanities or social studies; practitioners in
professions such as business, law, or medicine; \ldots; members of the
general population whose interests in computing are avocational.
Within this new world, many aspiring users of computers need only
specialized knowledge of mathematics in order to compute successfully.
But, the amount and direction of the specialization can vary greatly
based on a student's, or an academic program's, focus.

\medskip

The number of nontraditional students seeking technical education is
growing, and the ways in which their nontraditional goals and
backgrounds manifest themselves is increasing.  We have striven to
keep this diversity in mind as we have written this text.  We have
included a broad range of material, in both subject and level.

\bigskip

Our goal in this book is to endow each reader with an operational
understanding of an activity that we, hopefully evocatively, call
``{\em doing}'' mathematics.  We want the reader to be able to
recognize situations---especially relating to the world of
computing---wherein mathematical reasoning and analysis can make a
positive difference.  Of course, a large component of the process of
enabling this activity resides in the imparting of mathematical
knowledge of the fundamental concepts and tools and intellectual
artifacts that our mathematical forebears have given us access to.
Another part, though, requires transmitting the understanding of how
to use the knowledge effectively and creatively.

We have striven to have each chapter provide the early student of
mathematics with at least one new tool for progressing toward {\em
  operational} level of conceptual and methodological understanding of
the discrete mathematics that is useful for studying and understanding
both the activity of computing and the systems that enable that
activity.

Our approach toward achieving our goal is to present discrete
mathematics to the reader in a somewhat novel way.
\begin{itemize}
\item
{\em We often present several fundamentally different proofs for the
  same result.}

With tongues only mildly impinging on our cheeks, we encapsulate the
philosophy underlying this practice in the following ``self-evident
truth'', or, ``axiom''.  \index{The conceptual axiom}
\begin{quote}
{\bf The conceptual axiom}.
\index{conceptual axiom}
{\em
One's ability to think deeply about a complicated concept is always
enhanced by having more than one way to think about the concept.}
\end{quote}

An extreme example of this stratagem resides in our multiple
derivations of the sum of the first $n$ positive integers.  The reader
will encounter (in Chapter~\ref{ch:Summation}) derivations that adopt
each of the following worldviews:
  \begin{enumerate}
    \item
View the problem ``textually'', by writing out the summation
symbolically and manipulating the resulting string
  \item
View each positive integer $k$ as a height-$k$ unit-width rectangle
  \item
View each positive integer $k$ as a collection of $k$ tokens (pebbles)
that can be aggregated in various ways
  \item
View the process of summation as a combinatorial
object---specifically, the number of ways of selecting $2$ items from
a set of $n$ items
  \item
View the problem geometrically, by drawing a figure and calculating
its area
  \item
View the integers via their familiar names (say, in decimal notation),
and calculate their sum by replicating, rearranging, and counting the
number of occurrences of different numbers
  \end{enumerate}
While the number of derivations of this result exceeds our norm, the
{\em fact} of exploiting multiple viewpoints is a hallmark of our
approach.

\item
{\em We organize discrete mathematical topics in conceptual layers.}

Our treatment of the fundamental topic ``number'' illustrates this
strategem well.
  \begin{enumerate}
  \item
We describe the number system that we have inherited from our
mathematical forebears in Chapter~\ref{ch:numbers-numerals}.  The
treament in this chapter is mostly discursive but is peppered with
several important results and computational insights.
  \item
In Chapter~\ref{ch:numbers-advanced}, we take a more careful look
within our number system, addressing topics such as prime
numbers---and their use within encoding schemes and security.  In
order to do justice to this second layer of material, we insert
Chapters~\ref{ch:arithmetic} and~\ref{ch:Summation}, which cover
arithmetic and summations, respectively, between
Chapters~\ref{ch:numbers-numerals} and~\ref{ch:numbers-advanced}.
  \item
Finally, we turn in Chapter~\ref{ch:numerals} to the issue of how we
represent numbers in order to compute with them.  Once again, we
insert auxiliary material---in this case,
Chapter~\ref{ch:Recurrences}, which covers recurrences---between
number-oriented chapters, to lay the groundwork for material related
to systems of number representation.
  \end{enumerate}

%{\Denis why only the following chapters? I suggest to add few words about the basic materials -- chapters 3 sets and their algebra, 
%chapter 4 arithmetic, the status of chapter 5 is more ambiguous, not really basic material, targeting summations that are 
%a fundamental material for estimating the cost of algorithms.
%Another remark, the list starts by chapter 10 (the last one...)}
\item
{\em We continually point out mathematical insights into
 {\em computing-related} concepts, tools, and systems.}

While this book is unquestionably a text on {\em mathematics}, we
never lose sight of the computational motivation for our work.  Traces
of this focus are visible, for instance, in:
  \begin{itemize}
  \item
Chapters~\ref{ch:numbers-numerals},~\ref{ch:numbers-advanced},
and~\ref{ch:numerals}, where we develop material concerning numbers
and their representations that underlies much of the world of
computing:
     \begin{itemize}
     \item
how to efficiently encode complicated structures as integers

There are applications to security and cryptography lurking in this
subject.
     \item
how numbers and numerals can model {\em self-referentiality} in
languages

The consequences of self-referentiality abound throughout philosophy,
linguistics, and the foundations of computing.  The property leaves
its mark on computing-related topics ranging from programming
languages and compilers to the inherent complexity of broad families
of computations.
     \item
how to design an adder that does not expend time ``rippling carries''
     \end{itemize}
  \item
Chapter~\ref{ch:Recurrences}, where we discuss in great detail
material about recurrences that is particularly important in the
design and analysis of algorithms
  \item
Chapter~\ref{ch:Graphs-Trees}, as we discuss aspects of graphs and
networks that are particularly relevant to topics such as:
     \begin{itemize}
     \item
social networks
     \item
the interconnection networks of parallel computer architectures
     \item
the design of integrated electronic circuits
     \end{itemize}
  \end{itemize}

\item
{\em We sprinkle stories into the text, in the hope of inspiring the
  reader, while stimulating mathematical thinking and understanding.}

As we remarked at the end of our {\it Manifesto}, mathematics is
done---created, augmented, analyzed---by humans.  It is both inspiring
and illuminating to realize the variety of backgrounds, goals, and
motivations that our mathematical forebears had.  The digressions will
also provide momentary diversions to help the reader approach the
ensuing material reinvigorated.

\item
{\em We present a number of rather short and informal introductory
  essays on advanced material.}

We expect that much of this material will go beyond the level of
mathematical achievement that most readers will start out with.  We
hope that the citations we provide to accompany these essays will
inspire further reading, and often even further exploration into these
advanced topics.
\end{itemize}

\bigskip

\hfill Falmouth, Massachusetts

\hfill Grenoble, France



