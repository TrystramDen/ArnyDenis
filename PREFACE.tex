%version of 04-11-19 

\chapter*{PREFACE}

The goal of this book is to endow the reader with an operational
understanding of the activity we, hopefully evocatively, call
``doing'' mathematics.  We want the reader to be able to recognize
situations---especially within the world of computing---wherein
mathematical reasoning and analysis can make a positive difference.
Of course, part of enabling this activity resides in the imparting of
mathematical knowledge: what are the basic concepts and intellectual
artifacts that our mathematical forebears have given us access to.
Another part, though, requires transmitting the understanding of how
to use the knowledge effectively and ceatively.

**HERE

  A vast array of formal aids for the activities of
designing, analyzing, utilizing, and verifying computer systems has
been developed.  And, mathematical tools have always been at or near
the forefront of such aids.

Our goal for each chapter and section of this book is to endow every
reader with at least one new tool for achieving an {\em operational} level
of conceptual and methodological understanding of the discrete
mathematics that is used to study and understand both the activity of
computing and the design of the systems that enable that activity.  We
construe an ``operational'' level of understanding to be one that
enables the reader to ``do'' the relevant mathematics.







**ADD SHORT PARAGRAPH FOR EACH POINT

\begin{itemize}
\item
We aim for UNDERSTANDING rather than just knowledge
\item
We want to help students/readers THINK rather than just learn —
which explains multiple proofs
\item
We want to expose the reader to some culture (as an aid to thinking
and understanding) — which explains some digressions and stories
\item
Part of the culture comes via the $\oplus$ sections … even if they are
rather short and informal.
\end{itemize}




