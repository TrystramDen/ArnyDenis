%version of 10-28-19

\chapter*{Preface}

We live in a world of increasing professional and social diversity.  Computing curricula have responded to these changes by dramatically changing focus and scope.

\smallskip

Early computation-oriented academic programs focused narrowly on the principles and practices related to designing and using digital computers.  Discrete mathematics was ``automatically'' prepended to such programs, in recognition of the fact that every aspect of the design and use of computers built upon mathematical concepts and tools.  Today, these traditional programs have been joined by a broad range of curricula which focus on the computational concerns of an equally broad range of {\em consumers} of the craft of computing: from those interested in fields that touch upon technology, such as scientists or engineers, to those interested in nontechnological pursuits, such as scholars in the humanities or social studies, to practitioners in professions such as business, law, or medicine, and so on, all the way to members of the general population whose computing-related interests are avocational.  Within this new world, many aspiring users of computers need only very specialized knowledge of mathematics in order to compute successfully.  But, the amount and direction of the specialization can vary greatly
based on a student's, or an academic program's, focus; therefore, absent a broad background in mathematics, a student may lack the technical tools necessary to conceive of and to pursue truly bold novel ideas.

\bigskip

The number of nontraditional students seeking technical education is growing, and the ways in which their nontraditional goals and backgrounds manifest themselves is increasing.  We have striven to keep this diversity in mind while writing this text.  We have included a broad range of material, in both subject and level.  And, we have tried throughout to expound on the fundamentals of mathematical ideas, unadorned by superficial (from a mathematical perspective) details that arise in specific applications.

\bigskip

{\bf Our goal in this book is to endow each reader with an operational understanding of an activity that we---hopefully evocatively---call ``{\em doing}'' mathematics.}  We want the reader to be able to recognize situations---especially, but not exclusively, relating to the world of computing---wherein mathematical reasoning and analysis can make a positive difference.  Of course, a large component of enabling this activity requires imparting mathematical knowledge relating to the fundamental concepts and tools and intellectual artifacts that we have inherited from our mathematical forebears.  Another part, though, requires transmitting the understanding of how to use the knowledge effectively and creatively.

\smallskip

We have striven to design the text so that each chapter adds new computing-related mathematical concepts and techniques to the reader's toolkit.  We have designed each chapter to develop mathematical tools {\em in multiple ways}, with the hope that each reader, no matter what her goals, insights, and interests, will learn how to think mathematically and to achieve an {\em operational} understanding of both the activity of computing and the systems that enable that activity.

\smallskip

Our approach toward achieving our goal is embodied in the previous paragraph's promise to develop mathematical tools ``{\em in multiple ways}".
\begin{itemize}
\item
{\em We often present several fundamentally different proofs for the same result.}

\smallskip

\index{The conceptual axiom}
With tongues only mildly impinging on our cheeks, we encapsulate the philosophy underlying this practice in the following ``self-evident truth'', or, ``axiom''. 

\medskip

\hspace*{.2in}\begin{tabular}{l}
{\bf The conceptual axiom}.
{\em One's ability to think deeply about a complicated } \\
{\em concept is always enhanced by having more than one way to think about} \\
{\em the concept.}
\end{tabular}

\medskip

An extreme illustration of this stratagem resides in our multiple derivations of the sum of the first $n$ positive integers.  The reader will encounter (in Chapter~\ref{ch:Summation}) derivations that adopt each of the following worldviews:
  \begin{enumerate}
    \item
View the problem ``textually'', by writing out the summation symbolically and manipulating the resulting string.
  \item
View each positive integer $k$ as a height-$k$ unit-width rectangle.
  \item
View each positive integer $k$ as a collection of $k$ tokens (pebbles) which can be aggregated and manipulated in a variety of ways,
  \item
View the process of summation as a combinatorial object---specifically, the number of ways of selecting $2$ items from a set of $n$ items.
  \item
View the problem geometrically, by drawing a figure and calculating its area.
  \item
View the integers via their familiar names (say, in decimal notation), and calculate their sum by replicating, rearranging, and counting the number of occurrences of different numbers.
  \end{enumerate}
While the {\em number} of derivations of this result exceeds our norm, the {\em fact} of exploiting multiple viewpoints is a hallmark of our approach.

\item
{\em We organize discrete mathematical topics in conceptual layers.}

\smallskip

Our treatment of the fundamental topic ``number'' illustrates this stratagem well.
  \begin{enumerate}
  \item
We describe the number system that we have inherited from our mathematical forebears in Chapter~\ref{ch:numbers-numerals}.  The treatment in this chapter is mostly discursive but it is peppered with a number of important results and computational insights.
  \item
In Chapter~\ref{ch:numbers-advanced}, we take a more careful look inside our number system, addressing topics such as prime numbers---and their use within encoding schemes and security.  In order to do justice to this deeper layer of material, we insert Chapters~\ref{ch:arithmetic} and~\ref{ch:Summation}, which cover arithmetic and summations, respectively, between
Chapters~\ref{ch:numbers-numerals} and~\ref{ch:numbers-advanced}.
  \item
Finally, we turn in Chapter~\ref{ch:numerals} to the issue of how we represent numbers in order to compute with them.  Here again, we insert auxiliary material---in this case, Chapter~\ref{ch:Recurrences}, which covers recurrences---between number-oriented chapters, to lay the groundwork for material related to systems of number representation.
  \end{enumerate}

\item
{\em We continually point out mathematical insights into {\em computing-related} concepts, tools, and systems.}

\smallskip

While this book is unquestionably a text on {\em mathematics}, we never lose sight of the computational motivation for our work.  Traces of this motivation are visible, for instance, in:
  \begin{itemize}
  \item
Chapters~\ref{ch:numbers-numerals},~\ref{ch:numbers-advanced}, and~\ref{ch:numerals}, where we develop material concerning numbers and their representations that underlies much of the world of computing:
     \begin{itemize}
     \item
how to efficiently encode complicated structures as integers

\smallskip

There are applications to security and cryptography lurking in this subject.
     \item
how numbers and numerals can model {\em self-referentiality} in languages

\smallskip

The consequences of self-referentiality abound throughout philosophy, linguistics, and the foundations of computing.  The property leaves its mark on computing-related topics ranging from programming languages and compilers to the inherent complexity of broad families of computations.
     \item
how to design an adder that does not expend time ``rippling carries''

\smallskip

This is basically an engineering-oriented concern, but its resolution involves important mathematical ideas.
     \end{itemize}
  \item
Chapter~\ref{ch:Recurrences}, where we discuss in detail material about recurrences that is of crucial importance in the design and analysis of algorithms
  \item
Chapters~\ref{ch:Graphs1} and~\ref{ch:Graphs2}, where we discuss aspects of graphs and networks that are particularly relevant to topics such as:
     \begin{itemize}
     \item
social networks
     \item
the interconnection networks of parallel computer architectures
     \item
the design of integrated electronic circuits
     \end{itemize}
  \end{itemize}
  
\item
{\em We expound on explicit connections between the mathematics we develop here and the computational topics that have informed our choices of material.}

\smallskip

Section~\ref{sec:how-to-use} discusses how to excerpt from this book within courses that cover a broad range of computation-related topics, from {\em Algorithms} to {\em Digital Design} to {\em Programming Systems} to {\em Social Media}, and beyond.

\item
{\em We sprinkle stories into the text, in the hope of inspiring the reader, while stimulating mathematical thinking and understanding.}

\smallskip

As we remarked at the end of our {\it Manifesto}, mathematics is done---created, analyzed, applied---by humans.  It is both inspiring and illuminating to realize the variety of backgrounds, goals, and motivations that our mathematical forebears had.  The digressions also provide momentary diversions to help the reader approach the ensuing material reinvigorated.

\item
{\em We present a number of short, informal introductory essays on advanced topics.}

\smallskip

We expect that much of this material will go beyond the level of mathematical achievement that most readers will start out with.  We hope that the citations we provide to accompany these essays will inspire further reading, and often even further mathematical exploration.

\smallskip

In furtherance of our hope to excite the reader about further involvement with mathematics, we annotate the citations in our {\it References} section with tags that identify the roles that individual references play within the mathematical story that we are telling.
  \begin{itemize}
  \item
Citations tagged ``{\bf [HC]}" are identified as being of {\em historical} and/or {\em cultural} interest.  They expose how our mathematical forebears thought about the subjects being discussed and, thereby, how mathematical thinking has evolved since their writing.  Some of these references are, in fact, modern histories; others are (translations of) original texts.  It can be fascinating to read about Leonardo Pisano's interest in the demographics of rabbits and about John Arbuthnot's curiosity about the relative birth rates of the sexes in the London of his day.

  \item
Citations tagged ``{\bf [T]}" point to advanced and/or specialized textbooks.  They will lead the interested reader beyond the introductory level of this text.  For instance, our description in Appendix E of ``carry-free" addition can be a stepping stone to a study of the details of computer arithmetic.

  \item
Untagged citations identify ``modern" expositions of the material we describe in the text.  They also usually will lead the interested reader beyond the introductory level of this text.  Indeed, they are often quite technical companions to quite dramatic human interest stories.
  \end{itemize}

\item
{\em The text offers many (clearly labeled) opportunities for enrichment within the body of the text and within the exercises that accompany each chapter.}

  \begin{itemize}
  \item
Throughout the text, the symbols ``$\oplus$" and ``$\oplus \oplus$" identify sections that are either more advanced than the body of the text or targeted at a more specialized audience.
  \item
In similar ways, the symbols ``$\oplus$" and ``$\oplus \oplus$" identify exercises that are more challenging than their untagged kin.  We provide hints for the single-$\oplus$ problems, and we provide solutions for the double-$\oplus$ problems, in Chapter~\ref{ch:Exercises}: {\it Solutions to Selected Exercises}.
  \end{itemize}
\end{itemize}

\bigskip

\noindent {\bf The Authors}

\medskip

\noindent
{\bf Research.}
The authors have a long cumulative history of research on

\noindent \begin{tabular}{ll}
{\sc Mathematics:} &
Discrete math and Mathematical logic \\
{\sc Computer Science:} &
Algorithms, Computer architecture, Computation \\ 
   & theory, Data analytics, Information retrieval, \\
   & Resource management, Task scheduling \\
{\sc Computer Engineering:} &
Digital logic design, Energy conservation, \\
  & Fault tolerance, Testing of digital logic
\end{tabular}

\medskip

\noindent
They have cumulatively published more than 450 research articles in elite journals and conferences, as well as several research books and a few textbooks on subjects relating to
mathematics and/or computing.

\bigskip

\noindent {\bf Teaching.}
Over a cumulative span approaching 3/4 century, the authors have taught courses ranging from mathematics and mathematical logic to a broad spectrum of topics in computer science and engineering at the following institutions (listed alphabetically):

\smallskip

\noindent
Duke University, 
University of Grenoble,
University of Massachusetts,
University of Paris,
New York University (NYU),
Polytechnic Institute of New York,
The Technion (Israel Institute of Technology),
Tunis Scientific University,
University of Toronto,
Yale University

\medskip

\noindent
The courses taught by the authors have included:

\smallskip

\noindent
Algorithms,
Computation and Complexity, Approximation Theory,
Computer Architecture,
Mathematical Logic,
Research Methodology,
VLSI Design \\
in addition to numerous specialized seminars

\bigskip
\bigskip

\noindent
We owe a tremendous debt to the professional colleagues and the students, both undergraduate and graduate, who have enriched our lives over the decades.  We avoid attempting to list their names for fear of inadvertently omitting some.

\smallskip

Paraphrasing the many greats over the centuries who have acknowledged their debt to their predecessors,\footnote{See R.K.~Merton,  {\it On the Shoulders of Giants: A Shandean Postscript}.  The Free Press, NY.}~we have truly ``seen farther by standing on the shoulders of giants"!

\bigskip

\hfill \begin{tabular}{lll}
Arnold L.~Rosenberg & \hspace*{.1in} & Denis Trystram \\
{\em Falmouth} & \hspace*{.1in} & {\em Grenoble}
\end{tabular}
