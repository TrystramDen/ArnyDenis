
\chapter{A nice property on GCD}
\label{Appendix:FiboGCD}

This is a draft, in particular, the notation should be put in coherency with the other chapters
(F(n) instead of $F_n$)
\bigskip

Let $F_1=F_2=1$ and $F_{n+1}=F_n+F_{n-1}$.
\medskip

\noindent We want to prove that $GCD(F_n,F_m) = F_{GCD(n,m)}$.
{\Denis put it like a proposition?}

Without loss of generality, consider that $n \geq m$.

Check the property on particular cases (gain intuition)...

Let us denote $g=GCD(n,m)$ and $G=GCD(F_n,F_m)$.

As an example, let us check 

$GCD(F_{12},F_{18}) = GCD(144,2584) = 8 = F_6$ (where $6=GCD(12,18)$).
\medskip

This result is obtained by showing first that $F_g$ divides $G$ and then, $G$ divides $F_g$.

Let us first prove three technical lemmas.
\\

\noindent {\bf Lemma 1.}
The following relation holds for any integers $n$ and $k$:

$F_{n+k} = F_k.F_{n+1} + F_{k-1}.F_n$ \footnote{this relation assumes that we are able to define negative Fibonacci numbers. 
Well, there is a "natural" way of extending
the definition to negative numbers...}

The proof is straightforward  by induction on $k$ assuming that $n$ is fixed.
\\

\noindent {\bf Lemma 2.}
For any integer $k$ $F_{k.n}$ is a multiple of $F_n$.

The proof can be obtained as a consequence of the previous lemma.
\\

\noindent {\bf Lemma 3.}
If $a$ divides $b$ then $F_a$ divides $F_b$.
\\

We are able now to prove the final result:
\begin{enumerate}
\item $F_g$ divides $G$.

By definition of the GCD, $g$ divides $n$ and $m$. 
Using Lemma 3, that means that $F_g$ divides both $F_n$ and $F_m$.
thus, it divides their GCD.
\item $G$ divides $F_g$.

As $g$ is the GCD of $n$ and $m$, it can be written as a linear combination of them (in fact it is the smallest one):
$g = a.n + b.m$.

$F_g = F_{a.n + b.m}$ for some integers $a$ and $b$,
thus, according to lemma 1, it is a multiple of $n$ (and symmetrically of $m$).
Thus, it is a multiple of their GCD. 
\end{enumerate}


