\documentclass{article}[12pt]

\begin{document}

\begin{center}
PROPOSAL 

{\Large\bf\em UNDERSTAND MATHEMATICS, UNDERSTAND COMPUTING}
\end{center}

\begin{center}
\begin{tabular}{ccc}
{\large Arnold L.~Rosenberg} & & {\large Denis Trystram} \\
Distinguished Univ.~Professor Emeritus
  & & Distinguished Professor \\
University of Massachusetts  & & University of Grenoble Alpes \\
Amherst, MA 01003, USA       & & Grenoble, FRANCE \\
{\small\tt rsnbrg@cs.umass.edu} & & {\small\tt denis.trystram@imag.fr}
\end{tabular}
\end{center}

\section*{The Authors}

\subsection*{Research}

The authors have cumulatively published more than 300(?) research articles in elite journals on the areas of
\begin{itemize}
\item
Mathematics (discrete math and mathematical logic)
\item
Computer Science (algorithms, computation theory, architecture)
\item
Computer Engineering (digital logic design, fault tolerance, logic testing)
\end{itemize}
They have also a (?) research book on graph separators and their applications.

\subsection*{Education}

Over a cumulative span approaching 3/4 century, the authors have taught at the following universities (listed alphabetically).
\begin{itemize}
\item
Duke Univ.
\item
Univ. Grenoble
\item
Univ. Massachusetts
\item
New York University
\item
Univ. Paris
\item
Polytechnic Inst. of New York
\item
The Technion (Israel Inst. of Technology)
\item
Univ. Toronto
\item
Yale Univ.
\end{itemize}

The courses taught include---not listing special topics:
\begin{itemize}
\item
Algorithms
\item
Computation and Complexity Theory
\item
Computer Architecture
\item
Mathematical Logic
\item
VLSI Design
\end{itemize}

\section*{The Book: What Is Special}

Focus is not just on {\em learning topics in mathematics} but in learning {\em how to ``do mathematics"} --- how to {\em think like a mathematician} in terms of {\em problem solving} and {\em formulating ``real" problems mathematically}.  Here are some steps we take:
\begin{itemize}
\item
We have a chapter entitled ``Doing Mathematics".  We are not claiming that this is unique.  But this is one of the steps that we take.
\item
We provide numerous rather different explanations---including proofs---when we feel that the reader may benefit from learning multiple ways to think about the same issue.
\item
We strive to transmit important pieces of mathematical culture by telling stories that expose how mathematical greats of the past came to their discoveries.
\item
We go ``beyond the basics" so that the reader who is fascinated by a particular topic can 		test the waters" with more advanced and/or specialized material than one typically encounters in an introductory course.  We even include a small number of essays---in the appendix---that really give a peek into specialized topics.
\item
We have partitioned our references and our index to help the reader find {\em historical} material and references that are more {\em cultural} than technical (Euclid, Al-Khwarizmi).
\end{itemize}


\begin{itemize}
\item
Topics chosen not only from ``standard" syllabi, but also from our personal experience as researchers.
\item
Topics are organized with an eye toward the mathematical tools needed to master them.

Talk about the sequence: \\
NUMBERS1 \\
ARITHMETIC \\
SUMMATION \\
INFINITE \\
NUMBERS2 \\
RECURRENCES \\
NUMBERS3
\end{itemize}

\end{document}