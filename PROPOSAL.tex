\documentclass{article}[12pt]

\begin{document}

\begin{center}
PROPOSAL 

\bigskip

{\large\bf\em UNDERSTAND MATHEMATICS, UNDERSTAND COMPUTING}
\end{center}

\begin{center}
\begin{tabular}{ccc}
{\large Arnold L.~Rosenberg} & & {\large Denis Trystram} \\
Distinguished Univ.~Professor Emeritus
  & & Distinguished Professor \\
University of Massachusetts  & & University of Grenoble Alpes \\
Amherst, MA 01003, USA       & & Grenoble, FRANCE \\
{\small\tt rsnbrg@cs.umass.edu} & & {\small\tt denis.trystram@imag.fr}
\end{tabular}
\end{center}

\bigskip

\noindent
The authors propose a book on discrete mathematics.
\begin{itemize}
\item
{\em The motivating target audience is beginning university students.}  The presentation also provides material that would be suitable for ``advanced beginners".

\smallskip

Thereby, the book is suitable as a beginning text on discrete mathematics for students with a broad range of backgrounds, and it can also be used for a broad range of intermediate-level university students.

\item
{\em The book focuses squarely on discrete mathematics as a topic of study.}  However, the authors have called upon their decades-long experience as researchers in discrete mathematics, computer science, and computer engineering to inform their {\em choice of topics and examples} and the {\em organization of the material}.

\smallskip

That said, while the book can be used as {\em preparation} for the study of computational topics, it is {\em not an introduction} to such material: The authors have diligently selected mathematical material whose application goes beyond any single focus, even one as broad as computation.

\item
{\em The authors' emphasis on teaching students how to \underline{do mathematics} is discernible in certain unique features in the presentation of material.}  To cite just three important features: 
  \begin{itemize}
  \item
The first technical (i.e., non-introductory) chapter is entitled {\em Doing Mathematics}.  This chapter describes and amply illustrates topics such as: {\em the nature of mathematical argumentation}, including techniques of proof; {\em the centrality of representation} for the successful manipulation and analysis of mathematical concepts.

  \item
Many core topics are explained in multiple ways, using quite distinct representations and forms of reasoning.  (One particular summation is evaluated in five completely distinct ways.)

\smallskip

This pedagogical stratagem reflects a desire to empower readers to seek a way of thinking about topics that is congenial to them as individuals!  The authors' long history of doing mathematical research is full of examples of significant theorems and calculations that long evaded successful solution until the investigators discovered the ``right" way to think about a particular aspect of the target problem.

  \item
The book develops the all-important concept of ``number" via a sequence of {\em seven}intellectually interwoven chapters:
   \begin{enumerate}
   \item       
The first chapter describes {\em the basics of our number system}, including the historical needs that led to each successive augmentation of the system.  The differences in properties among the various components of our system (integers vs.~rationals vs.~reals vs.~complexes) are integrated with a discussion of topics such as order and divisibility.
  \item
The second chapter ``puts the numbers to work" by means of {\em arithmetic operations and important classes of numerical functions}.
  \item
The third chapter develops notions relating to {\em summation}, including techniques for exactly and approximately summing finite and infinite series.
  \item
The fourth chapter expands on a very important topic that rears its head when one works with infinite summations---{\em how to deal with infinity}, e.g.,  with the ``point at infinity'' and with infinite series.  The crucial topic of {\em asymptotics} is an important component of ``dealing with infinity".  The chapter closes with a discussions of the various types of paradoxes that arise when one works with the infinite.
  \item
The fifth chapter resumes the study of numbers.  Building on its predecessors, this chapter exposes the intricacies of: the {\em prime numbers} (building blocks of the integers); {\em pairing functions} (building blocks of the encodings that provide much power---and pitfalls---as one computes); {\em finite number systems}.
  \item
The sixth chapter is devoted to {\em recurrences}.  Coverage ranges from {\em techniques for solving recurrences}, to {\em important recursively defined sequences} (mainly, the binomial coefficients and Fibonacci numbers), to {\em sophisticated examples and applications}.
  \item
The final chapter in the sequence deals with {\em numerals: representations of numbers}.  Topics include: {\em Classes of ``positional" numerals}---the familiar $b$-ary numerals, the less-familiar $b$-adic numerals, the ``carry-free" signed-digit numerals.  The chapter also covers {\em proofs of uncountability}.  
\end{enumerate}

\item
The preceding sequence of chapters on number systems illustrate a philosophical pillar of our approach to teaching/learning mathematics.  We have structured the coverage of important topics---numbers, sets, graphs, combinatorics, etc.---so that a potential reader or instructor can choose coverage that will make a topic $A$ a ``casual acquaintance" or a ``good friend" or ``an intimate friend".
\end{itemize}

\item
Every topic is treated as a new vehicle for mathematical discovery and reasoning.  The recurring subject of binomial coefficients is a good example.  This class of integers is an important exemplar of the power of recursive definitions.  By their appearance in the Binomial Theorem, these numbers illustrate the power of compact representations---for both reasoning and computation.  By dint of their multiple uses within the study of combinatorics and probability, the numbers demonstrate completely unexpected interrelationships among seemingly unrelated concepts.

\item
Mathematics is invented by humans.  Motivated by superficial curiosity, humans exploit their innate powers to detect patterns.  Motivated by a deeper form of curiosity, humans develop conceptual tools in order to explicate the observed patterns.

\smallskip

By exposing the human side of what have turned out to be fruitful pursuits of curiosity, we have striven to motivate readers to launch their own journeys of discovery.  Important examples:

Leonardo Pisano's curiosity about the demographics of rabbits; John Arbuthnot's curiosity about the relative frequencies of male vs.~female births in humans. 
\end{itemize}

\bigskip

\noindent {\bf The Authors}

\medskip

\noindent
The authors have a long cumulative history of research on \\
$\cdot$ Mathematics (discrete math and mathematical logic) \\
$\cdot$ Computer Science (algorithms, architecture, computation theory, data analytics, resource management ) \\
$\cdot$ Computer Engineering (digital logic design, energy conservation, fault tolerance, logic testing)

\smallskip

\noindent
They have cumulatively published more than 300(?) research articles in elite journals, as well as several research books and textbooks on both mathematical and computational subjects

\bigskip

\noindent
Over a cumulative span approaching 3/4 century, the authors have taught courses ranging from mathematics and mathematical logic to a broad spectrum of topics in computer science and engineering at the following universities (listed alphabetically):

\smallskip

\noindent
Duke Univ., 
Univ. Grenoble,
Univ. Massachusetts,
New York University,
Polytechnic Inst. of New York,
The Technion (Israel Inst. of Technology),
Univ. Toronto,
Yale Univ.

\smallskip

\noindent
The courses taught by the authors include:

\smallskip

\noindent
Algorithms,
Computation and Complexity Theory,
Computer Architecture,
Mathematical Logic,
VLSI Design


\end{document}

\section*{The Book: What Is Special}

Focus is not just on {\em learning topics in mathematics} but in learning {\em how to ``do mathematics"} --- how to {\em think like a mathematician} in terms of {\em problem solving} and {\em formulating ``real" problems mathematically}.  Here are some steps we take:
\begin{itemize}
\item
We have a chapter entitled ``Doing Mathematics".  We are not claiming that this is unique.  But this is one of the steps that we take.
\item
We provide numerous rather different explanations---including proofs---when we feel that the reader may benefit from learning multiple ways to think about the same issue.
\item
We strive to transmit important pieces of mathematical culture by telling stories that expose how mathematical greats of the past came to their discoveries.
\item
We go ``beyond the basics" so that the reader who is fascinated by a particular topic can 		test the waters" with more advanced and/or specialized material than one typically encounters in an introductory course.  We even include a small number of essays---in the appendix---that really give a peek into specialized topics.
\item
We have partitioned our references and our index to help the reader find {\em historical} material and references that are more {\em cultural} than technical (Euclid, Al-Khwarizmi).
\end{itemize}


\begin{itemize}
\item
Topics chosen not only from ``standard" syllabi, but also from our personal experience as researchers.
\item
Topics are organized with an eye toward the mathematical tools needed to master them.

Talk about the sequence: \\
NUMBERS1 \\
ARITHMETIC \\
SUMMATION \\
INFINITE \\
NUMBERS2 \\
RECURRENCES \\
NUMBERS3
\end{itemize}

\end{document}