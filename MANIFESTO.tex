%version of 04-10-19 

\chapter*{MANIFESTO}

The technologies that enable both the hardware and software systems of
modern computers have grown in complexity at least as fast as they
have grown in power.  For decades, it has been impossible to design
and utilize computing systems without using mathematical tools.  For
too long, though, these tools have been used to create widgets which
are then assembled using only cleverness and untutored ingenuity.  The
net result is that many of the computing systems that run essential
aspects of our daily lives are neither well-understood nor
well-controlled.  We read daily about glaring breaches of security
and/or privacy that cannot be controlled or repaired because the
victimized system is too complex to be understood.  The lesson is
clear.  We must stop being satisfied with mere {\em knowledge} of how
to make a system take in inputs and emit outputs: we must settle for
nothing less than {\em understanding} of every system's structure and
behavior.

This book has a simple, yet fundamental, goal.  We want to endow each
reader with an {\em operational} conceptual and methodological
understanding of the discrete mathematics that can be used to study,
and understand, and perform computing.  We want each reader to {\em
  understand} the elements of computing, rather than just {\em know}
them.  Thereby, we hope that the interested reader will be able to
{\em develop new concepts} and {\em invent new techniques and
  technologies} that will expand the capabilities of the hardware that
perfrms computations and the software that controls the behavor of the
hardware.  We stress the word {\it operational:} we want the readers'
level of understanding to allow them to {\em ``do''} mathematics.

\bigskip

Lest the reader feel unworthy for the daunting task of ``doing''
mathematics, we invoke no less an authority than the great German
mathematician Leopold Kronecker, \index{Kronecker, Leopold} to brace
the spine and embolden the spirit.  In~\cite{Bell86} (page 477), we
are told that Kronecker assures us that ``God made the integers; all
else is the work of man.''  Therefore, although we may be standing on
the shoulders of giants when we ``do'' mathematics, we are not
insolently attempting to wrest fire from Olympus!

\ignore{***************
\bigskip

\noindent \fbox{
\begin{minipage}{0.95\textwidth}
The phrase ``standing on the shoulders of giants'' has been cited by
many, over many centuries, ranging from Sir Isaac Newton to Bernard of
Chartres, and beyond.  The treatise by Merton \cite{Merton}
\index{Merton, Robert K.} provides an entertaining, educational
history of the phrase.
\end{minipage}
}

****************}


