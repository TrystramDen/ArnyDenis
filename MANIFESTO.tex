\documentclass{article}

\input preamble

\begin{document}

\section{Our Goal}



The fundamental goal of this book / chapter / section is to endow the
reader with an operational level of conceptual and methodological
understanding of the discrete mathematics that is used to study and
understand computing.  We construe an ``operational'' level of
understanding to be one that enables the reader to ``do'' mathematics.

Somewhat surprising to the non-mathematician, a large portion of
``doing'' mathematics, the widely touted ``queen of the sciences'', is
pattern-matching---albeit of a monumentally sophisticated variety.
Mathematicians are trained to understand pieces of reality to a depth
that allows them to understand how apparently unrelated concepts $A$
and $B$ can be conceptualized via the same abstract representation,
and to analyze (computational, in our bailiwick) advantages to
exploiting such representations.

Toward the end of guiding the reader through this forest of
abstractions, we categorize our targets in three ways
\begin{enumerate}
\item
fundamental sums

of intrinsic intererest

e.g.: arithmetic sums, geometric sums, mathematically ``smooth'' sums

\item
fundamental techniques

that work in many distinct situations

e.g.: using integration for summation; grouping/replication; verification via induction

\item
fundamental representations

that enable one to study the same phenomenon in many seemingly unrelated ways

e.g.: numbers, Riemann sums, slices of pie, tokens arranged in stylized ways
\end{enumerate}



\end{document}
