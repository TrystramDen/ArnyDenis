
\chapter{Deriving and Solving Linear Recurrences}
\label{sec:linear-recurrences-2}

While we have discussed already discussed the use of linear
recurrences in Section~\ref{sec:linear-recurrences-1}, we derive the
underlying mathematics in this section

By the time the reader has reached this paragraph, she has the
mathematical tools necessary to prove and apply what is called {\it
  The Master Theorem for Linear Recurrences} \cite{CLRS}.  This level
of mathematical preparation should be adequate for most
early-undergrad courses on data structures and algorithms, as well for
for analyzing a large fraction of the algorithms that she is likely to
encounter in daily activities.

\begin{theorem}[The Master Theorem for Linear Recurrences]
\label{thm:master-thm}
\index{The Master Theorem for Linear Recurrences}
Let the function $F$ be specified by the following linear recurrence.
\begin{equation}
\label{eq:Lin-Recur:start}
F(n) \ = \ \left\{
\begin{array}{cl}
a F(n/b) + c & \mbox{for } n \geq b \\
c & \mbox{for } n < b
\end{array}
\right.
\end{equation}
Then the value of $F$ on any argument $n$ is given by
\begin{equation}
\label{eq:Lin-Recur:solve}
\begin{array}{lclll}
F(n) & = & (1 + \log_b n)c &  & \mbox{if } a=1 \\
     &   &                 &  & \\
     & = &
  {\displaystyle
  \frac{1-a^{\log_b n}}{1-a} \ \approx \ \frac{1}{1-a}
  }
 &  & \mbox{if } a<1 \\
    &   &                  & & \\
    & = &
  {\displaystyle
\frac{a^{\log_b n} -1}{a-1}
  }
 & & \mbox{if } a>1
\end{array}
\end{equation}
\end{theorem}

\begin{proof}
In order to discern the recurring pattern in
(\ref{eq:Lin-Recur:start}), let us begin to ``expand'' the specified
computation by replacing occurrences of $F(\bullet)$ as mandated in
(\ref{eq:Lin-Recur:start}).
\begin{equation}
\label{eq:Lin-Recur:expand}
\begin{array}{lcccc}
F(n) & = & a F(n/b) + c & & \\
     & = & a \left( a F(n/b^2) + c \right) + c
             & = & a^2 F(n/b^2) + (1 + a)c \\
     & = & a^2 \left( a F(n/b^3) + c \right) + (1+a)c
             & = & a^3 F(n/b^3) + (1 + a + a^2)c \\
     &   & \vdots & & \vdots \\
     & = & 
{\displaystyle
\left( 1 + a + a^2 + \cdots + a^{\log_b n} \right) c
} & &
\end{array}
\end{equation}
The segment of (\ref{eq:Lin-Recur:expand}) ``hidden'' by the vertical
dots betokens an induction that is left to the reader.  Equations
(\ref{eq:geom-sum:b>1}) and (\ref{eq:geom-sum:b<1}) now enable us to
demonstrate that (\ref{eq:Lin-Recur:solve}) is the case-structured
solution to (\ref{eq:Lin-Recur:start}).  \qed
\end{proof}
