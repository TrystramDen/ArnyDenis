%version of 08-10-18

\chapter{RECURRENCES}
\label{ch:Recurrences}

This chapter is devoted to derive and solve Linear Recurrences}.

While we have discussed already discussed the use of linear
recurrences in Section ***CHECK REFERENCE  %~\ref{sec:linear-recurrences-1}, 
we derive the
underlying mathematics in this section

\section{The Master Theorem}
\label{sec:mesterTheorem}

\subsection{A simplified form}

By the time the reader has reached this paragraph, she has the
mathematical tools necessary to prove and apply what is called {\it
  The Master Theorem for Linear Recurrences} \cite{CLRS}.  This level
of mathematical preparation should be adequate for most
early-undergrad courses on data structures and algorithms, as well for
for analyzing a large fraction of the algorithms that she is likely to
encounter in daily activities.

\begin{theorem}[The Master Theorem for Linear Recurrences]
\label{thm:master-thm}
\index{The Master Theorem for Linear Recurrences}
Let the function $F$ be specified by the following linear recurrence.
\begin{equation}
\label{eq:Lin-Recur:start}
F(n) \ = \ \left\{
\begin{array}{cl}
a F(n/b) + c & \mbox{for } n \geq b \\
c & \mbox{for } n < b
\end{array}
\right.
\end{equation}
Then the value of $F$ on any argument $n$ is given by
\begin{equation}
\label{eq:Lin-Recur:solve}
\begin{array}{lclll}
F(n) & = & (1 + \log_b n)c &  & \mbox{if } a=1 \\
     &   &                 &  & \\
     & = &
  {\displaystyle
  \frac{1-a^{\log_b n}}{1-a} \ \approx \ \frac{1}{1-a}
  }
 &  & \mbox{if } a<1 \\
    &   &                  & & \\
    & = &
  {\displaystyle
\frac{a^{\log_b n} -1}{a-1}
  }
 & & \mbox{if } a>1
\end{array}
\end{equation}
\end{theorem}

\begin{proof}
In order to discern the recurring pattern in
(\ref{eq:Lin-Recur:start}), let us begin to ``expand'' the specified
computation by replacing occurrences of $F(\bullet)$ as mandated in
(\ref{eq:Lin-Recur:start}).
\begin{equation}
\label{eq:Lin-Recur:expand}
\begin{array}{lcccc}
F(n) & = & a F(n/b) + c & & \\
     & = & a \left( a F(n/b^2) + c \right) + c
             & = & a^2 F(n/b^2) + (a+1)c \\
     & = & a^2 \left( a F(n/b^3) + c \right) + (a+1)c
             & = & a^3 F(n/b^3) + (a^2+a+1)c \\
     &   & \vdots & & \vdots \\
     & = & 
{\displaystyle
\left(a^{\log_b n} + \cdots +a^2+a+1 \right) c
} & &
\end{array}
\end{equation}
The segment of (\ref{eq:Lin-Recur:expand}) ``hidden'' by the vertical
dots betokens an induction that is left to the reader.  Equations
(\ref{eq:geom-sum:b>1}) and (\ref{eq:geom-sum:b<1}) now enable us to
demonstrate that (\ref{eq:Lin-Recur:solve}) is the case-structured
solution to (\ref{eq:Lin-Recur:start}).  \qed
\end{proof}

\subsection{A more complicated (but still regular) form}

We present in this section a more general recurring pattern:

\begin{equation}
\label{eq:Lin-Recur:general}
F(n) \ = \ \left\{
\begin{array}{cl}
a F(n/b) + f(n) & \mbox{for } n > b \\
1 & \mbox{for } n = b
\end{array}
\right.
\end{equation}

Let us first solve the case where function $f$ is linear, i.e. in the simplified form $f(n) = n$.
We also assume for clarity that $n$ is a power of $b$. 

We proceed as previously by replacing the successive occurrences of $F$:

\begin{equation}
\label{eq:Lin-Recur:expand}
\begin{array}{lcccc}
F(n) & = & a F(n/b) + n & & \\
     & = & a \left( a F(n/b^2) + n/b \right) + n
             & = & a^2 F(n/b^2) + (a.n/b+n) \\
     & = & a^2 \left( a F(n/b^3) + n/b^2 \right) + (a.n/b+n)
             & = & a^3 F(n/b^3) + (a^2/b^2+a/b+1)n \\
     &   & \vdots & & \vdots \\
     & = & 
{\displaystyle
\left(a^{\log_b n}F(1) + \sum_{i=0,\log_b (n)-1} (a/b)^i \right) n
} & &
\end{array}
\end{equation}

\begin{figure}[htb]
\begin{center}
       \includegraphics[scale=0.25]{FiguresRecurrences/MasterTheorem}
\caption{{\it Development of the computations. Summation on the right.}
\label{fig:masterTheorem}}
\end{center}
\end{figure}

Fig.~\ref{fig:masterTheorem} illustrates these computations.
The second term corresponds to a geometric sequence.

The interpretation is that if $a$ is larger than $b$, $F(n)$ is dominated by the first term while if it is smaller, the second term dominates.
The problem is perfectly balanced between the iterations when $a=b=2$, we obtain $F(n) = n.log_2(n)$. 


\section{The Fibonacci Recurrence and its Kin}
\label{sec:Fiboacci-plus}

\subsection{The Fibonacci sequence}

Leonardo of Pisa, ``the son of Bonaccio''



\subsection{Relatives of the Fibonacci sequence}

The Lucas' sequence

\section{The Token Game}
\label{sec:TokenGame}


