%version of 08-29-19

\chapter{Numbers III:
Operational Representations and Their Consequences}
\label{ch:numerals}
\index{numerals!operational}
\index{number representations!operational}
\index{number representations}


\begin{quote}
{\em What's in a name?} \\
\hspace*{2in}William Shakespeare, {\it Romeo and Juliet}
\end{quote}

\section{Historical and Conceptual Introduction}

Numbers are intangible idealizations.  One must endow numbers with
names before one can manipulate them and compute with them.
Historically, we have employed a broad range of mechanisms for naming
numbers.

{\ignore {\Arny I really like the idea of providing some cultural
    background to get the reader involved.}}
{\ignore {\Denis I also like the idea, may be we can shorten the Latin numbers
  and let some parts (examples) as an exercice...}}
{\ignore {\Arny Please propose what to omit.  I do want to point out the
  ``three levels'' of names/numerals: names that convey information
  only by cultural agreement; names that permit identification but no
  practical manipulation; {\em operational} names}}
\ignore{I am not sure, however, how much space to
  allocate.  For instance, I like the mention of Roman numerals and of
  the system that the Phoenicians used --- which I am familiar with
  mainly because of Hebrew --- but I am reluctant to go so far as to
  really discuss the formation rules of Roman numerals or the details
  of numeral formation in Hebrew and its kindred languages.}

\medskip

\noindent
{\it Nicknames for ``familiar'' numbers}.
%
We have endowed several numbers that are associated with concrete
entities (see the examples below) with names that do not even hint at
any aspect of the nature of the named number.  A few examples:
\begin{itemize}
\item
$\pi$: the ratio of the circumference of a circle to its diameter
\item
$e$: the base of so-called natural logarithms
\item
$\phi$: the {\it golden ratio} (one of several word-names for $\phi$)
  that can be observed in nature, e.g., in the leaf patterns of plants
  such as pineapples and cauliflower
\item
Avogadro's number: a fundamental quantity in chemistry and physics.
(This number-name indicates that not all numerical nicknames are
single letters.)
\end{itemize}
Nickname-based numerals give no information about the named number:
they do not help anyone (except the {\it cognoscenti:} the
``in-crowd'') {\em identify} the named number, and they do not help
anyone manipulate---e.g., compute with---it.  These names are valuable
only for {\em cultural} purposes, not mathematical ones.

To clarify our intended message: It is the {\em names} of these
special numbers that convey no operational information.  Each of these
is attached to valuable science and/or mathematics!  We shall expose
some of this mathematics as we discuss $e$ in the current chapter,
revisit $\pi$ and $e$ in Chapter~\ref{ch:Summation}, and revisit
$\phi$ in Chapter~\ref{ch:Recurrences}.

\medskip

\noindent
{\it Alphabet-based systems.} 
\index{alphabet-based number systems}
\index{number systems!alphabet-based}
%
Several cultures have developed systems for naming integers by using
their alphabets in some manner.  One such system that is still visible
in European cultures comprises {\it Roman numerals}.
\index{numerals!Roman} One encounters these within constrained
contexts, e.g., as hour markers on ``classical'' clocks and as
timestamps on the cornerstones of official buildings.  The numerals
are formed from a subset of the Latin alphabet:

{\small
\begin{tabular}{c|c}
{\it Letter} & {\it Numerical value} \\
\hline
I  & 1 \\
V  & 5 \\
X  & 10 \\
L  & 50
\end{tabular}
\hspace*{.5in}
\begin{tabular}{c|c}
{\it Letter} & {\it Numerical value} \\
\hline
C  & 100 \\
D  & 500 \\
M  & 1000
\end{tabular}
}

\noindent
The formation rules for Roman numerals of length exceeding $2$ are a
bit complicated, but {\em roughly}, a letter to the right of a higher
valued letter augments the value of the numeral (e.g., DCL $=650$, XVI
$=16$), while a letter to the left of a higher valued letter lowers
the value (e.g., MCM $=1900$, XLIV $=44$).

A rather different way to craft numerals from letters is observable in
the Hebrew system. \index{numerals!Hebrew} Assimilating ideas of the
ancient Egyptians, Phoenicians, and Greeks, this system assigns the
following values to the $22$ letters of the Hebrew alphabet.
\[ 1, 2, 3, 4, 5, 6, 7, 8, 9, 10,
20, 30, 40, 50, 60, 70, 80, 90, 100,
 200, 300, 400
\]
It then forms numerals as strings of single occurrences of letters, by
accumulating letters' numerical values.  Numbers that are too large to
be named via strings of single letter-instances often allow repeated
letter instances or incorporate auxiliary words, in a mixed-mode
manner similar to our writing $5000$ as ``$5$ thousand''.

\medskip

Alphabet-based systems for creating numerals are more useful than
nickname-based systems: they {\em do} allow anyone to {\em identify}
any named number. Indeed, one can (algorithmically) convert any Roman
numeral or any Hebrew numeral to a decimal numeral for the same
number.  However, any reader who is familiar with alphabet-based
systems will recognize a major drawback of such systems: It is {\em
  exceedingly difficult} to do any but the most trivial arithmetic
using such systems' numerals.  Two simple examples using Roman
numerals will make our case:
\begin{itemize}
\item
Square CC.  This is, of course, trivial using, e.g., decimal numerals:
an elementary school student can compute $200 \times 200 \ =
\ 40,000$.  But even in an early course on programming, one would not
assign the general ``multiply numbers using Roman numerals'' problem
as a first assignment.

\item
Substract MCMXCVIII from MMII.  Of course, the answer is IV, but one
would likely determine this by converting to decimal numerals
($2002-1998$).
\end{itemize}

\noindent
{\it Positional number systems.}
\index{positional number systems}
\index{number systems!positional number systems}
%
In our daily commerce, we deal almost exclusively with numerals that
are formed within a {\it base-$b$ positional number system},
\index{numerals in a base-$b$ positional number system} known also as
a {\it $b$-ary positional number system}.
\index{numerals in a $b$-ary positional number system}
The word {\it ``radix''} is often used in place of the word ``base'';
\index{numerals in a radix-$b$ positional number system}
we shall use the word ``base''.  The most common exemplars of $b$-ary
positional number systems are:

\smallskip

\begin{tabular}{llclcl}
base-$2$:  & the & $2$-ary  & or & {\em binary}      & system \\
base-$8$:  & the & $8$-ary  & or & {\em octal}       & system \\
base-$10$: & the & $10$-ary & or & {\em decimal}     & system \\
base-$12$: & the & $12$-ary & or & {\em duodecimal}  & system \\
base-$16$: & the & $16$-ary & or & {\em hexadecimal} & system
\end{tabular}

\smallskip

\noindent
Most of this chapter---specifically,
Section~\ref{sec:positional-numbers}---is devoted to studying {\em
  -ary} positional number systems in detail.

\medskip

\noindent
{\it Systems developed to ensure special properties.}
\index{number systems!formed to ensure special properties}
We briefly discuss two families of positional systems that were
invented to ensure number representations with specific properties.
Section~\ref{sec:bijective-adic} discusses {\em bijective number
  systems}, \index{positional number systems!bijective systems}
\index{positional number systems!-adic number systems}
\index{number systems!positional number systems!bijective systems}
\index{number systems!positional number systems!-adic number systems}
which are positional number systems in which distinct numerals name
distinct numbers.  This property can be significant in certain genres
of numeral-based encodings.  (Of course, -ary number systems do not
enjoy this property because of excess leading and trailing $0$s.)
Finally, Appendix Chapter~\ref{ch:carry-free} introduces number
systems that enable {\em carry-free addition}.  This property enables
the design of ``unit-time'' parallel adders of arbitrary-length
numerals.

\medskip

\noindent
{\it Systems formed using special families of numbers.}
\index{number systems!based on special families of numbers}
%
In Section~\ref{sec:numerals-special-families}
The final genre of number system that we discuss in this introduction
are those based on special families of numbers, such as the binomial
coefficients and the Fibonacci numbers.  Such systems are employed
most often when the properties of the underlying family expose
particular properties of named numbers.  We discuss one such system in
Section~\ref{sec:Fibo-numbers}.


\section{Positional Number Systems}
\label{sec:positional-numbers}
\index{positional number systems}
\index{number systems!positional number systems}

We turn now to the immensely successful family of {\em operational}
numerals, the $b$-ary number systems.  Each such system is built upon
a {\em number base $b$} which is an integer $b> 1$.\footnote{In
  rather specialized contexts one may encounter number bases that are
  not positive integers.}
\index{positional number system!number base}
The numerals in the system are strings of {\it digits} from the set
$\{0, 1, 2, \ldots, \overline{b-1}\}$, often embellished with other
symbols, such as a {\em radix point},\footnote{Using a period as the
  radix point is a US convention; in much of Europe, a comma is
  used.}~and sometimes a leading ``$+$'' or ``$-$'' to indicate,
respectively, the denoted number's positivity or negativity.

\medskip

\noindent \fbox{
\begin{minipage}{0.97\textwidth}
We employ the overline notation, ``$\overline{b-1}$'', to remind
ourselves that ``$b-1$'' is a digit \index{digit} here, not a string;
e.g., when $b = 10$ (the {\em decimal}, or, $10$-{\it ary} base),
$\overline{b-1}$ is the digit $9$.
\end{minipage}
}

\medskip

\noindent
We begin to discuss these systems with a few examples:
\begin{itemize}
\item
Most of our daily activities employ the number system usually called
{\it decimal} or {\it base-$10$}, or, unusually but also correctly,
{\it $10$-ary}.  \index{the decimal number system} 
\index{the base-$10$ number system} \index{the $10$-ary number system}
This system's digits comprise the set $\{0, 1, 2, 3, 4, 5., 6, 7, 8,
9\}$; its radix point is called a {\em decimal point}.
\index{decimal point}

\item
Because electrical and electronic circuitry are (for the most part)
built using {\it bistable} devices---e.g., switches that are either
{\em on} or {\em off}---the system most often employed when dealing
with such circuitry and its end products (say, computers) is the
{\it base}-$2$ system, which is also called {\it binary} of $2$-ary.
\index{the binary (base-2) number system}
The digits of this system comprise the set $\{0, 1\}$.  Each digit is
called a {\it bit}---a contraction of {\it binary digit}.
\index{bit: binary digit}

\item
Because of its small repertoire of digits, the binary system's
numerals are quite long---roughly $3$ times longer than decimal
numerals.  For instance, denoting the base-$b$ as a subscript to the
numeral (a common convention):
\[ 32,768_{10} \ \ = \ \ 1,000,000,000,000,000_2 \]
In order to make base-$2$e numerals easier for humans to deal with, we
usually aggregate small sequences of bits to form larger number
bases---but still powers of $2$.  Two aggregations have been
particularly popular:
  \begin{itemize}
  \item
By aggregating length-$3$ sequences of bits, one converts base-$2$
numerals to {\it base}-$8$ numerals, known also as {\it octal} or
$8$-{\it ary} numerals;
\index{the octal (base-$8$) number system} the octal digits comprise
the set $\{0, 1, 2, 3, 4, 5, 6, 7\}$.
  \item
By aggregating length-$4$ sequences of bits, one converts base-$2$
numerals to base-$16$ numerals, known also as {\it hexadecimal} or
{\it $16$-ary} numerals; \index{the hexadecimal (base-$16$) number system} 
the hexadecimal digits comprise the set 
\[ \{0, 1, 2, 3, 4, 5, 6, 7, 8, 9, \overline{10}, \overline{11},
\overline{12}, \overline{13}, \overline{14}, \overline{15}\}.
\]
{\it Note:} We have written the hexadecimal digits in decimal, to make
them easy to read, but we have placed overlines above the
$2$-decimal-digit numerals ``$10$'', ``$11$'', ``$12$'', ``$13$'',
``$14$'', and ``$15$'' as a reminder that each represents a single
hexadecimal digit, not a $2$-digit numeral.
  \end{itemize}
The success of these aggregations is attested to by the following
equation chain:
\[ 32,768_{10} \ \ = \ \ 1,000,000,000,000,000_2 \ \ = \ \
100,000_8 \ \ = \ \ 8,000_{16}
\]
\end{itemize}

\subsection{$b$-ary Number Systems}
\label{sec:b-ary-systems}

The {\it $b$-ary} number systems are by far the most commonly used
positional number systems.  The names of the different instances of
the system---each formed by choosing a specific number base
$b$---derive from the {\em Latin} name of the base number.
Regrettably, from a denotational point of view, the systems associated
with certain bases end with the suffix {\em -ary} (as in ``binary'')
while those associated with other bases end with the suffix {\em -al}
(as in ``decimal'').  The following table codifies the multiple names
of the systems associated with the most commonly used bases (for
mathematicians, scientists, and engineers).

\smallskip

\begin{tabular}{|c|l|l|}
\hline
{\bf Base} & \multicolumn{2}{c}{\bf {\em -ary} System Names}  \\
\hline
$2$   & binary     & $2$-ary  \\
\hline 
$4$   & quaternary & $4$-ary  \\
\hline
$8$   & octal      & $8$-ary \\
\hline
$10$  & decimal    & $10$-ary  \\
\hline
$12$  & duodecimal & $12$-ary  \\
\hline
$16$  & hexadecimal & $16$-ary \\
\hline
\end{tabular}

\bigskip

\index{positional number system!forming base-$b$ numeral}
\noindent {\bf The formation rules for $b$-ary numerals}.
As we turn to the {\it formation rules} for $b$-ary numerals, we want
to emphasize that these rules build in an essential way on the ideas
relating to summing {\em geometric summations}.  This is, therefore, a
good time to review Section~\ref{sec:geometric-sums}.

\medskip

%{\Denis the following should be put before in 4.5.2}

\index{base-$b$ numeral}
\index{positional number system!base-$b$ numeral}
\index{$b$-ary numeral}
\index{positional number system!$b$-ary numeral}
\noindent
A $b$-ary numeral is a string having three segments.
\begin{enumerate}
\item
The numeral begins with its {\em integer part},
\index{positional number system!base-$b$ numeral!integer part}
\index{base-$b$ numeral!integer part}
which is a {\em finite} string of base-$b$ digits, i.e., digits from
the set \index{$B_b$: $b$-ary digits}
\begin{equation}
\label{eq:b-ary-digits}
B_b \ \eqdef \ \{0, 1, 2, \ldots, \overline{b-1}\}
\end{equation}
(Recall that ``$\overline{b-1}$'' represents the integer $b-1$ as a
single digit.)  \index{$\overline{b-1}$: $b-1$ as a single digit}
We denote the integer-part string as: $\alpha_n \alpha_{n-1} \cdots
\alpha_1 \alpha_0$.

\item
The numeral continues with a single occurrence of the {\it
  radix point},
\index{positional number system!radix point}
\index{positional number system!base-$b$ numeral!radix point}
\index{base-$b$ numeral!radix point}
which is denoted ``$.$'' in the U.S.~and ``,'' in many other countries. 

\item
The numeral ends with its {\em fractional part},
\index{positional number system!fractional part of a numeral}
\index{positional number system!base-$b$ numeral!fractional part}
\index{base-$b$ numeral!fractional part}
which is a string---{\em finite or infinite}---of base-$b$ digits We
denote the fractional-part string as: $\beta_0 \beta_1 \beta_2
\cdots$.
\end{enumerate}
Our completed numeral now has the form
\begin{equation}
\label{eq:real-numeral}
\alpha_n \alpha_{n-1} \cdots \alpha_1 \alpha_0
\ . \ \beta_0 \beta_1 \beta_2 \cdots
\end{equation}

\noindent
This numeral has the {\it numerical value}\footnote{The notation
  ``$\mbox{\sc val}_{b}(x)$'' in (\ref{eq:real-numeral-number})
  denotes an operator that produces the {\em numerical value} of the
  base-$b$ numeral $x$.}
\begin{equation}
\label{eq:real-numeral-number}
\mbox{\sc val}_{b}(\alpha_n \alpha_{n-1} \cdots \alpha_1 \alpha_0                  
\ . \ \beta_0 \beta_1 \beta_2 \cdots)
\ \ \eqdef \ \
\sum_{i=0}^n \alpha_i \cdot b^i
\ + \ \sum_{j\geq 0} \beta_j \cdot b^{-j}.
\end{equation}
\index{positional number system!numerical value of base-$b$ numeral}
\index{base-$b$ numeral!numerical value}
\index{positional number system!base-$b$ numeral!numerical value}
For emphasis, we review:
\index{positional number system!numerical value of integer part}
\begin{itemize}
\item
The numerical value of the integer part of the numeral in
(\ref{eq:real-numeral}) is:
\[
\mbox{\sc val}_{b}(\alpha_n \alpha_{n-1}\cdots \alpha_1 \alpha_0)
\ \ = \ \
\sum_{i=0}^n \alpha_i \cdot b^i
\]
\item
The numerical value of the fractional part of the numeral in
(\ref{eq:real-numeral}) is:
\index{positional number system!numerical value of fractional part}
\[
\mbox{\sc val}_{b}(. \beta_0 \beta_1 \beta_2 \cdots)
\ \ = \ \
\sum_{j\geq 0} \beta_j \cdot b^{-j}
\]
\item
By prepending a ``minus sign'', or, ``negative sign'' ($-$) to a
numeral or a number, one renders the thus-embellished entity as
negative.
\end{itemize}

Note that {\em two types of sequences of $0$s do not affect the value
  of the number represented by a $b$ary numeral:}
\begin{itemize}
\item
an {\em initial} sequence of $0$s that reside to the {\em left} of the
radix point and of all non-$0$ digits;
\item
a {\em terminal} sequence of $0$s that reside to the {\em right} of
the radix point and of all non-$0$ digits.
\end{itemize}
One consequence of this fact is that we lose no generality by
insisting that every numeral have the following {\em normal form:}
\index{positional number system!numeral!normal form}
\index{normal form for for numeral in a positional number system}

\smallskip

\hspace*{.15in}
\begin{tabular}{l}
-- it begins with a finite sequence of digits, \\
-- it then has one occurrence of the radix point, \\
-- it ends with an infinite sequence of digits
\end{tabular}

\bigskip

We finish this section with an important consequence of the definition
of real numbers in Section~\ref{sec:define-Reals}, in terms of the
summations in (\ref{eq:real-defn}).

\begin{prop}
\label{thm:define-Reals-via-numerals}
A number $n$ is real, i.e., belongs to the set $\R$, if, and only if
it is the numerical value of a numeral of the form
(\ref{eq:real-numeral}).
\end{prop}

\subsection{A {\em Bijective} Number System}
\label{sec:bijective-adic}
\index{positional number system!bijective}

This section introduces a positional number system in which distinct
numerals name distinct numbers.  Of course, $b$-ary systems do not
enjoy this property because of the value neutrality of leading $0$s
for integer numerals and trailing $0$s for fractional numerals.  The
systems are often termed {\it bijective} because their unique numerals
for integers arise from a bijection between the integer numerals and
the set $\N^+$ of positive integers.  {\em The price that these
  systems pay for their bijectiveness is that they cannot represent
  the number $0$; they can represent all other integers.}  Each
bijective base-$b$ system is sometimes called the {\it $b$-adic number
  system}. \index{positional number system!$b$-adic} One also finds
some -adic number systems being named using Greek-inspired names for
the base $b$, in imitation of the Latin-inspired names of -ary
systems.  Most commonly one encounters the base-$2$ {\it dyadic}
\index{positional number system!dyadic: base-$2$}
\index{dyadic (base-$2$) number system} system as the -adic analogue
of the binary system.

For any number base $b > 1$, the base-$b$ bijective systems' numerals
are formed in exactly the same way as are the numerals of the $b$-ary
number system---see (\ref{eq:real-numeral-number})---but the bijective
systems' numerals are formed using the digit-set $B'_b \ = \ \{1, 2,
\ldots, b\}$, rather than the $b$-ary set $B_b$ of
(\ref{eq:b-ary-digits}).  In order to lend the reader some intuition,
we display the dyadic numerals that have one or two digits together
with their numerical values.

\smallskip

\begin{tabular}{|lc|ll|}
\hline
\multicolumn{2}{c}{\bf Dyadic Numeral} & \multicolumn{2}{c}{\bf Numerical Value} \\
\hline
$x=$ & $1$  & {\sc dyadic-val}$(x) =$ & $1$ \\
     & $2$  &                         & $2$ \\
     & $11$ &                         & $2 + 1 = 3$ \\
     & $12$ &                         & $2 + 2 = 4$ \\
     & $21$ &                         & $4 + 1 = 5$ \\
     & $22$ &                         & $4 + 2 = 6$ \\
\hline
\end{tabular}

\medskip
\noindent
The preceding table indicates how the numerical values of dyadic
numerals track the lexicographic order of the numerals.

\bigskip

Formally verifying the bijectiveness of $b$-adic number systems is a
valuable exercise in manipulating numerals.  The first appearance of
bijective number systems was in \cite{Foster47}, where the base-$10$
system is introduced and shown to be bijective.  A proof of
bijectiveness for arbitrary $b$-adic systems appears in
\cite{Smullyan61}, where the term {\em $b$-adic} is introduced.  The
motivating application in \cite{Smullyan61} was to the allied fields
of Mathematical Logic and Computation Theory: Encoded versions of a
program's computations play a central role in these theories.  While
crafting the required encodings using strings of symbols accomplished
many of the goals of the theories, the overarching reach of the
theories was fully appreciated only when mathematical logician Kurt
G\"{o}del \index{G\"{o}del, Kurt} showed, in 1931, that the encodings
could be achieved using integers and simple arithmetic
operations---see the discussions of {\it G\"{o}del numbers}
\index{G\"{o}del numbers} in the primary sources
\cite{Goedel31,Turing36} or texts such as \cite{Rosenberg12}.  The
introduction in \cite{Smullyan61} of -adic number systems to replace
the traditional -ary systems significantly simplified certain of the
theories' central proofs, because of the -adic systems' bijectiveness.

We turn now to a result that asserts the bijectiveness of -adic number
systems.  We provide a proof only for the case $b = 2$: this case
provides all of the ideas necessary for the general result.

\begin{prop}
\label{thm:adic-bijective}
Distinct $b$-adic numerals name distinct positive numbers.
\end{prop}

\begin{proof}[For dyadic integers]
We provide a proof only for dyadic, i.e., base $b=2$, integers.  This
will expose the basic ideas of a complete proof, with minimal
notation.  We begin by exposing the largest and smallest integers that
admit a $d$-digit dyadic numeral.

\begin{lemma}
\label{lem:big-small-dyadic}
The {\em smallest} integer representable by a $d$-digit dyadic
numeral is 
\[ \mbox{\sc min-integer}_d \ = \ 2^d + 2^{d-1} + \cdots + 1 \ = \ 2^{d+1} - 1
\]
The {\em largest} integer representable by a $d$-digit dyadic
numeral is
\[ \mbox{\sc max-integer}_d \ = \
2 \times (2^d + 2^{d-1} + \cdots + 1) \ = \ 2^{d+2} - 2
\]
\end{lemma}

\begin{proof}[Lemma]
By definition,
\begin{itemize}
\item
The smallest $d$-digit dyadic numeral is $11 \cdots 1$ ($d$ digits).
Its value is:
\[ \mbox{\sc dyadic-val}_2(11 \cdots 1)
 \ = \ 2^d + 2^{d-1} + \cdots + 1 \ = \ 2^{d+1} - 1 \]
\item
The {\em largest} $d$-digit dyadic numeral is 
$22 \cdots 2$ ($d$ digits).  Its value is:
\[ \mbox{\sc dyadic-val}_2(22 \cdots 2)
\ = \ 2 \times (2^d + 2^{d-1} + \cdots + 1) \ = \ 2^{d+2} - 2 \]
\end{itemize}
We derive the values of both extremal integers by invoking the
summation techniques in Section~\ref{sec:geometric-sums}.
\qed-Lemma
\end{proof}

\bigskip

We are now ready to prove the proposition.  To this end, let us be
given distinct dyadic numerals,
\[
x \ = \ \gamma_r \gamma_{r-1} \cdots \gamma_1
 \ \ \ \ \ \mbox{ and } \ \ \ \ \
y \ = \ \delta_s \delta_{s-1} \cdots \delta_1
\]

{\bf (a)} Say first that $r \neq s$.  With no loss of generality, say
that $r = s +c$ for some $c \geq 1$.  In this case, we have, by
Lemma~\ref{lem:big-small-dyadic}:
\[ \mbox{\sc dyadic-val}(x) \ \geq \ 2^{r+1} - 1 \ = \ 2^{s+c+1} -1 \]
while
\[ \mbox{\sc dyadic-val}(y) \ \leq \ 2^{s+2} - 2 \]
It follows, therefore, that
\[ \mbox{\sc dyadic-val}(x) \ \geq \ \mbox{\sc dyadic-val}(y) \ + \ 1 \]
In particular, numerals $x$ and $y$ behave in consistency with the
proposition.

{\bf (b)} The alternative to assumption (a) is that $r = s$.  Because
$x$ and $y$ are distinct numerals, there must be a largest index $m
\leq r$ such that $\gamma_m \neq \delta_m$; say, with no loss of
generality, that $\gamma_m = \delta_m + c$ for some base-$b$ dyadic
digit $c$.  We can, therefore, rewrite numeral $y$ as
\[ y \ = \ \gamma_r \cdots \gamma_{m+1} \delta_m \delta_{m-1} \cdots \delta_1
\]
Invoking Lemma~\ref{lem:big-small-dyadic}, we can infer the following
bounds on the difference between $\mbox{\sc dyadic-val}(x)$ and
$\mbox{\sc dyadic-val}(y)$.

\bigskip

$\mbox{\sc dyadic-val}(x) \ - \ \mbox{\sc dyadic-val}(y)$
\begin{eqnarray*}
  & =  &
2^m \cdot (\gamma_m - \delta_m) \ + \ 2^{m-1} \cdot (\gamma_{m-1} -
\delta_{m-1}) \ + \cdots + \  2 \cdot (\gamma_1 - \delta_1) \\
  & \geq &
2^m \cdot (\gamma_m - \delta_m) \ - \ \left( 2^{m-1} + 2^{m-2} +
\cdots + 1 \right) \\
  & = &
2^m \cdot (\gamma_m - \delta_m)\ - \ \left( 2^m -1 \right) \\
  & = &
2^m \cdot (c-1) +1 \\
  & \geq & 1
\end{eqnarray*}
Once again, numerals $x$ and $y$ behave in consistency with the
proposition.  The result follows.
\qed
\end{proof}

\ignore{********
\section{$\oplus$ A System Based on the Fibonacci Numbers}

{\Denis I moved the section in the appendix, let write a breaf introduction here and refer to the appendix
}

In a variety of application areas, the special properties of certain
families of numbers can be exploited if one uses the numbers in the
family to devise representations of all integers.  We illustrate this
fact while using the Fibonacci numbers---see
Section~\ref{sec:Fibonacci}---as our basis family.
***********}


%%%%%%%%%%%%%%%%%%%%%%%%%%%%%%%%%%%%%%%%%%

\section{Recognizing Integers and  Rationals from Their Numerals}
\label{sec:special-numerals-N-Q}

We have provided an adequate, albeit inelegant, characterization of
the real numbers: a number $r$ is real if, and only if, it can be
represented by an infinite-length numeral in a positional number
system.  Because every rational number---hence, also, every
integer---is also a real number, every rational number and every
integer can also be written as a $b$-ary numeral, in the form
(\ref{eq:real-numeral}).  For rational numbers and integers, we can
make much stronger statements about the forms of their positional
numerals.


\subsection{Positional numerals for integers}
\label{sec:special-numerals-N}

The following result slightly alters the usual way that we write
numerals for integers, in order to render explicit the familial
relationship between reals and integers.

\begin{prop}
\label{thm:integer-real}
A real number is an integer if, and only if, it can be represented by
a {\em finite-length} numeral all of whose nonzero digits are to the
left of the radix point.
\index{number!integer!as a real with a finite numeral}
\end{prop}

\begin{proof}
The result follows from definition (\ref{eq:real-numeral-number}).  In
the indicated form, if any $\beta_i$ is nonzero, then the numerical
value of the numeral is non-integral.  To wit, digit $\beta_i$
witnesses that the numeral has a nonzero fractional part, hence is not
an integer.  \qed
\end{proof}

We can go beyond the simple statement of
Proposition~\ref{thm:integer-real} and develop an efficient algorithm
that computes the base-$b$ numeral for an integer $n$ via a series of
integer divisions.

\bigskip

\noindent {\it To compute a (finite) base-$b$ numeral for integer $n$.}
\index{number!integer!computing an integer's finite numeral}
%
If we ignore the radix point and all of the $0$s to the right of it in
the base-$b$ numerals given by (\ref{eq:real-numeral}), then we see
that the base-$b$ numeral $a_d a_{d-1} \cdots a_1 a_0$ for an integer
$n$ is a polynomial

$P(x) \ \ = \ \ a_0 \ + \ a_1 x \ + \ a_2 x^2 \ + \cdots + \ a_{d-1}
x^{d-1} \ + \ a_d x^d$

\noindent
evaluated at the point $x=b$:

$\mbox{\sc val}_{b}(x) \ \ = \ \ a_d b^d \ + \ a_{d-1} b^{d-1} \ +
\cdots + \ a_1 b \ + \ a_0$.

\subsubsection{Horner's Rule and fast numeral evaluation}

It appears at first glance that $\Theta(d^2)$ multiplications are
needed to evaluate the degree-$d$ univariate polynomial $\mbox{\sc
  val}_{b}(x)$---which would not give us a very efficient procedure
for producing a numeral for $n$.  However, we can develop an
efficient, $O(d)$-multiplication, procedure by adapting a method of
rewriting univariate polynomials, which is known as {\it Horner's
  rule} (the name we shall use) or {\it Horner's scheme}
\cite{Horner}. \index{Horner's rule} \index{polynomial!Horner's rule}
\index{Horner's scheme} \index{polynomial!Horner's scheme} By
describing Horner's rule on a degree-$3$ polynomial, we should give
the reader adequate preparation for our $O(d)$-division recipe for
producing a $d$-digit base-$b$ numeral for integer $n$.

\medskip

\begin{itemize}
\item {\small\sf The ``standard'' way of writing the polynomial $P(x)$.}

\noindent General degree $d$:

$P(x) \ \ = \ \ a_0 \ + \ a_1 x \ + \ a_2 x^2 \ + \cdots + \ a_{d-1}
x^{d-1} \ + \ a_d x^d$

\noindent Degree $3$:

$P(x) \ \ = \ \ a_0 \ + \ a_1 x \ + \ a_2 x^2 \ + \ a_3 x^3$

\item {\small\sf Rewriting $P(x)$ using Horner's rule.}

\noindent General degree $d$:

$P(x) \ \ = \ \ a_0 \ + \ x \cdot (a_1 \ + \ x \cdot (a_2  \ +  \cdots
+ x \cdot (a_{d-2} \ + \ x \cdot (a_{d-1} \ + \ a_d x)) \cdots ))$  

\noindent Degree $3$:

$P(x) \ \ = \ \ a_0 \ + \ x \cdot (a_1 \ + \ x \cdot (a_2  \ + \ a_3 x))$ 
\end{itemize}

\noindent
We now describe, by example, our efficient procedure for computing a
base-$b$ numeral for a given integer $n$.  The procedure iteratively
divides $n$ by $b$, {\em using Euclidean division} (see
Section~\ref{sec:euclidian}).  We claim that the {\em remainder} upon
each consecutive division is the next lowest-order digit in the
base-$b$ numeral for $n$.  (We leave this verification as an
exercise.)

\bigskip

\noindent {\it Illustrating the procedure.}
We produce the base-$2$ (binary) numeral for $n = 143$, (binary) digit
by (binary) digit.

\medskip

\begin{tabular}{|c|r|r|}
\hline
Step &
Current Quotient &
Current Remainder \\
\hline
1. & $143$ & $1$ \\
2. & $71$  & $1$ \\
3. & $35$  & $1$ \\
4. & $17$  & $1$ \\
5. & $8$   & $0$ \\
6. & $4$   & $0$ \\
7. & $2$   & $0$ \\
8. & $1$   & $1$ \\
\hline
\end{tabular}

\medskip

\noindent
We thereby have the following equation which specifies the base-$2$
numeral for $143$.
\[ 143_{10} \ = \ 10001111_2 \]


\subsubsection{Horner's Rule and fast exponentiation} 
\index{fast exponentiation}

We now develop a procedure for exponentiating which is ``fast'' (the
popular term): It computes the power
\[ b^n \]
where both $b$ and $n$ are positive integers ($b, n \in \N^+$) using a
number of multiplications that is {\em logarithmic in} $n$, rather
than the linear number that a direct approach would use.

\medskip

\noindent \fbox{
\begin{minipage}{0.97\textwidth}
Although the problem we attack here falls strictly within the domain
of {\em algorithmics}, we discuss it in our mathematics text because:
($a$) it provides a vivid example of the importance of data {\em
  representation}; ($b$) it illustrates a direct, consequential
application of Horner's rule, which could be misinterpreted as an
esoteric puzzle rather than an important technique.
\end{minipage}
}

\bigskip

\noindent {\it The setup}.
Our ``fast'' exponentiator proceeds as follows:
\begin{enumerate}
\item
Compute the (shortest) base-$2$ numeral for the power $n$ that we
are raising the base $b$ to.

As a running example, if $n \leq 31$, then this numeral has the form
\[ a_4 a_3 a_2 a_1 a_0 \]
where each $a_i \in \{0,1\}$ and
\[ n \ = \
a_4 \cdot 2^4 \ + \  a_3 \cdot 2^3 \ + \  a_2 \cdot 2^2 \ + \  a_1
\cdot 2^1 \ + \ a_0
\] 

\item
Use Horner's Rule to rewrite the polynomial for $n$ as follows
\[ n \ = \
a_0 \ + \ 2 \cdot (a_1 \ + \ 2 \cdot (a_2 \ + \ 2 \cdot (a_3 \ + \
2 \cdot a_4 ))) \]

\item
Interpret the ``ground level'' computation of the numeral-polynomial
within the ``exponent level'' of base $b$.

Keep in mind that this ``promotes'' the operations of the polynomial
\begin{itemize}
\item
Each addition of the polynomial spawns a multiplication at the new
ground level, because $b^{c + d} \ = \ b^c \cdot b^d$.
\item
Each multiplication of the polynomialspawns an exponentiation at the
new ground level, because $b^{(c \cdot d)} \ = \ (b^c)^d$.
\end{itemize}

Using our running example, we have
\begin{eqnarray*}
b^n & = &
 b^{a_0 \ + \ 2 \cdot (a_1 \ + \ 2 \cdot (a_2 \ + \ 2 \cdot (a_3 \ + \
          2 \cdot a_4 )))} \\
    & = &
 b^{a_0} \ \times \ 
\left( b^{
(a_1 \ + \ 2 \cdot (a_2 \ + \ 2 \cdot (a_3 \ + \ 2 \cdot a_4 )))}
\right)^2 \\
    & = &
 b^{a_0} \ \times \
\left(
b^{a_1}  \ \times \
\left(
b^{(a_2 \ + \ 2 \cdot (a_3 \ + \ 2 \cdot a_4 ))}
\right)^2
\right)^2 \\
    & = &
b^{a_0} \ \times \
\left(
b^{a_1}  \ \times \
\left(
b^{a_2}  \ \times \
\left(
b^{(a_3 \ + \ 2 \cdot a_4 )}
\right)^2
\right)^2
\right)^2 \\
    & = &
b^{a_0} \ \times \
\left(
b^{a_1}  \ \times \
\left(
b^{a_2}  \ \times \
\left(
b^{a_3}  \ \times \
\left(
b^{a_4}
\right)^2
\right)^2
\right)^2
\right)^2
\end{eqnarray*}
\end{enumerate}
Note that the importance of our using base-$2$ numerals is that the
operation of squaring requires only a single multiplication.

Two simple examples should round out this section:
\begin{eqnarray*}
b^{19_{10}} & = & b^{10011_{2}} \\
        & = & b^{16 + 2 + 1} \\
        & = & b^{16} \times b^2 \times b \\
        & = & ((((b^2)^2)^2)^2) \times b^2 \times b \\
b^{31_{10}} & = & b^{11111_{2}} \\
    & = & b^{16 + 8 + 4 + 2 + 1} \\
    & = & b^{16} \times b^8 \times b^4 \times b^2 \times b^1 \\
    & = & ((((b^2)^2)^2)^2) \times (((b^2)^2)^2) \times ((b^2)^2) \times b^2 \times b
\end{eqnarray*} 


With only a few additional details, we could flesh out the preceding
discussion to a formal proof of the following result.

\index{fast exponentiation}

\begin{prop}[Fast Exponentiation]
\label{thm:fast-exponentiation}
Given positive integers $b$ and $n$, one can compute the number $b^n$
within $O(\log n)$ multiplications.
\end{prop}



\ignore{*****
\bigskip

\noindent
$\oplus$ {\bf Enrichment}.  {\it The dyadic system of numerals.}
\index{numerals!the dyadic system} 
\index{positional number systems!the dyadic system}
 **HERE

{\Denis We have to add a brief discussion about k-ary and k-adic systems...}
**********}

\subsection{Positional numerals for rationals}
\label{sec:special-numerals-Q}

We can completely characterize the positional numerals that represent
rational numbers, in terms of the following auxiliary notion.
\index{ultimately periodic sequence}
An infinite sequence $S$ of digits is {\em ultimately periodic} if
there exist two {\em finite} sequences of digits, $A$ and $B$, such
that $S$ can be written in the following form (we have added spaces to
enhance legibility):
\begin{equation}
\label{eq:ult-per-seq}
 S \ = \ A \ B \ B \ B \cdots B \ B \cdots
\end{equation}
The intention here is that the sequence $B$ is repeated {\it ad
  infinitum}.


\begin{prop}
\label{thm:rational-real}
A positional numeral denotes a rational number if, and only if, it is
ultimately periodic.
\end{prop}

\begin{proof}

\begin{enumerate}
\item 
{\small\sf Part 1: the ``if'' clause.}

Say first that the real number $r$ has an ultimately periodic infinite
base-$b$ numeral.

Since the exact lengths of the finite sequences $A$ and $B$ that
constitute the numeral, as in (\ref{eq:ult-per-seq}), are not germane
to the argument, we arbitrarily denote $r$ by the following
normal-form numeral (spaces added to enhance legibility):
\[  a_2 a_1 a_0 \ . \ b_0 b_1 \
c_0 c_1 c_2 \
c_0 c_1 c_2
\cdots
c_0 c_1 c_2
\cdots
\]
so that
\begin{eqnarray*}
A & = & a_2 a_1 a_0 \ . \ b_0 b_1 \\
B & = & c_0 c_1 c_2
\end{eqnarray*}
(Choosing specific lengths for $A$ and $B$ cuts down on the number of
``ellipsis dots'' we need to denote the numeral, as in ``$123 123
\cdots 123 \cdots$'', hence enhances legibility.)

If we now invoke the evaluation rules of (\ref{eq:real-numeral-number}),
we find that
\begin{eqnarray}
\nonumber
r & = &
\mbox{\sc val}_b(a_2 a_1 a_0 \ . \ b_0 b_1 \
c_0 c_1 c_2 \
c_0 c_1 c_2
\cdots
c_0 c_1 c_2
\cdots) \\
\nonumber
  & = &
\mbox{\sc val}_b(a_2 a_1 a_0)
 \ + \ \mbox{\sc val}_b(b_0 b_1) \cdot b^{-2}
 \ + \
\mbox{\sc val}_b(c_0 c_1 c_2) \cdot b^{-5} \\
\label{eq:sum-in-numeral}
  &  &
 \ + \
\mbox{\sc val}_b(c_0 c_1 c_2) \cdot b^{-8}
 \ + \
\mbox{\sc val}_b(\gamma_0 \gamma_1 \gamma_2) \cdot b^{-11}
\ + \cdots \\
\nonumber
  & = &
\mbox{\sc val}_b(a_2 a_1 a_0)
 \ + \ \mbox{\sc val}_b(b_0 b_1) \cdot b^{-2}
 \ + \
\mbox{\sc val}_b(c_0 c_1 c_2) \cdot \sum_{i=1}^\infty b^{-2-3i}
\end{eqnarray}

We learned in Section~\ref{sec:geometric-sums} that infinite
summations such as the one in (\ref{eq:sum-in-numeral}), namely,
$\sum_{i=1}^\infty b^{-2-3i}$, {\em converge}---meaning that {\em they
  have finite rational sums}---and we learn how to compute these sums.
For the purposes of the current proof, we just take this fact on
faith, and we denote the summation's finite rational sum by $p/q$.

Collecting all of this information, we find that there exist {\em
  integers} $m$, $n$, $p$, and $q$ such that
\[ r \ = \ m \ + \ n/ b^{2} \ + \ p/q \ = \
\frac{mqb^2 + nq + pb^2}{qb^2}. \]
The number $r$ is, thus, the ratio of two integers; hence, by
definition, it is rational.

\item 
{\small\sf Part 2: the `` only if'' clause.}

Say next that the real number $r$ is rational---specifically,
\[ r \ = \ s + \frac{t}{q} \]
for nonnegative integers $t < q$ and $s$.  It is only the fraction
$t/q < 1$ that can produce an infinite numeral, so it suffices for us
to verify the special case
\[ r \ = \ \frac{t}{q} \ < \ 1 \]
of the proposition.

We prove that $r = t/q$ has an ultimately periodic infinite numeral by
using \index{synthetic division} {\it synthetic division}---the
algorithm taught in elementary school---to compute the ratio $t/q$.
As we proceed, keep in mind that we are working in base $b$.  Each of
the following successive divisions produces one digit to the right of
the radix point, in addition to a possible {\it remainder} $r_i$ from
the set $\{0, 1, \ldots, q-1\}$.
\begin{equation}
\label{eq:build-rational-numeral}
\begin{array}{lclc|l|cc}
\multicolumn{3}{c}{\mbox{Division step}} & &  \hspace*{.15in} \mbox{Current numeral} & &
\mbox{Current remainder} \\
\hline
b \cdot t   & = & a_0 \cdot q \ + \ r_0 &
      & t/q \ = \ .a_0 \cdots &
      & r_0 < q \\
b \cdot r_0 & = & a_1 \cdot q \ + \ r_1 &
      & t/q \ = \ .a_0 a_1 \cdots &
      & r_1 < q \\
            & \vdots &  & & \hspace*{.3in} \vdots &  & \vdots \\
b \cdot r_i & = & a_{i+1} \cdot q \ + \ r_{i+1} &
      & t/q \ = \ .a_0 a_1 \cdots a_{i+1} \cdots &
      & r_{i+1} < q \\
b \cdot r_{i+1} & = & a_{i+2} \cdot q \ + \ r_{i+2} &
      & t/q \ = \ .a_0 a_1 \cdots a_{i+1} a_{i+2} \cdots &
      & r_{i+2} < q \\
            & \vdots &  & &  \hspace*{.3in}\vdots & & \vdots   \\
\end{array}
\end{equation}
Because of the possible values the remainders $r_j$ can assume, no
more than $q$ of the divisions in the (infinite) system
(\ref{eq:build-rational-numeral}) are distinct.  ({\em This is an
  application of the pigeonhole principle
  (Section~\ref{sec:pigeonhole}).})  Because of the way the system
proceeds, once we have encountered two remainders, say, $r_i$ and
$r_{i+k}$, that are equal---i.e., $r_i = r_{i+k}$---we must
thenceforth observe periodic behavior:
\[
\begin{array}{cccccccc}
r_i       & = & r_{i+k}    & = & r_{i+2k}   & = & r_{i+3k}   & = \ \cdots \\
r_{i+1}   & = & r_{i+k+1}  & = & r_{i+2k+1} & = & r_{i+3k+1} & = \ \cdots \\
\vdots    &   & \vdots     &   & \vdots     &   & \vdots     & \\
r_{i+k-1} & = & r_{i+2k-1} & = & r_{i+3k-1} & = & r_{i+4k-1} & = \ \cdots \\
\end{array}
\]
This will engender periodicity in the digits of $r$'s base-$b$ numeral:
\[ [\mbox{\sc initial segment}]
 [a_i a_{i+1} \cdots a_{i+k-1}]
          [a_i a_{i+1} \cdots a_{i+k-1}]
    \cdots  [a_i a_{i+1} \cdots a_{i+k-1}] \cdots 
\]
We are, thus, observing the claimed ultimately periodic behavior in
$r$'s base-$b$ numeral.
\end{enumerate}
This completes the proof. \qed
\end{proof}

\bigskip

We end this section by illustrating the process of generating numerals
for rationals via synthetic division.  We employ the fraction $t/q = 4/7$
and base $b = 10$.
\[
\begin{array}{lclc|l|cc}
\multicolumn{3}{c}{\mbox{Division step}} & &  \hspace*{.15in} \mbox{Current numeral} & &
\mbox{Current remainder} \\
\hline
10 \cdot 4   & = & 5 \cdot 7 \ + \ 5 &
      & 4/7 \ = \ .5 \cdots &
      & 5 \\
10 \cdot 5 & = & 7 \cdot 7 \ + \ 1 &
      & 4/7 \ = \ .57 \cdots &
      & 1 \\
10 \cdot 1 & = & 1 \cdot 7 \ + \ 3 &
      & 4/7 \ = \ .571 \cdots &
      & 3 \\
10 \cdot 3 & = & 4 \cdot 7 \ + \ 2 &
      & 4/7 \ = \ .5714 \cdots &
      & 2 \\
10 \cdot 2 & = & 2 \cdot 7 \ + \ 6 &
      & 4/7 \ = \ .57142 \cdots &
      & 6 \\
10 \cdot 6 & = & 8 \cdot 7 \ + \ 4 &
      & 4/7 \ = \ .571428 \cdots &
      & 4 \\
 & \vdots & & & \vdots & & \vdots
\end{array}
\]
The remainder $4$ in the last illustrated division step cycles us back
to the initial division step, where the ``$4$'' came from the
numerator of the target fraction.  This repetition signals that the
entire process cycles from this point on.  In other words, we have
determined that
\[ \frac{4}{7} \ = \ .[571428] \ [571428] \ [571428] \ \cdots \]

\bigskip

Propositions~\ref{thm:integer-real} and~\ref{thm:rational-real} show
us that the three sets of numbers we have defined are a nested
progression of successively more inclusive sets, in the sense that
{\em every integer is a rational number} and {\em every rational
  number is a real number}.  Those interested in the (philosophical)
foundations of mathematics might quibble about the verb ``is'' in the
highlighted sentences, but for all practical purposes, we can accept
the sentences as written.

\section{Sets that Are Uncountable, Hence, ``Bigger Than'' $\Z$ and $\Q$}
\label{sec:Q-Z-F-cardinality}
\label{sec:FNS-uncountable}

\subsection{$\F$ Is Uncountable}
\label{sec:F-uncountable}

Let $\F$ denote the set of all functions from $\N$ to $\{0,1\}$.  The
main result of this section establishes the uncountability of the set
$\F$.  We thereby have an argument that the infinitude of this set is
{\em of a higher order} than the infinitude of the set of integers.
In fact, Georg Cantor used variants of this result as the base of his
study of orders of infinity.

This section is dedicated to provin the following result.

\begin{prop}
\label{thm:F-uncountable}
The set $\F$ of binary-valued functions from $\N$ into $\{0,1\}$ is
not countable.  In particular, there is no injection $f: \F
\rightarrow \N$.
\end{prop}

In the next subsection, we prove the qualitatively similar but
technically more complicated companion of
Proposition~\ref{thm:F-uncountable}, which establishes that the set
$\R$ of real numbers is also uncountable.

\begin{proof}
Our multi-step proof of Proposition~\ref{thm:F-uncountable} shows that
assuming the countability of $\F$ leads to a contradiction.  Being
built around Georg Cantor's renowned {\it diagonalization
  construction},\index{diagonalization} upcoming development
\index{diagonalization argument} this upcoming development provides
the most sophisticated proofs by contradiction in our text.  The
reader might want to review the reasoning underlying such
argumentation, in Section~\ref{sec:Contradiction}, as a ``warm-up.''

\subsubsection{Plotting a strategy to prove uncountability}
\label{sec:the-diag-strategy}

Invoking the definition of countability, our proof begins with the
assumption that $|\F| \leq |\N|$ and demonstrates that this assumption
leads to a contradiction.  In fact, we can simplify our goal by
recasting the problem in terms of {\em bijections}.  We begin with two
simplifying lemmas.

\begin{lemma}
\label{lem:N-leq-F}
There exists an injection $f: \N \rightarrow \F$; i.e., $|\N| \leq
|\F|$.
\end{lemma}

\noindent {\it Verification}.
For each nonnegative integer $n \in \N$, define the function $f_n: \N
\rightarrow \{0,1\}$ as follows: for each integer $k \in \N$,
\[ f_n(k) \ = \ \mbox{\bf if } \ [k=n] \ \mbox{\bf then } \ 1
\ \mbox{\bf else } \ 0
\]
Clearly, specifying the integer $n$ uniquely identifies the function
$f_n$.  This means that the defined correspondence specifiess an
injection from $\N$ into $\F$.  \qed

\begin{lemma}
\label{lem:N-=-F}
If there exists an injection $g: \F \rightarrow \N$, then there exists
a bijection $h: \N \leftrightarrow \F$.  In other words, if $|\F| \leq
|\N|$, then $|\N| = |\F|$.
\end{lemma}

\noindent {\it Verification}.
This is an immediate consequence of the Schr\"{o}der-Bernstein Theorem
(Theorem~\ref{thm.S-B}).  \qed

\medskip

It is perhaps a bit surprising that our proof is simplified by
converting our initial assumption \\
\hspace*{.35in}$|\F| \leq |\N|$ \\
to the stronger assumption \\
\hspace*{.35in}$|\F| \ = \ |\N|$,  \\
but Cantor's diagonal argument deals quite gracefully with the latter
assumption.

\paragraph{\sf A. Seeking a bijection $g: \N \leftrightarrow \F$}

We assume, for contradiction, that a bijection $g: \N \leftrightarrow
\F$ exists.  As part of this two-way mapping, there exists an {\em
  injection}
\[ 
h: \N \ \rightarrow \ \F
\]
We view $h$ as an {\em enumeration}---i.e., an ordered listing---of
the elements of $\F$.  Specifically, for each integer $k \in \N$, we
can think of $h(k)$ as the ``$k$th binary-valued function in the set
$\F$.''  We thereby view $h$ as producing an ``infinite-by-infinite''
matrix $\Delta$ of bits (i.e., binary digits), whose $k$th row is the
infinite string of bits that is the characteristic vector of the
function $h(k)$.  Let us visualize $\Delta$:
\[ \Delta \ \ = \ \
\begin{array}{ccccccc}
\delta_0 = &
\delta_{0,0} & \delta_{0,1} & \delta_{0,2} & \delta_{0,3} &
	\delta_{0,4} & \cdots \\
\delta_1 = &
\delta_{1,0} & \delta_{1,1} & \delta_{1,2} & \delta_{1,3} &
	\delta_{1,4} & \cdots \\
\delta_2 = &
\delta_{2,0} & \delta_{2,1} & \delta_{2,2} & \delta_{2,3} &
	\delta_{2,4} & \cdots \\
\delta_3 = &
\delta_{3,0} & \delta_{3,1} & \delta_{3,2} & \delta_{3,3} &
	\delta_{3,4} & \cdots \\ 
\delta_4 = &
\delta_{4,0} & \delta_{4,1} & \delta_{4,2} & \delta_{4,3} &
	\delta_{4,4} & \cdots \\ 
\vdots &
\vdots  & \vdots  & \vdots  & \vdots  & \vdots  & \ddots
\end{array}
\]

\noindent We summarize, for emphasis:
\begin{itemize}
\item
Each row of $\Delta$ consists of the characteristic vector of a
function in the set $\F$ (which can be thought of as a ``name'' for
the function).

\item
Each function in the set $\F$ contributes its characteristic vector as
a row of $\Delta$.
\end{itemize}
We can, thus, view the successive rows of $\Delta$, $h(0)$, $h(1)$,
\ldots, as an enumeration of all of the functions in the set $\F$.  In
other words, we can view $\Delta$ as ``containing'' each function in
$\F$ precisely once.

\paragraph{\sf B. Every bijection $h: \N \leftrightarrow \F$ ``misses'' some function}

We are finally poised to find the contradiction to our assumption that
$|\F| \leq |\N|$.  Specifically, we define from $\Delta$ an infinite
bit-string
\[ \Psi \ = \ \psi_0 \ \psi_1 \ \psi_2 \ \psi_3 \ \psi_4 \cdots, \]
that {\em does not} appear in $\Delta$.  For each index $i \in \N$, we
define the $i$th (binary) digit $\psi_i$ of $\Psi$ from the $i$th {\em
  diagonal (binary) digit}\footnote{Our use of $\Delta$'s ``diagonal
  digits'' in this definition is the origin of the term ``{\em
    diagonal argument}'' \index{diagonal argument} to describe this
  proof and its intellectual kin.}~$\delta_{i,i}$ of $\Delta$ in the
following manner.
\[ \psi_i \ \ \eqdef \ \ \overline{\delta}_{i,i} \ \ = \ \
\Big[\mbox{\bf if } \ [\delta_{i,i} \ = \ 0] \ \mbox{\bf then } 1 \
\mbox{\bf else } \ 0 \Big]
\]
The important feature of the definition is the following.

\begin{lemma}
\label{lem:PSI-notin-DELTA}
The bit-string $\Psi$ does not occur as a row of $\Delta$.
\end{lemma}

\noindent {\it Verification}.
This is obvious because the bit-string
$\Psi$ differs from each row $k$ of $\Delta$ in the $k$th position; i.e.,
$\psi_k \neq \delta_{k,k}$.  \qed

\subsubsection{The denouement: There is no bijection  $h: \N \leftrightarrow \F$}

The infinite binary string $\Psi$ differs from every row of $\Delta$,
even though $\Psi$ is the characteristic vector of a binary valued
function.  Therefore, $\Delta$ {\em does not} contain as a row {\em
  every} element of $\F$, i.e., the characteristic vector of every
binary-valued function.  But this contradicts $\Delta$'s assumed
defining characteristic!

Where could we have gone wrong?  Every step of our argument, save one,
is backed up by a proof---so the one step that is not so bolstered
must be the link that has broken the argument.  This one
unsubstantiated step is our assumption that the set $\F$ is countable.
Since this assumption has led us to a contradiction, we must conclude
that the set $\F$, is {\em not} countable!  \qed
\end{proof}

\subsection{$\oplus$ $\R$ Is Uncountable}
\label{sec:R-uncountable}
\label{sec:Reals-uncountable}

This section is a companion to Section~\ref{sec:FNS-uncountable}, in
which we establish the uncountability of the set $\R$ of real numbers.  We thereby
have an argument that the infinitude of the set of real numbers is
{\em of a higher order} than the infinitude of the set of integers.
We accomplish by providing a proof of the following result.

\begin{prop}
\label{thm:R-uncountable}
The set $\R$ of real numbers is not countable.  In particular, there
is no injection $g: \R \rightarrow \N$.
\end{prop}

\begin{proof}
We simplify our exposition by making extensive use of the proof in
Section~\ref{sec:FNS-uncountable} that the set $\F$ of functions from
$\N$ into $\{0,1\}$ is not countable (Proposition~\ref{thm:F-uncountable}).

Our proof begins with the following lemmas.

\begin{lemma}
\label{lem:N-leq-R}
There exists an injection $f: \N \rightarrow \R$; i.e., $|\N| \leq |\R|$.
\end{lemma}

\noindent {\it Verification}.
For each nonnegative integer $n \in \N$, let $\mbox{\sc name}_b(n)$
denote the shortest base-$b$ numeral for $n$, i.e., a numeral with no
leading $0$s.  The mapping that associates each $n \in \N$ with the
infinite string
\[ \mbox{\sc name}_b(n) \ . \ 00 \cdots \]
is an injection from $\N$ into $\R$.  To wit, when given a real
numeral that has only $0$s to the right of its radix point, one
produces the integer $n$ by stripping the numeral of its radix point
and all $0$s to the right of the point, and then evaluating the
remaining string of digits, which is $\mbox{\sc name}_b(n)$, according
to Section~\ref{sec:define-Reals}'s rules for evaluating integer numerals.  \qed

\begin{lemma}
\label{lem:N-=-R}
If there exists an injection $g: \R \rightarrow \N$, then there exists a bijection 
$h: \N \leftrightarrow \R$.  In other words, if $|\R| \leq |\N|$, then $|\N| = |\R|$.
\end{lemma}

\noindent {\it Verification}.
This is an immediate consequence of the Schr\"{o}der-Bernstein Theorem
(Theorem~\ref{thm.S-B}).  \qed

\medskip

We make a few technical assumptions that simplify our proof without weakening the result.

\noindent
{\em We henceforth focus on the proper subset of $\R$ that consists of
  the set $\R_{(0,1)}$ of real numbers between $0$ and $1$.}

This simplifies our argument because every real number in the set
$\R_{(0,1)}$ has an infinite decimal numeral of the form
\[ 0 \ . \ \delta_0 \delta_1 \delta_2 \delta_3 \cdots \]
where each $\delta_i$ is a decimal digit: $\delta_i \in \{0, 1, 2, 3,
4, 5, 6, 7, 8, 9\}$.\footnote{We choose {\em decimal} numerals for
  convenience: converting the argument to other number
  bases---especially to the base $b=2$---slightly complicates clerical
  details.}

\smallskip

Of course, if we prove that the proper subset $\R_{(0,1)} \subset \R$
is uncountable, then it will follow that $\R$ is uncountable.  (In
informal terms which can be made formal, any putative injection $f: \R
\rightarrow \N$ ``contains'' an injection $f_{(0,1)}: \R_{(0,1)} \rightarrow \N$.)

\bigskip

Now we begin to seek a bijection $h: \N \leftrightarrow \R_{(0,1)}$.

We assume, for contradiction, that the targeted bijection $g$ exists.
As part of this two-way mapping, there exists an {\em injection}
\[ 
h: \N \ \rightarrow \ \R_{(0,1)},
\]
which we view as an {\em enumeration} of the elements of $\R_{(0,1)}$.
Specifically, for each integer $k \in \N$, we can think of $h(k)$ as
the ``$k$th number in the set $\R_{(0,1)}$.''  We thereby view $h$ as
producing an ``infinite-by-infinite'' matrix $\Delta$ of decimal
digits, whose $k$th row is the infinite string of decimal digits
$\mbox{\sc name}_{10}(h(k))$.  Let us visualize $\Delta$:
\[ \Delta \ = \
\begin{array}{ccccccc}
\delta_0 = &
\delta_{0,0} & \delta_{0,1} & \delta_{0,2} & \delta_{0,3} &
	\delta_{0,4} & \cdots \\
\delta_1 = &
\delta_{1,0} & \delta_{1,1} & \delta_{1,2} & \delta_{1,3} &
	\delta_{1,4} & \cdots \\
\delta_2 = &
\delta_{2,0} & \delta_{2,1} & \delta_{2,2} & \delta_{2,3} &
	\delta_{2,4} & \cdots \\
\delta_3 = &
\delta_{3,0} & \delta_{3,1} & \delta_{3,2} & \delta_{3,3} &
	\delta_{3,4} & \cdots \\ 
\delta_4 = &
\delta_{4,0} & \delta_{4,1} & \delta_{4,2} & \delta_{4,3} &
	\delta_{4,4} & \cdots \\ 
\vdots &
\vdots  & \vdots  & \vdots  & \vdots  & \vdots  & \ddots
\end{array}
\]

\noindent We summarize, for emphasis:
\begin{itemize}
\item
Each row of $\Delta$ consists of the decimal numeral for a number in
the set $\R_{(0,1)}$.

\item
Each number in the set $\R_{(0,1)}$ contributes at least one numeral
to the rows of $\Delta$.

A number may contribute more than one numeral because of an artifact
of positional number systems, which is exemplified by equations such
as
\[ 0.25 \ = \ 0.24999\cdots \]
\end{itemize}
We can, thus, view the successive rows of $\Delta$: $h(0)$, $h(1)$,
\ldots, as an ordered listing (with possible repetitions) of all of
the real numbers in the set $\R_{(0,1)}$.

\medskip

We show now that every bijection $h: \N \leftrightarrow \R$ ``misses''
some real $x \in \R_{(0,1)}$.

We are finally poised to find the contradiction to our assumption that
$|\R_{(0,1)}| \leq |\N|$.  Specifically, we define from $\Delta$ an
infinite decimal numeral
\[ \Psi \ = \ \psi_0 \ \psi_1 \ \psi_2 \ \psi_3 \ \psi_4 \cdots, \]
that {\em does not} appear in $\Delta$, even though $\mbox{\sc
  val}_{10}(\Psi) \in \R_{(0,1)}$.  For each index $i \in \N$, we
define the $i$th digit $\psi_i$ of $\Psi$ from the $i$th {\em diagonal
  digit}\footnote{Our use of $\Delta$'s ``diagonal digits'' in this
  definition is the origin of the term ``{\em diagonal argument}'' to
  describe this proof and its intellectual kin.}~$\delta_{i,i}$ of
$\Delta$ in the following manner.
\[ \psi_i \ \eqdef \
\left\{
\begin{array}{cc}
0 & \mbox{ if } \ \delta_{i,i} \ > \ 5 \\
9 & \mbox{ if } \ \delta_{i,i} \ \leq \ 5 \\
\end{array}
\right.
\]
The important feature of the definition is the following.

\begin{lemma}
\label{lem:PSI-notin-DELTA-num}
The numeral $\Psi$ does not occur as a row of $\Delta$.
\end{lemma}

\noindent {\it Verification}.
Focus on an arbitrary row of $\Delta$, say row $k$, and on the numeral,
$\delta_k$, in that row.

\begin{tabular}{lclc}
If & $\delta_{k,k} \ > \ 5$ & then & \ \
$\mbox{\sc name}_{10}(\delta_k) \ - \ \mbox{\sc name}_{10}(\Psi) \ > \ 4
\cdot 10^{-k}$ \\
If & $\delta_{k,k} \ \leq \ 5$ & then & \ \
$\mbox{\sc name}_{10}(\Psi) \ - \ \mbox{\sc name}_{10}(\delta_k) \ > \ 4
\cdot 10^{-k}$
\end{tabular}

\noindent
In either case, we have $\mbox{\sc name}_{10}(\Psi) \ \neq \ \mbox{\sc
  name}_{10}(\delta_k)$ so that $\Psi$ does not appear as row $k$ of
$\Delta$.  Since $k \in \N$ is an arbitrary row-index of $\Delta$, we
conclude that $\Psi$ does not occur as any row of $\Delta$.  \qed

\medskip

We are now ready for the denouement of our argument: There is no
bijection $h: \N \leftrightarrow \R$

Because the infinite decimal string $\Psi$ differs from every row of
$\Delta$, even though $\mbox{\sc name}_{10}(\Psi) \in \R_{(0,1)}$, we
have shown that $\Delta$ {\em does not} contain as a row {\em every}
infinite decimal numeral of a number in $\R_{(0,1)}$.  But this
contradicts $\Delta$'s assumed defining characteristic!

Where could we have gone wrong?  Every step of our argument, save one,
is backed up by a proof---so the one step that is not so bolstered
must be the link that has broken.  This one unsubstantiated step is
our assumption that the set $\R_{(0,1)}$ is countable.  Since this
assumption has led us to a contradiction, we must conclude that the
set $\R_{(0,1)}$, and hence, the set $\R$, is {\em not} countable!  \qed
\end{proof}



\section{Scientific Notation}
\label{sec:scientific-notation}
\index{scientific notation}

There is a familiar game in which one is challenged to guess how many
beans there are in a jar. The wild ranges of guesses that players make
indicate eloquently what is one of the main starting points in the
popular-science book {\it Innumeracy} \cite{Paulos}: While we ``know''
a lot about even {\em very} large numbers and {\em very} small
numbers, we often lack {\em operational} command of the numbers.  This
fact can be illustrated in at least two ways.

\noindent {\bf 1.}
The ability to compare the magnitudes of big numbers:
\begin{itemize}
\item
Can you compare the probabilities of a person's being hit by lightning
(say, in Mexico City) or by a car (say, crossing Fifth Avenue in Times
Square)?
\item
Do you know whether you have more hairs on your body than there are
grains of sand on the beach at Ipanema, Brazil?
\item

Do you know whether there were more Homo Sapiens alive on December 31,
1999, than had lived from the Big Bang to December 31, 1945?
\item
Can you compare the number of rhinoviruses that can populate a square
of side-length $1$mm with the number of stars visible on a clear night
at the summit of Mount Everest?
\end{itemize}

\noindent {\bf 2.}
The ability to delineate ``how much information'' a number tells us:
Many of us know---or can calculate---that (in some sense) the distance
between Earth and its closest star, the Sun, is, very roughly,
\[ \begin{array}{rl}
93,000,000 & \mbox{ miles} \\
148,800,000 & \mbox{ km} \\
491,040,000,000 & \mbox{ feet} \\
5,892,480,000,000 & \mbox{ inches}
\end{array}
\]

\medskip

\noindent \fbox{
\begin{minipage}{0.97\textwidth}
One might read an argument in favor of the metric system into the
preceding listing of ``miles'' and ``feet'' and ``inches'', whose
interrelationships require a lexicon, in contrast to the singular
``km'' whose relationships to ``cm'' and ``meter'' are transparent.
\end{minipage}
}

\medskip

All of these numbers are coarse approximations.  In some sense, they
all convey exactly the same information, since all are obtained from
the first number (the number of miles to the Sun) by simple scaling.
Yet, while the first number projects a modest two (decimal) digits of
accuracy, the others project, respectively, four digits, five digits,
and six digits.  Do all of these numbers convey the same (level of)
truth?

\medskip

Scientists and pedagogues and philosophers have grappled with the
problems engendered by innumeracy throughout time; see, e.g.,
Footnote~\ref{foot:Pascal}.  One ingenious approach within the domain
of astronomy has been to establish a new standard unit of distance to
express the {\em very} large distances from Earth to stars beyond our
solar system: A {\em light year} is the distance that light travels in
an Earth-year, roughly $9.4607 \times 10^{12}$ km.  By using this
measure, one can describe enormous (well, astronomical) numbers
without unwarranted appearances of inflated accuracy.  The notion of
light year plays an important role for astronomy, but it does not port
gracefully to other domains, for two reasons: (1) The use of the speed
of light as a frame of reference has no meaning when one is, for
instance counting grains of sand or numbers of viruses.  (2) The
scaling factor inherent in a light year is not appropriate for other
domains.  The widely accepted general alternative to a new scaling
unit is {\em scientific notation}.  \index{scientific notation}

Within scientific notation, one specifies an arbitrary number, of
arbitrary magnitude via a {\em rational approximation} of the form
\[ . \beta_0 \beta_1 \beta_2 \cdots \beta_{a-1} \times b^s \]
The interpretation is that
\begin{itemize}
\item
$\beta_0 \beta_1 \beta_2 \cdots \beta_{a-1}$ represents the $a$
  base-$b$ {\em digits of accuracy} that are warranted by the accuracy
  of one's level of knowledge about the number being specified.

\item
$b^s$ is the base-$b$ {\em scaling factor} that adjust the digits of
  accuracy relative to the radix point.
\end{itemize}
Within this system of specification, we thus have
\[ \begin{array}{lcll}
.93    & \times & 10^8     & \mbox{ miles from Earth to the Sun} \\
.94607 & \times & 10^{13}  & \mbox{ kilometers traveled by light in an Earth-year} \\
.31415 & \times & 10       & \mbox{ value of $\pi$ to $5$ digits of accuracy} \\
.166   & \times & 10^{-23} & \mbox{ grams of weight of a proton, to $3$ digits of accuracy}
\end{array}
\]

