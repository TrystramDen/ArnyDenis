%version of 02-27-19

\chapter{EXERCISES}
\label{ch:Exercises}






\subsubsection{A fun result: complex
  multiplication via $3$ real multiplications}
\index{complex number!multiplication via 3 real multiplications}

\begin{prop}
\label{thm:complex-mult-3real}
One can compute the product of two complex numbers using {\em three}
real multiplications rather than four.
\end{prop}

\begin{proof}
Although implementing (\ref{eq:complex-mult}) ``directly'' correctly
produces the product $\kappa = (a+bi) \cdot (c+di)$, there is another
implementation that is {\em more efficient}.  Specifically, the
following recipe computes $\kappa$ using only {\em three} real
multiplications instead of the four real multiplications of the
``direct'' implementation.  We begin to search for this recipe by
noting that our immediate goal is to compute both Re$(\kappa) = ac-bd$
and Im$(\kappa) = ad+bc$.  We can accomplish this by computing the
{\em three} real products
\begin{equation}
\label{eq:complex-mult-3a}
(a+b) \cdot (c+d); \ \ \ \ \
ac;  \ \ \ \ \ bd
\end{equation}
and then noting that
\begin{equation}
\label{eq:complex-mult-3b}
\begin{array}{lcl}
\mbox{Im}(\kappa) & = & (a+b) \cdot (c+d) - ac -bd, \\
\mbox{Re}(\kappa) & = & ac -bd
\end{array}
\end{equation}
We thereby achieve the result of the complex multiplication described
in (\ref{eq:complex-mult}) while using only {\em three} real
multiplications.

Of course, a full reckoning of the costs of the two implementations we
have discussed exposes the fact that the implementation that invokes
(\ref{eq:complex-mult-3a}) and (\ref{eq:complex-mult-3b}) uses {\em
  three} real additions rather than the {\em two} real additions of
the ``direct'' implementation.  But this entire exercise was
predicated on the observation that each real addition is much less
costly than a real multiplication, so trading one multiplication for
one addition is an unqualified ``win''.  \qed
\end{proof}

{\Denis I added a sentence to refer to an exercice dealing with karatsuba which uses the same idea...}
Notice that this technique is classical and it has been used in many other situations.
For instance while multiplying two integers in base 2 (see exercice~{Karatsuba}).




\subsection{A Fun Result: A ``Trick'' for Squaring Certain Integers}

Sometimes only basic knowledge is needed to craft amusing
``tricks''---we know that they are not really tricks at all!---that
are really rigorous applications of principles that we have learned.
Here is an ``old chestnut'' example that may inspire you to design
your own. 

If someone presents you with a number that has a numeral that ends in
$5$, then there is a simple way to square the number mentally.  For
instance, if someone says

\hspace{.25in}``$n = 25$''

\noindent
then you can instantly respond

\hspace{.25in}``$n^2 = 625$''

\noindent
If the challenge is

\hspace{.25in}``$n = 75$''

\noindent
then your response is

\hspace{.25in}``$n^2 = 5625$''

\noindent
Let's make this ``game'' mathematical.

\begin{prop}
\label{thm:75x65=4925}
Let $n$ be any number that has a $2$-digit decimal numeral of the form

\hspace{.25in}$\delta \ 5$ \ \ \ \ $(\delta \in \{ 0,1,2,3,4,5,6,7,8,9\})$.

\noindent
Then the square of $n$ is the integer

\hspace{.25in}$25 \ + \ \delta \cdot (\delta +1)$. 
\end{prop}

\begin{proof}
We can rewrite the premise of the proposition in the form
\[ n \ = \ 10 \cdot \delta + 5 \]
It is now easy to invoke Proposition~\ref{prop:(a+b)(c+d)} and the
distributive law to compute that

\[ n^2 \ = \ 100 \cdot \delta \cdot (\delta+1) + 25. \]
To wit: 
\[
\begin{array}{lclll}
n^2 & = & (10 \cdot \delta + 5)^2 & & \mbox{Given} \\
    & = & 100 \cdot \delta^2 \ + \ 100 \cdot delta \ + \ 25
              & & \mbox{the proposition} \\
    & = & 100 \cdot (\delta^2 \ + \ \delta) \ + \ 25
              & & \mbox{factoring: distributive law} \\
    & = & 100 \cdot \delta \cdot (\delta + 1) \ + \ 25
              & & \mbox{factoring: distributive law} \\
\end{array}
\]
A parlor trick has become a mathematical demonstration!
\qed
\end{proof}


\subsubsection{A fun result via geometric sums: When is integer  $n$
  divisible by $9$?}
\label{sec:divisible-by-9}

We now exploit our ability to evaluate geometric summations to
illustrate a somewhat surprising, nontrivial fact.  One can deduce
information about the divisibility of an integer $n$ from $n$'s
positional numerals.  We hope that this ``fun'' result will inspire
the reader to seek kindred numeral-encoded properties of numbers.

\begin{prop}
\label{thm:div-by-b-bar}
An integer $n$ is divisible by an integer $m$ if, and only if, $m$
divides the sum of the digits in the base-$(m+1)$ numeral for $n$.
\end{prop}

The most familiar instance of this result is phrased in terms of our
traditional use of base-$10$ (decimal) numerals. \\
{\it An integer $n$ is divisible by $9$ if, and only if, the sum of
  the digits of $n$'s base-$10$ numeral is divisible by $9$.}

\smallskip

\begin{proof}
({\it Argument for general number-base $b$}).
%
Of course, we lose no generality by focusing on numerals without
leading $0$'s, because leading $0$'s do not alter a numeral's sum of
digits.

Let us focus on the base-$b$ numeral for a number $n$ (so $b = m+1$ in
the statement of the proposition).  There therefore exist base-$b$
digits---i.e., integers from the set $\{0, 1, \ldots, b-1\}$---call
them $\delta_k \neq 0$, $\delta_{k-1}$, \ldots $\delta_1$, $\delta_0$,
such that
\[ n \ = \ \delta_k \cdot b^k + \delta_{k-1} \cdot b_{k-1} + \cdots +
\delta_1 \cdot b + \delta_0. \]
The sum of the digits of $n$'s base-$b$ numeral is, then
\[ s_b(n) \ \eqdef \ \delta_k + \delta_{k-1} + \cdots + \delta_1 +
\delta_0. \]
Let us calculate the difference $n - s_b(n)$ in the following manner,
digit by digit.
\begin{equation}
\label{eq:sum-of-digits}
\begin{array}{ccccccccccc}
n & = &
\delta_k \cdot b^k & + & \delta_{k-1} \cdot b^{k-1} & + & \cdots
  & + & \delta_1 \cdot b & + & \delta_0 \\
s_b(n) & = &
\delta_k & + & \delta_{k-1} & + & \cdots & + & \delta_1 & + & \delta_0 \\
\hline
n - s_b(n) & = &
\delta_k \cdot (b^k -1) & + &
\delta_{k-1} \cdot (b^{k-1} -1) & + &
\cdots & + &
\delta_1 \cdot (b-1) & & 
\end{array}
\end{equation}

\medskip

We now revisit summation (\ref{eq:geom-sum:b>1}).  Because $b$ is a
positive integer, so that $1 + b + \cdots + b^{a-2} + b^{a-1}$ is also
a positive integer, we infer that {\em the integer $b^a -1$ is
  divisible by $b-1$.}

We are almost home.  Look at the equation for $n - s_b(n)$ in the
system (\ref{eq:sum-of-digits}).  As we have just seen, every term on
the righthand side of that equation is divisible by $b-1$.  It follows
therefore, that the lefthand expression, $n - s_b(n)$, is also
divisible by $b-1$.
An easy calculation, which we leave to the reader, now shows that this
final fact means that $n$ is divisible by $b-1$ if, and only if,
$s_b(n)$ is.
\end{proof}
