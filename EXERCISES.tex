%version of 07-03-19

\chapter{EXERCISES}
\label{ch:Exercises}

*************

For all $\langle x,y \rangle \in \N^+ \times \N^+$

$\a(x,y) \ \eqdef \ 2^{x-1} \cdot (2y -1)$

We leave to the reader the exercise of verifying function $\a$'s
bijectiveness.

*************

As an exercise, the reader
should define the following two operations from union and set
difference: {\it intersection} ($S \cap T$), which isolates all
elements that sets $S$ and $T$ share, and {\it symmetric difference}
($S+T$), which isolates all elements that belong to precisely one of
$S$ and $T$.

*************

We leave the following as an exercise for the reader.  After
consulting Fig.~\ref{fig:defns-via-tables}:

\begin{tabular}{llcl}
1. &
{\it Prove that the Propositional expression} &
$\sim P \ \Rightarrow \ P$ &
{\it is a tautology.} \\
2. &
{\it Prove that the Propositional expression} &
$P \ \Rightarrow \ \sim P$ &
{\it is satisfiable.}
\end{tabular}



**************

Analyze the game of craps in the same way as we did for sum-of-three

Verify the table in Fig.~\ref{fig:dice-ordered-configs}

In the ordered version of sum-of-three, sum$k$ is engendered by the
same number of configurations as is sum $21 - k$.  Why is this the
case?


***************

An interesting variant of the birthday puzzle: determine the
probability of a student to be born the same day as me.

****************

Say that you are told that the sum of the first $n$ perfect squares is
a {\em cubic} polynomial in $n$.  Use the mthod of undetermined
coefficients to derive the exact formula (\ref{eq:sum-1-to-nsq1}) for
the sum.

*****************

Prove that when you do Euclidian division, then the {\em remainder}
upon each consecutive division is the next lowest-order digit in the
base-$b$ numeral for $n$.

******************

Prove Proposition~\ref{thm:pascal-binom} via the double induction
outlined in the text.


*******************

Complete the proof of Fig. 7.17 and similar geometric arguments: show
that all points in the areas are captured

********************

Prove Proposition~\ref{thm:|Q|=|N|}

\begin{prop}
$|\Q| \ = \ |\N|$.
\end{prop}

from Proposition~\ref{thm:|NxN|=|N|}.


*********************

Prove that the views of SAT are equivalent: (1) a set of clauses; (2)
a disjunction of clauses.


**********************

 Prove the following

\begin{theorem}
The equality relation, $=$, on a set $S$ is the {\em finest}
equivalence relation on $S$, in the sense that $=$ refines every
equivalence relation on $S$.
\end{theorem}

*********************

 Play with negating strong (weak) relations to get weak (strong) ones.

*********************

 Use De Morgan-type exercises.


*********************

 The biggest and smallest elements of
${\cal P}(S)$ are, respectively, the set $S$ itself and the empty set
$\emptyset$.


{\Arny
I think that solving specific examples of cubics via radicals, although specialized, does
involve useful skills for manipulating polynomials.  Therefore, I
propose to stop the general case here and leave a few exercises.  WHAT DO YOU THINK?}



{\Arny Because your geometric solution below involves just a single
  special quadratic --- and one that seems not to have other special
  interest --- I propose that we include it in some form of ENRICHMENT
  section.  }

{\ignore {\Denis It is interesting here to stress how hard it was to solve such equations. Algebra raised only in the second half of the last millenium.
However, there are several nice examples of geometrical solutions.
One of the oldest comes from babylonians in the 18th century BC.
The numeral system was in base 60, and the problem was to determine the length of the side of a square which was part of a larger rectangle.
The following figure details the process.}}

\begin{figure}[htb]
\begin{center}
       \includegraphics[scale=0.4]{FiguresArithmetic/tabletteMesopotamie}
\caption{Solving $x^2 + x = 45$.
The idea of the proof is to represent the left hand side by the square $x^2$ beside a rectangle $60 \times x$.
Then, split the right rectangle into two equal parts and move one part a the bottom of the left square.
The final figure shows the whole square whose surface is equal to $45$ plus the surface of the white square
whose surface is equal to $30 \times 30$.
In base $60$, this is $15$. 
$45+15 = 60$, thus, the big square is the unit square, its side is $60$.
Thus, the length of the initial square is equal to $60-30=30$.}
\label{fig:equationBabillon}
\end{center}
\end{figure}


\section{misc}


\subsection{The average length of a carry in a binary counter}

\noindent {\it The problem.}
%
You add from $1$ to $n$, in increments of $1$ using a counter of
binary (or, base-$2$) numerals.  Each time you increment the counter,
there is a {\it carry}.  These carries have varying lengths; for
instance, when $n = 32 = 100000_2$, the carry-lengths range
from $0$---whenever you increment an even integer---to $5$---when you
increment $31 = 11111_2$ to achieve $32 = 100000_2$. \\
{\em Prove that the average carry as you go from $1$ to $n$ has length $2$.}

\medskip

\noindent {\it The solution.}

\noindent
Half of the increments add $1$ to an even number, i.e., a number whose
binary numeral ends in ``$ \ldots 0$''.  These increments generate no
carry---or, equivalently, a carry of length $0$.

\noindent
One-quarter of the increments, which form half of the remaining
increments, execute a carry of length $1$, because they add $1$ to a
numeral that ends in ``$ \ldots 01$''.

\noindent
One-eighth of the increments, which form half of the remaining
increments, execute a carry of length $2$, because they add $1$ to a
numeral that ends in ``$ \ldots 011$''.

Continuing in this way, one can show that the average length of a
carry can be expressed in the form
\[ 
\frac{1}{2} \cdot 0 \ + \ \frac{1}{4} \cdot 1 \ + \ \frac{1}{8} \cdot
3 \ + \ \frac{1}{16} \cdot 4 \ + \ \cdots
\]
Using techniques that we cover in Chapter~\ref{ch:Summation}, one
verifies that this infinite series converges with the sum $2$.  \qed




\section{Summations}


\subsection{Compute $\Delta_n$ by means of sum of squares}

$\Delta_n = \sum_{i=1}^{n} i = \frac{n(n+1)}{2}$

The idea here is to write the sum by extracting the first and the last element
of the sum of squares.
%We loose since the coefficient of the $\Delta_n$ is the same after these manipulations, but we can manage if we compute the \textit{next} sum, that is sum of the squares.
\medskip

$S_{n+1} = 1 + \sum_{i=1}^{n+1} i^2$

$S_{n+1} = (\sum_{i=1}^{n} i^2) + (n+1)^2$

where $\sum_{i=1}^{n+1} i^2 
= \sum_{i=0}^{n} (i-1)^2 
= \sum_{i=0}^{n} (i^2-2i+1) 
= \sum_{i=0}^{n} i^2- 2 ( \sum_{i=0}^{n} i) + (n+1)
= \sum_{i=1}^{n} i^2- 2 ( \sum_{i=1}^{n} i) + (n+1)$

Thus, 
$\sum_{i=1}^{n} i^2- 2 ( \sum_{i=1}^{n} i) + (n+1) = (\sum_{i=1}^{n} i^2) + (n+1)^2$

$-2 \Delta_n + (n+1) =  (n+1)^2$

$\Delta_n =  (n+1)^2-(n+1) = n(n+1)$


\subsection{Tetrahedral numbers}

\noindent {\it The problem.}
The sum of the triangular numbers $\Delta_n$ is denoted by $\Theta_n$ and it is called a tetrahedral number:

Our objective here is to prove:

$\Theta_n =  \sum_{k=1}^{n} \Delta_k = \frac{n.(n+1).(n+2)}{6}$

\medskip

\noindent {\it The solution.}
We proved the expression of $\Delta_n$ by mirroring the developed expression and adding term by term.
Similarly, a way to prove the expression of $\Theta_n$ is to consider three copies and organize them 
in order to obtain the expected result.
A tetrahedral number can be arranged as a triangle (see Figure~\ref{fig:TetrahedralBasic}).
\begin{figure}[h]
\begin{center}
        \includegraphics[scale=0.5]{FiguresArithmetic/TetrahedralBasic}
        \caption{Computing $\Theta_n$: basic triangle pattern.}
        \label{fig:TetrahedralBasic}
\end{center}
\end{figure}

The proof is obtained by Fubini's principle by rotating this triangle as shows in Figure~\ref{fig:Tetrahedral}.
\begin{figure}[h]
\begin{center}
        \includegraphics[scale=0.5]{FiguresArithmetic/Tetrahedral}
        \caption{Computing $\Theta_n$ using an adequate arrangement of $3$ triangles.}
        \label{fig:Tetrahedral}
\end{center}
\end{figure}

Sum up all the numbers in each row.

\begin{itemize}
\item 
The first row is equal to $1+1+n = n+2$.
\item
The second one is equal to $3 + 3 + 2(n-1) = 2(n+2)$. 
\item
Let us sum up the elements in row $k$: 

$\Delta_k + \Delta_k + k(n-k+1)  = k(k+1) + kn-k^2+k = k(n+2)$.
\end{itemize}

Conclusion:
the global sum is equal to $n+2$ times $(1+2+...+n)$.

Finally, $3 \Theta_n = (n+2) \Delta_n$.
\bigskip

\noindent {\it Going further}

Summary: we proved the following results:
\begin{itemize}
\item $Id_n = 1+1+ ... +1 = n$
\item $\Delta_n = 1+2+3+ ... +n = \frac{1}{2}.Id_n.(n+1)$
\item $\Theta_n = \Delta_1 + \Delta_2 + ... + \Delta_n = \frac{1}{3} .\Delta_n.(n+2)$
\end{itemize}

A natural question is if we can go further following the same pattern for computing 
$ \sum_{k=1}^{n} \Theta_k$, and so on.

Are you able to consider the challenge?


\subsection{Sum of perfect cubes}

This exercice presents an alternative proof of the result of Section~\ref{sec:sumOfOdds},
that establishes that the sum of $n$ first cubes is equal to a perfect square, and more precisely, $\Delta_n^2$.

The proof is based on an hold and simple pattern that we learned in elementary school.
\medskip

\index{$n^2$ as sum of first $n$ odd integers!a proof from elementary school}
\begin{proof}
%
Consider the following reasoning which emerges from the way
multiplication tables are developed in elementary school.  
Let us first illustrate the idea using the case $n=5$.
\begin{equation}
\label{eq:Fubini-table}
\begin{array}{rrrrr}
1  &  2 &  3 &  4 &  5 \\
2  &  4 &  6 &  8 & 10 \\
3  &  6 &  9 & 12 & 15 \\
4  &  8 & 12 & 16 & 20 \\
5  & 10 & 15 & 20 & 25 \\
\end{array}
\end{equation}
Write the integers $1, 2, \ldots, n$ in a row.  Below this row, write
the doubles of these integers.  Below the ``double'' row, write the
triples of the integers.  Below the ``triple'' row, write the
quadruples of the integers, then the quintuples, and so on.  Note that
the resulting table is {\em symmetric:} its rows are identical to its
columns.
\medskip

Using again Fubini's rearrangement stratagem, we now count all the integers in
the table in two different ways.
\begin{enumerate}
\item
We sum the entries of our table by peeling away successively larger
reversed instances of the letter ``$L$'' (as in our earlier
``pictorial'' proof of
Proposition~\ref{thm:squares-odd-integers-Gauss}).  We find that the
integers in each ``$L$'' sum to a perfect cube.
Actually, the diagonal is (by definition) equals to the square.
\[
\begin{array}{rrrrrrrrr|rrc}
1  &    &    &    &    &   &     &    &   & 1   & = 1^3 \\
2  &  4 &  2 &    &    &   &     &    &   & 8   & = 2^3 \\
3  &  6 &  9 &  6 &  3 &   &     &    &   & 27  & = 3^3 \\
4  &  8 & 12 & 16 & 12 &  8 &  4 &    &   & 64  & = 4^3 \\
5  & 10 & 15 & 20 & 25 & 20 & 15 & 10 & 5 & 125 & = 5^3
\end{array}
\]

\item
We sum the successive rows of the $n \times n$ table (\ref{eq:Fubini-table}).  
The first row of the table sums to $\Delta_n$; the second row sums to $2
\Delta_n$; the third row sums to $3 \Delta_n$; \ldots; the last row sums
to $n \Delta_n$.  
Thus, the aggregate sum of the table's rows is 
\[ (1 + 2 + \cdots + n) \cdot \Delta_n \ = \ \left(\Delta_n \right)^2 \]
\end{enumerate}
We conclude that
\[
\sum_{i=1}^n i^3 \ = \  \left(\Delta_n \right)^2
\]
\end{proof}



\section{More exercises}

\subsection{A first graphical proof}

\noindent {\it The problem.}
Compute the sum of $(\frac{1}{4})^k = \frac{1}{3} $ using a graphical argument.
\medskip

\noindent {\it The solution.}
The solution is simply depicted in Fig.~\ref{Fig:SUmgeo1sur4}. 
\begin{figure}
\begin{center}
        \includegraphics[scale=0.3]{FiguresArithmetic/SumGeometric1sur4}
        \caption{Graphical construction. Assuming the total area is 1, the area of the grey internal triangle (left) is $\frac{1}{4}$.
        As the grey area is one third at each layer (right), the whole area is $\frac{1}{3}$.
        By Fubini's principle, this area is the sum of the $\frac{1}{4^k}$ (for $k \geq 1$).}
        \label{Fig:SUmgeo1sur4}
\end{center}
\end{figure}


\subsection{Another graphical proof}

A nonobvious identity for arithmetic sums.

We close this subsection by using a ``picture'' to verify an identity for
arithmetic sums that one would be unlikely to come upon by purely
textual thinking.

\begin{prop} 
\label{thm:an-arithmetic-identity}
For any positive integer $n$,
\[ \Delta_{2n-1} \ = \ n + 4 \Delta_{n-1}. \]
\end{prop} 

\begin{proof} 
Consider the arithmetic series in (\ref{eq:arith-seq}) for the case
$a=1$ and $b=4$.  By Proposition~\ref{thm:sum-of-arithmetic-series},
this series, call it $S^{(1,4)}(n)$, has the sum
\begin{equation}
\label{eq:triangles}
S^{(1,4)}(n) \ = \ n + 4 \Delta_{n-1}.
\end{equation}

Let us represent the sum $\Delta_{n-1}$ in the natural way as a
triangle of tokens.  This triangle has a base of $n-1$ tokens, upon
which sits a row of $n-2$ tokens, upon which sits a row of $n-3$
tokens, \ldots, all the way to the apex, which has a single token.

Now, let us view equation (\ref{eq:triangles}) as giving us access to four
copies of the preceding triangle of tokens.  Let us arrange these
triangles in the manner depicted in Fig.~\ref{fig:Delta(n)4}.
\begin{figure}[ht]
\begin{center}
       \includegraphics[scale=0.5]{FiguresMaths/Delta4}
 \caption{Arranging the four triangles plus a row to obtain a new (bigger) triangle.}
       \label{fig:Delta(n)4}
\end{center}
\end{figure}
Now, ``complete the picture'' by adding an ``extra'' row of $n$
tokens at row $n$ of the figure (these are depicted in dark gray in
the figure).  The four small triangles, augmented by the ``extra'' row
of $n$ tokens has clearly become a representation  of $\Delta_{2n-1}$
by tokens.
\ignore{******
The sum is $S(n)=n+4 \Delta_{n-1}$.  From the arrangement given in
Fig.~\ref{fig:Delta(n)4}, it is easy to see that it is equal to
$\Delta_{2n-1}$ (4 triangles of size $n$ plus a row of $n$ tokens in
dark grey form a bigger triangle of size $2n-1$).
*******}

We now have a purely pictorial proof of the proposition. \qed
\end{proof} 


\subsection{A variant about $\Delta_n$}

Consider again the sum of the first integers $\Delta_n$. 

Show graphically that $\Delta_n$ is congruent to $1$ modulo $8$.

In other words: $8 \Delta_n = K^2 -1$
\begin{figure}[ht]
\begin{center}
       \includegraphics[scale=0.4]{FiguresMaths/Delta8}
\caption{8 copies of $\Delta_n$ are filling a big square but one token.}
       \label{fig:Sum8deltas}
\end{center}
\end{figure}



\subsection{Another proof for irrationality of $\sqrt{2}$}

Put the explicit question here.

The solution is depicted in Fig.~\ref{Fig:sqrtbisInit} and~\ref{Fig:sqrtbisFin} . 
\begin{figure}
\begin{center}
        \includegraphics[scale=0.3]{FiguresArithmetic/sqrtbisInit}
        \caption{First step: folding the big triangle along the side.}
        \label{Fig:sqrtbisInit}
\end{center}
\end{figure}
\begin{figure}
\begin{center}
        \includegraphics[scale=0.3]{FiguresArithmetic/sqrtbisFin}
        \caption{Second step. The sides of the small isocel triangle are integers.}
        \label{Fig:sqrtbisFin}
\end{center}
\end{figure}

\subsection{Harmonic series}

$H_{n} = \sum_{k=1}^{n} \frac{1}{k}$.

Another way to prove that the sum is infinite:
 
The analysis is as follows.
Group the terms according to powers of $2$. 
The sum within each group is between $\frac{1}{2}$ and $1$, thus,
$H_n > \frac{1}{2}.n$
\\

Another (more precise) way is to gather the terms 3 by 3 as follows:

$S_k = (\frac{1}{3k-1} + \frac{1}{3k} + \frac{1}{3k+1} )$ for $k\geq1$, 

$H = 1 + S_1 + ... + S_k + ... > 1 + 3.\frac{1}{3} + 3.\frac{1}{6} + ... + 3.\frac{1}{3k} + ... $

since $S_k > 3.\frac{1}{k} $.

The proof is by contradiction, if $H$ is finite, from the previous relation we have: $H > 1 + H$, which is obviously impossible.
\bigskip

Moreover, the first way of  bounding the sum tells us about its value (actually, we know the value at a factor of $2$):

$\frac{log(n)+1}{2} < H_n < log(n)+1$. Thus, $H_n = O(log(n))$

{\Denis Change the writing of O}

\subsection{On meeting new people}

Illustration of pigeon's hole...
\medskip

You are attending a cocktail party that is populated by $n$ couples.
In order to create a warm atmosphere, the host requests that each
attendee shake the hand of every attendee that he or she does not
know.  \\
{\em Prove that some two attendees shake the same number of hands.}

\medskip

\noindent {\it Solution.}
%
This observation follows from the {\it pigeonhole principle}, which
states the following.

{\it If $n+1$ pigeons occupy $n$ pigeonholes, then some hole contains
  $2$ pigeons.}

\noindent
This principle guarantees that some two attendees shake the same
number of hands.  To wit, the number of people that each attendee {\em
  does not know} belongs to the set $\{ 0, 1, \ldots, 2n-2 \}$,
because each person knows him/herself and his/her partner.  Because
there are $2n$ handshakers (the pigeons) and $2n-1$ numbers of hands
to shake (the boxes), some two shakers must shake the same numbers of
hands.  \qed



\section{Arithmetic}

\subsection{A fun result: complex
  multiplication via $3$ real multiplications}
\index{complex number!multiplication via 3 real multiplications}

\begin{prop}
%\label{thm:complex-mult-3real}
One can compute the product of two complex numbers using {\em three}
real multiplications rather than four.
\end{prop}

\begin{proof}
Although implementing (\ref{eq:complex-mult}) ``directly'' correctly
produces the product $\kappa = (a+bi) \cdot (c+di)$, there is another
implementation that is {\em more efficient}.  Specifically, the
following recipe computes $\kappa$ using only {\em three} real
multiplications instead of the four real multiplications of the
``direct'' implementation.  We begin to search for this recipe by
noting that our immediate goal is to compute both Re$(\kappa) = ac-bd$
and Im$(\kappa) = ad+bc$.  We can accomplish this by computing the
{\em three} real products
\begin{equation}
\label{eq:complex-mult-3a}
(a+b) \cdot (c+d); \ \ \ \ \
ac;  \ \ \ \ \ bd
\end{equation}
and then noting that
\begin{equation}
\label{eq:complex-mult-3b}
\begin{array}{lcl}
\mbox{Im}(\kappa) & = & (a+b) \cdot (c+d) - ac -bd, \\
\mbox{Re}(\kappa) & = & ac -bd
\end{array}
\end{equation}
We thereby achieve the result of the complex multiplication described
in (\ref{eq:complex-mult}) while using only {\em three} real
multiplications.

Of course, a full reckoning of the costs of the two implementations we
have discussed exposes the fact that the implementation that invokes
(\ref{eq:complex-mult-3a}) and (\ref{eq:complex-mult-3b}) uses {\em
  three} real additions rather than the {\em two} real additions of
the ``direct'' implementation.  But this entire exercise was
predicated on the observation that each real addition is much less
costly than a real multiplication, so trading one multiplication for
one addition is an unqualified ``win''.  \qed
\end{proof}

%%%%%%%%%%%%%%%%%%%%%%%%%%%%%%%%%%

\subsection{Karatsuba (Extension of the same method)} 


Notice that the previous technique is classical and it has been used in many other situations.
For instance while multiplying two integers in base 2.

Let us first recall the generic divide-and-conquer for designing algorithms.
\bigskip

\noindent \fbox{
\begin{minipage}{0.95\textwidth}
{\it Divide and conquer} is a paradigm for designing efficient algorithms for solving problems that can be
decomposed into sub-problems.

Let consider a problem of size $n$ that can be decomposed into $p$ sub-problems.
(notice, all sub-problems have the same size n/q).
\bigskip

\noindent {\bf Principle:}
\begin{enumerate}
\item Decompose the problem into $a$ sub-problems of size $\frac{n}{q}$.
\item Solve those sub-problems.
\item Reconstruct the solution of the initial problem.
\end{enumerate}

In general, the sub-problems are solved recursively (at least until a certain threshold).
The cost is governed by the following expression:

$T(n) = p.T(\frac{n}{q}) + c_1(n) + c_3(n)$ 

where $c_1(n)$ et $c_3(n)$ are the costs of phases (1) et (3).

Note: the resolution of this kind of equation is using the Master theorem of section~\ref{sec:linear-recurrence-general}.
\end{minipage}
}
\bigskip

Let $A=(a_na_{n-1}\ldots a_1)_2$ and $B=(b_nb_{n-1}\ldots b_1)_2$
be two long integers, and assume $n=2^k$ for some positive integer $k$.
The goal is to compute the product $A \times B$.

The standard (and naive) method is to compute the $n$ partial products 
$A=(a_na_{n-1}\ldots a_1)_2$ by each of the $b_i$, leading to $O(n^2)$ basic operations. 

The divide-and-conquer version of this problem is to break each integer into two parts
of $\frac{n}{2}$ bits each:

$A=(a_n\ldots a_{n/2+1})_2 2^{n/2}\ + (a_{n/2}\ldots a_1)_2$.

Denoting by $A_1$ and $A_2$ the two separate terms, we get:

$A.B = (A_1.B_1) 2^n + (A_1.B_2 + A_2.B_1) 2^{n/2} + A_2.B_2$
\bigskip

\textbf{Cost analysis} :
The previous computation requires 4 multiplications of integers of $n/2$ bits 
and 3 additions of integers with at most $2n$ bits. The multiplication of integers in base 2 by powers of 2 
correspond to simple shifts to the left. 

The cost is given by the following expression: $T(n) = 4.T(\frac{n}{2}) + f(n)$ where $f$ is a linear function
and $T(1) = 1$.

We obtain $T(n) = \Theta(n^2)$, same as for the naive algorithm.
\bigskip
 

The idea of Karatsuba is to decrease the number of multiplications 
(at the price of slightly increasing the additions/substractions) by using the following identity:

$A_1.B_1 + A_2.B_2 + (A_1-A_2).(B_2-B_1)$
\bigskip

We obtain:

$A.B = (A_1.B_1) 2^n + (A_1.B_1 + A_2.B_2 + (A_1-A_2).(B_2-B_1)) 2^{n/2} + A_2.B_2$


Cost analysis: 3 multiplications of $n/2$ bits

4 additions and 2 substractions of integers at most $2n$ bits. Again, using the Master Theorem leads to:

$T(n) = 3.T(\frac{n}{2}) + \theta (n) = n^{log_2 3}$


\subsection{A property about binomial coefficients}

Prove the following property by two ways (recurrence and by using a cominatorial argument
\begin{prop}
$\forall n,k$ $1 \leq k \leq n$

$k.{n \choose k} = n.{{n-1} \choose {k-1}}$
\end{prop}

\begin{proof}
to complete
\end{proof}



\subsection{Another fun result dealing with divisibility}

This exercice was a favorite question by Paul Erdos,
he often use this question to test the young students, apprentis mathematicians...
\bigskip

Let consider the $2n$ first integers.

Take any $n+1$ integers in this set and prove that there exists a pair $(p,q)$
such that $p$ divides $q$. 

The sketch of the proof is as follows.

\begin{enumerate}
\item
Let $\alpha_i$ be the elements of this set of cardinality $n+1$.

Write $\alpha_i = 2^k \times m$ where $m$ is odd (and $k \geq 0$).

Then, $m$ belongs to $\{1,3,5, \ldots, 2n-1 \}$
\item
From the pigeon hole principle, there are two numbers with the same value of $m$. 
\item 
Thus, $2^{k1} \times m$ and $2^{k2} \times m$.

$p$ is the smallest one, which divides $q$ (the largest one).
\end{enumerate}


\subsection{A Fun Result: A ``Trick'' for Squaring Certain Integers}

Sometimes only basic knowledge is needed to craft amusing
``tricks''---we know that they are not really tricks at all!---that
are really rigorous applications of principles that we have learned.
Here is an ``old chestnut'' example that may inspire you to design
your own. 

If someone presents you with a number that has a numeral that ends in
$5$, then there is a simple way to square the number mentally.  For
instance, if someone says

\hspace{.25in}``$n = 25$''

\noindent
then you can instantly respond

\hspace{.25in}``$n^2 = 625$''

\noindent
If the challenge is

\hspace{.25in}``$n = 75$''

\noindent
then your response is

\hspace{.25in}``$n^2 = 5625$''

\noindent
Let's make this ``game'' mathematical.

\begin{prop}
\label{thm:75x65=4925}
Let $n$ be any number that has a $2$-digit decimal numeral of the form

\hspace{.25in}$\delta \ 5$ \ \ \ \ $(\delta \in \{ 0,1,2,3,4,5,6,7,8,9\})$.

\noindent
Then the square of $n$ is the integer

\hspace{.25in}$25 \ + \ \delta \cdot (\delta +1)$. 
\end{prop}

\begin{proof}
We can rewrite the premise of the proposition in the form
\[ n \ = \ 10 \cdot \delta + 5 \]
It is now easy to invoke Proposition~\ref{prop:(a+b)(c+d)} and the
distributive law to compute that

\[ n^2 \ = \ 100 \cdot \delta \cdot (\delta+1) + 25. \]
To wit: 
\[
\begin{array}{lclll}
n^2 & = & (10 \cdot \delta + 5)^2 & & \mbox{Given} \\
    & = & 100 \cdot \delta^2 \ + \ 100 \cdot \delta \ + \ 25
              & & \mbox{the proposition} \\
    & = & 100 \cdot (\delta^2 \ + \ \delta) \ + \ 25
              & & \mbox{factoring: distributive law} \\
    & = & 100 \cdot \delta \cdot (\delta + 1) \ + \ 25
              & & \mbox{factoring: distributive law} \\
\end{array}
\]
A parlor trick has become a mathematical demonstration!
\qed
\end{proof}


\subsection{A fun result via geometric sums: When is integer  $n$
  divisible by $9$?}
\label{sec:divisible-by-9}

We now exploit our ability to evaluate geometric summations to
illustrate a somewhat surprising, nontrivial fact.  One can deduce
information about the divisibility of an integer $n$ from $n$'s
positional numerals.  We hope that this ``fun'' result will inspire
the reader to seek kindred numeral-encoded properties of numbers.

\begin{prop}
\label{thm:div-by-b-bar}
An integer $n$ is divisible by an integer $m$ if, and only if, $m$
divides the sum of the digits in the base-$(m+1)$ numeral for $n$.
\end{prop}

The most familiar instance of this result is phrased in terms of our
traditional use of base-$10$ (decimal) numerals. \\
{\it An integer $n$ is divisible by $9$ if, and only if, the sum of
  the digits of $n$'s base-$10$ numeral is divisible by $9$.}

\smallskip

\begin{proof}
({\it Argument for general number-base $b$}).
%
Of course, we lose no generality by focusing on numerals without
leading $0$'s, because leading $0$'s do not alter a numeral's sum of
digits.

Let us focus on the base-$b$ numeral for a number $n$ (so $b = m+1$ in
the statement of the proposition).  There therefore exist base-$b$
digits---i.e., integers from the set $\{0, 1, \ldots, b-1\}$---call
them $\delta_k \neq 0$, $\delta_{k-1}$, \ldots $\delta_1$, $\delta_0$,
such that
\[ n \ = \ \delta_k \cdot b^k + \delta_{k-1} \cdot b_{k-1} + \cdots +
\delta_1 \cdot b + \delta_0. \]
The sum of the digits of $n$'s base-$b$ numeral is, then
\[ s_b(n) \ \eqdef \ \delta_k + \delta_{k-1} + \cdots + \delta_1 +
\delta_0. \]
Let us calculate the difference $n - s_b(n)$ in the following manner,
digit by digit.
\begin{equation}
\label{eq:sum-of-digits}
\begin{array}{ccccccccccc}
n & = &
\delta_k \cdot b^k & + & \delta_{k-1} \cdot b^{k-1} & + & \cdots
  & + & \delta_1 \cdot b & + & \delta_0 \\
s_b(n) & = &
\delta_k & + & \delta_{k-1} & + & \cdots & + & \delta_1 & + & \delta_0 \\
\hline
n - s_b(n) & = &
\delta_k \cdot (b^k -1) & + &
\delta_{k-1} \cdot (b^{k-1} -1) & + &
\cdots & + &
\delta_1 \cdot (b-1) & & 
\end{array}
\end{equation}

\medskip

We now revisit summation (\ref{eq:geom-sum:b>1}).  Because $b$ is a
positive integer, so that $1 + b + \cdots + b^{a-2} + b^{a-1}$ is also
a positive integer, we infer that {\em the integer $b^a -1$ is
  divisible by $b-1$.}

We are almost home.  Look at the equation for $n - s_b(n)$ in the
system (\ref{eq:sum-of-digits}).  As we have just seen, every term on
the righthand side of that equation is divisible by $b-1$.  It follows
therefore, that the lefthand expression, $n - s_b(n)$, is also
divisible by $b-1$.
An easy calculation, which we leave to the reader, now shows that this
final fact means that $n$ is divisible by $b-1$ if, and only if,
$s_b(n)$ is.
\end{proof}


%%%%%%%%%%%%%%%%%%%%%%%%%%%%%%%%%%%%%

\section{Infinite}


\subsection{What's wrong?}

In this exercice, we investigate a proof which leads to a surprising result...

\begin{enumerate}
\item
Let consider the infinite sum $A = 1-1+1-1+ \ldots$

and show that $A=\frac{1}{2}$ (hint: compute $1-A$)
\item
Let now consider the other infinite sum $B=2-3+4-5+6 \ldots$

and show that $B=\frac{1}{4}$ (hint: compute $A+B-1$)
\item 
Compute the sum of the integers $C=1+2+3+4+ \ldots$

and show that $C=-\frac{1}{12}$ (hint: compute $C-B=4+8+12+16+ \ldots$)
\end{enumerate}

What's wrong?
First, summing up positive number should be positive, 
and second, the sum of the integers should be infinite...

Then, how to analyze the previous result/proof?


%%%%%%%%%%%%%%%%%%%%%%%%%%%%%%%%%%%%%

\section{Recurrences}



%%%%%%%%%%%%%%%%%%%%%%%%%%%%%%%%%%%

\section{Graphs}

\subsection{Graph Isomorphism}
\label{Exercice:isomorphism}

Recall here the problem:
Formal proof of Proposition~\ref{prop.graphIsomorphism}. 

\begin{prop}
The order-$4$ hypercube $\q_4$ is \textit{isomorphic} to the $4 \times
4$ torus $\widetilde{\m}_{4,4}$.
\end{prop}

\medskip

The solution:

To do, describe the both codings. 



\subsection{Spanning Trees}
\label{Exercice:spanningTrees}

Recall here the problem
\medskip

There are mainly two ways for constructing such a MST, each one
emphasizes a different propriety of the MST, namely, avoid cycles and
minimize the span.  In both cases, the edges are sorted in increasing
order of weights.  More precisely, the first one constructs a subtree
which partially spans the graph by adding at each step the minimum
neighboring edge while the other add successively the edges of minimal
weights that do not create a cycle.


\subsection{Formal definition of mesh graphs}
\label{Exercice:FormalDefinitionMesh}

The definition of mesh is very easy to capture with a drawing.
{\Denis Refer to the figure in GRAPHS1 here or put again the figure in the section?}

For positive integers $m, n \in \N^+$, both the $m \times n$ {\it mesh
  (network)} $\m_{m,n}$ and the $m \times n$ {\it toroidal network}
(or, {\it torus}) $\widetilde{\m}_{m,n}$ have {\it vertex-set}
\begin{eqnarray*}
\n_{\fm_{m,n}} \ = \ \n_{\widetilde{\fm}_{m,n}}
  & = & 
\{1, \ 2, \ldots, \ m\} \ \times \ \{1, \ 2, \ldots, \ n\} \\
  & = & 
\big\{ \langle i, \ j \rangle \ \ | \ \ 
\big[ i \in \{1, \ 2, \ldots, \ m\} \big], \ \
\big[ j \in \{1, \ 2, \ldots, \ n\} \big]
\big\}
\end{eqnarray*}


$\m_{m,n}$ has $(m-1)n \ + \ (n-1)m$ edges; its {\it edge-set} is
\begin{eqnarray*}
\e_{\fm_{m,n}} & = & 
\big\{
\{ \{ i, j \}, \ \{ i+1, j \} \ \ | \ \
1 \leq i < m, \ \ 1 \leq j \leq n \} \\
  &  & \hspace*{.1in} \cup
\{ \{ i, j \}, \ \{ i, j+1 \} \ \ | \ \
1 \leq i \leq m, \ \ 1 \leq j < n \}
\big\}
\end{eqnarray*}

\bigskip

The subgraph of $\m_{m,n}$ defined by the vertex-set
\[ \{ \langle i, \ j \rangle  \ \ | \ \ \left[i \in \{1, 2, \ldots,
  m\}\right], \ \ \left[1 \leq j < n\right]\}
\]
and all edges both of whose endpoints belong to that set is called the
$i$th {\it row} of $\m_{m,n}$
Dually, the subgraph of $\m_{m,n}$ defined by the vertex-set
\[ \{ \langle i, \ j \rangle  \ \ | \ \ \left[j \in \{1, 2, \ldots,
  n\}\right], \ \ \left[1 \leq i < m\right] \}
\]
and all edges both of whose endpoints belong to that set is called the
$j$th {\it column} of $\m_{m,n}$.


\begin{itemize}
     \item
Vertices $\langle 1, \ 1 \rangle$, $\langle 1, \ n \rangle$, $\langle m,
\ 1 \rangle$, and $\langle m, \ n \rangle$ are the {\it corner vertices}
(or, just {\it corners}) of $\m_{m,n}$.
     \item
The path-graph consisting of the vertex-set
\[ \{ \langle 1, \ 1 \rangle, \ \langle 1, \ 2 \rangle, \ldots, \
\langle 1, \ n \rangle \}
\]
together with all edges of $\m_{m,n}$ both of whose endpoints belong
to this set, is the {\it top edge} of $\m_{m,n}$.

The other edges of $\m_n$ are defined analogously:

\medskip

The {\it bottom edge} of $\m_{m,n}$ is the path-graph built upon the
vertex-set
\[ \{ \langle m, \ 1 \rangle, \ \langle m, \ 2 \rangle, \ldots, \
\langle m, \ n \rangle \}
\]

The {\it left edge} of $\m_{m,n}$ is the path-graph built upon the
vertex-set
\[ \{ \langle 1, \ 1 \rangle, \ \langle 2, \ 1 \rangle, \ldots, \
\langle m, \ 1 \rangle \}
\]

The {\it right edge} of $\m_{m,n}$ is the path-graph built upon the
vertex-set
\[ \{ \langle 1, \ n \rangle, \ \langle 2, \ n \rangle, \ldots, \
\langle m, \ n \rangle \}
\]

\end{itemize}

Same for the torus graphs:

The subgraph of $\widetilde{\m}_{m,n}$ defined by the vertex-set
\[ \{ \langle i, \ j \rangle  \ \ | \ \ \left[i \in \{1, 2, \ldots,
  m\}\right], \ \ \left[1 \leq j \leq n\right]\}
\]
and all edges both of whose endpoints belong to that set is called the
$i$th {\it row} of $\widetilde{\m}_{m,n}$
Dually, the subgraph of$\widetilde{\m}_{m,n}$ defined by the vertex-set
\[ \{ \langle i, \ j \rangle  \ \ | \ \ \left[j \in \{1, 2, \ldots,
  n\}\right], \ \ \left[1 \leq i \leq m\right] \}
\]
and all edges both of whose endpoints belong to that set is called the
$j$th {\it column} of $\widetilde{\m}_{m,n}$.


