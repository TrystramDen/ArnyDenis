%version of 08-29-18

\begin{thebibliography}{999}

%A


%B

\bibitem{Bell86}
E.T.~Bell (1986):
{\it Men of Mathematics}.
Simon and Schuster, New York.

\bibitem{Bernstein05}
F.~Bernstein (1905): Untersuchungen aus der Mengenlehre. {\it
Math.~Ann.~61}, 117--155.

\bibitem{Birkhoff-MacLane53}
G.~Birkhoff and S.~Mac Lane (1953): {\it A Survey of Modern Algebra},
Macmillan, New York.

\bibitem{Bishop67}
E.~Bishop (1967): {\it Foundations of Constructive Analysis},
McGraw Hill, New York.

\bibitem{BooneCL73}
W.W.~Boone, F.B.~Cannonito, R.C.~Lyndon (1973):
{\it Word Problems: Decision Problem in Group Theory}, North-Holland,
Amsterdam.

\bibitem{BuyyaAG01}
R.~Buyya, D.~Abramson, J.~Giddy (2001): A case for economy Grid
architecture for service oriented Grid computing.  {\it 10th
Heterogeneous Computing Wkshp.}

%C

\bibitem{Cantor74}
G.~Cantor (1874): \"{U}ber eine Eigenschaft des Inbegriffes aller
reellen algebraischen Zahlen.  {\it J.~Reine und Angew.~Math.~77},
258--262.

\bibitem{Cantor78}
G.~Cantor (1878): Ein Beitrag zur Begr\"{u}ndung der transfiniter
Mengenlehre.  {\it J.~Reine Angew.~Math.~84}, 242--258.

\bibitem{Cantor87}
G.~Cantor (1887): Mitteilungen zur Lehre vom Transfiniten.
{\it Zeitschrift f\"{u} Philosophie und Philosophische Kritik 91}
81-–125.

\bibitem{Cauchy21}
A.L.~Cauchy (1821): {\it Cours d'analyse de l'\'{E}cole Royale
Polytechnique, 1\`{e}re partie: Analyse alg\'{e}brique}.
l'Imprimerie Royale, Paris.  Reprinted: Wissenschaftliche
Buchgesellschaft, Darmstadt, 1968.

\bibitem{Chomsky56}
N.~Chomsky (1956): Three models for the description of language.
{\it IRE Trans.~Information Theory 2}, 113--124.

\bibitem{Chomsky59}
N.~Chomsky (1959): On certain formal properties of grammars.  {\it
Inform.~Contr.~2}, 137--167.

\bibitem{Church41}
A.~Church (1941):
{\it The Calculi of Lambda-Conversion.}
{\it Annals of Math.~Studies 6}, Princeton Univ.~Press, Princeton, NJ.

\bibitem{Church44}
A.~Church (1944):
{\it Introduction to Mathematical Logic, Part I}.
{\it Annals of Math.~Studies 13}, Princeton Univ.~Press, Princeton, NJ.

\bibitem{CirneM99}
W.~Cirne and K.~Marzullo (1999): The Computational Co-Op: gathering
clusters into a metacomputer.  {\it 13th Intl.~Parallel Processing
Symp.}, 160--166.

\bibitem{Codd70}
E.F.~Codd (1970):  {\it A relational model of data for large shared
data banks}.  {\it Comm.~ACM 13}, 377--387.

\bibitem{Cook71}
S.A.~Cook (1971): The complexity of theorem-proving procedures.  {\it
ACM Symp.~on Theory of Computing}, 151--158.

\bibitem{CopiEW58}
I.M.~Copi, C.C.~Elgot, J.B.~Wright (1958): Realization of events by
logical nets.  {\it J.~ACM 5}, 181--196.

\bibitem{CLRS}
T.H.~Cormen, C.E.~Leiserson, R.L.~Rivest, C.~Stein (2001):
{\it Introduction to Algorithms (2nd ed.)}.
MIT Press, Cambridge, MA.

\bibitem{Curry34}
H.B.~Curry (1934): Some properties of equality and implication in
combinatory logic.  {\it Annals of Mathematics, 35}, 849--850.

\bibitem{CurryFC58}
H.B.~Curry, R.~Feys, W.~Craig (1958):
{\it Combinatory Logic.  Studies in logic and the foundations of
mathematics}.  North-Holland, Amsterdam.

%D

\bibitem{Davis58}
M.~Davis (1958):
{\it Computability and Unsolvability.}
McGraw-Hill, New York.

\bibitem{Deiser2010}
O.~Deiser (2010): Einf\"{u}hrung in die Mengenlehre – Die Mengenlehre
Georg Cantors und ihre Axiomatisierung durch Ernst Zermelo (3rd,
corrected ed.), Berlin/Heidelberg: Springer, pp. 71, 501,
doi:10.1007/978-3-642-01445-1, ISBN 978-3-642-01444-4.


%F


\bibitem{Fubini}
G.~Fubini (1907): Sugli integrali multipli.
{\it Rom.~Acc.~L.~Rend.~(5), 16(1)}, pp.~608–614.  In {\it zbMATH
  38.0343.02}.  Reprinted in
G.~Fubini (1958): {\it Opere scelte, 2}, Cremonese, pp. 243–249.

\bibitem{FueterP23}
R.~Fueter and G.~P\'{o}lya (1923):
Rationale Abz\"{a}hlung der Gitterpunkte.  {\it
Vierteljschr.~Naturforsch.~Ges.~Z\"{u}rich 58}, 380--386.

%G

\bibitem{GareyJ79}
M.R.~Garey and D.S.~Johnson (1979):
{\it Computers and Intractability}.
W.H.~Freeman and Co., San Francisco.

\bibitem{Gilat04}
A.~Gilat (2004):
{\it MATLAB: An Introduction with Applications (2nd ed.)}.
J.~Wiley \& Sons, New York.

\bibitem{Goedel31}
K.~G\"{o}del (1931): \"{U}ber Formal Unentscheidbare S\"{a}tze der
Principia Mathematica und Verwandter Systeme, I.  {\it Monatshefte
f\"{u}r Mathematik u.~Physik 38}, 173--198.

\bibitem{Goldberg96}
B.~Goldberg (1996): Functional programming languages,  {\it ACM
  Computing Surveys 28}, 249--251.

\bibitem{Goldreich06}
O.~Goldreich (2006): On teaching the basics of complexity theory.  In
{\it Theoretical Computer Science: Essays in Memory of Shimon Even.}
{\it Springer Festschrift series, Lecture Notes in Computer Science
3895}, Springer, Heidelberg.

\bibitem{GolubV96}
G.H.~Golub, C.F.~Van Loan (1996):
{\it Matrix Computations} (3rd ed.)
Johns Hopkins Press,  Baltimore.

%H

\bibitem{Halmos60}
P.R.~Halmos (1960):
{\it Naive Set Theory}.
D.~Van Nostrand, New York.

\bibitem{Harel87}
D.~Harel (1987): {\it Algorithmics: The Spirit of Computing}.
Addison-Wesley, Reading, MA.

%\bibitem{HartmanisS66}
%J.~Hartmanis and R.E.~Stearns (1966): {\it Algebraic Structure Theory
%of Sequential Machines}.  Prentice Hall, Englewood Cliffs, NJ.

\bibitem{HeathLR92}
L.S.~Heath, F.T.~Leighton A.L.~Rosenberg (1992): ``Comparing queues
and stacks as mechanisms for laying out graphs.''  {\it SIAM
  J.~Discr.~Math.~5}, 398--412.

%\bibitem{Hennie66}
%F.C.~Hennie (1966): On-line Turing machine computations.  {\it IEEE
%Trans.~Electronic Computers, EC-15}, 35--44.

%\bibitem{HennieS66}
%F.C.~Hennie and R.E.~Stearns (1966): Two-tape simulation of multitape
%Turing machines. {\it J.~ACM 13}, 533--546.

%\bibitem{HomerS01}
%S.~Homer and A.L.~Selman (2001):
%{\it Computability and Complexity Theory}.
%Springer, New York

%\bibitem{HopcroftMU01}
%J.E.~Hopcroft, R.~Motwani, J.D. Ullman (2001):
%{\it Introduction to Automata Theory, Languages, and Computation}
%(2nd ed.).
%Addison-Wesley, Reading, MA.

%\bibitem{HopcroftU79}
%J.E.~Hopcroft and J.D. Ullman (1979):
%{\it Introduction to Automata Theory, Languages, and Computation}
%(1st ed.)
%Addison-Wesley, Reading, MA.

\bibitem{Horner}
W.G.~Horner (1819): 
A new method of solving numerical equations of all orders, by
continuous approximation. {\it Philosophical Transactions. Royal
Society of London}, 308-–335.

%I

\bibitem{Immerman88}
N.~Immerman (1988): Nondeterministic space is closed under
complementation.  {\it SIAM J.~Comput.~17}, 935--938.

\bibitem{Intel01}
{\it The Intel Philanthropic Peer-to-Peer program.}  $\langle${\tt
www.intel.com/cure}$\rangle$.

\bibitem{Iverson62}
K.E.~Iverson (1962):
{\it A Programming Language}.
J.~Wiley \& Sons, New York.

%J

\bibitem{Jaffe78}
J.~Jaffe (1978): A necessary and sufficient pumping lemma for regular
languages.  {\it SIGACT News}, 48--49.

%K

\bibitem{Karp67}
R.M.~Karp (1967): Some bounds on the storage requirements of
sequential machines and Turing machines.  {\it J.~ACM 14}, 478--489.

\bibitem{Karp72}
R.M.~Karp (1972): Reducibility among combinatorial problems.  In {\it
Complexity of Computer Computations} (R.E.~Miller and J.W.~Thatcher,
eds.)  Plenum Press, NY, pp.~85--103.

\bibitem{Kleene36}
S.C.~Kleene (1936): General recursive functions of natural numbers.
{\it Math.~Annalen 112}, 727--742.

\bibitem{Kleene52}
S.C.~Kleene (1952):
{\it Introduction to Metamathematics.}
D.~Van Nostrand, Princeton, NJ.

\bibitem{Kleene56}
S.C.~Kleene (1956): Realization of events in nerve nets and finite
automata.  In {\it Automata Studies} (C.E.~Shannon and J.~McCarthy,
Eds.) {\it [Ann.~Math.~Studies 34]}, Princeton Univ.~Press, Princeton,
NJ, pp.~3--42.

\bibitem{Konig36}
D.~K\"onig (1936):
{\it Theorie der endlichen und unendlichen Graphen.}  Lipzig: Akad.~Verlag.

\bibitem{Knuth73}
D.E.~Knuth (1973): {\it The Art of Computer Programming: Fundamental
Algorithms} (2nd ed.)  Addison-Wesley, Reading, MA.

\bibitem{KorpelaWACL00}
E.~Korpela, D.~Werthimer, D.~Anderson, J.~Cobb, M.~Lebofsky (2000):
SETI@home: massively distributed computing for SETI.  In {\it
Computing in Science and Engineering} (P.F.~Dubois, Ed.)  IEEE
Computer Soc.~Press, Los Alamitos, CA.

%L

\bibitem{Landin64}
P.J~Landin (1964): The mechanical evaluation of expressions.  {\it
  Computer J.~6}, 308--320.

\bibitem{Levin73}
L.~Levin (1973): Universal search problems.  {\it Problemy Peredachi
Informatsii 9}, 265--266.  Translated in, B.A.~Trakhtenbrot (1984): A
survey of Russian approaches to perebor (brute-force search)
algorithms.  {\it Annals of the History of Computing 6}, 384--400.

\bibitem{LewR78a}
J.S.~Lew and A.L.~Rosenberg (1978): Polynomial indexing of integer
lattices, I.  {\it J.~Number Th.~10}, 192--214.
 
\bibitem{LewR78b}
J.S.~Lew and A.L.~Rosenberg (1978): Polynomial indexing of integer
lattices, II.  {\it J.~Number Th.~10}, 215--243.

\bibitem{LewisP81}
H.R.~Lewis and C.H.~Papadimitriou (1981):
{\it Elements of the Theory of Computation}.
Prentice-Hall, Englewood Cliffs, NJ.

\bibitem{Linz01}
P.~Linz (2001): {\it An Introduction to Formal Languages and Automata}
(3rd ed.)  Jones and Bartlett Publ., Sudbury, MA.

%M

\bibitem{Markov49}
A.A.~Markov (1949): On the representation of recursive functions (in
Russian).  {\it Izvestiya Akademii Nauk S.S.S.R. 13}.  English
translation: Translation 54, Amer.~Math.~Soc., 1951.

\bibitem{McCullochP43}
W.S.~McCulloch and W.H~Pitts (1943): A logical calculus of the ideas
immanent in nervous activity.  {\it Bull.~Mathematical Biophysics 5},
115--133.

\bibitem{McNaughtonY64}
R.~McNaughton and H.~Yamada (1964): Regular expressions and state
graphs for automata.  In {\it Sequential Machines: Selected Papers}
(E.F.~Moore, ed.)~Addison-Wesley, Reading, MA, PP.~157--176.

\bibitem{Mealy55}
G.H.~Mealy (1955): A method for synthesizing sequential circuits.
{\it Bell Syst.~Tech.~J.~34}, 1045--1079.

\bibitem{Mehlhorn84}
K.~Mehlhorn (1984):
{\it Data Structures and Algorithms 2: Graph Algorithms and
  NP-Completeness.}  Springer-Verlag, Berlin.

\bibitem{Miller91}
C.F.~Miller (1991): Decision problems for groups -- survey and
reflections.  In {\it Algorithms and Classification in Combinatorial
  Group Theory}, Springer, New York, pp.~1--60.

\bibitem{Minsky67}
M.~Minsky (1967):
{\it Computation: Finite and Infinite Machines.}
Prentice-Hall, Inc., Englewood Cliffs, NJ. 

\bibitem{Moore56}
E.F.~Moore (1956): Gendanken experiments on sequential machines.  In
{\it Automata Studies} (C.E.~Shannon and J.~McCarthy, eds.) {\it
[Ann.~Math.~Studies 34]}, Princeton Univ.~Press, Princeton, NJ,
pp.~129--153.

\bibitem{Moret97}
B.M.~Moret (1997):
{\it The Theory of Computation}.
Addison-Wesley, Reading, MA.

\bibitem{Myhill57}
J.~Myhill (1957): Finite automata and the representation of events.
WADD TR-57-624, Wright Patterson AFB, Ohio, pp.~112--137.

%N

\bibitem{Naur06}
T.~Naur (2006): Letter to the Editor.  {\it Comm.~ACM 49}, 13.

\bibitem{Nerode58}
A.~Nerode (1958): Linear automaton transformations.  {\it Proc.~AMS
9}, 541--544.

\bibitem{NivenZ80}
I.~Niven and H.S.~Zuckerman (1980):
{\it An Introduction to the Theory of Numbers.} (4th ed.)
J.~Wiley \& Sons, New York.

%O

\bibitem{Olson01}
{\it The Olson Laboratory Fight AIDS@Home project.}  $\langle${\tt
www.fightaidsathome.org}$\rangle$.

%\bibitem{Ortega88}
%J.M.~Ortega (1988):
%{\it Introduction to Parallel and Vector Solution of Linear Systems}.
%Plenum Press, New York.

\bibitem{OttF61}
G.H.~Ott, N.H.~Feinstein (1961): Design of sequential machines from
their regular expressions.  {\it J.~ACM 8}, 585--600.

%P

\bibitem{Papadimitriou94}
C.H.~Papadimitriou (1994):
{\it Computational Complexity}.
Addison-Wesley, Reading, MA.

%R

\bibitem{Rabin63}
M.O.~Rabin (1963): Probabilistic automata.  {\it Inform.~Control 6},
230--245.

\bibitem{Rabin64}
M.O.~Rabin (1964): The word problem for groups.  {\it J.~Symbolic
Logic 29}, 205--206.

\bibitem{RabinScott59}
M.O.~Rabin and D.~Scott (1959): Finite automata and their decision
problems.  {\it IBM J.~Res.~Develop.~3}, 114--125.

\bibitem{Rogers67}
H.~Rogers, Jr.~(1967):
{\it Theory of Recursive Functions and Effective Computability}.
McGraw-Hill, New York.  Reprinted in 1987 by MIT Press, Cambridge, MA.

\bibitem{Rosenberg71}
A.L.~Rosenberg (1971): Data graphs and addressing schemes.  {\it
J.~CSS 5}, 193--238.

\bibitem{Rosenberg74}
A.L.~Rosenberg (1974): Allocating storage for extendible arrays.  {\it
J.~ACM 21}, 652--670.

\bibitem{Rosenberg75}
A.L.~Rosenberg (1975): Managing storage for extendible arrays.  {\it
SIAM J.~Comput.~4}, 287--306.

\bibitem{Rosenberg02}
A.L.~Rosenberg (2003): Accountable Web-computing.  {\it IEEE
Trans.~Parallel and Distr.~Systs.~14}, 97--106.

\bibitem{Rosenberg03}
A.L.~Rosenberg (2003): Efficient pairing functions---and why you
should care.  {\it Intl.~J.~Foundations of Computer Science 14},
3--17.

\bibitem{Rosenberg06}
A.L.~Rosenberg (2006): State.  In {\it Theoretical Computer Science:
Essays in Memory of Shimon Even} (O.~Goldreich, A.L.~Rosenberg,
A.~Selman, eds.)  {\it Springer Festschrift series, Lecture Notes in
Computer Science 3895}, Springer, Heidelberg, pp.~375--398.

\bibitem{Rosenberg09}
A.L.~Rosenberg (2009):
{\it The Pillars of Computation Theory: State, Encoding,
  Nondeterminism}.
Universitext Series, Springer, New York 

\bibitem{RosenbergH01}
A.L.~Rosenberg and L.S.~Heath (2001):
{\it Graph Separators, with Applications}.
Kluwer Academic/Plenum Publishers, New York.

\bibitem{RosenbergS77}
A.L.~Rosenberg and L.J.~Stockmeyer (1977): Hashing schemes for
extendible arrays.  {\it J.~ACM 24}, 199--221.

\bibitem{Rosser53}
J.B.~Rosser (1953):
{\it Logic for Mathematicians.}
McGraw-Hill, New York.

\bibitem{RSAbyWeb95}
{\it The RSA Factoring by Web Project.}
$\langle${\tt http://www.npac.syr.edu/factoring}$\rangle$ (with
Foreword by A.~Lenstra).  Northeast Parallel Architecture Center.

\bibitem{Russel03}
B.~Russell, Bertrand (1903). Principles of Mathematics. Cambridge:
Cambridge University Press. 


%S

\bibitem{Schonfinkel24}
M.~Sch\"onfinkel (1924): \"{U}ber die Bausteine der mathematischen
Logik.  {\it Math.~Annalen 92}, 305--316.

\bibitem{Schroeder98a}
E.~Schr\"{o}der (1898): \"{U}ber zwei Definitionen der Endlichkeit und
G. Cantor'sche S\"{a}tze.  {\it Nova Acta Academiae Caesareae
Leopoldino-Carolinae (Halle a.d.~Saale) 71}, 303--362.

\bibitem{Schroeder98b}
E.~Schr\"{o}der (1898): Die selbst\"{a}ndige Definition der
M\"{a}chtigkeiten 0, 1, 2, 3 und die explicite
Gleichzahligkeitsbedingung.  {\it Nova Acta Academiae Caesareae
Leopoldino-Carolinae (Halle a.d.~Saale) 71}, 365--376.

\bibitem{Sipser97}
M.~Sipser (1997):
{\it Introduction to the Theory of Computation}.
PWS Publishing, Boston, MA.

\bibitem{Stearnsetal72}
R.E.~Stearns, J.~Hartmanis, P.M.~Lewis, II (1972): Hierarchies of
memory limited computations.  {\it J.~Symbolic Logic 37}, 624--625.

\bibitem{Stockmeyer73}
L.J.~Stockmeyer (1973): Extendible array realizations with additive
traversal.  IBM Research Report RC-4578.

\bibitem{Szelepcsenyi87}
R.~Szelepcs\'{e}nyi (1987): The method of forcing for nondeterministic
automata.  {\em Bull.~EATCS 33}, 96--100.

%T

\bibitem{Tarjan72}
R.E.~Tarjan (1972): Sorting using networks of queues and stacks.  {\it
  J.~ACM 19}, 341--346.

\bibitem{TauferACB05}
M.~Taufer, D.~Anderson, P.~Cicotti, C.L.~Brooks (2005): Homogeneous
redundancy: A technique to ensure integrity of molecular simulation
results using public computing.  {\it 19th Intl.~Parallel and
Distributed Processing Symp.}

\bibitem{Tewarson73}
R.P.~Tewarson (1973):
{\it Sparse Matrices}.  In {\it Mathematics in Science \&
  Engineering}.
Academic Press, New York.

\bibitem{Turing36}
A.M.~Turing (1936): On computable numbers, with an application to the
Entscheidungsproblem.  {\it Proc.~London Math.~Soc.} (ser.~2, vol.~42)
230--265; Correction {\it ibid.}~(vol.~43) 544--546.

%W

\bibitem{Warshall62}
S.~Warshall (1962): A theorem on Boolean matrices.  {\it J.~ACM 9},
11-–12.

%\bibitem{WhiteheadR}
%A.N.~Whitehead and B.~Russell (1910-13):
%{\it Principia Mathematica,} 3 vols, Cambridge University Press, 1910,
%1912, and 1913.  Abridged as {\it Principia Mathematica to *56},
%Cambridge University Press, 1962.

\bibitem{Wikipedia05}
Wikipedia: The Free Encyclopedia (2005): \\
{\tt http://en.wikipedia.org/wiki/Pumping\_lemma}

%Z

\bibitem{Zach99}
R.~Zach (1999): Completeness before Post: Bernays, Hilbert, and the
development of propositional logic.  {\it Bull.~Symbolic Logic 5},
331--366.

\end{thebibliography}

