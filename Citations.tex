%version of 12-18-19

\begin{thebibliography}{999}
 
%A

%\bibitem{AielloCL00}
%W.~Aiello, F.R.K.~Chung, L.~Lu (2000): A random graph model for massive graphs.  {\it 32nd Ann.~Symp.~on the Theory of Computing}.

\bibitem{Al-Khwarizmi}
Al-Khwarizmi [Muhammad ibn Musa al-Khwarizmi] (9th century):
{\it The Compendious Book on Calculation by Completion and Balancing}.

%\bibitem{Amatoetal17}
%F.~Amato, F.~di Lillo, V.~Moscato, A.~Picariello, G.~Sperl (2017):
%Influence analysis in online social networks using hypergraphs.
%{\it IEEE Int'l Conf.~on Information Reuse and Integration}.

%\bibitem{AnnexsteinBR90}
%F.~Annexstein, M.~Baumslag, A.L.~Rosenberg (1990):
%Group action graphs and parallel architectures.  {\it SIAM J.~Comput.~19}, 544--569.

\bibitem{AppelH77a}
K.~Appel, W.~Haken (1977):
Every planar map is four colorable, I: Discharging.
{\it Illinois J.~Mathematics, 21}(3), 429-–490.

\bibitem{AppelH77b}
K.~Appel, W.~Haken (1977):
Every planar map is four colorable, II: Reducibility.
{\it Illinois J.~Mathematics, 21}(3), 491-–567.

\bibitem{AppelH77c}
K.~Appel, W.~Haken (October 1977):
Solution of the four color map problem.
{\it Scientific American, 237}(4), 108-–121.

\bibitem{AppelH89}
K.~Appel, W.~Haken [with the collaboration of J.~Koch] (1989):
Every planar map is four-colorable.
{\it Contemporary Mathematics, 98},
American Mathematical Society, Providence, RI.

\bibitem{Arbuthnot}
J.~Arbuthnot (1710): 
An argument for Divine Providence taken from the constant regularity observed in the births of both sexes.
{\it Philosophical Transactions}, The Royal Society,

\bibitem{Argand}
J.-R Argand (1806):
R\'{e}flexions sur la nouvelle th\'{e}orie d'analyse.  Manuscript.

%B

\bibitem{Backus-etal57}
J.W.~Backus, R.J.~Beeber, S.~Best, R.~Goldberg, L.M.~Haibt,
H.L.~Herrick, R.A.~Nelson, D.~Sayre, P.B.~Sheridan, H.~Stern,
L.~Ziller, R.A.~Hughes, R.~Nutt (1957):
The FORTRAN automatic coding system.  {\it Western Joint Computer
Conf.}, 188-–198.

%\bibitem{BarabasiA99}
%A.L.~Barab\'{a}si, R.~Albert (1999):
%Emergence of scaling in random networks.
%{\it Science (286)}, 509--512.

\bibitem{Basin63}
S.L.~Basin (1963): The Fibonacci Sequence as it appears in nature.
{\it Fibonacci Quart.~1}, 53--57.

\bibitem{Bayes}
T.~Bayes, R.~Price (1763): ``An Essay towards solving a
Problem in the Doctrine of Chance.  By the late Rev.~Mr.~Bayes,
communicated by Mr.~Price, in a letter to John Canton,
A.M.F.R.S.'' (PDF).  {\it Philosophical Transactions of the Royal
Society of London 53}(0): 370-–418.

\bibitem{Bell86}
E.T.~Bell (1986):
{\it Men of Mathematics}.
Simon and Schuster, New York.

\bibitem{Berge73}
C.~Berge (1973):
{\it Graphs and Hypergraphs}.
North-Holland, Amsterdam.

\bibitem{BermondP89}
J.-C.~Bermond, C.~Peyrat (1989):
The de Bruijn and Kautz networks: a competitor for the hypercube?
In {\it Hypercube and Distributed Computers} (F.~Andre and
J.P.~Verjus, eds.)  North-Holland, Amsterdam, 279--293.

\bibitem{Bernoulli}
J.~Bernoulli (1713):
{\it Ars conjectandi} (opus posthumum).
Impensis Thurnisiorum Fratum, Basel. 

\bibitem{Bernstein05}
F.~Bernstein (1905): Untersuchungen aus der Mengenlehre. {\it
Math.~Ann.~61}, 117--155.

\bibitem{Birkhoff-MacLane53}
G.~Birkhoff and S.~Mac Lane (1953): {\it A Survey of Modern Algebra},
Macmillan, New York.

\bibitem{Bishop67}
E.~Bishop (1967): {\it Foundations of Constructive Analysis},
McGraw Hill, New York.

\bibitem{BlumS77}
M.~Blum, W.~Sakoda (1977):
On the capability of finite automata in $2$- and $3$- dimensional
space.  {\it 18th IEEE Symp. on Foundations of Computer Science
  (FOCS)}, 147--161.

%\bibitem{Bollobas85}
%B.~Bollobas (1985):
%{\it Random Graphs}.
%Academic Press, N.Y.

\bibitem{Boole54}
G.~Boole (1854):
{\it An Investigation of the Laws of Thought.}  
Walton \& Moberly, London.  Reprinted 2003, Prometheus Books. ISBN
978-1-59102-089-9.

\bibitem{BooneCL73}
W.W.~Boone, F.B.~Cannonito, R.C.~Lyndon (1973):
{\it Word Problems: Decision Problem in Group Theory}, North-Holland,
Amsterdam.

\bibitem{Bremaud17}
P.~Br�maud (2017):
{\it Discrete Probability Models and Methods}.
Springer, Switzerland.

%\bibitem{BuT02}
%T.~Bu, D.~Towsley (2002): On distinguishing between internet power-law generators.  {\it IEEE INFOCOM'02}.

%C

\bibitem{Cantor74}
G.~Cantor (1874): \"{U}ber eine Eigenschaft des Inbegriffes aller
reellen algebraischen Zahlen.  {\it J.~Reine und Angew.~Math.~77},
258--262.

\bibitem{Cantor78}
G.~Cantor (1878): Ein Beitrag zur Begr\"{u}ndung der transfiniter
Mengenlehre.  {\it J.~Reine Angew.~Math.~84}, 242--258.

\bibitem{Cantor87}
G.~Cantor (1887): Mitteilungen zur Lehre vom Transfiniten.
{\it Zeitschrift f\"{u} Philosophie und Philosophische Kritik 91}
81-–125.

\bibitem{Cauchy21}
A.L.~Cauchy (1821): {\it Cours d'analyse de l'\'{E}cole Royale
Polytechnique, 1\`{e}re partie: Analyse alg\'{e}brique.}
l'Imprimerie Royale, Paris.  Reprinted: Wissenschaftliche
Buchgesellschaft, Darmstadt, 1968.

\bibitem{ChartrandB67}
G.~Chartrand, F.~Harary (1967):
Planar permutation graphs.
{\it Annales de l'Institut Henri Poincar\'{e} B, 3}(4), 433-–438.

\bibitem{ChartrandK69}
G.~Chartrand, S.F.~Kapoor (1969):
The cube of every connected graph is 1-hamiltonian.
{\it J.~Research of the National Bureau of Standards, 73B(1)}.  DOI:
10.6028/jres.073B.007

%\bibitem{ChenCGJSW}
%Q.~Chen, H.~Chang, R.~Govindan, S.~Jamin, S.~Shenker,
%W.~Willinger (2002): The origin of power laws in internet topologies revisited.  {\it IEEE INFOCOM'02}.

\bibitem{Christofides76}
N.~Christofides (1976):
Worst-case analysis of a new heuristic for the travelling salesman
problem.  Report 388, Graduate School of Industrial Administration,
Carnegie-Mellon Univ.

\bibitem{Codd70}
E.F.~Codd (1970):
A relational model of data for large shared data banks.
{\it Commun. of the ACM 13}(6), 377-–387.

\bibitem{Cook71}
S.A.~Cook (1971): The complexity of theorem-proving procedures.  {\it
  ACM Symp.~on Theory of Computing}, 151--158.

\bibitem{CLRS}
T.H.~Cormen, C.E.~Leiserson, R.L.~Rivest, C.~Stein (2001):
{\it Introduction to Algorithms (2nd ed.)}.
MIT Press, Cambridge, MA.

\bibitem{Curry34}
H.B.~Curry (1934): Some properties of equality and implication in
combinatory logic.  {\it Annals of Mathematics, 35}, 849--850.

\bibitem{CurryFC58}
H.B.~Curry, R.~Feys, W.~Craig (1958):
{\it Combinatory Logic.  Studies in logic and the foundations of
mathematics}.  North-Holland, Amsterdam.

%D

\bibitem{Davis58}
M.~Davis (1958):
{\it Computability and Unsolvability.}
McGraw-Hill, New York.  Reprinted in 1982 by Dover Press.

\bibitem{Davis73}
M.~Davis (1973):
Hilbert's tenth problem is unsolvable.
{\it American Math.~Monthly 80}, 233–-269.  Reprinted as an appendix
in M.~Davis (1982): {\it Computability and Unsolvability}, Dover
reprint, 1982.

\bibitem{DavisH73}
M.~Davis, R.~Hersh (1973): 
Hilbert's 10th problem.  {\it Scientific American 229},
84-–91.  doi:10.1038/scientificamerican1173-84.

\bibitem{DavisMR76}
M.~Davis, Y.~Matiyasevich, J.~Robinson (1976): Hilbert's Tenth
Problem: Diophantine Equations: Positive Aspects of a Negative
Solution.  In F.E.~Browder.  {\it Mathematical Developments Arising
  from Hilbert Problems.}  {\it Proceedings of Symposia in Pure
  Mathematics, XXVIII.2}, American Math.~Soc., pp.~323-–378.
Reprinted in S.~Feferman, ed.~(1996): {\it The Collected Works of
  Julia Robinson}, {\it American Math.~Soc.}, pp. 269–-378.

\bibitem{Deiser2010}
O.~Deiser (2010): Einf\"{u}hrung in die Mengenlehre – Die Mengenlehre
Georg Cantors und ihre Axiomatisierung durch Ernst Zermelo (3rd,
corrected ed.), Berlin/Heidelberg: Springer, pp. 71, 501,
doi:10.1007/978-3-642-01445-1, ISBN 978-3-642-01444-4.

\bibitem{Poussin}
C.-J.~de la Vallée Poussin (1896):
Recherches analytiques la th\'{e}orie des nombres premiers.
{\it Ann.~Soc.~Scient.~Bruxelles 20}, 183--256

\bibitem{DeMilloLP79}
R.A.~De Millo, A.J.~Perlis, R.J.~Lipton (1979):
Social processes and proofs of theorems and programs.
{\it Comm.~ACM 22}(5), 271--280.

\bibitem{DeMoivre}
A.~de Moivre (1718):
{\it The Doctrine of Chances}.
W.~Pearson, printer, London.


%F

%\bibitem{FaloutsosFF99}
%M.~Faloutsos, P.~Faloutsos, C.~Faloutsos (1999):
%On power-law relationships of the internet topology.  {\it ACM SIGCOMM'99}.

\bibitem{Fleischner74}
H.~Fleischner (1974):
The square of every two-connected graph is hamiltonian.
{\it J.~Combinatorics Theory (B) 16}, 29--34.

\bibitem{Foster47}
J.E.~Foster (1947): A number system without a zero symbol.
{\it Mathematics Magazine 21}(1), 39-–41.

\bibitem{Fubini}
G.~Fubini (1907): Sugli integrali multipli.
{\it Rom.~Acc.~L.~Rend.~(5), 16(1)}, pp.~608–614.  In {\it zbMATH
  38.0343.02}.  Reprinted in
G.~Fubini (1958): {\it Opere scelte, 2}, Cremonese, pp. 243–249.

\bibitem{FueterP23}
R.~Fueter and G.~P\'{o}lya (1923):
Rationale Abz\"{a}hlung der Gitterpunkte.  {\it
Vierteljschr.~Naturforsch.~Ges.~Z\"{u}rich 58}, 380--386.

%G

\bibitem{GareyJ79}
M.R.~Garey and D.S.~Johnson (1979):
{\it Computers and Intractability}.
W.H.~Freeman and Co., San Francisco.

\bibitem{Goedel31}
K.~G\"{o}del (1931): \"{U}ber Formal Unentscheidbare S\"{a}tze der
Principia Mathematica und Verwandter Systeme, I.  {\it Monatshefte
f\"{u}r Mathematik u.~Physik 38}, 173--198.


%H

\bibitem{Hadamard}
J.~Hadamard (1896):
Sur la distribution des z\'{e}ros de la fonction $\zeta(s)$ et ses
cons\'{e}quences arithm\'{e}tiques.
{\it Bull.~Soc.~Math.~France 24}, 199--220.


\bibitem{Halmos60}
P.R.~Halmos (1960):
{\it Naive Set Theory}.
D.~Van Nostrand, New York.

\bibitem{Hazewinkel}
M.~Hazewinkel, ed. (2001): %[1994],
Vi\`{e}te theorem.  In {\it Encyclopedia of Mathematics}.
Springer Science+Business Media B.V. / Kluwer Academic Publishers,
ISBN 978-1-55608-010-4.

\bibitem{Heawood90}
P.J.~Heawood (1890):
Map-colour theorems.  {\it Quarterly J.~Mathematics, Oxford, 24}
332--338.

\bibitem{Hilbert02}
D.~Hilbert (1902): Mathematical Problems.  (Translated by
M.F.W.~Newson).  {\it Bull.~American Math.~Soc.~8}(10), 437-–479.

\bibitem{Hoel58}
P.G.~Hoel (1958):
{\it Introduction to Mathematical Statistics} (2nd Ed.)
John Wiley \& Sons, New York.

\bibitem{Horner}
W.G.~Horner (1819): 
A new method of solving numerical equations of all orders, by
continuous approximation. {\it Philosophical Transactions. Royal
Society of London}, 308--335.

\bibitem{Hutchins00}
W.J.~Hutchins [ed.] (2000):
Machine translation at Harvard: Interview with A.G.~Oettinger.
In {\it Early Years in Machine Translation: Memoirs and Biographies of
  Pioneers.  (Studies in the History of the Language Sciences)}.
John Benjamins Publishing Co., p.~86

\bibitem{Hwang79}
K.~Hwang (1979):
{\it Computer Arithmetic: Principles, Architecture, and Design}.
John Wiley \& Sons, New York.


%I


%J

%\bibitem{JaiswalRT04}
%S.~Jaiswal, A.L.~Rosenberg, D.~Towsley (2004):
%Comparing the structure of power-law graphs and the Internet AS graph.
%{\it 12th IEEE Int'l Conf.~on Network Protocols (ICNP'04)}.

\bibitem{JohnssonH1989}
S.L.~Johnsson, C.T.~Ho (1989):
Optimum broadcasting and personalized communication in hypercubes.
{\it IEEE Trans.~Computers 38}, 1249--1268.


%K

\bibitem{Karp72}
R.M.~Karp (1972): Reducibility among combinatorial problems.  In {\it
Complexity of Computer Computations} (R.E.~Miller and J.W.~Thatcher,
eds.)  Plenum Press, NY, pp.~85--103.

\bibitem{KennedyE95}
J.~Kennedy, R.~Eberhart (1995): 
Particle swarm optimization.  {\it IEEE Int'l Conf.~Neural
Networks, IV}, 1942--1948.

\bibitem{KirkpatrickGV83}
S.~Kirkpatrick, C.D.~Gelatt Jr., M.P.~Vecchi (1983):
Optimization by simulated annealing.  {\it Science 220}(4598),
671�--680.

\bibitem{Kleene36}
S.C.~Kleene (1936): General recursive functions of natural numbers.
{\it Math.~Annalen 112}, 727--742.

\bibitem{Kleene52}
S.C.~Kleene (1952):
{\it Introduction to Metamathematics.}
D.~Van Nostrand, Princeton, NJ.

\bibitem{Knuth69}
D.E.~Knuth (1969):
{\it The Art of Computer Programming, Vol.~2: Seminumerical
  Algorithms.}  Addison-Wesley, Reading, MA.

\bibitem{Konig36}
D.~K\"onig (1936):
{\it Theorie der endlichen und unendlichen Graphen.}  Lipzig: Akad.~Verlag.

\bibitem{Kuhn55}
H.W.~Kuhn (1955): The Hungarian method for the assignment problem.
{\it Naval Research Logistics Quarterly 2}, 83--��97.

\bibitem{Kuratowski30}
K.~Kuratowski (1930):
Sur le probl\`{e}me des courbes gauches en topologie.
{\it Fundamenta Mathematica 15}, 271--283. 

%\bibitem{Kwan60}
%M.-K.~Kwan (1960): Graphic programming using odd or even points.
%{\it Acta Mathematica Sinica, 10} (in Chinese), 263--266.  Translated
%into English in {\it Chinese Mathematics 1} (1962) 273--277.


%L

\bibitem{Lamport12}
L.~Lamport (2012): How to write a 21st century proof.
{\it J.~Fixed Point Theory and Applications, 11}(1), 43-–63.

\bibitem{Laplace}
P.S.~Laplace (1814):
{\it Essai philosophique sur les probabilit\'{e}s}.
Bachelier, Paris.

\bibitem{Lee12}
P.M.~Lee (2012): {\it Bayesian Statistics: An Introduction}.  Wiley,
Hoboken, NJ.

\bibitem{Leibnitz}
G.W.~ Leibniz (Leibnitz) (1666):
{\it Dissertatio de Arte Combinatoria}.
{\it S\"{a}mtliche Schriften und Briefe}.
Akademie Verlag, Berlin.

\bibitem{Leibniz}
G.W.~ Leibniz (Leibnitz) (1674-76):
{\it S\"{a}mtliche Schriften und Briefe, Reihe VII: Mathematische
  Schriften, vol.~5: Infinitesimalmathematik}.
Akademie Verlag, Berlin.

%\bibitem{Leiserson85}
%C.E.~Leiserson (1985):
%Fat-trees: Universal networks for hardware-efficient supercomputing.
%{\it IEEE Trans.~on Computers, C-34}(10), 892--201.

\bibitem{LewR78a}
J.S.~Lew and A.L.~Rosenberg (1978): Polynomial indexing of integer
lattices, I.  {\it J.~Number Th.~10}, 192--214.
 
\bibitem{LewR78b}
J.S.~Lew and A.L.~Rosenberg (1978): Polynomial indexing of integer
lattices, II.  {\it J.~Number Th.~10}, 215--243.

%\bibitem{LiptonT79}
%R.J.~Lipton, R.E.~Tarjan (1979):
%A separator theorem for planar graphs.
%{\it SIAM J.~Applied Mathematics 36}(2), 177-–189.

\bibitem{Littlewood-misc}
J.E.~Littlewood (1953):
{\it A Mathematician's Miscellany.}
Methuen \& Co, Ltd.
Reprinted as {\it Littlewood's Miscellany} (B.~Bollob\`{a}s, ed.),
1986, University Press, Cambridge.

%\bibitem{LiuBV10}
%D.~Liu, N.~Blenn, P.~Van Mieghem (2010):
%Modeling social networks with overlapping communities using hypergraphs and their line graphs.
%Report arXiv:1012.2774, Dec.~2010, {\tt http://cds.cern.ch/record/1314107}.

%\bibitem{Lovasz73}
%L.~Lovasz (1973): Coverings and colorings of hypergraphs.  {\it 4th Southeast Conf.~on Combinatorics, Graph Theory, and Computing},
%3--12.

%M

\bibitem{Marchese96}
F.~Marchese (1996): Cellular automata in robot path planning.
{\it EUROBOT'96}, 116--125.

\bibitem{Matiyasevich93}
Y.~Matiyasevich (1993):
{\it Hilbert's Tenth Problem.}
MIT Press, Cambridge, MA.

\bibitem{MacDuffee}
C.C.~MacDuffee (1954):
{\it Theory of Equations}.
John Wiley \& Sons, New York

%\bibitem{Mead-Conway}
%C.~Mead and L.~Conway (1979):
%{\it Introduction to VLSI Systems}.
%Addison-Wesley, Reading, MA., (ISBN 0201043580).

\bibitem{Merton}
R.K.~Merton (1965):
{\it On the Shoulders of Giants: A Shandean Postscript}.
The Free Press, New York.

\bibitem{MillerPRS79}
R.E.~Miller, N.~Pippenger, A.L.~Rosenberg, L.~Snyder (1979): Optimal
2,3-trees.  {\it SIAM J.~Comput.~8}, 42--59.


%N

\bibitem{Nash50}
J.~Nash (1950): Equilibrium points in $n$-person games.
{\it Proc.~U.S.~Nat'l Acad.~Sciences 36}(1), 48--49.

\bibitem{Newton}
I.~Newton (1687): {\it Philosophiæ Naturalis Principia Mathematica}
(known popularly as {\it Principia Mathematica}).
Royal Society.


\bibitem{NivenZ80}
I.~Niven and H.S.~Zuckerman (1980):
{\it An Introduction to the Theory of Numbers.} (4th ed.)
J.~Wiley \& Sons, New York.

%O


%P

\bibitem{Paulos}
J.A.~Paulos (1990):
{\it Innumeracy: Mathematical Iliteracy and Its Consequences.}
Vintage Books (Random House), New York.

\bibitem{PetersonW81}
W.W.~Peterson, E.J.~Weldon, Jr.~(1981):
{\it Error-Correcting Codes.}~(2nd Ed.)
MIT Press, Cambridge, Mass.

%\bibitem{PhatakK94}
%D.S.~Phatak, I.~Koren (1994):
%Hybrid signed-digit number systems: a unified framework for redundant
%number representations with bounded carry propagation chains.
%{\it IEEE Trans.~Computers 43}(8), 880-–891.

%R

\bibitem{RabinS59}
M.O.~Rabin, D.~Scott (1959): Finite automata and their decision
problems.  {\it IBM J.~Research and Development 3}(2), 114-–125.

\bibitem{Rosenberg74}
A.L.~Rosenberg (1974): Allocating storage for extendible arrays.  {\it
J.~ACM 21}, 652--670.

\bibitem{Rosenberg75}
A.L.~Rosenberg (1975): Managing storage for extendible arrays.  {\it
SIAM J.~Comput.~4}, 287--306.

\bibitem{Rosenberg79}
A.L.~Rosenberg (1979): Profile numbers.  {\it Fibonacci Quart.~17}(3),
259--264.

%\bibitem{Rosenberg85a}
%A.L.~Rosenberg (1985): A hypergraph model for fault-tolerant VLSI processor arrays.  {\it IEEE Trans.~Comput.~C-34}, 578--584.

%\bibitem{Rosenberg89a}
%A.L.~Rosenberg (1989): Interval hypergraphs.  In {\it Graphs and Algorithms} (R.B.~Richter, ed.) {\it Contemporary Mathematics 89}, Amer.~Math.~Soc., 27--44.

\bibitem{Rosenberg91}
A.L.~Rosenberg (1991): Cycles in networks.  Tech.~Rpt.~COINS-91-20, Univ.~Massachusetts.

\bibitem{Rosenberg02}
A.L.~Rosenberg (2003): Accountable Web-computing.  {\it IEEE
Trans.~Parallel and Distr.~Systs.~14}, 97--106.

\bibitem{Rosenberg03}
A.L.~Rosenberg (2003): Efficient pairing functions---and why you
should care.  {\it Intl.~J.~Foundations of Computer Science 14},
3--17.

\bibitem{Rosenberg09}
A.L.~Rosenberg (2009):
{\it The Pillars of Computation Theory: State, Encoding, Nondeterminism}.
Universitext Series, Springer, New York 

\bibitem{Rosenberg12}
A.L.~Rosenberg (2012): The parking problem for finite-state robots.  {\it J.~Graph Algorithms and Applications 16}(2), 483--506.

%\bibitem{RosenbergH01}
%A.L.~Rosenberg and L.S.~Heath (2001):
%{\it Graph Separators, with Applications}.
%Kluwer Academic/Plenum Publishers, New York.

\bibitem{RosenbergS78}
A.L.~Rosenberg and L.~Snyder (1978):Minimal-comparison 2,3-trees.
{\it SIAM J.~Comput.~7}, 465--480.

%\bibitem{RosenbergS77}
%A.L.~Rosenberg and L.J.~Stockmeyer (1977): Hashing schemes for extendible arrays.  {\it J.~ACM 24}, 199--221.

\bibitem{Ross76}
S.M.~Ross (1976):
{\it A First Course in Probability}.
Pearson Education.

\bibitem{Rosser53}
J.B.~Rosser (1953):
{\it Logic for Mathematicians.}
McGraw-Hill, New York.

\bibitem{Russell02}
B.~Russell (1902): Letter to Frege.  In, Jean van Heijenoort,
ed. (1967):  {\it From Frege to G\"{o}del}.  Harvard University
Press, Cambridge, MA, pp.~124--125.

%S

\bibitem{SaadS89}
Y.~Saad, M.H.~Schultz (1989):
Data communication in hypercubes.
{\it J.~Parallel and Distributed Computing 6}, 115--135.

%\bibitem{Sammet69}
%J.E.~Sammet (1969)
%String and list processing languages.
%In {\it Programming Languages: History and Fundamentals}.
%Prentice-Hall.  ISBN 0-13-729988-5.

\bibitem{Sammet78}
J.E.~Sammet (1978):
The early history of COBOL.  In {\it History of Programming
  Languages}.  Academic Press (published 1981).

\bibitem{Schonfinkel24}
M.~Sch\"onfinkel (1924): \"{U}ber die Bausteine der mathematischen
Logik.  {\it Math.~Annalen 92}, 305--316.

\bibitem{Schroeder98a}
E.~Schr\"{o}der (1898): \"{U}ber zwei Definitionen der Endlichkeit und
G. Cantor'sche S\"{a}tze.  {\it Nova Acta Academiae Caesareae
Leopoldino-Carolinae (Halle a.d.~Saale) 71}, 303--362.

\bibitem{Schroeder98b}
E.~Schr\"{o}der (1898): Die selbst\"{a}ndige Definition der
M\"{a}chtigkeiten 0, 1, 2, 3 und die explicite
Gleichzahligkeitsbedingung.  {\it Nova Acta Academiae Caesareae
Leopoldino-Carolinae (Halle a.d.~Saale) 71}, 365--376.

\bibitem{Schwartz80}
J.T.~Schwartz (1980):
Ultracomputers.
{\it ACM Trans.~Programming Languages 2}, 484--521.

\bibitem{Shannon38}
C.E.~Shannon (1938): A symbolic analysis of relay and switching
circuits.  {\it Trans.~American Inst.~of Electrical Engineers 57}(12),
713-–723.

\bibitem{Shannon48}
C.E.~Shannon (1948): A mathematical theory of communication.  {\it
  Bell System Technical Journal 27}(July and October), pp.~379--423
and 623--656.

\bibitem{Shinahr74}
I.~Shinahr (1974): Two- and three-dimensional firing-squad
synchronization problems.  {\it Inform. and Control 24}, 163-–180.

\bibitem{Smullyan61}
R.M.~Smullyan (1961):  Lexicographical ordering; $n$-adic
representation of integers.  
{\it Theory of Formal Systems, Annals of Mathematics Studies 47},
Princeton University Press, pp.~34–-36.

%T

%\bibitem{TangmunarunkitGJSW02}
%H.~Tangmunarunkit, R.~Govindan, S.~Jamin, S.~Shenker, W.~Willinger (2002):
%Network topology generators: Degree-based vs. structural.  {\it ACM SIGCOMM'02}.

\bibitem{Tonien07}
D.~Tonien (2007):
A simple visual proof of the Schr\"{o}der-Bernstein theorem.
{\it Elemente der Mathematik 62}, 118--120.

\bibitem{Turing36}
A.M.~Turing (1936): On computable numbers, with an application to the
Entscheidungsproblem.  {\it Proc.~London Math.~Soc.} (ser.~2, vol.~42)
230--265; Correction {\it ibid.}~(vol.~43) 544--546.

%U

%\bibitem{Ullman84}
%J.D.~Ullman (1984): {\it Computational Aspects of VLSI.}  Computer Science Press, Rockville, Md.

%V

\bibitem{vonNeumann-Morg}
J.~von Neumann, O.~Morgenstern (1944):
{\it Theory of Games and Economic Behavior.}
Princeton University Press.

%W

\bibitem{WagnerF74}
R.~Wagner, M.J.~Fischer (1974):
The string-to-string correction problem.
{\it Journ.~Assoc.~Computing Machinery (ACM) 21}, 168-–178.

\bibitem{Weber91}
H.L.~Weber (1891-2): Kronecker.
{\it Jahresbericht der Deutschen Mathematiker-Vereinigung 2}, 5--23.

\bibitem{Russell03}
A.N.~Whitehead, B.~Russell (1903).  {\it Principles of Mathematics}.
Cambridge University Press. 

\bibitem{Wiles95}
A.J.~Wiles (May 1995): ``Issue 3''.  {\it Annals of Mathematics 141}, 1--��551.


%Y

\bibitem{Yngve}
V.~Yngve (1958): 
A programming language for mechanical translation.
In {\it Mechanical Translation 5}(1): 25--41.  MIT, Cambridge, MA.

\bibitem{Yoeli62}
M.~Yoeli (1962): Binary ring sequences.  {\it Amer.~Math.~Monthly 69},
852--855.

%Z

\bibitem{Zeckendorf72}
E.~Zeckendorf (1972): 
{\it Repr\'{e}sentation des nombres naturels par une somme de nombres de Fibonacci ou de nombres de Lucas.}
{\it Bull.~Soc.~R.~Sci.~Li\`{e}ge 41} (in French), 179?182.

%\bibitem{ZeguraCD97}
%E.~Zegura, K.L.~Calvert, M.J.~Donohoo (1997):
%A quantitative comparison of graph-based models for internetworks.
%{\it IEEE/ACM Trans.~on Networking, 5}(6), 770--783.
\end{thebibliography}

