%version of 12-27-18

\begin{thebibliography}{999}

%A

\bibitem{AnnexsteinBR90}
F.~Annexstein, M.~Baumslag, A.L.~Rosenberg (1990):
Group action graphs and parallel architectures.
{\it SIAM J.~Comput.~19}, 544--569.

%B

\bibitem{Basin63}
S.L.~Basin (1963): The Fibonacci Sequence as it appears in nature.
{\it Fibonacci Quart.~1}, 53--57.

\bibitem{Bell86}
E.T.~Bell (1986):
{\it Men of Mathematics}.
Simon and Schuster, New York.

\bibitem{Berge73}
C.~Berge (1973):
{\it Graphs and Hypergraphs}.
North-Holland, Amsterdam.

\bibitem{BermondP89}
J.-C.~Bermond, C.~Peyrat (1989):
The de Bruijn and Kautz networks: a competitor for the hypercube?
In {\it Hypercube and Distributed Computers} (F.~Andre and
J.P.~Verjus, eds.)  North-Holland, Amsterdam, 279--293.

\bibitem{Bernstein05}
F.~Bernstein (1905): Untersuchungen aus der Mengenlehre. {\it
Math.~Ann.~61}, 117--155.

\bibitem{Birkhoff-MacLane53}
G.~Birkhoff and S.~Mac Lane (1953): {\it A Survey of Modern Algebra},
Macmillan, New York.

\bibitem{Bishop67}
E.~Bishop (1967): {\it Foundations of Constructive Analysis},
McGraw Hill, New York.

\bibitem{BooneCL73}
W.W.~Boone, F.B.~Cannonito, R.C.~Lyndon (1973):
{\it Word Problems: Decision Problem in Group Theory}, North-Holland,
Amsterdam.


%C

\bibitem{Cantor74}
G.~Cantor (1874): \"{U}ber eine Eigenschaft des Inbegriffes aller
reellen algebraischen Zahlen.  {\it J.~Reine und Angew.~Math.~77},
258--262.

\bibitem{Cantor78}
G.~Cantor (1878): Ein Beitrag zur Begr\"{u}ndung der transfiniter
Mengenlehre.  {\it J.~Reine Angew.~Math.~84}, 242--258.

\bibitem{Cantor87}
G.~Cantor (1887): Mitteilungen zur Lehre vom Transfiniten.
{\it Zeitschrift f\"{u} Philosophie und Philosophische Kritik 91}
81-–125.

\bibitem{Cauchy21}
A.L.~Cauchy (1821): {\it Cours d'analyse de l'\'{E}cole Royale
Polytechnique, 1\`{e}re partie: Analyse alg\'{e}brique}.
l'Imprimerie Royale, Paris.  Reprinted: Wissenschaftliche
Buchgesellschaft, Darmstadt, 1968.

\bibitem{ChartrandK69}
G.~Chartrand, S.F.~Kapoor (1969):
The cube of every connected graph is 1-hamiltonian.
{\it J.~Research of the National Bureau of Standards, 73B(1)}.  DOI:
10.6028/jres.073B.007

\bibitem{CLRS}
T.H.~Cormen, C.E.~Leiserson, R.L.~Rivest, C.~Stein (2001):
{\it Introduction to Algorithms (2nd ed.)}.
MIT Press, Cambridge, MA.

\bibitem{Curry34}
H.B.~Curry (1934): Some properties of equality and implication in
combinatory logic.  {\it Annals of Mathematics, 35}, 849--850.

\bibitem{CurryFC58}
H.B.~Curry, R.~Feys, W.~Craig (1958):
{\it Combinatory Logic.  Studies in logic and the foundations of
mathematics}.  North-Holland, Amsterdam.

%D

\bibitem{Davis58}
M.~Davis (1958):
{\it Computability and Unsolvability.}
McGraw-Hill, New York.

\bibitem{Deiser2010}
O.~Deiser (2010): Einf\"{u}hrung in die Mengenlehre – Die Mengenlehre
Georg Cantors und ihre Axiomatisierung durch Ernst Zermelo (3rd,
corrected ed.), Berlin/Heidelberg: Springer, pp. 71, 501,
doi:10.1007/978-3-642-01445-1, ISBN 978-3-642-01444-4.


%F

\bibitem{Fleischner74}
H.~Fleischner (1974):
The square of every two-connected graph is hamiltonian.
{\it J.~Combinatorics Theory (B) 16}, 29--34.

\bibitem{Fubini}
G.~Fubini (1907): Sugli integrali multipli.
{\it Rom.~Acc.~L.~Rend.~(5), 16(1)}, pp.~608–614.  In {\it zbMATH
  38.0343.02}.  Reprinted in
G.~Fubini (1958): {\it Opere scelte, 2}, Cremonese, pp. 243–249.

\bibitem{FueterP23}
R.~Fueter and G.~P\'{o}lya (1923):
Rationale Abz\"{a}hlung der Gitterpunkte.  {\it
Vierteljschr.~Naturforsch.~Ges.~Z\"{u}rich 58}, 380--386.

%G

\bibitem{Goedel31}
K.~G\"{o}del (1931): \"{U}ber Formal Unentscheidbare S\"{a}tze der
Principia Mathematica und Verwandter Systeme, I.  {\it Monatshefte
f\"{u}r Mathematik u.~Physik 38}, 173--198.


%H

\bibitem{Halmos60}
P.R.~Halmos (1960):
{\it Naive Set Theory}.
D.~Van Nostrand, New York.

\bibitem{Hazewinkel}
M.~Hazewinkel, ed. (2001): %[1994],
Vi\`{e}te theorem.  In {\it Encyclopedia of Mathematics}.
Springer Science+Business Media B.V./Kluwer Academic Publishers,
ISBN 978-1-55608-010-4.

\bibitem{Horner}
W.G.~Horner (1819): 
A new method of solving numerical equations of all orders, by
continuous approximation. {\it Philosophical Transactions. Royal
Society of London}, 308-–335.

%I


%J

\bibitem{JohnssonH1989}
S.L.~Johnsson, C.T.~Ho (1989):
Optimum broadcasting and personalized communication in hypercubes.
{\it IEEE Trans.~Computers 38}, 1249--1268.


%K

\bibitem{Kleene36}
S.C.~Kleene (1936): General recursive functions of natural numbers.
{\it Math.~Annalen 112}, 727--742.

\bibitem{Kleene52}
S.C.~Kleene (1952):
{\it Introduction to Metamathematics.}
D.~Van Nostrand, Princeton, NJ.

\bibitem{Konig36}
D.~K\"onig (1936):
{\it Theorie der endlichen und unendlichen Graphen.}  Lipzig: Akad.~Verlag.

\bibitem{Kuhn55}
H.W.~Kuhn (1955): The Hungarian method for the assignment problem.
{\it Naval Research Logistics Quarterly, 2}, 83-–97.

%L

\bibitem{Leibniz}
G.W.~ Leibniz (Leibnitz) (1674-76):
{\it S\"{a}mtliche Schriften und Briefe, Reihe VII: Mathematische
  Schriften, vol.~5: Infinitesimalmathematik}.
Akademie Verlag, Berlin.

\bibitem{LewR78a}
J.S.~Lew and A.L.~Rosenberg (1978): Polynomial indexing of integer
lattices, I.  {\it J.~Number Th.~10}, 192--214.
 
\bibitem{LewR78b}
J.S.~Lew and A.L.~Rosenberg (1978): Polynomial indexing of integer
lattices, II.  {\it J.~Number Th.~10}, 215--243.

\bibitem{Littlewood-misc}
J.E.~Littlewood (1953):
{\it A Mathematician's Miscellany.}
Methuen \& Co, Ltd.
Reprinted as {\it Littlewood's Miscellany} (B.~Bollob\`{a}s, ed.),
1986, University Press, Cambridge.


%M

\bibitem{MillerPRS79}
R.E.~Miller, N.~Pippenger, A.L.~Rosenberg, L.~Snyder (1979): Optimal
2,3-trees.  {\it SIAM J.~Comput.~8}, 42--59.


%N

\bibitem{Newton}
I.~Newton (1687): {\it Philosophiæ Naturalis Principia Mathematica}
(known popularly as {\it Principia Mathematica}).
Royal Society.


\bibitem{NivenZ80}
I.~Niven and H.S.~Zuckerman (1980):
{\it An Introduction to the Theory of Numbers.} (4th ed.)
J.~Wiley \& Sons, New York.

%O

%P

\bibitem{Paulos}
J.A.~Paulos (1990):
{\it Innumeracy: Mathematical Iliteracy and Its Consequences.}
Vintage Books (Random House), New York.

\bibitem{PetersonW81}
W.W.~Peterson, E.J.~Weldon, Jr.~(1981):
{\it Error-Correcting Codes.}~(2nd Ed.)
MIT Press, Cambridge, Mass.


%R

\bibitem{Rosenberg74}
A.L.~Rosenberg (1974): Allocating storage for extendible arrays.  {\it
J.~ACM 21}, 652--670.

\bibitem{Rosenberg75}
A.L.~Rosenberg (1975): Managing storage for extendible arrays.  {\it
SIAM J.~Comput.~4}, 287--306.

\bibitem{Rosenberg79}
A.L.~Rosenberg (1979): Profile numbers.  {\it Fibonacci Quart.~17}(3),
259--264.

\bibitem{Rosenberg02}
A.L.~Rosenberg (2003): Accountable Web-computing.  {\it IEEE
Trans.~Parallel and Distr.~Systs.~14}, 97--106.

\bibitem{Rosenberg03}
A.L.~Rosenberg (2003): Efficient pairing functions---and why you
should care.  {\it Intl.~J.~Foundations of Computer Science 14},
3--17.

\bibitem{Rosenberg09}
A.L.~Rosenberg (2009):
{\it The Pillars of Computation Theory: State, Encoding,
  Nondeterminism}.
Universitext Series, Springer, New York 

\bibitem{RosenbergH01}
A.L.~Rosenberg and L.S.~Heath (2001):
{\it Graph Separators, with Applications}.
Kluwer Academic/Plenum Publishers, New York.

\bibitem{RosenbergS78}
A.L.~Rosenberg and L.~Snyder (1978):Minimal-comparison 2,3-trees.
{\it SIAM J.~Comput.~7}, 465--480.


\bibitem{RosenbergS77}
A.L.~Rosenberg and L.J.~Stockmeyer (1977): Hashing schemes for
extendible arrays.  {\it J.~ACM 24}, 199--221.

\bibitem{Ross76}
S.M.~Ross (1976):
{\it A First Course in Probability}.
Pearson Education.

\bibitem{Rosser53}
J.B.~Rosser (1953):
{\it Logic for Mathematicians.}
McGraw-Hill, New York.

\bibitem{Russel03}
B.~Russell (1903).  {\it Principles of Mathematics}.
Cambridge University Press. 


%S

\bibitem{SaadS89}
Y.~Saad, M.H.~Schultz (1989):
Data communication in hypercubes.
{\it J.~Parallel and Distributed Computing 6}, 115--135.

\bibitem{Schonfinkel24}
M.~Sch\"onfinkel (1924): \"{U}ber die Bausteine der mathematischen
Logik.  {\it Math.~Annalen 92}, 305--316.

\bibitem{Schroeder98a}
E.~Schr\"{o}der (1898): \"{U}ber zwei Definitionen der Endlichkeit und
G. Cantor'sche S\"{a}tze.  {\it Nova Acta Academiae Caesareae
Leopoldino-Carolinae (Halle a.d.~Saale) 71}, 303--362.

\bibitem{Schroeder98b}
E.~Schr\"{o}der (1898): Die selbst\"{a}ndige Definition der
M\"{a}chtigkeiten 0, 1, 2, 3 und die explicite
Gleichzahligkeitsbedingung.  {\it Nova Acta Academiae Caesareae
Leopoldino-Carolinae (Halle a.d.~Saale) 71}, 365--376.

\bibitem{Schwartz80}
J.T.~Schwartz (1980):
Ultracomputers.
{\it ACM Trans.~Programming Languages 2}, 484--521.


%T

\bibitem{Turing36}
A.M.~Turing (1936): On computable numbers, with an application to the
Entscheidungsproblem.  {\it Proc.~London Math.~Soc.} (ser.~2, vol.~42)
230--265; Correction {\it ibid.}~(vol.~43) 544--546.

%U

\bibitem{Ullman84}
J.D.~Ullman (1984):
{\it Computational Aspects of VLSI.}
Computer Science Press, Rockville, Md.

%V

%W

%Y

%Z


\end{thebibliography}
