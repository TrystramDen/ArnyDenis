
\chapter{Lucas numbers}
\label{sec:Lucas-numbers}
\index{Lucas numbers}
\index{Lucas sequence}

A constant preoccupation of mathematicians is to understand why
important mathematical structures exhibit their observed properties.
A common way to seek such understanding is to perturb the definition
of the important structure and study the effects of the perturbation.
While this stratagem leads to interesting, valuable results only
sometimes, it is an invaluable tool in the hands of a gifted
mathematician.  This chapter is devoted to a brief survey of such a
study by the 19th-century French mathematician Fran\c{c}ois Edouard
Anatole Lucas (commonly known as Edouard Lucas).\index{Lucas, Edouard}

\paragraph{\small\sf A. Definition}

Lucas, who is credited with giving the name ``Fibonacci numbers'' to
the sequence discovered by Leonardo Pisano,
\index{Fibonacci numbers!origin of name}
investigated the consequences of perturbing the initial conditions,
$F(0) = F(1) = 1$, in the definition (\ref{eq:Fibonacci-defn}) of the
Fibonacci sequence.

{\Denis we may add an observation here. There is another "natural" choice of perturbation with 1 and 2, but this leads to the same sequence as before (shifted)}

Lucas's overall goal was simply to replace the Fibonacci sequence's
initial values $\langle 1,1 \rangle$, with the values $\langle 2,1
\rangle$.  It turns out to be much more fruitful---in terms of more
striking results and simpler proofs---to make a somewhat more drastic
perturbation:

The {\it Lucas sequence} \index{Lucas sequence!definition} is the
infinite sequence of positive integers
\[ L(-1), \ L(0), \ L(1), \ L(2), \ldots \]
generated by the recurrence
\begin{eqnarray}
\nonumber
L(-1) & = & 2 \\
\label{eq:Lucas-defn-1}
L(0) & = & 1 \\
\nonumber
L(n) & = & L(n-1) \ + \ L(n-2) \ \ \ \mbox{ for all } n \geq 1
\end{eqnarray}
Because we conventionally index sequences by {\em nonnegative}
numbers, we henceforth ignore $L(-1)$ and use the following {\em
  standard definition} of the Lucas sequence.
\begin{eqnarray}
\nonumber
L(0) & = & 1 \\
\label{eq:Lucas-defn-2}
L(1) & = & 3 \\
\nonumber
L(n) & = & L(n-1) \ + \ L(n-2) \ \ \ \mbox{ for all } n > 1
\end{eqnarray}

\medskip

The following finite sequences present the first few elements of both
the Lucas sequence (for illustration) and the Fibonacci sequence
(for comparison).
\[
\begin{array}{r|rrrrrrrrrrr}
n: &
 0, & 1, & 2, & 3, &  4, &  5, &  6, &  7, &  8, &   9, & \ldots \\
\hline
L(n): &
 1, & 3, & 4, & 7, & 11, & 18, & 29, & 47, & 76, & 123, & \ldots \\
F(n): &
 1, & 1, & 2, & 3, &  5, &  8, & 13, & 21, & 34, &  55, & \ldots
\end{array}
\]

We begin our  brief study of the Lucas sequence by noting that just a
minor tweak converts the Fibonacci-related identity revealed in
Proposition~\ref{thm:FiboSum-1} to an identity about Lucas numbers.

\begin{prop}
\label{thm:LucasSum-1}
For all integers $n \geq 0$,
\begin{equation}
\label{eq:multilinear-Lucas-1}
L(n+2) \ = \
1 \ + \ L(-1) \ + \ L(0) \ + \ L(1) \ + \ L(2) \ + \cdots + \ L(n)
\end{equation}
\end{prop}

\begin{proof}[Sketch]
We can literally repeat the proof of Proposition~\ref{thm:FiboSum-1},
with only a change in the induction's base case, which becomes
\[ L(2) \ = \ L(-1) + L(0) + 1 \ = \ 2 + 1 + 1 \ = \ 4. \]
The body of the inductive argument holds for the Lucas sequence as
well as for the Fibonacci sequence.
\qed
\end{proof}

\paragraph{\small\sf B. Relating the Lucas and Fibonacci numbers}

There are several simple equations that relate the Lucas and Fibonacci
numbers.  We present a few of the most aestetically pleasing
ones,\footnote{Aesthetically pleasing, that is, to the authors.  As
  noted by the author Margaret Wolfe Ungerford in {\it Molly Bawn}
  (1878),\index{Ungerford, Margaret Wolfe} ``Beauty is in the eye of
  the beholder.''}~in terms of their exposing an intimate
relationship between the two sequences.
\index{Fibonacci numbers!relations with Lucas numbers}
\index{Lucas numbers!relations with Fibonacci numbers}

\begin{prop}
\label{thm:Lucas-n:2Fibs}
For all $m, n \geq 1$
\begin{eqnarray}
\label{eq:L-F-a}
\mathbf{(a) } \ \ 
L(n) & = & F(n+1) + F(n-1) \\
\label{eq:L-F-b}
\mathbf{(b) } \ \
F(n+1) & = & \frac{1}{2} \big(F(n) \ + \ L(n) \big) \\
\label{eq:L-F-c}
\mathbf{(c) } \ \
F(m + n-1) & = & \frac{1}{2} \big( F(m) \cdot L(n) \ + \ F(n) \cdot L(m) \big) \\
\label{eq:L-F-d}
\mathbf{(d) } \ \
F(2n) & = & F(n) \cdot L(n).
\end{eqnarray}
\end{prop}

\begin{proof}
We consider the three identities in turn.

\noindent {\bf (a)}
We proceed by induction.

\medskip

\noindent
{\it The base case.}
Equation (\ref{eq:L-F-a}) holds when $n=1$ because
\[ L(1) = 3 = F(2) + F(0) = 2+1. \]

\medskip

\noindent
{\it The inductive hypothesis}.
Assume that equation (\ref{eq:L-F-a}) holds for $L(2), L(3), \ldots,
L(n)$.

\medskip

\noindent
{\it The inductive extension}. 
Let us compute $L(n+1)$:
\begin{itemize}
\item
By definition (\ref{eq:Lucas-defn-2}),
\begin{equation}
\label{eq:L-FF-1}
L(n+1) \ = \ L(n) \ + \ L(n-1).
\end{equation}
\item
When we apply the inductive hypothesis to both addends in
(\ref{eq:L-FF-1}), we obtain (after rearranging terms):
\begin{equation}
\label{eq:L-FF-2}
L(n+1) \ = \  F(n+1) \ + \ F(n) \ + \ F(n-1) \ + \ F(n-2)
\end{equation}
\item
Finally, we invoke the defining recurrence (\ref{eq:Fibonacci-defn})
of the Fibonacci numbers on the first two addends in (\ref{eq:L-FF-2})
and on the last two addends.  We thereby transform (\ref{eq:L-FF-2})
to equation (\ref{eq:L-F-a}), which validates the latter identity.
\end{itemize}

\bigskip

\noindent \fbox{
\begin{minipage}{0.95\textwidth}
Notice that a proof similar to the preceding one yields the identity
$L(n) = F(n+2) + F(n-2)$.  Similar, but more complicated, identities
hold for larger arguments.  For the cases $n+3$ and $n+4$, for
instance, one can establish the following pair of identities.
\begin{eqnarray}
\label{eq:LF:n+3}
L(n) & = & \frac{1}{2} (F(n+3)+F(n-3)) \\
\label{eq:LF:n+4}
L(n) & = & \frac{1}{3} (F(n+4)+F(n-4))
\end{eqnarray}
\end{minipage}
}
\bigskip

\noindent {\bf (b)}
By direct calculation, we derive the desired result:
\begin{eqnarray*}
2 F(n+1) & = & F(n+1) \ + \ F(n) \ + \ F(n-1) \\
         & = & L(n) \ + \ F(n).
\end{eqnarray*}

\bigskip

\noindent {\bf (c)}
This identity is verified via a somewhat complicated induction.  We
fix parameter $n$ in the argument $F(m+n)$ and induce on parameter $m$.

\medskip

\noindent
{\it The base case.}
Because $L(0) = F(0)= 1$ the instance $m=0$ of identity
(\ref{eq:L-F-c}) reduces to identity (\ref{eq:L-F-b}), which we have
just proved.  To wit,
\[ F(n+1) \ = \ \frac{1}{2} \big( L(n) \ + \ F(n) \big)
\ = \ \frac{1}{2} \big( F(0) \cdot L(n) \ + \ F(n) \cdot L(0) \big).
\]

\medskip 

\noindent
{\it The inductive hypothesis}.
Let us assume that identity (\ref{eq:L-F-c}) holds for all $m \leq k$.

{\Denis Again, we should check for every proof the indices, the basic expressions are on $n$, fine.
The induction is on $m$ or $k$ up to $n$, and then, we derive for $n+1$...}
\medskip

\noindent
{\it The inductive extension}.
Let us focus on instance $m = k+1$ of identity (\ref{eq:L-F-c}).  Note
first that the classical Fibonacci recurrence
(\ref{eq:Fibonacci-defn}) implies that
\[ F(n + k +1) \ = \ F(n + k) \ + \ F(n + k - 1). \]
When we apply the inductive hypothesis to both $F(n + k)$ and $F(n + k
- 1)$, we obtain the following two identities.
\begin{eqnarray*}
F(n + k) & = & \frac{1}{2} \big( F(k-1) \cdot L(n) \ + \ F(n) \cdot
L(k-1) \big) \\
F(n + k - 1) & = & \frac{1}{2} \big( F(k-2) \cdot L(n) \ + \ F(n) \cdot L(k-2) \big).
\end{eqnarray*}
Because both the Fibonacci and Lucas sequences obey the body of
recurrence (\ref{eq:Fibonacci-defn}), the preceding equations combine
to extend the induction.  To wit,
\begin{eqnarray*}
2 F(n + k +1) & = & 2 F(n + k) \ + \ 2 F(n + k - 1) \\
              & = & 
\big( F(k-1) \cdot L(n) \ + \ F(n) \cdot L(k-1) \big)
\ + \
\big( F(k-2) \cdot L(n) \ + \ F(n) \cdot L(k-2) \big) \\
              & = &
L(n) \cdot \big( F(k-1) \ + \ F(k-2) \big)
\ + \
F(n) \cdot \big( L(k-1) \ + \ L(k-2) \big) \\
              & = &
L(n) \cdot F(k) \ + \ F(n) \cdot L(k)
\end{eqnarray*}
The thus-extended induction verifies identity (\ref{eq:L-F-c}).

\bigskip

\noindent {\bf (d)}
Identity ((\ref{eq:L-F-d}) is actually the case $m=n$ of identity
((\ref{eq:L-F-c}).

This validates our final identity, which completes the proof.  \qed
\end{proof}

