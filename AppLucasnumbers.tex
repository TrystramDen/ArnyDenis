%version of 07-21-20

A continuing preoccupation of mathematicians is to understand why important mathematical structures exhibit their observed properties.  A common way to seek such understanding is to perturb the definition of a structure and study the effects of the perturbation.  While this stratagem leads to interesting, valuable results only sometimes, it is an invaluable tool in the hands of a gifted mathematician.  This section presents a brief survey of such a study by \'{E}douard Lucas.  \index{Lucas, \'{E}douard (Fran\c{c}ois \'{E}douard Anatole Lucas)} \index{Lucas numbers}

\subsection{Definition}

\index{Fibonacci numbers!origin of name}
Lucas, who is credited with giving the name ``Fibonacci numbers'' to the sequence discovered by Leonardo Pisano, investigated the consequences of perturbing the initial conditions, $F(0) = F(1) = 1$, in the classical definition (\ref{eq:Fibonacci-defn}) of the Fibonacci sequence.

\medskip

Lucas's approach was simply to replace the Fibonacci sequence's initial values $\langle 1,1 \rangle$, with the values $\langle 2,1 \rangle$.  It turns out to be much more fruitful---in terms of more striking results and simpler proofs---to make a somewhat more drastic perturbation:

\smallskip

\index{Lucas sequence!definition} \index{Lucas numbers!definition}

The {\it Lucas sequence} (or, {\it the sequence of Lucas numbers}) is the infinite sequence of positive integers
\[ L(-1), \ L(0), \ L(1), \ L(2), \ldots \]
which is generated by the recurrence
\begin{eqnarray}
\nonumber
L(-1) & = & 2 \\
\label{eq:Lucas-defn-1}
L(0) & = & 1 \\
\nonumber
L(n) & = & L(n-1) \ + \ L(n-2) \ \ \ \mbox{ for all } n \geq 1
\end{eqnarray}
Because we conventionally index sequences by {\em nonnegative} numbers, we henceforth ignore $L(-1)$ and use the following {\em standard definition} of the Lucas sequence.
\begin{eqnarray}
\nonumber
L(0) & = & 1 \\
\label{eq:Lucas-defn-2}
L(1) & = & 3 \\
\nonumber
L(n) & = & L(n-1) \ + \ L(n-2) \ \ \ \mbox{ for all } n > 1
\end{eqnarray}

\medskip

The following finite sequences present the first few elements of both the Lucas sequence (for illustration) and the Fibonacci sequence (for comparison).
\[
\begin{array}{r|rrrrrrrrrrr}
n &
 0, & 1, & 2, & 3, &  4, &  5, &  6, &  7, &  8, &   9, & \ldots \\
\hline
L(n) &
 1, & 3, & 4, & 7, & 11, & 18, & 29, & 47, & 76, & 123, & \ldots \\
F(n) &
 1, & 1, & 2, & 3, &  5, &  8, & 13, & 21, & 34, &  55, & \ldots \\
\hline
\end{array}
\]

We begin our brief study of the Lucas sequence by noting that just a minor tweak converts the Fibonacci-related identity revealed in Proposition~\ref{thm:FiboSum-1} to an identity for Lucas numbers, as originally defined---beginning with $L(-1)$.

\begin{prop}
\label{thm:LucasSum-1}
For all integers $n \geq 0$,
\begin{equation}
\label{eq:multilinear-Lucas-1}
L(n+2) \ = \
1 \ + \ L(-1) \ + \ L(0) \ + \ L(1) \ + \ L(2) \ + \cdots + \ L(n)
\end{equation}
\end{prop}

\begin{proof}[Sketch]
We can literally repeat the proof of Proposition~\ref{thm:FiboSum-1}, with only a change in the induction's base case, which now becomes
\[ L(2) \ = \ L(-1) + L(0) + 1 \ = \ 2 + 1 + 1 \ = \ 4. \]
The body of the inductive argument holds for the Lucas sequence as well as for the Fibonacci sequence.  \qed
\end{proof}


\subsection{Relating the Lucas and Fibonacci Numbers}

\index{Hungerford, Margaret Wolfe}

There are several simple equations that relate the Lucas and Fibonacci numbers.  We present a few of the most aesthetically pleasing ones,\footnote{Aesthetically pleasing, that is, to the authors.  As noted by the author Margaret Wolfe Hungerford in {\it Molly Bawn} (1878), ``Beauty is in the eye of the beholder.''}~in terms of their exposing an intimate relationship between the two sequences.
\index{Fibonacci numbers!relations with Lucas numbers}
\index{Lucas numbers!relations with Fibonacci numbers}

\begin{prop}
\label{thm:Lucas-n:2Fibs}
For all $m, n \geq 1$
\begin{eqnarray}
\label{eq:L-F-a}
{\bf (a)} \  \hspace*{.49in}
L(n) & = & F(n+1) + F(n-1) \\
\label{eq:L-F-b}
{\bf (b)} \  \hspace*{.25in}
F(n+1) & = & \frac{1}{2} \big(F(n) \ + \ L(n) \big) \\
\label{eq:L-F-c}
{\bf (c)} \  
F(m + n-1) & = & \frac{1}{2} \big( F(m) \cdot L(n) \ + \ F(n) \cdot L(m) \big) \\
\label{eq:L-F-d}
{\bf (d)} \ \hspace*{.37in}
F(2n) & = & F(n) \cdot L(n)
\end{eqnarray}
\end{prop}

\begin{proof}
We consider the identities in turn.

\noindent {\bf (a)}
We proceed by induction.

\medskip

\noindent
{\sf Base case.}
Equation (\ref{eq:L-F-a}) holds when $n=1$ because
\[ L(1) = 3 = F(2) + F(0) = 2+1 \]

\medskip

\noindent
{\sf Inductive hypothesis}.
Assume that equation (\ref{eq:L-F-a}) holds for $L(2), L(3), \ldots, L(n)$.

\medskip

\noindent
{\sf Inductive extension}. 
Let us compute $L(n+1)$:
\begin{itemize}
\item
By definition (\ref{eq:Lucas-defn-2}),
\begin{equation}
\label{eq:L-FF-1}
L(n+1) \ = \ L(n) \ + \ L(n-1)
\end{equation}
\item
When we apply the inductive hypothesis to both addends in (\ref{eq:L-FF-1}), we obtain (after rearranging terms):
\begin{equation}
\label{eq:L-FF-2}
L(n+1) \ = \  F(n+1) \ + \ F(n) \ + \ F(n-1) \ + \ F(n-2)
\end{equation}
\item
Finally, we invoke the defining recurrence (\ref{eq:Fibonacci-defn}) of the Fibonacci numbers on the first two addends in (\ref{eq:L-FF-2}) and on the last two addends.  We thereby transform (\ref{eq:L-FF-2}) to equation (\ref{eq:L-F-a}), which validates the latter identity.
\end{itemize}

\bigskip

\noindent \fbox{
\begin{minipage}{0.95\textwidth}
{\bf Explanatory note}.

\smallskip

Notice that a proof similar to the preceding one yields the identity $L(n) = F(n+2) + F(n-2)$.  Similar, but more complicated, identities hold for larger arguments.  For the cases $n+3$ and $n+4$, for instance, one can establish the following pair of identities.
\begin{eqnarray}
\label{eq:LF:n+3}
L(n) & = & \frac{1}{2} (F(n+3)+F(n-3)) \\
\label{eq:LF:n+4}
L(n) & = & \frac{1}{3} (F(n+4)+F(n-4))
\end{eqnarray}
\end{minipage}
}
\bigskip

\noindent {\bf (b)}
By direct calculation, we derive the desired result:
\[ 2 F(n+1) \ = \ F(n+1) \ + \ F(n) \ + \ F(n-1) \ = \ L(n) \ + \ F(n)  \]

\bigskip

\noindent {\bf (c)}
This identity is verified via a somewhat complicated induction.  We fix parameter $n$ in the argument $F(m+n)$ and induce on parameter $m$.

\medskip

\noindent
{\sf Base case.}
Because $L(0) = F(0)= 1$, the instance $m=0$ of identity (\ref{eq:L-F-c}) reduces to identity (\ref{eq:L-F-b}), which we have just proved.  To wit,
\[ F(n+1) \ = \ \frac{1}{2} \big( L(n) \ + \ F(n) \big)
\ = \ \frac{1}{2} \big( F(0) \cdot L(n) \ + \ F(n) \cdot L(0) \big)
\]

\medskip 

\noindent
{\sf Inductive hypothesis}.
Let us assume that identity (\ref{eq:L-F-c}) holds for all $m \leq k$.

\medskip

\noindent
{\sf Inductive extension}.
Let us focus on instance $m = k+1$ of identity (\ref{eq:L-F-c}).  Note first that the classical Fibonacci recurrence (\ref{eq:Fibonacci-defn}) implies that
\[ F(n + k +1) \ = \ F(n + k) \ + \ F(n + k - 1). \]
When we apply the inductive hypothesis to both $F(n + k)$ and $F(n + k - 1)$, we obtain the following two identities.
\begin{eqnarray*}
F(n + k) & = & \frac{1}{2} \big( F(k-1) \cdot L(n) \ + \ F(n) \cdot L(k-1) \big) \\
F(n + k - 1) & = & \frac{1}{2} \big( F(k-2) \cdot L(n) \ + \ F(n) \cdot L(k-2) \big)
\end{eqnarray*}
Because both the Fibonacci and Lucas sequences obey the body of recurrence (\ref{eq:Fibonacci-defn}), the preceding equations combine to extend the induction.  To wit,
\begin{eqnarray*}
2 F(n + k +1) & = & 2 F(n + k) \ + \ 2 F(n + k - 1) \\
              & = & 
\big( F(k-1) \cdot L(n) \ + \ F(n) \cdot L(k-1) \big)
\ + \
\big( F(k-2) \cdot L(n) \ + \ F(n) \cdot L(k-2) \big) \\
              & = &
L(n) \cdot \big( F(k-1) \ + \ F(k-2) \big)
\ + \
F(n) \cdot \big( L(k-1) \ + \ L(k-2) \big) \\
              & = &
L(n) \cdot F(k) \ + \ F(n) \cdot L(k)
\end{eqnarray*}
The thus-extended induction verifies identity (\ref{eq:L-F-c}).

\bigskip

\noindent {\bf (d)}
Identity (\ref{eq:L-F-d}) is actually the case $m=n$ of identity (\ref{eq:L-F-c}).

\smallskip

This validates our final identity, which completes the proof.  \qed
\end{proof}


\subsection{Lucas Numbers} 

%\noindent \textit{The aim.}
%Fibonacci progression is the most popular and the most simple Definition of Lucas' numbers
%
\textbf{Definition.}
Given the two starting numbers $L(0) = 1$ and $L(1) = 3$, the Lucas numbers are obtained by the same progression as the Fibonacci numbers: 
\[ L(n+1) = L(n)+L(n-1) \]

\medskip

In order to gain intuition on the relationship between the Fibonacci and Lucas sequences, let us compare the first elements of the two sequences.
\begin{figure}[htb]
\[
\begin{array}{c||r|r|r|r|r|r|r|r|r|r|r}
{\displaystyle n } & k=0 & k=1 & k=2 & k=3 & k=4 & k=5 &
k=6 & k=7 & k=8 & k=9 & \ldots \\
\hline
F(n) & 1 & 1 &  2  &  3  &   5  &   8  &  13  &  21  & 34  & 55  & \ldots \\
\hline
L(n) & 1 & 3 &  4 &  7  &  11  &  18  &  29 & 47  & 76  & 123 & \ldots \\
\hline
\end{array}
\] 
\caption{Comparing Fibonacci and Lucas numbers.}
\label{fig:fiboLucas}
\end{figure}


\medskip

Based on the patterns we observe the following.

\begin{prop}
For all positive integers $n$, and all $k \leq n$:
\[ F(k-1) \cdot L(n) \ = \ F(n+k) \ + \ (-1)^{k-1} F(n-k) \]
\end{prop}

\begin{proof}[Sketch]
Let us use induction on $k$ to study the asserted products for fixed $n$.
\begin{itemize}
\item
{\sf Base.} 
The case, $k=1$, follows by the defining recurrence of the Lucas numbers.

\item
{\sf Extension.}
Using an approach similar to the base case, we can observe that
\[ L(n) = F(n+2)-F(n-2) \] 
Let us see why.

\smallskip

Easy calculations yield the following equations:
\begin{eqnarray*}
2 L(n) & = & F(n+3) + F(n-3) \\
3 L(n) & = & F(n+4) - F(n-4) \\
5 L(n) & = & F(n+5) + F(n-5)
\end{eqnarray*}
 
Observing that $2$, $3$, and $5$ are consecutive Fibonacci numbers, we obtain the intuition for the general case,
\[  F(k-1) \cdot L(n) \ = \ F(n+k) \ + \ (-1)^{k-1}F(n-k) \ \ \ \mbox{ for  } k \leq n \]
which is proved (again) by induction on $k$.

\smallskip

The base case is straightforward (see case $k=1$).

\smallskip

Compute $F((k+1)-1) \cdot L(n)$ and apply the definition of Fibonacci number:
\[ F((k+1)-1) \ = \ F(k-1) \ + \ F(k-2) \]
If we replace the two last terms by using the induction hypothesis, we observe the following.
\begin{eqnarray*}
F(k) \cdot L(n) & = & F(k-1) \cdot L(n \ + \ F(k-2) \cdot L(n) \\
                        & = & F(n+k) \ + \ (-1)^{k-1} F(n-k) \ + \ F(n+k-1) \\
                        &    & \ \ \ \ \ \ + \ (-1)^{k-2} F(n-(k-1)) \\
                        & = & F(n+k) \ + \ F(n+k-1) \ + \ (-1)^{k-1} (F(n-k) \ -  F(n-k+1))
\end{eqnarray*}
The sought result is finally obtained by invoking the definition of the Fibonacci numbers twice. \qed
\end{itemize}
\end{proof}
