%version of 10-04-19

\chapter*{List of Symbols}

\section*{Symbols Related to Sets and Their Algebras}

\begin{tabular}{c|cl}
{\bf Symbol} &  & {\bf Meaning} / {\bf Usage} \\ \hline
$\{ \circ, \circ, \circ \}$ 
  & & grouping symbols for listed elements of an {\em unordered} set \\
  & & $\{a,b\} \ = \ \{b,a\}$ \\ \hline
$\{ \circ \ | \ \circ \}$
  & & the symbol ``$|$" introduces a specification clause \\
  & & $\{x \ | \ P(x) \}$: the set of $x$ {\em such that} proposition $P(x)$ holds
   \\ \hline
$\langle \circ, \circ, \circ \rangle$
  & & grouping symbols for listed elements of an {\em ordered} set \\
  & & $\langle a,b\rangle \ \neq \ \langle b,a\rangle$ \\ \hline
$\in$
  & & set membership \\
  & & $s \in S$: $s$ belongs to, or is a member of $S$ \\ \hline
$\cap$
  & & set intersection \\
  &  & $S \cap T$: elements common to $S$ and $T$ \\ \hline
$\cup$
  &  & set union \\
  &  & $S \cup T$: elements belonging to at least one of $S$ and $T$ \\ \hline
$\setminus$
  & & set difference \\ 
  & & $S \setminus T$: elements belonging to $S$ but not to $T$ \\
$-$
  & & set difference \\ 
  & & $S - T$: elements belonging to $S$ but not to $T$ \\ \hline
$\times$
  & & direct product, or, Cartesian product \\
  & & $S \times T$: ordered pairs $\langle s,t \rangle$ where $s \in S$ and $t \in T$ \\ \hline
$|S|$
  & & set cardinality \\
  & & $|S|$: the number of elements of set $S$ \\ \hline
$\emptyset$
  & & the empty, or, null, set \\
  & & $\emptyset$ is a subset of every set \\ \hline
$\p(S)$
  & & power set \\
  & & $\p(S)$: the set of all subsets of set $S$ \\ \hline
\end{tabular}

\section*{Symbols Related to Logic}

\begin{tabular}{c|cl}
{\bf Symbol} & & {\bf Meaning} / {\bf Usage} \\ \hline
\hline
$\wedge$
  & & conjunction, or logical product \\
  & & $x \wedge y$: this assertion is true iff both $x$ and $y$ are true \\
{\sc and}
  & & conjunction, or, logical product \\
  & & $x \ \mbox{\sc and } y$: this assertion is true iff both $x$ and $y$ are true \\ \hline
$\vee$
  & & disjunction, or, logical sum \\
  & & $x \vee y$:  this assertion is true iff at least one of $x$ and $y$ is true \\
{\sc or}
  & & disjunction, or, logical sum \\
  & & $x \ \mbox{\sc or } y$:  this assertion is true iff at least one of $x$ and $y$ is true  \\ \hline
$\neg$
  & & negation \\
  & & $\neg x$: the negation of logical variable $x$; true iff $x$ is false \\ 
{\sc not}
  & & negation \\
  & & {\sc not}  $x$: the negation of logical variable $x$; true iff $x$ is false \\ \hline
\end{tabular}

\section*{Symbols Related to Numbers and Arithmetic}

\begin{tabular}{c|cl}
{\bf Symbol} & & {\bf Meaning} / {\bf Usage} \\ \hline
\hline
$\leq$
  & & less than or equal to  \\
  & & $x \leq y$: the number $x$ is no larger (or, greater) than the number $y$ \\ \hline
$<$
  & & strictly less than  \\
  & & $x < y$: the number $x$ is smaller than the number $y$ \\ \hline
$\geq$
  & & greater than or equal to  \\
  & & $x \geq y$: the number $x$ is no smaller than the number $y$ \\ \hline
$>$
  & & strictly greater than  \\
  & & $x > y$: the number $x$ is greater (or, larger) than the number $y$ \\ \hline
$+$
  & & unary plus sign \\
  & & $+x$: the nonnegative version of number $x$ \\ \hline
$+$
  & & binary addition sign \\
  & & $x+y$: the sum of numbers $x$ and $y$ \\ \hline
$-$
  & & unary minus sign \\
  & & $-x$: the negative version of number $x$ \\ \hline
$-$
  & & binary subtraction sign \\
  & & $x-y$: the difference of numbers $x$ and $y$ \\ \hline
$\cdot$
  & & multiplication sign \\
  & & $x \cdot y$: the product of numbers $x$ and $y$ \\
$\times$
  & & multiplication sign \\
  & & $x \times y$: the product of numbers $x$ and $y$ \\ \hline
$/$
  & & division sign \\
  & & $x / y$: the ratio of number $y$ to number $x$ \\
  & & in display mode, we write $\displaystyle \frac{x}{y}$ \\
$\div$
  & & division sign \\
  & & $x \div y$: the ratio of number $y$ to number $x$ \\ \hline
$|x|$
  & & magnitude, or, absolute value \\
  & & $|x| =$ {\bf if} $x<0$ {\bf then} $-x$ {\bf else} $x$ \\ \hline
$\displaystyle \sum_{i=a}^b$
  & & summation between limits \\
  & & $\sum_{i=a}^b \ c_i$: the summation $c_a + c_{a+1} + \cdots + c_b$ \\
  & & in display mode, we write $\displaystyle \sum_{i=a}^b \ c_i$  \\ \hline
$\sqrt{}$
  & & radical (square root) sign \\
  & & $\sqrt{x}$: number whose square is $x$: $\sqrt{x} \times \sqrt{x} = x$ \\ \hline
$!$
  & & factorial sign \\
  & & $n! = n \times (n-1) \times \cdot \times 2 \times 1$ \\ \hline
$\displaystyle {n \choose k}$ 
  & & the binomial coefficient choose-$k$-out-of-$n$ \\
  & & $\displaystyle {n \choose k} = \frac{n!}{k! (n-k)!}$ \\ \hline
$\lfloor \circ \rfloor$
  & & the floor, or, integer part \\
  & & $\lfloor n/m \rfloor$: the largest integer $k$ such that $m \cdot k \leq n$ \\ \hline
$\lceil \circ \rceil$
  & & the ceiling \\
  & & $\lceil n/m \rceil$: the smallest integer $k$ such that $m \cdot k \geq n$ \\ \hline
$\bmod$
  & & modulo sign\\
  & & $n \bmod m$: the unique integer $r < m$ such that $n =  \lfloor n/m \rfloor \cdot m + r$. \\ \hline
$\infty$
  & & the {\it leminscate curve} used to represent (the point at) infinity \\
  & & $1/n$ tends to $0$ as $n$ tends to $\infty$ \\
  & & Symbolically: $1/n \rightarrow 0$ as $n \rightarrow \infty$ \\ \hline
\end{tabular}

\section*{Symbols Related to Combinatorics and Probability}

\begin{tabular}{c|cl}
{\bf Symbol} & & {\bf Meaning} / {\bf Usage} \\ \hline
\hline
$Pr[E]$
  & & the probability of event $E$ \\ \hline
$Pr[X=x]$
  & & the probability that random variable $X$ assumes value $x$ \\ \hline
$Pr[E \ | \ F]$
  &  & the conditional probability of event $E$ {\em given that} event $F$ \\
  &  & has occurred \\ \hline
$Pr[X=x \ | \ Y=y]$
  & & the probability that random variable $X$ assumes value $x$ \\
  &. & {\em given that} random variable $Y$ assumes value $y$ \\ \hline
\end{tabular}


\section*{Symbols Related to Graphs}
