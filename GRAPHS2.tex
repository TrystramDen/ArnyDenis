%version of 04-02-20

\chapter{Graphs II:
Graphs for Computing and Communicating}
\label{ch:Graphs2}
\index{graph}

\begin{quote}
``Many hands make light work" \\ 
\hspace*{1in}John Heywood, 15th century English writer
\end{quote}

\bigskip

Chapter~\ref{ch:Graphs1} has laid the groundwork for an intensive study of the many  ways in which graphs and graph-related concepts find application within the fields of computing and communicating.  That chapter focused mainly on properties of graphs---and families of graphs---that are local and descriptive: What is the essence of a graph's being a tree? a hypercube?  The current chapter takes a further step and studies how the described local structure influences some of the dynamically applicable characteristics of graphs.  Two major foci in this chapter are {\em vertex-coloring} (Section~\ref{sec:graph-color}) and {\em path discovery} (Section~\ref{sec:path-cycle-problems}).  Both of these topics are used algorithmically in myriad applications of graphs, from circuit layout to computation scheduling to inter-agent communication scheduling.  We provide a layered presentation to both topics: We completely cover the basic aspects of both topics; we offer extended material in appendices, providing further study opportunities for the interested reader; and we provide pointers to the literature for a number of fascinating more-advanced topics.  We close the chapter with capsule discussions of several specialized topics which go beyond the scope of any introductory text.  We hope that the reader will be excited by these preliminary excursions into some of the wats that graph theory plays a role in ``real life".


%%%%%%%%%%%%%%%%%%%%%%%%%%%%%

\section{Graph Coloring and Chromatic number}
\label{sec:graph-color}
\index{graph!vertex-coloring}

\index{graph!chromatic number}

%This section introduces the notion of {\it graph coloring}, and its associated notion of the {\it chromatic number} of a graph.

\index{graph!vertex-coloring}
\index{graph!chromatic number}
\index{graph!$c$-colorable} 
 
A {\it vertex-coloring} of a graph $\g$ is an assignment of labels (the ``{\em colors}") to $\g$'s vertices,  in such a way that all of a vertex $v$'s neighbors get different labels than $v$'s.  The {\it chromatic number} of a graph $\g$ is the smallest number of colors that one can use in crafting a legal vertex-coloring of $\g$.  In traditional parlance, the assertions

\smallskip

``$\g$ has chromatic number $c$'' \ \ \ and \ \ \  ``$\g$ is {\it $c$-colorable}''

\smallskip

\noindent
are viewed as synonymous.

\bigskip

\noindent \fbox{ \begin{minipage}{0.96\textwidth}
{\bf Enrichment note}.

The notion of graph coloring can be used to computational advantage in a broad variety of situations.

\medskip

One extremely important, and illustrative, use of graph coloring is to model independence among agents in {\it distributed} computing: The vertices of a computation-graph $\g$ represent {\it agents}, such as, e.g., processing elements in a multicomputer; and $g$'s edges represent {\it communication links} that enable each vertex $u$ to check its neighbors' states before taking any action and to inform its neighbors of state-changes occasioned by an action by $u$.  

\smallskip

The prohibition against ``monochrome'' edges---i.e., edges both of whose incident vertices have the same color---guarantees that vertex $u$ and all of its like-colored vertices can act at the same instant with no fear of missing an important input to those actions.  Indeed, one often encounters programs for distributed computing that look something like

\smallskip

1. All {\em red} vertices perform simultaneously an action

2. All {\em green} vertices perform simultaneously an action

3. All {\em blue} vertices perform simultaneously an action

$\cdots$

\medskip

In a quite different setting, graph coloring can be used to represent either compatibility or incompatibility between pairs of graph vertices.  As but example: The graph-theoretic formulation in Section ~\ref{sec:graph-model-2SAT} of the 2SAT problem employs adjacencies in a graph to signal incompatible literals in a POS expression.
\end{minipage}
}

\bigskip

Most of the ``named'' graphs in Section~\ref{sec:graphs-important-families} have quite small chromatic numbers.  This is no accident: These graphs were invented (or, at least, placed in the spotlight) because of their importance to the arena of parallel and distributed computing.  As such, vertex-colorings of these graphs are an important tool for identifying the dependencies, or the lack thereof, as one schedules the concurrent execution of the graphs' task-vertices.

\smallskip

Motivated by such applications of graph models, we devote the current section to studying graphs that have small chromatic numbers.

\subsection{Graphs with Chromatic Number $2$}
\label{sec:2-color-graphs}

We begin our study of graphs having small chromatic numbers by focusing on {\em $2$-colorable} graphs, which enjoy the smallest nontrivial chromatic number.  It is not difficult to characterize these graphs structurally.

\medskip

 \index{graph!leveled}
 
A graph $\g$ is {\it leveled} if there exists an assignment of {\it level-numbers} $\{ 1, 2, \ldots, \lambda\}$ to the vertices of $\g$ in such a way that every neighbor of a vertex having level-number $\ell$ has either level-number $\ell +1$ or level-number $\ell -1$.  A vertex that has level-number $\ell$ is said to {\it reside on level $\ell$} of $\g$.

\begin{prop}
\label{thm:leveled=2-color}
A graph $\g$ has chromatic number $2$ if, and only if, it is leveled.
\end{prop}

\begin{proof}
Say first that $\g$ is a leveled graph.  Then labeling each vertex of $\g$ with the (odd-even) parity of its level provides a valid $2$-coloring of $\g$.

\smallskip

Say next that $\g$ is $2$-colorable.  Pick any vertex $v$ of $\g$ and assign it to be the unique vertex on level $1$.  Let all neighbors of $v$ be assigned to level $2$.  Continuing iteratively, say that the largest level-number that we have employed---i.e., assigned vertices to---is $\ell$.  Then we now assign level-number $\ell +1$ all neighbors of level-$\ell$ vertices that have not yet been assigned to a level of $\g$.  Because $\g$ is $2$-colorable, each of the levels we have specified is monochromatic, so that each edge of $\g$ connects a vertex of one color with a vertex of the other color.  \qed
\end{proof}

We can now show that the following named graphs are $2$-colorable.

\begin{corol}
\label{thm:list-2-colorables}
The following graphs are leveled, hence have chromatic number $2$:

\smallskip

{\bf (a)}
every tree (which includes any path-graph $\p_n$)

\smallskip

{\bf (b)}
every cycle-graph $\cc_n$ that has an even number $n$ of vertices

\smallskip

{\bf (c)}
every mesh-graph $\m_{m,n}$

\smallskip

{\bf (d)}
every torus-graph $\widetilde{\m}_{m,n}$ with even $m+n$, i.e., evenly many diagonals

\smallskip

{\bf (e)}
every hypercube $\q_n$
\end{corol}

\begin{proof}
We provide a detailed sketch for each of the five graph families in turn.

\smallskip

\noindent {\bf (a)}
The procedure from the second half of the proof of Proposition~\ref{thm:leveled=2-color} exposes a level structure in any tree $\t$, as follows.  Pick any vertex $v$ of $\t$ and make it the
unique vertex on level $1$.  Let all neighbors of $v$ be assigned to level $2$.  Continuing iteratively, say that the largest level-number that we have employed is $\ell$.  Then we now assign level-number $\ell +1$ to all neighbors of level-$\ell$ vertices that have not yet been
assigned to a level of $\t$.

\smallskip

Of course this process can be simplified when $\t$ is a path-graph $\p$, by choosing one of $\p$'s end-vertices as vertex $v$.  We thereby have precisely one vertex on each level.  (Starting with an internal vertex would create some levels with $2$ vertices.)  A lesson here is that {\em a graph may admit distinct level structures}.

\medskip

\noindent {\bf (b)}
When we apply the procedure of part (a) to an even-length cycle, $\cc_{2q}$, we produce a level structure in which levels $1$ and $q+1$ have one vertex apiece, while all other levels have two vertices apiece.

\medskip

\noindent {\bf (c)}
The edge-structure of mesh-graphs ensures that the labeling of each vertex $\langle i,j \rangle$ of $\m_{m,n}$ with the odd-even parity of the number $i+j$ is a $2$-coloring of $\m_{m,n}$.

\medskip

\noindent {\bf (d)}
The labeling in part (d) provides a $2$-coloring of any torus-graph $\widetilde{\m}_{m,n}$ with even $m+n$.

\medskip

\noindent {\bf (e)}
Each edge of a hypercube $\q_n$ connects a vertex $v = \beta_1 \beta_2 \cdots \beta_n$, where each $\beta_i \in \{0,1\}$, to a vertex $v' = \beta'_1 \beta'_2 \cdots \beta'_n$ where  $\beta_j \neq \beta'_j$ for precisely one $j$.  Therefore, the following aggregation of vertices of $\q_n$ into sets $S_0, S_1, \ldots, S_n$ provides a valid leveling of $\q_n$.

\smallskip

Assign vertex $v = \beta_1 \beta_2 \cdots \beta_n$ to set $S_k$ precisely if $k$ of the bits $\beta_i$ equal $1$.

\medskip

\noindent
The preceding explanations complete the proof. \qed
\end{proof}

A simple argument verifies that no odd-length cycle $\cc_{2q+1}$ with $q \geq 1$ is $2$-colorable.  One can craft such an argument by trying to $2$-color such a cycle.  One can then extend this argument to prove that no graph $\g$ that {\em contains} an odd-length cycle (as a subgraph) can be $2$-colored.

{\em This is an example of an {\em excluded subgraph} condition:}  Containing odd-length cycles is the feature that prevents $2$-colorings of many graphs, including the torus $\widetilde{\m}_{m,n}$ when $m+n$ is odd and all de Bruijn networks.

\index{excluded subgraph}

\smallskip

Let us focus momentarily on de Bruijn networks because they have quite interesting cyclic sub-digraphs.  One observes the directed $3$-cycle
\[ 00 \ \rightarrow \ 01 \ \rightarrow \ 10  \rightarrow \ 00 \]
in $\d_2$ in Fig.~\ref{fig:dB2by2} and the $3$-cycle
\[ 001 \ \rightarrow \ 010 \ \rightarrow \ 100 \ \rightarrow \ 001 \]
in $\d_3$ in Fig.~\ref{fig:dB2by3}.  In fact, these small odd-length cycles are only the proverbial tip of the proverbial iceberg for de Bruijn networks.  In fact, de Bruijn networks are {\it directed-pancyclic}, in the sense of the following result from \cite{Yoeli62}.

\index{graph!pancyclic} \index{de Bruijn network!pancyclicity} 
\index{de Bruijn graph!pancyclicity}

\begin{prop}
\label{thm:DB-pancyclic}
For all $n$, the order-$n$ de Bruijn network $\d_n$ is directed-pancylic, i.e., it contains directed cycles of all possible lengths $1, 2, \ldots, 2^n$.
\end{prop}

The proof of this result is beyond the scope of an introductory text, but its treatment in \cite{Yoeli62} should be accessible to the motivated reader.

\subsection{Planar and Outerplanar Graphs}
\label{sec:planar+outerplanar-color}
\index{planar graph} \index{outerplanar graph}
\index{graph!planar} \index{graph!outerplanar}

In this section, we focus on two graph families that are defined in terms of the way they can be drawn (on a two-dimensional medium, such as a piece of paper).

\bigskip

\index{VLSI}  \index{Very Large Scale Integrated Circuit technology}
\index{VLSI:Very Large Scale Integrated Circuit technology} 

\noindent \fbox{
\begin{minipage}{0.96\textwidth}
{\bf Enrichment note}.

Paying attention to how a graph can be drawn is not just an abstract game.  The  process of designing and implementing circuits within the constraints of {\it VLSI}, {\it Very Large Scale Integrated Circuit} technology, is very similar to drawing a circuit on a two-dimensional medium. 

\smallskip

We refer the reader to the revolutionary 1979 text by Mead and Conway \cite{Mead-Conway} for an introduction to this fascinating technology; the text requires technical literacy but little specialized knowledge.
\end{minipage}
}

\bigskip

\index{planar graph}
\index{graph!planar}
\index{graph!outerplanar}
\index{outerplanar graph}

\noindent
A graph is {\it planar} precisely if it can be drawn {\em without any crossing edges}.  A graph $\g$ is {\it outerplanar} precisely if it can be drawn by {\em placing its vertices along a circle in such a way that its edges can be drawn as non-crossing chords of the circle}.  The latter condition is
equivalent to demanding that $\g$'s edges can be drawn within the circle without any crossings.

\smallskip

We urge the reader to garner intuition about the graphs in these families by experimenting with drawing some specific, rather complex graphs.
\begin{itemize}
\item
The first set of graphs to draw are cliques, as defined in Section~\ref{sec:clique}.  The cliques $\k_3$, $\k_4$, and $\k_5$ will help expose the nature of the planar and outerplanar graphs, because:
  \begin{itemize}
  \item
$\k_3$ is outerplanar; 
  \item
$\k_4$ is planar but not outerplanar;
  \item
$\k_5$ is not planar.
  \end{itemize}

\item
The {\em bipartite} cousins of the cliques also provide valuable insights.  For positive integers $m$ and $n$, the $m \times n$ bipartite clique $\k_{m,n}$ is the graph whose vertex-set comprises the ordered pairs of integers:
\[  \n_{\fk_{m,n}} \ = \
\{ \langle i,j \rangle \ \ | \ \ 1 \leq i \leq m; \ \ 1 \leq j \leq n\}
\]
and whose edges connect each vertex $\langle i,j \rangle$ to every vertex $\langle i,k \rangle$ where $1 \leq k \leq n$ and to every vertex $\langle h,j \rangle$ where $1 \leq h \leq m$.

\index{bipartite clique}

\smallskip

The second set of graphs to draw are the bipartite cliques $\k_{1,3}$, $\k_{2,3}$, and $\k_{3,3}$.  These graphs will also help expose the nature of the planar and outerplanar graphs, because:
  \begin{itemize}
  \item
$\k_{1,3}$ is outerplanar; 
  \item
$\k_{2,3}$ is planar but not outerplanar;
  \item
$\k_{3,3}$ is not planar.
  \end{itemize}
\end{itemize}

\smallskip

\index{forbidden subgraph} \index{forbidden subgraph!characterization of planar graphs}
\index{forbidden subgraph!characterization of outerplanar graphs}
\index{graph!homeomorphism} \index{graph!homeomorph}

We selected the preceding cliques and bipartite cliques to ``play with'' very carefully.  Using arguments that go beyond the scope of an introductory text, one can prove the following result, which characterizes each of our graph families by identifying {\it forbidden subgraphs}.   The notion of {\it graph homeomorphism} plays a fundamental role in the characterization.

\smallskip

\begin{tabular}{l}
{\it Homeomorphism} is a daunting technical term which is easily understood \\
informally.  A {\it homeomorph} of a graph $\g$ is obtained by adding (degree-$2$) \\
vertices along one or more edges of $\g$.
\end{tabular}

\smallskip

\index{Kuratowski, Kazimierz}

\noindent
The characterization of planar graphs via forbidden subgraphs constitutes a celebrated theorem by the Polish mathematician and logician Kazimierz Kuratowski; the analogous result for outerplanar graphs was derived by the French mathematician Gary Chartrand and the American mathematician Frank Harary (who invented the name ``outerplanar", for reasons described in Section~\ref{sec:planar-graphs}).

\index{Chartrand, Gary} \index{Harary, Frank}
 
\begin{theorem}
\label{thm:planar+outerplanar-exclusion}
{\bf (a)} {\rm \cite{ChartrandB67}}
A graph is outerplanar if, and only if, it does not have a subgraph that is homeomorphic to either $\k_4$ or $\k_{2,3}$.

\smallskip

\noindent
{\bf (b)} {\rm \cite{Kuratowski30}}
A graph is planar if, and only if, it does not have a subgraph that is homeomorphic to either $\k_5$ or $\k_{3,3}$.
\end{theorem}


\subsubsection{On $3$-coloring {\em outerplanar} graphs}

\index{Kuratowski, Kazimierz}

We look first at the smaller of this section's graph families, namely, the {\it outerplanar graphs}. 
We begin our journey toward $3$-colorings of these graphs with some basic facts about the family.

\begin{prop}
Every tree is outerplanar.
\end{prop}

We leave the challenge of providing the formal proof of this property to the reader.   To aid in finding such drawings, we graphically describe in Figs.~\ref{fig:treeoutplanar1}--\ref{fig:treeoutplanar3} the process of developing an outerplanar drawing of a directed tree.  The first two figures begin to distribute the vertices of $\t$ around a circle in a way that allows one to draw $\t$'s edges in a noncrossing manner.
\begin{figure}[hbt]
\begin{center}
       \includegraphics[scale=0.45]{FiguresGraph/TreeOutplanar1}
       \caption{Beginning an outerplanar drawing of a rooted tree $\t$: Placing $\t$'s root on a circle.}
  \label{fig:treeoutplanar1}
\end{center}
\end{figure}
\begin{figure}[hbt]
\begin{center}
       \includegraphics[scale=0.45]{FiguresGraph/TreeOutplanar2}
       \caption{Continuing an outerplanar drawing of $\t$: Placing $\t$'s root's children around the circle}
  \label{fig:treeoutplanar2}
\end{center}
\end{figure}
The third figure depicts an entire outerplanar drawing of $\t$.
\begin{figure}[hbt]
\begin{center}
       \includegraphics[scale=0.45]{FiguresGraph/TreeOutplanar3}
       \caption{A complete outerplanar drawing of $\t$.}
  \label{fig:treeoutplanar3}
\end{center}
\end{figure}

\begin{prop}
\label{thm:basic-outerplanar-stuff}
Let $\g$ be an outerplanar graph.  Then:

\smallskip

\noindent {\bf (a)}
$\g$ is planar.


\smallskip

\noindent {\bf (b)}
Every subgraph of $\g$ is outerplanar.

\smallskip

\noindent {\bf (c)}
At least one of $\g$'s vertices has degree $\leq 2$.
\end{prop}

\begin{proof}
{\bf (a)}
$\g$'s planarity can be inferred from our ability to draw $\g$'s edges as noncrossing chords of the circle.

\medskip

\noindent {\bf (b)} 
We can produce a drawing of any subgraph  $\g'$ of $\g$ that witnesses $\g'$'s outerplanarity by erasing some vertices and/or some edges from our outerplanarity-witnessing drawing of $\g$.  These erasures cannot introduce any edge-crossings.

\medskip

\noindent {\bf (c)}
One verifies easily that part (c) holds for all outerplanar graphs having $\leq 3$ vertices.  Focus, therefore, on an arbitrary outerplanar graph $\g$ that has $> 3$ vertices.  Since adding more edges to a graph cannot decrease the degree of any vertex, we lose no generality by focusing on a graph $\g$ that is {\em maximally} outerplanar, in the sense that adding any new edge to $\g$ would destroy our ability to draw $\g$'s edges in a noncrossing manner.
\index{maximal outerplanar graph}

\smallskip

Because $\g$ has more than $3$ vertices, and because all of its vertices lie on a circle (in the drawing that witnesses $\g$'s outerplanarity), there must be pairs of vertices of $\g$ that are not adjacent along the circle.  Let $u$ and $v$ be nonadjacent (in the drawing) vertices of $\g$ such 
that the distance between $u$ and $v$ (measured in term of number of edges that one must traverse to reach one vertex from the other) is {\em minimal} among pairs of nonadjacent vertices.  We must consider two cases.
\begin{itemize}
\item
If the distance between $u$ and $v$ were {\em exactly} $2$, then the unique vertex that lies between $u$ and $v$ along the circle would have degree $2$.
\item
If the distance between $u$ and $v$ {\em exceeded} $2$, then there would be at least {\em two} vertices that lie between $u$ and $v$ in either direction around the circle.  But in this case, there would be two nonadjacent vertices that were closer to one another than $u$ and $v$---which contradicts our choice of $u$ and $v$ as a pair of {\em closest} nonadjacent vertices.
\end{itemize}
We conclude that $\g$ must have a vertex of degree $\leq 2$.  \qed
\end{proof}

\medskip

We return now to our primary concern---vertex-colorings in graphs.

\medskip

The $3$-vertex cycle $\cc_3$ witnesses the fact that not every outerplanar graph is $2$-colorable. Therefore, the chromatic number for the family of outerplanar graphs can be no smaller than $3$.  We now show that $3$ is, in fact, the chromatic number for this family.  Our inductive proof of this fact can easily be turned into an efficient $3$-coloring algorithm.

\begin{prop}[The $3$-Color Theorem for Outerplanar Graphs]
\label{thm:OP-3-colorability}
Every outerplanar graph is $3$-colorable.
\end{prop}

\begin{proof}
We proceed by induction on the number of vertices in the outerplanar graph to be colored.

\smallskip

\noindent {\sf Base case}.
We leave to the reader the task of finding $3$-colorings for small outerplanar graphs---say those having $\leq 3$ vertices.

\medskip

\noindent {\sf Inductive assumption}.
Assume that every outerplanar graph having $< n$ vertices is $3$-colorable.

\medskip

\noindent {\sf Inductive extension}.
Focus on an arbitrary $n$-vertex outerplanar graph $\g$.

\smallskip

\index{incident edges}

By Proposition~\ref{thm:basic-outerplanar-stuff}(d), $\g$ has a vertex $v$ of degree $\leq 2$.  Let us remove vertex $v$ from $\g$, along with its {\it incident} edges, i.e., those that connect $v$ to the rest of $\g$; call the resulting graph $\g'$.  Now: 
\begin{itemize}
\item
$\g'$ is clearly outerplanar.

\smallskip

This will be clear once we ``stitch'' together the circle that we ``damaged'' by removing $v$.
\item
$\g'$ has {\em fewer than} $n$ vertices.
\end{itemize}
By our inductive hypothesis, $\g$ is $3$-colorable.

\smallskip

But now we can reattach vertex $v$ to $\g'$ by replacing the edges that attach $v$ to $\g$. Moreover, we can now color $v$ using whichever of the $3$ colors on $\g$ that is {\em not} used for $v$'s neighbors in $\g$.  Once we so color $v$, we will have a $3$-coloring of $\g$.

\smallskip

Our induction is, thus extended, which completes the proof.  \qed
\end{proof}


\subsubsection{On vertex-coloring {\em planar} graphs}
\label{sec:planar-graphs}

 \index{planar graphs} \index{graph!planar}
 
The larger of this section's two graph families comprises {\it planar graphs}: graphs that can be drawn (on a two-dimensional medium) with no crossing edges.

\bigskip

\noindent \fbox{
\begin{minipage}{0.96\textwidth}
{\bf Historical note}.

The historically original focus on planar graphs and their vertex-colorings stemmed from viewing the graphs as abstractions of geographical maps.  The political units on a map (countries and/or cities and/or \ldots) became the vertices of a graph $\g$.  And, political units that shared a border led to an edge connecting the corresponding vertices of $\g$.

\smallskip

Of course, every abstraction idealizes reality in some way.  In this geographical setting, the major idealization is the assumption that ``adjacent" units shared a boundary that had non-zero length.  Adjacencies such as one observes with the US states of Arizona, Utah, Colorado, and New Mexico---the famous ``four corners states"---which meet at a point were not allowed in the abstraction.

\smallskip

The chromatic number of $\g$ then became, quite literally, the numbers of shades of ink that one would need in order to print the map in a way that assigned different colors to units that shared a border.
\end{minipage}
}

\bigskip

\noindent
In a clique, every pair of vertices are mutually adjacent.  Therefore:

\smallskip

{\em For all $n$, the $n$-vertex clique $\k_n$ can be colored with $n$ colors but no fewer.}

\smallskip

\noindent
Thus, the $4$-vertex clique $\k_4$ witnesses the fact that {\em not every planar graph is $3$-colorable}.  But it also raised the question, {\em Is every planar graph $4$-colorable}?


\index{Appel, Kenneth}  \index{Haken, Wolfgang}

\paragraph{A. On the $4$-Color Theorem for planar graphs}

A century-plus attempt to prove that $4$ colors suffice for planar graphs culminated in one of the most fascinating dramas in modern mathematics:  American mathematicians Kenneth Appel and Wolfgang Haken enlisted the help of their families---{\em and of their computer!}---as they crafted a proof of their renowned {\it $4$-Color Theorem for Planar Graphs}.  Their 1974 proof was long enough to consume {\em two} journal articles---both appearing in volume 21 of the {\it Illinois Journal of Mathematics} \cite{AppelH77a,AppelH77b}.

\index{The $4$-Color Theorem for Planar Graphs}
\index{coloring planar graphs!the $4$-Color Theorem}

\begin{theorem}[The $4$-Color Theorem for Planar Graphs~\cite{AppelH77a,AppelH77b}]
\label{thm:Four-ColorTheorem}
Every planar graph is $4$-colorable.
\end{theorem}

\index{The $4$-Color Problem for planar graphs} 

The proof of Theorem~\ref{thm:Four-ColorTheorem} is beyond the scope of any textbook, even an advanced one, but the backstory of the proof is truly fascinating!  And, the backstory supplies ample motivation for the proofs we present of the $6$-color and $5$-color analogues of the Theorem.

\medskip

Beginning with a failed attempt in 1875 to prove that every planar map can be $4$-colored, the so-called {\it $4$-Color Problem} held the world of discrete  mathematics in thrall for roughly a century before Appel and Haken announced their proof of Theorem~\ref{thm:Four-ColorTheorem} in 1974.  But, this proof notwithstanding, the drama surrounding the $4$-Color Problem persisted, because of the Appel-Haken proof's reliance---in a fundamental way---on a computer program that checked more than a thousand essential, but clerical, assertions (about forbidden subgraphs).  It took the
mathematics community years before the Appel-Haken proof, with its massive complexity and unprecedented employment of ``collaboration'' by computer, was generally accepted.

\smallskip

Even readers who might be daunted by the primary references \cite{AppelH77a,AppelH77b} that accompany our statement of the Theorem may well enjoy the much more accessible articles \cite{AppelH77c,AppelH89} in which the authors summarize---and, at a rather sophisticated level, popularize---this marvelous mathematical tale.

\bigskip

We turn now to the eminently accessible proofs of the $6$- and $5$-color analogues of Theorem~\ref{thm:Four-ColorTheorem}.  There are significant lessons within the proofs of these 
analogues.
\begin{itemize}
\item
The weaker, $6$-color, version of the Theorem can be proved in much the same way as its
outerplanar-graph cousin, Proposition~\ref{thm:OP-3-colorability}.  We
present this proof in detail in Paragraph B.  
\item
The proof of the stronger, $5$-color, version of the Theorem already requires us to break the world into multiple cases---but only a single-digit number of cases, in contrast to the four-digit list of cases engendered by the proof of Theorem \cite{AppelH77a,AppelH77b}.  That said, the complexity of the $5$-color theorem has led us to include its proof only as an ``Enrichment Topic''; see Paragraph C.
\end{itemize}


\paragraph {B. The $6$-Color Theorem for planar graphs}

The first step in showing that every planar graph can be vertex-colored using $6$ colors resides in the following analogue for planar graphs of Proposition~\ref{thm:basic-outerplanar-stuff}(d), which asserts that every outerplanar graph has a vertex of degree $2$.

\begin{lemma}
\label{thm:PlanarGraph-degree5}
Every planar graph has a vertex of degree $\leq 5$.
\end{lemma}

\index{graph!planar!face in a drawing} \index{planar graph!face in a drawing} 

\begin{proof} {\em (Lemma~\ref{thm:PlanarGraph-degree5})}
Let us focus on a planar drawing of a (perforce) planar graph $\g$ which has $n$ vertices, $e$ edges, and $f$ {\it faces}.  A {\it face} in a drawing of $\g$ is a polygon whose sides are edges of
$\g$, whose points are vertices of $\g$, and whose interiors are ``empty"---no edge of $\g$ crosses through a face.

\bigskip

\noindent \fbox{
\begin{minipage}{0.96\textwidth}
{\bf Historical note.}

\index{outerplanar graph!origin of name}

\smallskip

Now that we know about faces, we can finally describe the origin of the term {\it outerplanar}.  A graph $\g$ is outerplanar if it can be drawn (on a $2$-dimensional medium) in the following manner.  All of $\g$'s vertices are placed around a circle, and all of $\g$'s edges are drawn as noncrossing chords of the circle.  The region outside the vertex-bearing circle is, thus, the {\em outer} face of this special planar drawing of $\g$.
\end{minipage}
}
\bigskip

The following auxiliary result derives a celebrated ``formula'' attributed to Euler.

\index{Euler, Leonhard} \index{Euler's Formula}

\begin{prop} [Euler's Formula for Planar Graphs]
\label{thm:Euler-Formula}
Let $\g$ be a planar graph having $n$ vertices and $e$ edges.  For every $f$-face planar drawing of $\g$, we have
\begin{equation}
\label{eqn:Eulers-formula}
n \ - \ e \ + \ f \ \ = \ \ 2
\end{equation}
\end{prop}

\medskip

We defer proving Proposition~\ref{thm:Euler-Formula} so that we can proceed with our ongoing proof of Lemma~\ref{thm:PlanarGraph-degree5}.  We devote Subsection~\ref{subsec:validationEulerFormula} to two quite different proofs of Euler's formula.

\medskip

\index{graph!planar!maximal}

As we approach the next step in the proof of Lemma~\ref{thm:PlanarGraph-degree5}, we  simplify the setting by assuming henceforth that $\g$ is {\em connected} and that it is a {\em maximal} planar graph---meaning that one cannot add any new edge to the drawing without crossing an existing edge (and, thereby, destroying planarity).  Fig.~\ref{fig:K5andK3by3} illustrates the notion of maximality by providing planar drawings of two connected {\em maximal} planar graphs.
\begin{figure}[hbt]
\begin{center}
       \includegraphics[scale=0.55]{FiguresGraph/K5andK3by3}
\caption{Planar drawings of the maximal planar graphs $\k_5$ (left) and $\k_{3,3}$ (right).  One cannot add an edge to either drawing without destroying planarity---by introducing crossing edges.  For reference, we add dotted edges in the figure, which would augment the depicted planar graphs to the (inherently nonplanar) graphs $\k_5$ (on the left) and $\k_{3,3}$ (on the right).}
  \label{fig:K5andK3by3}
\end{center}
\end{figure}
The assumption of maximality only strengthen's the Lemma's conclusion by (apparently) making it more difficult to find a small-degree vertex.

\smallskip

With the maximality assumption in place, we now adapt a pedagogical tool from \cite{Berge73}, in order to make the following counting argument easier to follow.  We construct a {\em directed bipartite} graph {\bf G} which exposes certain features of $\g$'s structure.  On one side of {\bf G} are the $f$ faces of $\g$; on the other side are $\g$'s $e$ edges.  {\bf G} contains an arc from each face of $\g$ to each edge of $\g$ that forms a ``side'' of the polygonal drawing of the face.  Because $\g$ is a {\em maximal} planar graph, we have:
\begin{itemize}
\item
Each face of $\g$ is a $3$-cycle, hence involves three vertices.
\item
Each edge of $\g$ touches two faces.
\item
Each edge of $\g$ touches two vertices.
\end{itemize}

Let us now put these facts together, and assume, for contradiction, that every vertex of $\g$ had degree $\geq 6$.  We would then find that
\[ \left[ f \ \ \leq \ \ \frac{2}{3} e \right] \ \ \ \ \mbox{ and } \ \ \ \ \Big[ e \ \ \geq \ \ 3n \Big] \]
Incorporating these two bounds into Euler's Formula (\ref{eqn:Eulers-formula}), we arrive at the following contradiction.
\[ 2 \ \ = \ \ n \ - \ e \  + \ f \ \ \leq \ \ \frac{1}{3} e \ - \ e \ + \ \frac{2}{3} e \ \ = \ \ 0 \]
This contradiction proves that every planar graph must have a vertex of degree $\leq 5$.
 \qed-Lemma~\ref{thm:PlanarGraph-degree5}
\end{proof}

\medskip

We finally have the tools to color any planar graph using $6$ colors.

\index{The $6$-Color Theorem for Planar Graphs}
\index{coloring planar graphs!the $6$-Color Theorem}

\begin{prop}[The $6$-Color Theorem for Planar Graphs]
\label{thm:P-6-colorability}
Every planar graph is $6$-colorable.
\end{prop}

\begin{proof}
The $2$-Color Theorem for Outerplanar Graphs (Proposition~\ref{thm:OP-3-colorability}) and this result follow via almost-identical inductions on the number of vertices in the graph $\g$ that is being colored.  Both arguments:
\begin{enumerate}
\item
remark that the target number of colors is adequate for small graphs

\smallskip

For outerplanar graphs, ``small'' means ``$\leq 3$ vertices''.  For planar graphs, it means ``$\leq 4$ vertices''.

\item
remove from $\g$ a vertex $v$ of smallest degree $d_v$, together with all incident edges

\smallskip

For outerplanar graphs, we guarantee that $d_v \leq 2$ (Proposition~\ref{thm:basic-outerplanar-stuff}(d)).  For planar graphs, we guarantee that $d_v \leq 5$ (Lemma~\ref{thm:PlanarGraph-degree5}).

\item
inductively color the vertices of the graph left after the removal of $v$

\smallskip

Let us denote by $\g'$ the graph obtained by removing $v$ from $\g$.  Then: 

For outerplanar graphs, we color $\g'$ with $\leq 3$ colors (Proposition~\ref{thm:OP-3-colorability}).  For planar graphs, we use an inductive assumption that $\g'$ can be colored with $\leq 6$ colors. 

\item
reattach $v$ via its $d_v$ edges and then color $v$.

\smallskip

Note that the coloring guarantee in both results---Proposition~\ref{thm:OP-3-colorability} for outerplanar graphs and the current result for planar graphs---allows us to use $d_v +1$ colors to color $\g$.  Because $v$ has degree $d_v$, it can have no more than $d_v$ neighboring vertices in $\g'$, so our access to $d_v +1$ colors guarantees that we can successfully color $v$.
\end{enumerate}
The proofs of the $3$-colorability of outerplanar graphs and the $6$-colorability of planar graphs thus differ only in the value of $d_v$.  \qed
\end{proof}


\paragraph{C. $\oplus$ The $5$-Color Theorem for planar graphs}
\index{The $5$-Color Theorem for Planar Graphs}
\index{coloring planar graphs!the $5$-Color Theorem}

\begin{prop}[The $5$-Color Theorem for Planar Graphs]
\label{thm:5colors}
\label{thm:P-5-colorability}
Every planar graph is $5$-colorable.   \cite{Heawood90}
\end{prop}

\bigskip

%The proof, being rather technical, is relegated to the Appendix, as Chapter~\ref{Appendix:5colors}.

\noindent \fbox{
\begin{minipage}{0.96\textwidth}
{\bf Enrichment note}.

The case analysis in the following proof is a bit more complex than in most of the results in the text, but a methodical reading should make the proof accessible to the reader.  Moreover,  the {\em roadmap} of the case analysis is a valuable lesson in how mathematics is really done!

\smallskip

The motivated reader will be able to recast the totally positive proof we present into the form of a proof by contradiction.  The positive version should be more to the taste of a computation-oriented reader---but both proofs are equally correct.
\end{minipage}
}

\ignore{*************
The proof again (I mean like for 6 colors) is by recurrence but we add
here a contradiction argument:
Similarly, we focus on a vertex with degree 5 (at some points, we will
have to justify this)
This vertex (call it x) has 5 neighbors
The problem is when they are colored by the 5 colors, otherwise it is
colored by a missing one.
Let now label these 5 vertices from 1 to 5.
Consider the vertices 1 and 3, colored by two different colors and the
sub-graph composed of vertices with colors 1 and 3
If these two vertices are in disjoint connected components, (case 1), it is
easy to color x by reverting the colors 1 and 3 in one component
(see figure)
Thus, the problem is when there exists an alternate path (in term of
colors) between vertex 1 and vertex 3
in this case (case 2), consider vertices 2 and 4, using a similar process
as in case 1, if there are two connected components
we are done (x can be colored by reverting the colors along the path
in one component), otherwise, there is an alternate path
between 2 and 4.
However, since the graph is planar, then, both alternate paths will
intersect, which is impossible
************************}

\begin{proof}
We henceforth discuss, without explicit mention, only {\em valid} vertex-colorings---in which neighboring vertices get different colors.  We proceed by induction.

\medskip

\noindent {\sf Base case.}
Because the $5$-clique $\k_5$ is obviously $5$-colorable, so also must be all graphs having $\leq 5$ vertices.  Therefore, we know that any non-$5$-colorable graph would have $\geq 6$ vertices.

\medskip

\noindent {\sf Inductive hypothesis.}
Assume, for induction, that every planar graph having $\leq n$ vertices is $5$-colorable.

\medskip

\noindent {\sf Inductive extension.}
If the proposition were false, then there would exist a planar graph $\g$ having $n+1$ vertices which is not $5$-colorable.  By Lemma~\ref{thm:PlanarGraph-degree5}, $\g$ would have a vertex $v$ of degree $\leq 5$.  The remainder of the proof focuses on the graph $\g$, its minimal-degree vertex $v$, and on $v$'s $(d_v \leq 5)$ neighbors in $\g$.

\smallskip

If there were a coloring of $\g$'s vertices in which $\leq 4$ colors were used to color $v$'s neighbors, then the following analogue of the coloring strategy of Proposition~\ref{thm:P-6-colorability} would produce a $5$-coloring of $\g$.
\begin{enumerate}
\item
Remove vertex $v$ and its incident edges from $\g$, thereby producing the $n$-vertex planar graph $\g'$.
\item
Produce a $5$-coloring of $\g'$ that uses only $4$ colors for the vertices that are neighbors of $v$ in $\g$.
\item
($a$) Reattach vertex $v$ and its edges to $\g'$, thereby reconstituting $\g$.

\smallskip

($b$) Color $v$ with any of the $5$ available colors that is not used for $v$'s neighbors.
\end{enumerate}

\smallskip

In order to proceed in pursuit of a contradiction, we must understand what structural features of $\g$ make it impossible to use only $4$ colors on $v$'s neighbors when $5$-coloring $\g$.  There are three important situations to recognize.
\begin{description}
\item[{\sf Case 1}.]
Vertex $v$ has degree $\leq 4$.

\smallskip

By definition, no more than $4$ colors are used to color $v$'s neighbors in this case.
\end{description}
Note that, in all remaining cases, vertex $v$ has precisely $5$ neighbors---or else, we would have invoked Case 1 to color $\g$ with $5$ colors.
\begin{description}
\item[{\sf Case 2}.]
For some $5$-coloring of $\g$, at least $2$ neighbors of $v$ get the same color.

\smallskip

Because $v$ has exactly $5$ neighbors, in this case, only $4$ colors are used to color these neighbors.
\end{description}
In all remaining cases, the $5$ neighbors of $v$ receive distinct colors.
\begin{description}
\item[{\sf Case 3}.]
For some $5$-coloring of $\g$, some two neighbors of $v$---call them $v_1$ and $v_2$---reside in distinct components of $\g$ once $v$ and its incident edges are removed from $\g$.

\smallskip

As before, let $\g'$ be the (in this case, disconnected) graph that results when $v$ and its incident edges are removed from $\g$.  For $i = 1,2$ Let $\g_i$ be the component of $\g'$ that contains vertex $v_i$.

Say that, under the $5$-coloring of $\g$ that we are focusing on, $v_1$ is colored {\it red} and $v_2$ is colored {\it green}.

\smallskip

Let us recolor the vertices of $\g_1$ so that vertex $v_1$ is now colored {\it green}.  (One needs only switch the colors {\it red} and {\it green} in the existing coloring of $\g_1$ to achieve this.)  It is always possible to do this in a way that does not affect the valid coloring of $\g_2$ because $\g_1$ and $\g_2$ are mutually disjoint.

\smallskip

Once we have thus-recolored $\g_1$, we have a $5$-coloring of $\g$ for which Case 2 holds.  (In fact, we can color vertex $v$ {\em red} when we reattach it to $\g'$.)
\end{description}

\smallskip

\noindent
We now see that Cases 1--3 cannot prevent us from $5$-coloring $\g$, so we are left with the following minimally constrained situation.
\begin{description}
\item[{\sf Case 4}.]
\begin{itemize}
\item
Every minimum-degree vertex of $\g$ has $5$ neighbors.

\smallskip

For the minimum-degree vertex $v$, let us call these neighbors $v_1$, $v_2$, $v_3$, $v_4$, $v_5$, {\em in clockwise order within the planar drawing}.
\item
In every $5$-coloring of $\g$, the neighbors of every minimum-degree vertex receive distinct colors.

\smallskip

For vertex $v$, let us say that neighbor $v_i$ receives color $c_i$.
\end{itemize}
The leftmost graph in Fig.~\ref{fig:5colorsInit} depicts the portion of $\g$ comprising vertex $v$ and its neighbors.  .
\begin{figure}[hbt]
\begin{center}
   \includegraphics[scale=0.45]{FiguresGraph/5colorsInit}
\caption{A vertex with degree $5$.}
  \label{fig:5colorsInit}
\end{center}
\end{figure}
In the Fig.~\ref{fig:5colorsCase2}, we use integer $i$ to denote, ambiguously, vertex $v_i$ and its assigned color $c_i$.  
The question mark ``?'' that ``colors'' vertex $v$ indicates that we do not yet know what color to assign to $v$.  

The other two graphs in the figure depict schematically how we have dealt with Case 3 above.
\begin{figure}[hbt]
\begin{center}
   \includegraphics[scale=0.4]{FiguresGraph/5colorsCase1}
\caption{A vertex-coloring for Case 1.}
  \label{fig:5colorsCase1}
\end{center}
\end{figure}
\begin{itemize}
\item
All neighbors of vertex $v$ remain in the same component of $\g$ when $v$ and its incident edges are removed.
\end{itemize}
\end{description}
To analyze Case 4, we focus on vertices $v_1$ and $v_3$ in Fig.~\ref{fig:5colorsCase2}.
\begin{figure}[hbt]
\begin{center}
   \includegraphics[scale=0.4]{FiguresGraph/5colorsCase2}
\caption{A vertex-coloring for Case 2.}
  \label{fig:5colorsCase2}
\end{center}
\end{figure}
Importantly, these vertices have received distinct colors ($c_1$ and $c_3 \neq c_1$, respectively), and these vertices are not adjacent to one another along a clockwise sweep around vertex $v$.

\smallskip

Now take $\g$ and focus only on the vertices that are colored $c_1$ or $c_3$ (as are $v_1$ and $v_3$, respectively) and on the vertices that are colored $c_2$ or $c_4$ (as are $v_2$ and $v_4$, respectively).  One sees from Figure~\ref{fig:5colorsCase2} that:
\begin{itemize}
\item
$\g$ can---{\em but need not}---contain a path whose vertices alternate  colors $c_1$ and $c_3$. Call this a ``$c_1$-$c_3$ path'' between vertices $v_1$ and $v_3$.
\item
$\g$ can---{\em but need not}---contain a path whose vertices alternate colors $c_2$ and $c_4$.  Call this a ``$c_2$-$c_4$ path'' between vertices $v_2$ and $v_4$.
\item
$\g$ {\em cannot} contain both of the paths just described, i.e., a $c_1$-$c_3$ path between $v_1$ and $v_3$ {\em and} a $c_2$-$c_4$ path between $v_2$ and $v_4$.

\smallskip

These two paths, if they existed, would cross one another---which is forbidden because $\g$ is a {\em planar} graph,  See Figure~\ref{fig:5colorsCase2}.


\end{itemize}
It follows that {\em either} $\g$ does not contain a $c_1$-$c_3$ path between $v_1$ and $v_3$ {\em or} $\g$ does not contain a $c_2$-$c_4$ path between $v_2$ and $v_4$.  Say, with no loss of generality, that the former path does not exist.  Then we can switch colors $c_1$ and $c_3$ beginning with vertex $v_1$ and obtain a coloring of $\g$ in which $v_1$ and $v_3$ both receive the color $c_3$.  We can then proceed as in Case 2 to get a $5$-coloring of $\g$.

\smallskip

This four-case analysis shows that we can always produce a $5$-coloring of $\g$, which completes the proof.  \qed
\end{proof}

\bigskip

We close our study of vertex-colorings of planar graphs with an overview of the intellectual cost-benefit tradeoff that we have exposed:
\begin{itemize}
\item
A straightforward recursive coloring strategy suffices if one is willing to settle for a $6$-color palette when coloring planar graphs (Proposition~\ref{thm:P-6-colorability}).
\item
A four-case analysis, in which one case comprises several subcases, is needed in order to eliminate one of the colors from our palette, i.e., to achieve a $5$-coloring strategy (Proposition~\ref{thm:5colors}).
\item
An analysis involving close to 2000 cases is needed in order to achieve the provable optimal, a $4$-color palette that always works (Theorem~\ref{thm:Four-ColorTheorem}).
\end{itemize}

\subsubsection{Two validations of Euler's Formula}
\label{subsec:validationEulerFormula}
\index{Euler's Formula}

We now develop two quite different proofs of Proposition~\ref{thm:Euler-Formula}, which expose different ways to think about the identity.

\medskip

\index{Euler's Formula!validation via structural induction}

\noindent {\bf Validation via structural induction}. 
Our first approach validates (\ref{eqn:Eulers-formula}) by growing a planar graph $\g$
inductively, edge by edge.  Note that we formulate our induction a bit differently than our earlier, simpler ones, particularly in the Inductive hypothesis.

\smallskip

\noindent {\sf Base case}.
The Formula clearly holds for the smallest planar graphs, including the smallest interesting one, $\cc_3$, which has $n = e = 3$ and $f =2$ (the inner and outer faces of the ``triangle'').

\smallskip

\noindent {\sf Inductive hypothesis}.
Assume that the Formula holds for a given graph $\g$.

\smallskip

\noindent {\sf Inductive extension}.
We extend our induction by growing the current version of $\g$, by adding a new edge.  Two cases arise.
\begin{itemize}
\item
{\it The new edge connects existing vertices.}
In this case, this augmentation of $\g$ increases the number of edges ($e$) and the number of faces ($f$) by $1$ each, while keeping the number of vertices ($n$) unchanged.

\item
{\it The new edge adds a new vertex, which is appended to preexisting vertex.}
In this case, this augmentation of $\g$ keeps the number of faces ($f$) unchanged while
it increases by $1$ both the number of edges ($e$) and the number of vertices ($n$). 
\end{itemize}
In both cases Euler's Formula (\ref{eqn:Eulers-formula}) continues to hold as we augment $\g$.  

The augmentation thus extends the induction, hence validates the Formula.  \qed

\bigskip

\index{Euler's Formula!validation via deconstruction}

\noindent {\bf Validation via deconstruction}.
Let us be given a planar graph $\g$ that has $n$ vertices, $e$ edges, and $f$ faces.  We validate Formula (\ref{eqn:Eulers-formula}) by deconstructing $\g$ and showing that each step in the process preserves as {\it invariant} the expression
\begin{equation}
\label{eq:phi-in-euler-formula}
 \phi(n,e,f) \ = \ n-e+f
\end{equation}

\bigskip

\index{invariants}
\noindent \fbox{
\begin{minipage}{0.96\textwidth}
{\bf Explanatory note}.

\smallskip

\textit{Invariants} are extremely important conceptual tools when crafting proofs.  The idea underlying their use is to find an expression $\phi(\circ)$ whose value is preserved---i.e., it ``holds
invariant"---as some relevant process proceeds.

\smallskip

With many mathematical concepts---including the notion of invariant---it is easier to grasp the concept by experiencing examples than by internalizing abstract explanations.  Therefore, we recommend that the reader follow this second validation of Euler's Formula while keeping track of the invariance of expression $\phi(n,e,f)$ of (\ref{eq:phi-in-euler-formula}).
\end{minipage}
}
\bigskip

\noindent
Focus on the following two-phase process.
\begin{description}
\item[{\bf Phase 1}.]
Iterate the process of removing edges from $\g$ until some edge-removal reduces $\g$ to a graph with a single face.

\smallskip

The reader should verify that this termination condition is equivalent to stopping when the remaining graph is a (connected) tree.

\medskip

{\sf The action}.
If the graph remaining at some step contains an edge that is shared by two distinct faces, then remove any such edge.

\smallskip

Fig.~\ref{fig:planarStep1} illustrates Phase 1 for a (residual) graph having $n=8$ vertices, $e=11$ edges, and $f=5$ faces.
\begin{figure}[hbt]
\begin{center}
   \includegraphics[scale=0.35]{FiguresGraph/planarStep1}
\caption{Illustrating Phase 1: From top left to bottom right, each transformation preserves the invariant $\phi(n,e,f)$.  The first transformation removes the edge shared by faces (1) and (2), creating a new, merged, face (1)'.}
  \label{fig:planarStep1}
\end{center}
\end{figure}

\medskip

{\sf The analysis}.
\begin{itemize}
\item
The graph remaining after an edge-removal is still planar, so we can continue the process.
\item
The process preserves the value of function $\phi$; i.e.,
\[ \phi(n,e,f) \ = \ \phi(n,e-1,f-1) \]
This is because $n$ is unchanged, while $e$ and $f$ are each reduced by $1$.
\end{itemize}

\item[{\bf Phase 2}.]
Iterate the following process of removing vertices from the tree produced by Phase 1, until only one vertex remains.

\medskip

{\sf The action}.
Remove any leaf of the current tree, together with its incident edge.

\smallskip

Fig.~\ref{fig:planarStep2} illustrates the action of Phase  2 for a tree having $n=8$ vertices---hence, $e=7$ edges and $f=1$ face.
\begin{figure}[hbt]
\begin{center}
   \includegraphics[scale=0.35]{FiguresGraph/planarStep2}
 \caption{Illustrating Phase 2: We remove a sequence of leaves, each with its incident edge, until we reach a single vertex.  Here again, $\phi$ remains invariant after each leaf-removal.}
  \label{fig:planarStep2}
\end{center}
\end{figure}

\medskip

{\sf The analysis}.
The value of $\phi$ remains invariant under each leaf-removal:
\[ \phi(n,e,f) \ = \ \phi(n-1,e-1,f) \]
To wit, $f$ is unchanged ($f \equiv 1$ for a tree), while $n$ and $e$ are each reduced by $1$.
\end{description}

\medskip

\noindent {\sf The cumulative analysis}.
\begin{itemize}
\item
The described process executes a number of steps exactly equal to the number of edges of the graph $\g$ that we begin with.  Specifically, the edge-removing Phase 1 removes $e-n$ edges, while the leaf-removing Phase 2 removes $n-1$ vertices.
\item
At the end of the process, $e=0$ and $n=f=1$, so that the residual value of $\phi$ is $2$.  Because each action prescribed by the process preserves the value of $\phi$, we know that $\phi$ has the value $2$ also {\em before} each edge-removal and leaf-removal.
\end{itemize}
We conclude, in summation, that at the start of the process, $\phi(n,e,f)$ had the value $2$.  In other words, $n-e+f = 2$, as asserted by Euler's Formula.   \qed

\bigskip

\noindent \fbox{
\begin{minipage}{0.96\textwidth}
{\bf Enrichment note}.

Our study of vertex-coloring in graphs has focused almost entirely on planar and outerplanar graphs.  This is because---aside from the importance of such graphs in applications---their vertex-coloring problems can be studied entirely in terms of the mathematical structure of the graphs being colored.  Vertex-coloring for broader families of graphs usually cannot be studied solely by focusing on the target graph's structure:  Algorithmic issues arise quite early in the coloring process (although of course, mathematics is employed to analyze the algorithms).

\smallskip

In closing our treatment of this topic, though, we remark that there are many interesting algorithmic problems relating to vertex-coloring arbitrary graphs.  The following such result dates to the earliest days of the study of {\sf NP}-completeness \cite{GareyJ79,Karp72}.

\smallskip

\hspace*{.075in}
{\em For any fixed $k \geq 3$, the problem ``Is graph $\g$ $k$-colorable" is {\sf NP}-complete}.

\smallskip

This complexity-theoretic result suggests that {\em optimal} vertex-coloring is computationally inherently difficult.  The ``sting" of the result is moderated by the existence of ``greedy'', hence efficient, vertex-coloring algorithms that give quite good results in practice \cite{CLRS}.  The term ``greedy" here refers to algorithms that allocate a yet-unused color to a vertex only when no uncolored vertex can be validly colored with an already-used color.  (See Footnote~\ref{foot:greedy} in Chapter~\ref{ch:Graphs1}.)
\end{minipage}
}


\section{Path and Cycle Discovery Problems in Graphs}
\label{sec:path-cycle-problems}

Just as various genres of {\it spanning trees} are used to ``summarize'' aspects of the {\em connectivity} structure of a  graph $\g$, various genres of {\it paths} and {\it cycles} in $\g$ are often useful to ``summarize'' aspects of $\g$'s {\em traversal} structure.  This section is devoted to a range of problems related to determining the existence in a graph $\g$ of a path or a cycle that {\em completely} ``summarizes'' $\g$'s traversal structure---either by containing every edge of $\g$ precisely once (the {\it Eulerian} version of ``traversal summarization") or by containing every vertex of $\g$ precisely once (the {\it Hamiltonian} version of ``traversal summarization").

We generally focus in this section only on {\em undirected} paths and cycles in {\em undirected}, {\em unweighted} graphs.  Extending the notions we discuss to their directed analogues in directed and/or weighted graphs, will be accomplished via carefully crafted exercises.  In a similar, but simpler, vein, we generally discuss only problems concerning {\em cycles}, leaving to the reader the analogous path-related notion.  We begin by delimiting the two main classes of cycle-discovery problems that we study.

\bigskip

\index{graph!Eulerian cycle} \index{Eulerian cycle}
\index{graph!Eulerian circuit} \index{Eulerian circuit}  \index{Euler, Leonhard}

{\it Eulerian cycles (or, tours)}.
A cycle in a graph $\g$ that traverses each of $\g$'s {\em edges} precisely once is called an {\it Eulerian cycle} (or, often, an {\it Eulerian circuit}).  The edge-exhausting cycles/circuits/tours referred to by these several names were introduced as a topic of study in 1736 by the Swiss mathematician Leonhard Euler, whose name we have already encountered multiple times.  Euler allegedly identified the topic while contemplating how to devise a tour of the town of K\"{o}nigsberg that would cross each of the town's bridges precisely once.  The quest for Eulerian cycles in graphs is certainly one of the oldest problems---perhaps literally the oldest problem---in the fields now called {\it operations research} and {\it graph theory}.  (An edge-exhausting {\em path} in $\g$ is referred to in the obvious analogous way.)  When one views an Eulerian cycle as a ``map'' for traversing a graph---as did Euler when contemplating this problem---one often calls the cycle an {\it Eulerian tour}.  Traditionally, a graph that admits an Eulerian cycle is said to be {\it Eulerian}. 
 \index{graph!Eulerian tour} \index{Eulerian tour}  \index{graph!Eulerian} \index{Eulerian graph}

\medskip

\index{graph!Hamiltonian cycle} \index{Hamiltonian cycle}
\index{graph!Hamiltonian circuit} \index{Hamiltonian circuit}
\index{Hamilton, William Rowan} \index{Kirkman, Thomas Pennington}

{\it Hamiltonian cycles (or, circuits, or, tours)}.
A cycle in a graph $\g$ that encounters each of $\g$'s {\em vertices} precisely once is called a {\it Hamiltonian cycle}, (or, often, a {\it Hamiltonian circuit}).  This cycle-discovery problem is named in honor of the British mathematician Sir William Rowan Hamilton, who is credited with inventing the
concept in the mid-19th century.  (In fact, Hamilton's work on his eponymous cycles was anticipated by many decades in the work of fellow British mathematician, Thomas Pennington Kirkman.)  When one views a Hamiltonian cycle as a ``map'' for traversing a graph, one often calls the cycle a {\it Hamiltonian tour}.  Traditionally, a graph that admits a Hamiltonian cycle is said to be {\it Hamiltonian}.  (A vertex-exhausting {\em path} in a graph is referred to in the obvious analogous way.)  
 \index{graph!Hamiltonian tour}\index{Hamiltonian tour} \index{graph!Hamiltonian} \index{Hamiltonian graph}

\bigskip

Despite the conceptual duality between the edge-exhausting goal that underlies Eulerian paths/cycles and the vertex-exhausting goal that underlies Hamiltonian paths/cycles, these two graph-traversing goals differ from one another in virtually every significant mathematical and algorithmic respect.  It is rather easy to characterize the family of graphs that admit Eulerian tours and to find such a tour if it exists (Section~\ref{sec:EulerianCycle}); in sharp contrast, there is no known characterization of the family of graphs that admit Hamiltonian tours, and the computational problem of efficiently determining whether a graph admits such a tour is one of the major classical problems in the field of computational complexity  (Section~\ref{sec:Hamiltonian-cycle}).

%%%%%%%%%%%%%%%%%%%%%%%%%%%%%%%%%%%%%%%%%%%%%

\subsection{Eulerian Cycles and Paths}
\label{sec:EulerianCycle}

The main results in this section characterize the families of directed and undirected graphs that admit {\it Eulerian cycles} or {\it paths}.  The proofs of these characterizations are constructive: they are built upon algorithms that efficiently find such a cycle or such a path if one exists.  We focus on graphs that are {\em connected}---but our algorithms can actually be adapted to find an Eulerian cycle or path in each connected component of a general graph.

\smallskip

We open our adventure with the following amusing fact:

\bigskip

\index{graph!drawing without lifting pencil}

\noindent \fbox{ \begin{minipage}{0.96\textwidth}
{\bf Enrichment note}.

The problem of discovering an Eulerian cycle in a graph $\g$ is (algorithmically) equivalent to the problem of drawing $\g$ ``cyclically"---i.e., starting and ending with the same vertex---without ever lifting one's pencil. 
\end{minipage}
}

\bigskip

The next two propositions develop---for both cycles and paths, in both undirected and directed graphs---simple, elegant characterizations of graphs that admit Eulerian cycles and paths.
These characterization are validated via simple and efficient algorithms that determine whether a given graph admits such a cycle or such a path and that discover such a cycle or path when one exists.  To avoid ``special cases" that disrupt the flow of a proof, we restrict attention to graphs and digraphs that are {\em nontrivial} in the sense that
\begin{itemize}
\item
each (di)graph has at least three vertices;
\item
each (di)graph is connected.
\end{itemize}
Within this domain, we have the following characterizations.

\begin{prop}[Eulerian Cycles]
\label{thm:eulerian-cycle}
{\bf (a)}
A connected undirected graph $\g$ admits an Eulerian cycle if, and only if, every vertex of $\g$ has even degree.

\smallskip

\noindent {\bf (b)}
A connected directed graph $\g$ admits a directed Eulerian cycle if, and only if, $\mbox{\sc indegree}(v) \ = \ \mbox{\sc outdegree}(v)$ for every vertex $v$ of $\g$.
\end{prop}

\begin{proof}
{\bf Necessity}.
The conditions asserted in parts (a) and (b) are {\em necessary} because of the following fact.  Every time any cycle traverses a vertex $v$ of a graph $\g$, that encounter accounts for {\em two} edges that are incident to $v$: the cycle must ``enter'' $v$ using one edge and ``exit'' $v$ using a different edge.   For part (b), $\g$ is a digraph, and the encounter with vertex $v$ traverses arcs rather than edges.

\medskip

\noindent {\bf Sufficiency}.
We now verify that Proposition's conditions are also {\em sufficient}.  We provide  a complete proof of sufficiency only for part (a), for undirected graphs; we provide only hints regarding the sufficiency of the conditions for part (b).

\smallskip

We have a choice of proving sufficiency either by induction on the number of vertices in a 
graph or by induction on the number of edges in the graph.  While we invite the reader to 
develop the vertex-based proof, we focus only on the technically more streamlined edge-based proof. 

\medskip

Focus on a connected undirected graph $\g$ that has $m$ edges and $n$ vertices, each vertex having nonzero even degree.  We prove by induction on $m$ that $\g$ is Eulerian.

\smallskip

\noindent {\sf Base case.}
We begin with the base case $m=3$.  (You should consider why graphs with $m=1$ or $m=2$ edges cannot be Eulerian.)  For $m=3$, there is only one Eulerian graph, namely, 
$\cc_3 \ = \ \k_3$.

\smallskip

\noindent {\sf Inductive hypothesis.}
Assume that the Proposition holds for any number of edges $k \leq m$.

\smallskip

\noindent {\sf Inductive extension.}
Consider a connected $(m+1)$-edge graph $\g$ all of whose vertices have nonzero even degree.
By Proposition~\ref{thm:cycle-in-graph}, $\g$ contains a nonempty cycle $\cc$ (see Fig.~\ref{fig:eulerianProof1}). 
\begin{figure}[hbt]
\begin{center}
       \includegraphics[scale=0.45]{FiguresGraph/EulerianProof1}
       \caption{A cycle in graph $\g$ (ignoring its other vertices and their incident edges).}
  \label{fig:eulerianProof1}
\end{center}
\end{figure}
Focus on the subgraph $\g'$ of $\g$ that is obtained by removing cycle $\cc$.  Clearly $\g'$ contains fewer than $m-2$ edges, because $\g$ has $m+1$ edges and $\cc$ contains at least $3$ edges.

\smallskip

Now,  graph $\g'$ may have multiple connected components because the process of removing cycle $\cc$ could have disconnected the residual graph $\g'$; see Fig.~\ref{fig:eulerianProof2}.
\begin{figure}[hbt]
\begin{center}
       \includegraphics[scale=0.45]{FiguresGraph/EulerianProof2}
 \caption{Subgraph $\g'$ and its connected components after we removed the edges of cycle $\cc$ (whose edges are dotted lines).  In this example, there are four connected components, $C_1$, $C_2$, $C_3$, $C_4$; two of these, namely, $C_2$ and $C_4$ are isolated vertices.}
  \label{fig:eulerianProof2}
\end{center}
\end{figure}
But, no matter how many components $\g'$ has, each such component has no more than $m-2$ edges.  This means that each connected component of $\g'$ is either an isolated vertex---i.e., a graph with $1$ vertex and $0$ edges---or a {\em nontrivial component}.  In the latter case, our inductive hypothesis guarantees that $\g'$ has an Eulerian cycle.

\smallskip

The final step in extending the induction is to ``knit" the structure we have derived---namely, cycle $\cc$ plus graph $\g'$---into a single Eulerian cycle for $\g$.  We accomplish this via the following {\em gendanken} process.  We begin to traverse cycle $\cc$, continuing to walk until we encounter a nontrivial component, call it $\g"$, which we have never before encountered.  We now pause in our walk along $\cc$ to traverse the Eulerian cycle in $\g"$ that the induction hypothesis guarantees.  When we conclude the traversal of $\g"$---perforce (since it is a cycle) at the same vertex where the traversal began---we resume our walk along $\cc$.  Eventually, the walk along $\cc$---punctuated by traversals of Eulerian cycles in connected components---terminates.  At this point, we are guaranteed to have crossed every edges of $\g$ exactly once.

\medskip

The described walk exposes an Eulerian cycle in $\g$, as claimed.   \qed
\end{proof}

\medskip

\noindent {\em What we have {\em not} done}.
(1) Adapting our proof of Proposition~\ref{thm:eulerian-cycle} to directed graphs requires only  ensuring that we always enter a vertex $v$ along an in-arc and leave via an out-arc.  Because the in-degree of each vertex $v$ of $\g$ equals $v$'s out-degree, the required changes to the proof are clerical in nature.

\noindent (2)
The challenge of algorithmically producing an Eulerian cycle in graph $\g$ requires nontrivial data structuring that is outside the scope of a mathematics text.  However, our proof of Proposition~\ref{thm:eulerian-cycle} does give major hints about how to produce a {\em recursive} algorithm that constructs such a cycle.

\bigskip

The simplicity of the preceding characterization degrades a trifle when
one seeks Eulerian {\em paths} rather than {\em cycles}.

\begin{prop}[Eulerian Paths]
\label{thm:eulerian-path}
{\bf (a)}
A connected undirected graph $\g$ admits an Eulerian path if, and only if, at most two vertices of $\g$ have odd degree.

\smallskip

\noindent {\bf (b)}
A connected directed graph $\g$ admits an Eulerian path if, and only if: either $\g$ admits an 
Eulerian cycle, or $\g$ contains one vertex $u$ such that

\hspace*{.35in}$\mbox{\sc indegree}(u) \ = \ \mbox{\sc outdegree}(u) +1$

\noindent
and one vertex $v$ such that

\hspace*{.35in}$\mbox{\sc indegree}(v) \ = \ \mbox{\sc outdegree}(v) -1$.
\end{prop}

The proof of Proposition~\ref{thm:eulerian-path} shares its overall structure with the proof of  Proposition~\ref{thm:eulerian-cycle}, with one major difference.  Whereas a cycle has neither beginning nor end, a path has both.  Proposition~\ref{thm:eulerian-path}(a) asserts that an undirected graph which admits an Eulerian path but not an Eulerian cycle has precisely two vertices of odd degree, and these odd-degree vertices play the roles of the endpoints of the Eulerian path.  In similar fashion, Proposition~\ref{thm:eulerian-path}(b) asserts that a directed graph which admits an Eulerian path but not an Eulerian cycle contains one vertex, $u$, whose out-degree exceeds its in-degree and one vertex, $v$, whose in-degree exceeds its out-degree: vertex $u$ plays the role of the beginning vertex of the Eulerian path, and vertex $v$ plays the role of the end vertex of the path.  With these hints, we invite the reader to adapt the proof of Proposition~\ref{thm:eulerian-cycle} to obtain a proof of Proposition~\ref{thm:eulerian-path}.

\bigskip

We close this subsection by applying Proposition~\ref{thm:eulerian-cycle} to the ``named graphs" of Section~\ref{sec:graphs-important-families}.

\medskip

\begin{corol}
\label{corol:eulerian-named-graphs}
The following facts follow from Proposition~\ref{thm:eulerian-cycle}.

\medskip

\noindent
\begin{tabular}{lclcl}
\underline{Graph} & & \underline{Eulerian?} & & \underline{Explanation} \\ 
Cycle               & & Yes                     & & Each vertex has degree $2$ \\
Clique              & & Odd index only   & & $\k_{2n}$ has odd vertex-degrees \\
2-Dim.~Mesh   & & No                      & & Each non-corner top, side vertex has degree $3$ \\                      
2-Dim.~Torus   & & Yes                     & & Each vertex has degree $4$ \\
Hypercube       & & Even index only & & Each vertex of $\q_n$ has degree $n$ \\
de Bruijn           & &                           & &  \\
\hspace*{.15in}directed  & & Yes  & & Vertices have equal {\sc indegree} and {\sc outdegree} \\
\hspace*{.15in}undirected
         & &  Yes & & For $\d_n$: Vertices $\overline{0}$, $\overline{1}$ have
                                                  degree $2n-2$; \\
         & &         & & \hspace*{.5in}Each other vertex has degree $2n$
\end{tabular}
%}
\end{corol}

%%%%%%%%%%%%%%%%%%%%%%%%%%%%%%%%%%%%%%%%%

\subsection{Hamiltonian Paths and Cycles/Tours}
\label{sec:Hamiltonian-cycle}

We turn now to the problem of determining when a connected graph $\g$ has a {\it Hamiltonian cycle}---and the allied problem of finding such a cycle when one exists.  Supplementary material related to cycles in general graphs, as well as in our ``named'' graphs can be found in the survey \cite{Rosenberg91}.

\medskip

One can envision a number of benefits rendered accessible by the presence of a Hamiltonian cycle in a graph $\g$.  Most obviously, the cycle specifies a tour of $\g$ (or of a map whose structure $\g$ abstracts) which visits each of $\g$'s vertices precisely once.  This is the sense in which the cycle ``summarizes" $\g$'s traversal structure.

\subsubsection{More inclusive notions of Hamiltonianicity}

\index{graph!Hamiltonianicity}

Many graphs---even ones with ``nice'' structures---do not admit Hamiltonian cycles.  As an exercise, the reader can generate {\it mesh-graphs} (Section~\ref{sec:mesh}) that admit no Hamiltonian cycle.  The existence of such non-Hamiltonian graphs has spawned several
independent paths of inquiry.  One path seeks ``modest'' ways to weaken the property of {\it Hamiltonianicity} in a way that retains many of the benefits of Hamiltonianicity's while encompassing a broader range of graph structures.  We describe two avenues toward weakened, more inclusive, notions of Hamiltonianicity.

\medskip

\index{graph!Hamiltonian path} \index{Hamiltonian path}

\noindent {\it Be satisfied with paths, rather than cycles}.
A {\it Hamiltonian path} in a graph $\g$ is a path that passes through each of $\g$'s vertices precisely once.  Hamiltonian paths can easily be shown to be a strictly weaker notion than Hamiltonian cycles:  Clearly, every graph which admits a Hamiltonian cycle also admits a Hamiltonian path: one just drops any single edge of such a cycle to obtain such a path.  However, there are many graphs which admit a Hamiltonian path but {\em not} any Hamiltonian cycle. Mesh-graphs can supply easy examples.  You have produced (in the preceding paragraph) mesh-graphs
which admit no Hamiltonian cycle, even though every mesh-graph easily admits a Hamiltonian path---just traverse rows {\it seriatim}, in alternating directions.

\medskip

\index{graph!Hamiltonian $k$-cycle}

\noindent {\it Be satisfied with paths constructed from short paths, rather than edges}.
A Hamiltonian cycle in graph $\g$ is a circular enumeration of $\g$'s vertices in which adjacent vertices are at unit distance from one another---i.e., are connected by an edge.  We can weaken (or, generalize) this notion by introducing the notion of a {\it Hamiltonian $k$-cycle} in $\g$, for any positive integer $k$:  This is a circular enumeration of $\g$'s vertices in which adjacent vertices are at distance $\leq k$ from one another---so a Hamiltonian $1$-cycle is what we have been calling a Hamiltonian cycle.  The following result shows that one need not let $k$ be very big before one finds that all connected graphs admit Hamiltonian $k$-cycles.  The proof of this result is based on traversals of a graph's spanning trees, an algorithmic stratagem which is beyond the current text.

\begin{prop}
\label{thm:weak-Hamiltonianicity}
{\bf (a)} {\rm \cite{ChartrandK69}}
Every connected graph admits a Hamiltonian $3$-cycle.

\smallskip

\index{graph!$2$-connected} \index{graph!biconnected}

\noindent {\bf (b)} {\rm  \cite{Fleischner74}}
Let $\g$ be any graph that is {\em $2$-connected} (or, {\it biconnected}) in the sense  that, for every two vertices, $u, v \in \n_{\fg}$, there exist at least two vertex-disjoint paths in $\g$ that connect $u$ and $v$.  Every biconnected graph $\g$ admits a Hamiltonian $2$-cycle.
\end{prop}

\subsubsection{Hamiltonianicity in ``named'' graphs}
\label{sec:hamiltonian-named-graphs}

Yet another direction of inquiry is to determine whether specific graphs of interest are Hamiltonian.  We illustrate this avenue by reviewing the five ``named" families of graphs
Section~\ref{sec:graphs-important-families}.

\begin{prop}
\label{thm:named-graph-Hamiltonian}
\label{thm:deBruijn-Hamiltonian}
{\bf (a)}
Every cycle-graph $\cc_n$ is Hamiltonian.

\smallskip

\noindent {\bf (b)}
Every clique-graph $\k_n$ is Hamiltonian.

\smallskip

\noindent {\bf (c)}.1.
For all $m,n$: the mesh-graph $\m_{m,n}$:

(i)  is path-Hamiltonian.

(ii) contains no odd-length cycle; hence, is not Hamiltonian if $mn$ is odd.

(iii) is Hamiltonian whenever $mn$ is even 

\smallskip

\noindent {\bf (c)}.2.
For all $m,n$: the torus-graph $\widetilde{\m}_{m,n}$ is Hamiltonian.

\smallskip

\noindent {\bf (d)}
Every hypercube $\q_n$  is Hamiltonian.

\smallskip

\noindent {\bf (e)}
Every de Bruijn network $\d_n$ is both Hamiltonian and directed-Hamiltonian.
\end{prop}
 
\index{graph!pancyclic} \index{digraph!directed-pancyclic} \index{pancyclic}
\index{directed-pancyclic}

Before we embark on our proof Proposition~\ref{thm:named-graph-Hamiltonian}, we remark that some of the assertions in the Proposition can be strengthened to assert the presence in the target graph of cycles of many lengths; cf.~\cite{Rosenberg91}.  In particular, we devote Section~\ref{Appendix:deBruijn-Pancyclic} to proving that the $2^n$-vertex de Bruijn network $\d_n$ is {\it directed-pancyclic}---meaning that it contains directed cycles of {\em every} length, from length $1$ to length $2^n$.

\begin{proof}[Proposition~\ref{thm:named-graph-Hamiltonian}]
\noindent {\bf (a)}
This is a tautology, by definition of $\cc_n$.

\medskip

\noindent {\bf (b)}
This is immediate because, by definition, $\k_n$ contains every $n$-vertex graph---including $\cc_n$---as a subgraph.

\medskip

\noindent {\bf (c)}.1.i.
As we noted earlier (when discussing Hamiltonian paths vs.~cycles), one can ``snake'' a path through $\m_{m,n}$, row by row, traversing adjacent rows in alternating directions.

\smallskip

\index{graph!bipartite}

\noindent {\bf (c)}.1.ii.
This is a consequence of the fact that $\m_{m,n}$ is {\it bipartite}, so that every cycle has even length.

\smallskip

\noindent {\bf (c)}.1.iii.
We sketch the construction of a Hamiltonian cycle in $\m_{m,n}$ when $mn$ is even.  Say, with no loss of generality that $m$ is even, so that $\m_{m,n}$ has an even number of rows.  Temporarily remove column $1$ of $\m_{m,n}$, and construct the ``snaking'' Hamiltonian path described in part {\bf (c)}.1.i of this proof.  Because $m$ is even, this path begins and ends in column $2$ of $\m_{m,n}$.  One can, therefore, reinstate column $1$ and use it to connect the ends of the ``snaking'' Hamiltonian path.  This describes a Hamiltonian cycle in $\m_{m,n}$.

\smallskip

\noindent {\bf (c)}.2.
When $mn$ is even, the Hamiltonianicity of $\widetilde{\m}_{m,n}$ follows from the fact that $\m_{m,n}$ is a spanning subgraph of $\widetilde{\m}_{m,n}$.  (Think about it!)  When $mn$ is odd, one needs just traverse $\widetilde{\m}_{m,n}$'s vertices row by row, going to the cyclically next vertex after completing each row (see Fig.~\ref{fig:HamiltonTorus}).  Details are left to the reader.
\begin{figure}[hbt]
\begin{center}
       \includegraphics[scale=0.45]{FiguresGraph/HamiltonTorus}
\caption{Illustrating the principles for building Hamiltonian cycles in $m \times n$ tori when both $m$ and $n$ are even (left) and when both are odd (right).}
  \label{fig:HamiltonTorus}
\end{center}
\end{figure}

\smallskip

\index{Gray code} \index{binary reflected Gray code} \index{Gray, Frank}

\noindent {\bf (d)}
One can craft a Hamiltonian cycle in $\q_n$ by generating an {\it order-$n$ binary reflected Gray code}---so named for its inventor, Bell Laboratories researcher Frank Gray; see \cite{PetersonW81}.   Such a ``code'' is a cyclic enumeration of all $2^n$ binary strings of length $n$, having the property that cyclically adjacent strings differ in only one bit-position.  Length-$n$ 
strings $x_i$ and $x_j$ are {\it cyclically adjacent} in the Gray code $\langle x_0, \ x_1, \ldots, x_{2^n-1} \rangle$ if $j = i+1 \bmod 2^n$.
\index{strings!cyclically adjacent} 

\smallskip

\noindent
It is computationally easy to generate an order-$n$ Gray code from an order-$(n-1)$ Gray code, as follows.

\smallskip

We note first that the order-$1$ code is just the sequence $\langle 0, 1 \rangle$.

\smallskip

Inductively, to generate the order-$(k+1)$ Gray code from the order-$k$ code:
\begin{itemize}
\item
Concatenate the order-$k$ code with a {\em reversed} copy of itself.  (It is the code-sequence that is reversed, not the individual strings.  For instance, as we go from the order-$2$ code $\langle x_0, \ x_1, \ x_2, \ x_3 \rangle$ to the order-$3$ code, we concatenate that sequence with $\langle x_3, \ x_2, \ x_1, \ x_0 \rangle$.)

\item
Augment each length-$k$ string in one copy of the order-$k$ Gray code to length $(k+1)$ by prepending a $0$ to each string; and, augment each length-$k$ string in the other (reversed) copy of the order-$k$ Gray code to length $(k+1)$ by prepending a $1$ to each string.
\end{itemize}
Table~\ref{tab:gray-code} illustrates Gray codes of the first few orders.

\begin{table}
\caption{Gray codes of orders $1$--$4$.}
\label{tab:gray-code}
\begin{equation}
%\label{eq:gray-code}
 {\small
\begin{array}{|c|c|c|c|}
\hline
\mbox{Order } \ 1
  & \mbox{Order } \ 2
  & \mbox{Order } \ 3
  & \mbox{Order } \ 4 \\
\hline
0   & 00   & 000  &  0000 \\ 
1   & 01   & 001  &  0001 \\
    & 11   & 011  &  0011 \\
    & 10   & 010  &  0010 \\
    &      & 110  &  0110 \\
    &      & 111  &  0111 \\
    &      & 101  &  0101 \\
    &      & 100  &  0100 \\
    &      &      &  1100 \\  
    &      &      &  1101 \\  
    &      &      &  1111 \\  
    &      &      &  1110 \\  
    &      &      &  1010 \\  
    &      &      &  1011 \\  
    &      &      &  1001 \\  
    &      &      &  1000 \\  
\hline
\end{array} }
\end{equation}
\end{table}

We now sketch a proof that for each index $n \in \N^+$, the order-$n$ Gray code sequence specifies a Hamiltonian cycle in $\q_n$; exercises will give the reader the opportunity to fill in details.  We verify the following two assertions in turn:
\begin{enumerate}
\item
{\it The order-$n$ Gray code contains all $2^n$ length-$n$ binary strings.}
\item
{\it Every pair of cyclically adjacent strings in the order-$n$ Gray code differ in a single bit-position.}
\end{enumerate}
{\it Verification}.

\noindent {\it Assertion} 1.
We sketch an inductive verification.

The order-($n=1$) Gray code consists of the two distinct strings $0$ and $1$.

Assume that the assertion holds for all orders $n \leq k$.

The order-$(k+1)$ code is obtained by taking two copies of the order-$k$ code and prepending $0$ to the strings in one copy and $1$ to the strings in the other copy.  The $2^k$ distinct binary strings from the order-$k$ code thereby produce $2^{k+1}$ distinct binary strings in the order-$(k+1)$ code.

\medskip

\noindent {\it Assertion} 2.
We distinguish three situations.  Let the adjacent strings be string $x$, which appears in position $i$ of the code, and string $y$, which appears in position $i+1 \bmod 2^n$ of the code.
  \begin{itemize}
  \item
Say that $i = 2^n-1$.  In this case $x$ is the last string in the code, and $y$ is the first string.  By the ``reflective'' nature of the construction of the code, we know that $x = 1z$ and $y = 0z$ for some length-$(n-1)$ binary string $z$.  Strings $x$ and $y$ therefore differ in precisely one bit-position, namely, bit-position $0$.

  \item
Precisely the same argument shows that when $i = 2^{n-1} -1$, strings $x$ and $y$ again differ precisely in bit-position $0$.

  \item
In all other cases---i.e., when $i \in \{0,1, \ldots, 2^n-1\} \setminus \{2^{n-1} -1, 2^n-1\}$---strings  $x$ and $y$ share the same first bit-position.  In fact, for some bit $\beta \in \{0,1\}$, $x = \beta u$ and $y = \beta v$ for length-$(n-1)$ binary strings $u$ and $v$ which are cyclically adjacent in the order-$(n-1)$ Gray code.  By an inductive argument which we leave to the reader, $u$ and $v$ differ in precisely one bit-position---which means that $x$ and $y$ also differ in precisely one bit-position.
  \end{itemize}
The previous analysis is summarized in Fig.~\ref{fig:HamiltonHypercude} for $n=4$.
  \begin{figure}[hbt]
\begin{center}
       \includegraphics[scale=0.5]{FiguresGraph/HamiltonHypercube}
       \caption{A Hamiltonian cycle (bold) in $\q_4$ built using a binary reflected Gray code.}
  \label{fig:HamiltonHypercude}
\end{center}
\end{figure}

\noindent {\bf (e)}
de Bruijn networks require the most complex analysis of Hamiltonianicity among our  ``named" graphs .  We begin to deal with them by restating the result in more detail in order to establish some notation.

\medskip

\noindent
{\em
For all $n \in \N^+$, $\d_n$ contains a {\em directed Hamiltonian cycle}, i.e., a length-$2^n$ 
directed cycle of the form
\begin{equation}
\label{eq:deBruijn-cycle}
 x \ \rightarrow \ y_1 \ \rightarrow \ y_2 \ \rightarrow \cdots  \rightarrow \ y_{2^n-1} \ \rightarrow \ x
\end{equation}
which contains every vertex of $\d_n$ precisely once; i.e.:
\begin{itemize}
\item
$\{x, \ y_1, \ y_2, \ldots, \ y_{2^n-1}\} \ = \ \n_{\fd_n}$.
\item
All of the vertices $y_j$ that appear in cycle (\ref{eq:deBruijn-cycle}) differ from $x$ and from each other.
\end{itemize}
}

The simplest proof of this result has two steps.  The first step introduces a significant, rather
sophisticated, concept, the {\it line (di)graph} of a (di)graph.

\bigskip

\index{line graph} \index{line digraph}

\noindent {\bf (1)}
For any directed graph $\g$, the {\it line digraph} of $\g$, denoted $\Lambda(\g)$, is the following directed graph.
\begin{itemize}
\item
The vertices of $\Lambda(\g)$ are the arcs of $\g$:
\[ \n_{{\Lambda}({\cal G})} \ = \ \a_{\fg} \]

\item
For each pair of arcs of $\g$ of the form
\[ \big[a_{x,y} = (x \ \rightarrow \ y) \big] \ \ \ \mbox{ and } \ \ \ 
\big[a_{y,z} = (y \ \rightarrow \ z) \big]
\]
i.e, arcs such that the target of the first arc is the source of the second arc, $\Lambda(\g)$ contains an arc $(a_{x,y} \ \rightarrow \ a_{y,z})$.
\end{itemize}
The relevance of this concept to this section is that the line graph of every de Bruijn network $\d_n$ is the ``next bigger'' de Bruijn network, $\d_{n+1}$.  Let us verify this claim.

\begin{lemma}
\label{thm:deBruin-linegraph}
For all $n \in \N^+$, $\d_{n+1}$ is the line digraph of $\d_n$; symbolically,
\[ \d_{n+1} \ = \ \Lambda(\d_n) \]
\end{lemma}

\begin{proof}[Lemma~\ref{thm:deBruin-linegraph}]
Each vertex of $\Lambda(\d_n)$ is an arc of $\d_n$, hence has the form
\[ (\beta x \ \rightarrow \ x \gamma) \]
where $x$ is a length-$(n-1)$ binary string and $\beta, \gamma \in \{0,1\}$.  Let us associate vertex $\beta x \gamma$ of $\d_{n+1}$ with this vertex of $\Lambda(\d_n)$.  See Fig.~\ref{fig:dBlabelEdge-App}.
\begin{figure}[hbt]
\begin{center}
       \includegraphics[scale=0.45]{FiguresGraph/dBlabelEdge}
\caption{Illustrating how to label each arc of a de Bruijn network by concatenating the labels of the vertices incident to the arc and compacting the common intermediate bits.  In the depicted example, the vertex-labels $01$ and $11$ combine to yield the arc-label $011$.}
  \label{fig:dBlabelEdge-App}
\end{center}
\end{figure}

\smallskip

Note first that each arc of $\d_{n+1}$ has the form
\[ (\delta y \varepsilon \ \rightarrow \ y \varepsilon \varphi), \]
where $y$ is a length-$(n-2)$ binary string and $\delta, \varepsilon, \varphi \in \{0,1\}$.  By our association of vertices of $\d_{n+1}$ with arcs of $\d_n$, this arc of $\d_{n+1}$ does, indeed, correspond to two successive arcs of $\d_n$.   The first of these successive arcs {\em enters} vertex $y \varepsilon$ of $\d_n$; the second {\em leaves} that vertex.

\smallskip

Note next that, given any two successive arcs of $\d_n$, say
\[
(\rho \sigma z \ \rightarrow \ \sigma z \tau) \ \ \ \mbox { and } \ \ \
(\sigma z \tau \ \rightarrow \  z \tau \xi)
\]
where $z$ is a length-$(n-2)$ binary string and $\rho, \sigma, \tau, \xi \in \{0,1\}$, there is, indeed, an arc of $\d_{n+1}$ of the form
\[ (\rho \sigma z \tau \ \rightarrow \ \sigma z \tau \xi) \]
This means that the digraph $\d_{n+1}$ is identical to the digraph $\Lambda(\d_n)$, except for a renaming of vertices and arcs.\footnote{Technically, we are asserting that the digraphs ${\cal D}_{n+1}$ and ${\Lambda}({\cal D}_n)$ are {\it isomorphic} to one another.  The topic of graph isomorphism is beyond the scope of this text, but our informal description provides all the details one would need to formalize the described isomorphism.}

\smallskip

The just-described correspondence between the vertices and arcs of digraphs $\d_{n+1}$ and $\Lambda(\d_n)$ completes the proof.  \qed-Lemma~\ref{thm:deBruin-linegraph}
\end{proof}

\medskip

\index{Eulerian cycle} \index{Eulerian tour} 

\noindent {\bf (2)}
The table following Proposition~\ref{thm:eulerian-cycle} contains the seeds of a proof of the following corollary to the proposition.

\begin{corol}
\label{thm:deBruijn-Eulerian}
Every de Bruijn network $\d_n$ admits a (directed) Eulerian cycle.
\end{corol}

This corollary combines with Lemma~\ref{thm:deBruin-linegraph} to complete the proof of 
Proposition~\ref{thm:named-graph-Hamiltonian}(e).  To wit:  Each $\d_n$ is the line-digraph of  $\d_{n+1}$.  Therefore, by definition of  ``line (di)graph'', the fact that $\d_n$ is (directed)-Eulerian
(Corollary~\ref{thm:deBruijn-Eulerian}) means that $\d_{n+1}$ is (directed)-Hamiltonian.  \qed
\end{proof}


\subsubsection{Testing general graphs for Hamiltonianicity}
\label{sec:Hamiltonian-unweighted}

The techniques we use in Section~\ref{sec:hamiltonian-named-graphs} to investigate the Hamiltonianicity of our ``named'' graphs exploit the detailed structures of the individual graphs.  Thus, we cannot expect the proof of Proposition~\ref{thm:named-graph-Hamiltonian} to suggest avenues for determining whether an arbitrary given graph is Hamiltonian.  In fact, quite sophisticated results proved in the early 1970s make a strong mathematical argument that no set of case studies is likely to have a major impact on the problem of testing general graphs for Hamiltonianicity.  This is because, in common with the Satisfibility problem {\sf 3SAT} of Section~\ref{sec:Satisfiability}, the Hamiltonianicity-detection problem is {\sf NP}-complete.  We repeat from our discussion in Section~\ref{sec:Satisfiability} that the details of the theory of {\sf NP}-completeness are beyond the scope of this text, but we do want the reader to recognize the following.
\begin{description}
\item
{\it The problem of deciding, given a graph $\g$ that is presented via a list of vertices and a list of edges, whether $\g$ admits a Hamiltonian path or a Hamiltonian cycle is {\sf NP}-complete.}
\end{description}



%%%%%%%%%%%%%%%%%%%%%%%%%%%%%%%%%%%%%%%%%%%%%

\section{Exercises: Chapter 13}

Throughout the text, we mark each exercise with 0 or 1 or 2 occurrences of the symbol $\oplus$, as a rough gauge of its level of challenge.  The 0-$\oplus$ exercises should be accessible by just reviewing the text.  We provide {\em hints} for the 1-$\oplus$ exercises; Chapter~\ref{ch:Exercises} provides {\em solutions} for the 2-$\oplus$ exercises.  Additionally, we begin each exercise with a brief explanation of its anticipated value to the reader.
 

\begin{enumerate}
\item
{\bf Leveling---hence, $2$-coloring, of torus-graphs}

{\sc Lesson:} Practice reasoning about graphs

\smallskip

{\em Provide a detailed proof of Corollary~\ref{thm:list-2-colorables}(d):  The torus-graph $\widetilde{\m}_{m,n}$ is leveled---hence, $2$-colorable---if and only if $m+n$ is even.}

\medskip
\item
{\bf Appreciating de Bruijn networks}

{\sc Lesson:} Practice reasoning about graphs

\smallskip

Reasoning about de Bruijn networks quickly goes beyond the elementary, but many notions are quite accessible in small, specific examples.  Here are two such, which refer to the order-$3$ de Bruijn network $\d_3$ and the order-$4$ de Bruijn network $\d_4$.

  \begin{enumerate}
  \item
$\oplus$
{\em Identify, by coloring the edges of $\d_3$ and $\d_4$, how each network can be viewed as two trees which are ``embracing" one another.}

\smallskip

\index{Escher, M.~C.}
These drawings are very reminiscent of the lithograph {\it Drawing Hands} by the Dutch artist M.~C.~Escher.  Studying the drawing may be useful in solving this problem.

  \item 
{\em Prove Proposition~\ref{thm:DB-pancyclic} for $\d_3$ and $\d_4$: Describe the eight directed paths in $\d_3$ and the sixteen directed paths in $\d_4$.}
  \end{enumerate}

\medskip
\item
{\bf Colored-triangle subgraphs}

{\sc Lesson:} Practice reasoning about graphs

\smallskip

{\em Prove the following assertion.}

\begin{prop}
Let us be given a copy $\g$ of $\k_6$ each of whose edges has been colored either {\em red} or {\em blue}.  Graph $\g$ contains either a {\em red} ``triangle" or a blue ``triangle" as a subgraph.  (A ``triangle" is a copy of $\cc_3$.)
\end{prop}

\item
{\bf A fundamental result about outerplanar graphs}

{\sc Lesson:} Practice reasoning about graph-theoretic properties.

\smallskip

{\em Prove the following assertion.}

\begin{prop}
The complete bipartite graph $K_{3,2}$ is not outerplanar.
\end{prop}

\smallskip

{\em Hint:} Use the following figure to garner intuition.
\begin{figure}[h]
\begin{center}
        \includegraphics[scale=0.35]{FiguresGraph/outerplanarK3,2init} 
\caption{$K_{3,2}$: Its two types of vertices are identified by color---white and shaded.}
\end{center}
\end{figure}
Perform a case-by-case analysis of all possible ways to distribute $K_{3,2}$'s vertices around a circle.

\medskip
\item
{\bf Outerplanarity and Hamiltonianicity}

{\sc Lesson:} Practice reasoning about graph-theoretic properties.

\smallskip

{\em Prove the following assertion.}

\begin{prop}
Every outerplanar graph is a subgraph of a Hamiltonian graph.
\end{prop}





\end{enumerate}


\subsection{Euler's Formula}

\noindent \textit{The aim.}
Complete a result of the termination with a connected tree.
\medskip

\noindent \textit{The problem.}
%Prove the termination with a connected tree should be an exercise.
Phase 1 of the proof by deconstruction of Euler's formula
\medskip



%%%
\subsection{Spanning Trees}
\label{Exercice:spanningTrees}

%{\Denis I don't think it is mandatory to have such an exercice, the objective is not clear for me
%and it is too complicated -- and more algorithms than Maths...}
%
Recall here the problem
\medskip

There are mainly two ways for constructing such a MST, each one
emphasizes a different propriety of the MST, namely, avoid cycles and
minimize the span.  In both cases, the edges are sorted in increasing
order of weights.  More precisely, the first one constructs a subtree
which partially spans the graph by adding at each step the minimum
neighboring edge while the other add successively the edges of minimal
weights that do not create a cycle.


