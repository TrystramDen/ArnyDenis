%version of 07-02-19
\chapter{Coloring a Planar Graph with Five Colors}
\label{Appendix:5colors}

\begin{prop}[The $5$-Color Theorem for Planar Graphs \cite{Heawood90}]
\label{thm:P-5-colorability}
Every planar graph is $5$-colorable.
\end{prop}


\begin{proof}
The proof combines a sophisticated analysis of the structure of the
planar graph we wish to color with an induction that can be converted
into an efficient algorithm actually achieving the $5$-coloring.

\smallskip

For brevity, let us henceforth discuss only {\em valid} colorings,
i.e., colorings of a graph's nodes in which neighboring nodes get
different colors.

\smallskip

\noindent {\em Base of our induction.}
Because the $5$-clique $\k_5$ is obviously $5$-colorable, so also must
be all graphs having $\leq 5$ nodes.  Therefore, we know that any
non-$5$-colorable graph would have $\geq 6$ nodes.

\smallskip

\noindent {\em Inductive hypothesis.}
Assume, for induction, that every planar graph having $\leq n$ nodes
is $5$-colorable.

\smallskip

\noindent {\em Inductive extension.}
If the proposition were false, then there would exist a planar graph
$\g$ having $n+1$ nodes which is not $5$-colorable.  By
Lemma~\ref{thm:PlanarGraph-degree5}, $\g$ would have a node $v$ of
degree $\leq 5$.  The remainder of the proof focuses on the graph
$\g$, its minimal-degree node $v$, and on $v$'s $(d_v \leq 5)$
neighbors in $\g$.

Now, if there were a coloring of $\g$'s nodes in which $\leq 4$ colors
were used to color $v$'s neighbors, then the following analogue of the
coloring strategy of Proposition~\ref{thm:P-6-colorability} would
produce a $5$-coloring of $\g$.
\begin{enumerate}
\item
Remove node $v$ and its incident edges from $\g$, thereby producing
the $n$-node planar graph $\g'$.
\item
Produce a $5$-coloring of $\g'$ that uses only $4$ colors for the
nodes that are neighbors of $v$ in $\g$.
\item
($a$) Reattach node $v$ and its edges to $\g'$, thereby reconstituting
  $\g$.  ($b$) Color $v$ with whichever of the $5$ available colors is
  not used to color $v$'s neighbors.
\end{enumerate}

In order to proceed in pursuit of a contradiction, we must understand
what structural features of $\g$ make it impossible to use only $4$
colors on $v$'s neighbors when $5$-coloring $\g$.  There are three
important situations to recognize.
\begin{description}
\item[{\sf Case 1}.]
Node $v$ has degree $\leq 4$.

\smallskip

By definition, $\leq 4$ colors are used to color $v$'s neighbors in
this case.
\end{description}
Note that, in all remaining cases, node $v$ has precisely $5$
neighbors---or else, we would have invoked Case 1 to color $\g$ with
$5$ colors.
\begin{description}
\item[{\sf Case 2}.]
For some $5$-coloring of $\g$, $\geq 2$ neighbors of $v$ get the same
color.

\smallskip

Because $v$ has exactly $5$ neighbors, in this case, only $4$ colors
are used to color these neighbors.
\end{description}
In all remaining cases, the $5$ neighbors of $v$ receive distinct
colors.
\begin{description}
\item[{\sf Case 3}.]
For some $5$-coloring of $\g$, some two neighbors of $v$, call them
$v_1$ and $v_2$, reside in distinct components of $\g$ once $v$ and
its incident edges are removed from $\g$.

\smallskip

As before, let $\g'$ be the (in this case, disconnected) graph that
results when $v$ and its incident edges are removed from $\g$.  For $i
= 1,2$ Let $\g_i$ be the component of $\g'$ that contains node $v_i$.

Say that, under the $5$-coloring of $\g$ that we are focusing on,
$v_1$ is colored {\it red} and $v_2$ is colored {\it green}.

Let us recolor the nodes of $\g_1$ so that node $v_1$ is now colored
{\it green}.  (One needs only switch the colors {\it red} and {\it
  green} in the existing coloring of $\g_1$.)  It is always possible
to do this in a way that does not affect the valid coloring of $\g_2$
because $\g_1$ and $\g_2$ are mutually disjoint.

Once we have thus-recolored $\g_1$, we have a $5$-coloring of $\g$ for
which Case 2 holds.  (In fact, we can color node $v$ {\em red} when we
reattach it to $\g'$.)
\end{description}

\noindent
We now see that Cases 1--3 cannot prevent us from $5$-coloring $\g$, so
we are left with the following minimally constrained situation.
\begin{description}
\item[{\sf Case 4}.]
\begin{itemize}
\item
Every minimum-degree node of $\g$ has $5$ neighbors.

For the minimum-degree node $v$, let us call these neighbors $v_1$,
$v_2$, $v_3$, $v_4$, $v_5$, in clockwise order within the planar
drawing.
\item
In every $5$-coloring of $\g$, the neighbors of every minimum-degree
node receive distinct colors.

For node $v$, let us say that neighbor $v_i$ receives color $c_i$.
\end{itemize}
{\Denis Check carefully the ref to the figures...}

\begin{figure}[hbt]
\begin{center}
       \includegraphics[scale=0.4]{FiguresGraph/5colorsCase1}
\caption{to be completed}
  \label{fig:5colorsCase1}
\end{center}
\end{figure}
The leftmost graph in Fig.~\ref{fig:5colorsCase1} depicts the portion of $\g$ comprising
node $v$ and its neighbors.  In the figure, we use integer $i$ to
denote, ambiguously, node $v_i$ and its assigned color $c_i$.  The
question mark ``?'' that ``colors'' node $v$ indicates that we do not
yet know what color to assign to $v$.  The other two graphs in the
figure depict schematically how we have dealt with Case 3 above.
{\Denis Clarify which are the concerned graphs here}
\begin{itemize}
\item
All neighbors of node $v$ remain in the same component of $\g$ when
$v$ and its incident edges are removed.
\end{itemize}
\end{description}
To analyze Case 4, we focus on nodes $v_1$ and $v_3$ in Fig.~\ref{fig:5colorsCase2}.
Importantly, these nodes have received distinct colors ($c_1$ and $c_3
\neq c_1$, respectively), and these nodes are not adjacent to one
another as one makes a clockwise sweep around node $v$.
\begin{figure}[hbt]
\begin{center}
       \includegraphics[scale=0.4]{FiguresGraph/5colorsCase2}
\caption{{\Denis to be completed}}
  \label{fig:5colorsCase2}
\end{center}
\end{figure}

Now take $\g$ and focus only on the nodes that are colored $c_1$ or
$c_3$ (as are $v_1$ and $v_3$, respectively) and on the nodes that are
colored $c_2$ or $c_4$ (as are $v_2$ and $v_4$, respectively).  One
sees from Fig.~\ref{fig:5colorsCase2} that:
\begin{itemize}
\item
$\g$ can, {\em but need not}, contain a path whose nodes alternate
  colors $c_1$ and $c_3$---call this a ``$c_1$-$c_3$ path'' between
  nodes $v_1$ and $v_3$.
\item
$\g$ can, {\em but need not}, contain a path whose nodes alternate
  colors $c_2$ and $c_4$---call this a ``$c_2$-$c_4$ path'' between
  nodes $v_2$ and $v_4$.
\item
$\g$ {\em cannot} contain both of the paths just described, i.e., a
  $c_1$-$c_3$ path between $v_1$ and $v_3$ {\em and} a $c_2$-$c_4$ path
  between $v_2$ and $v_4$.

{\em These two paths, if they existed, would cross one another---which
  is forbidden because $\g$ is a {\em planar} graph.}  See Figure 2.
\end{itemize}
It follows that {\em either} $\g$ does not contain a $c_1$-$c_3$ path
between $v_1$ and $v_3$ {\em or} $\g$ does not contain a $c_2$-$c_4$
path between $v_2$ and $v_4$.  Say, with no loss of generality, that
the former path does not exist.  Then we can switch colors $c_1$ and
$c_3$ beginning with node $v_1$ and obtain a coloring of $\g$ in which
$v_1$ and $v_3$ both receive the color $c_3$.  We can then proceed as
in Case 2 to get a $5$-coloring of $\g$.

This four-case analysis shows that we can always produce a
$5$-coloring of $\g$, which completes the proof.  \qed
\end{proof}
