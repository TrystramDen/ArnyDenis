%version of 08-12-20
\chapter{$\oplus \oplus$ Signed-Digit Numerals: Carry-Free Addition}
\label{ch:carry-free}

\noindent \fbox{
\begin{minipage}{0.96\textwidth}
{\bf Topic-specific reference}

\smallskip

K.~Hwang (1979):
{\it Computer Arithmetic: Principles, Architecture, and Design}.
John Wiley \& Sons, New York.

\smallskip

D.E.~Knuth (1969):
{\it The Art of Computer Programming, Vol.~2: Seminumerical
  Algorithms.}  Addison-Wesley, Reading, Mass.
\end{minipage}
}


\bigskip

\index{carry-ripple adder}

\noindent
Say that you have a box that contains a {\it counter}.  There are two buttons on top of the box: one {\em red} and one {\em green}.  Each time the green button is pushed, the counter increments---i.e., adds $+1$ to---its tally; each time the red button is pushed, the counter
decrements---i.e., adds $-1$ to---its tally.  Now, in order for the tally on the counter to {\em always be correct}, you must insert delays between button-pushes: you must always wait until the
electronic circuitry inside the box settles into a stable configuration with all necessary carries and borrows up to date.  Now, if the electronic circuitry that implements the counter was designed to mimic the binary representation of the number---in technical jargon,  the counter implements a {\it carry-ripple adder}; see [Hwang, 1979] for details---then you observe the following.  While you keep pushing only the green (increment) button, you do not feel that the delays between
button-pushes are onerous, for the reason exposed in the following analysis.
\begin{itemize}
\item
Roughly half the time, you have no enforced delays between successive button-pushes: the update of the tally engenders no carry.
\item
Roughly one-quarter of the time, you have a minuscule delay: the update engenders a carry of only one place.
\item
Roughly one-eighth of the time, you have a slightly longer delay: the update engenders a carry of two places.
\item
\ldots Continuing this progression: For each integer $k$ roughly the fraction $2^{-k}$ of the button-pushes engender a delay commensurate with a carry of $k-1$ places.
\end{itemize}
Stated differently, the {\em average} delay you must suffer is bounded by the time required for a two-place carry.  In fact, you can even improve on the described pattern of delays, by using more
sophisticated circuitry for your counter's adder.  If you replace the carry-ripple adder by a {\it carry-lookahead adder}---see [Hwang, 1979] for details---you can thereby reduce the aggregate
number of carries engendered by the first $n$ button-pushes from the $\Theta(n^2)$ carries of the carry-ripple adder to $O(n \log n)$ carries for a carry-lookahead adder.

\smallskip

However, the picture changes markedly as soon as you (or an opponent) start to use the red button as well as the green one.  If you wait until the counter has tallied some number of green-button pushes of the form $2^k$, hence contains the binary numeral $100 \cdots 00$ with $k$ $0$s, then a push of the red button engenders a delay of $k$ carry-units---and a subsequent push of the green button incurs the same delay!  In fact, if you (and your opponent) begin to toggle the two buttons---one red push, then one green push, then one red push, then one green push, \ldots---then you incur a delay of $k$ carry-units with each successive button-push.  If you do this for long enough, then your {\em average} delay starts growing toward $k$ carry-units.

\medskip

\index{signed-digit number representation} \index{carry-free addition}
\index{positional number representation!signed-digit representation}

If (the risk of) long delays such as those just described is more than you want to deal with, then you can employ mathematical and electronic technology to replace your counter's adder with one that (almost) {\em eliminates these delays completely}.  The ``silver bullet'' resides in using a {\em signed-digit} (positional) number representation to design a counter that implements {\it carry-free addition}.  By changing the form of the numerals your counter uses, you can design adders whose constituent digit-adders can operate {\em in parallel}.  While these digit-adders are roughly twice as complex as those of the more familiar carry-ripple adders, they do guarantee bounded---i.e., $O(1)$---delay between button-pushes, no matter how long you keep
pushing a button or which one you choose to push.

\medskip

\index{base-$b$ signed-digit number system}
\index{radix-$b$ signed-digit number system}

The signed-digit number systems which we describe now\footnote{There are many such systems and many places to learn about them; see, e.g., the encyclopedic work [Knuth, 1969] on ``seminumerical algorithms'' and the comprehensive text [Hwang, 1979] on computer arithmetic.}~are positional systems of the same genre as -ary systems: Their numerals are also evaluated by the {\sc val} function of (\ref{eq:real-numeral-number}).  These systems achieve their buffered carries via augmented digit-sets that include {\em negative} digits as well as positive ones.  The base-$b$ (or, {\it radix}-$b$) redundant number system  has the following set of digits.\footnote{($a$) Recall that the overline notation, as in ``$\overline{b-2}$'' is to indicate that we are referring to a single digit.  ($b$) We add parentheses when we describe $\widehat{B}_b$ to enhance legibility.}
\[ \widehat{B}_b \ = \ \{(-\overline{b-1}), (-\overline{b-2}), \ldots,
(-1), 0 , 1, \ldots, \overline{b-2}, \overline{b-1}\}
\]
For reasons that are purely technical, the simple carry-free adder that we describe requires that the number base $b$ be no smaller than $3$.  For the smallest relevant base, $b=3$, one can use the digit-set $\widehat{B}_3 \ = \ \{ (-2), (-1), 0, 1, 2\}$ (parentheses added to enhance legibility).  The ``redundancy'' in this system is witnessed by facts such as the following:  The number $1$ is the numerical value of {\em every} numeral of the form
\[ 1 \ (-2) \ (-2) \ \cdots \ (-2) \ (-2) \]
because the $d$-digit instance of these numerals has numerical value

\smallskip

$\displaystyle \mbox{\sc val}_3 \big( 1 \ (-2) \ (-2) \ \cdots \ (-2) \ (-2) \big)$

\smallskip

\hspace*{.25in}$= \ 3^{d-1} \ - \ 2 \cdot \big(3^{d-2} \ + \ 3^{d-3} \ + \cdots + \ 3 \ + \ 1 \big)$

\hspace*{.25in}$= \ 1$

\bigskip

We now provide a schematic description\footnote{Detailed descriptions of both genres of adders---including both operational and implementational matters---can be found in sources such as [Hwang, 1979].}~of ($a$) a carry-ripple adder and ($b$) a carry-free adder adding the $b$-ary integer-specifying numerals
\[ x_n x_{n-1} \cdots x_1 x_0 \ \ \ \mbox{ and } \ \ \ 
y_n y_{n-1} \cdots y_1 y_0
\]
In both descriptions, we describe the function of the $i$th digit-adder, emphasizing the information that it receives from the $(i-1)$th digit-adder and the information that it transmits to the $(i+1)$th digit-adder.  Of course, this information is at the heart of the issue of carries or no-carries.

\begin{figure}[hbt]
\[
\begin{array}{|lcccr|}
\hline
 & & x_i \ \ \ y_i & & \\
 & & \downarrow \ \ \ \downarrow & & \\
c_{i+1} \ = \ \big( (x_i + y_i + c_i) \ominus \bar{b} \big)
 & & \longleftarrow \ \fbox{$x_i + y_i + c_i$} \ \longleftarrow & &  c_i 
  \\
 & & \downarrow & &
  \\
 & & z_i \ = \ \min \big( \bar{b}, \ x_i + y_i + c_i \big)  & & 
  \\
\hline
\end{array}
\]
\caption{Digit $i$ of a $b$-ary carry-ripple adder.  The input digits $x_i$ and $y_i$ are summed with the in-carry $c_i$.  If the sum can be represented by a single $b$-ary digit, then it is the output from this digit-adder, and the out-carry $c_{i+1}$ is set to $0$.  If the sum is too large, then the maximum $b$-ary digit $\bar{b}$ is the output from this digit-adder, and the out-carry $c_{i+1}$ is set to the portion of the sum that exceeds $\bar{b}$}
\label{fig:carry-ripple-digit}
\end{figure}

\bigskip

\index{carry-ripple adder}

\noindent {\it A Carry-ripple adder}.  See Fig.~\ref{fig:carry-ripple-digit}.
\begin{itemize}
\item
All digits---input digits $x_i$ and $y_i$, and carry digit $c_i$ (for $i>0$) come from $B_b = \{ 0, 1, \ldots, \bar{b}\}$; $c_0 = 0$.
\item
Addition is a {\em single-pass} process.  For each digit-index $i$:
  \begin{itemize}
  \item
Digit-adder $i$ admits inputs $x_i$ and $y_i$ (from the ``outside'') and carry-in $c_i$: When $i=0$, $c_0 = 0$ by convention; for all other indices $i$, $c_i$ is the carry-in from digit-adder $i-1$.
  \item
The $i$th sum-digit, $z_i$, and the carry-out, $c_{i+1}$, are evaluated as follows.
\[
\begin{array}{ccccccc}
 z_i    & = &  \min \big( \bar{b}, \ x_i + y_i + c_i \big) & & & & \\
c_{i+1} & = &  (x_i + y_i + c_i) \ominus \bar{b} &
\mbox{ where } &
m \ \ominus \ n &  \eqdef & \left\{
{\displaystyle
\begin{array}{cl}
m-n & \mbox{ if } \ \ m \geq n \\
0   & \mbox{ if } \ \ m \leq n
\end{array}
} \right.
\end{array}
\]
  \end{itemize}
\end{itemize}

\bigskip


\begin{figure}[hbt]
\[
\begin{array}{|llcccr|}
\hline
 & & & x_i \ \ \ y_i & & \\
 & & & \downarrow \ \ \ \downarrow  & & \\
\mbox{\bf stage $1$:   } & t_{i+1} & \longleftarrow &
\fbox{
\begin{tabular}{c}
Compute \\
$\tau_i$ \ \ and \ \ $t_{i+1}$ \\
via the relation \\
$b \cdot t_{i+1} + \tau_i \ = \ x_i + y_i$
\end{tabular}
} & & \\
  & & & \downarrow & & \\  
  & & & \tau_i & & \\
  & & & \downarrow & & \\ 
\mbox{\bf stage $2$:   } & & &  \fbox{Compute $\tau_i + t_i$} & \longleftarrow & t_i 
  \\
  & & & \downarrow & &
  \\
  & & & z_i & &
  \\
\hline
\end{array}
\]
\caption{Digit-adder $i$ of a $(b > 2)$-ary carry-free adder.  The outgoing {\it transfer digit} $t_{i+1}$ and the {\it tentative sum} $\tau_i$ are computed from the input digits $x_i$ and $y_i$, using the relation specified in the upper box.  These results are transmitted, respectively, to stage $1$ of digit-adder $i+1$ and stage $2$ of the current digit-adder $i$}
\label{fig:carry-free-digit}
\end{figure}

\index{carry-free adder} \index{carry-free adder!tentative-sum digits} 
\index{carry-free adder!transfer digit} 

\newpage

\noindent {\it A Carry-free adder}. 
See Fig.~\ref{fig:carry-free-digit}.
\begin{itemize}
\item
This simple carry-free adder operates in {\em two stages}.  It operates on base-$b$ numerals but requires $b > 2$ (for technical reasons).
\item
There is no ``carry digit''.  In its place, there is a {\it transfer digit} $t_i \in \{-1, 0, 1\}$ which is generated in stage 1 of the adder and consumed in stage 2.
\item
In addition to transfer digits, there are {\it tentative sum} digits $\tau_i$, which are also generated in stage $1$ and consumed in stage $2$.
\end{itemize}
\begin{enumerate}
\item
Stage $1$ of the adder
\begin{enumerate}
\item
computes {\it transfer digit} $t_{i+1} \in \{-1, 0 , 1\}$ and {\it tentative-sum digit} $\tau_i$ via the equation
\[ b \cdot t_{i+1} + \tau_i \ = \ x_i + y_i \]
\item
transmits $t_{i+1}$ to digit-adder $i+1$
\item
transmits $\tau_i$ to stage $2$.
\end{enumerate}
\item
Stage 2 of the adder computes the $i$th output digit $z_i$ by adding $\tau_i$ to $t_i$ (which came from digit-adder $i-1$).
\end{enumerate}
The important fact to note is that this adder does not have a chain of propagated carries!  Digit-adder $i$ computes transfer digit $t_{i+1}$---which is the only information that it transmits to
digit-adder $i+1$---based only on the $i$th input digits $x_i$ and $y_i$.  Consequently: {\em All of the adder's digit-adders can operate in parallel!}
