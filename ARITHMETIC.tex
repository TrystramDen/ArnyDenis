%version of 03-14-18

\chapter{ARITHMETIC}
\label{ch:arithmetic}

Many, perhaps most, of us take for granted the brilliant notations
that have been developed for the myriad arithmetic constructs that we
use daily.  Our mathematical ancestors have bequeathed us notations
that are not only perspicuous but also convenient for computing and
for discovering and verifying new mathematical truths.  This chapter
is dedicated to sharing this legacy with the reader.

\section{Numbers and Numerals}
\label{sec:numbers-numerals}

Every reader will be familiar with the notion of {\it number} and with
the familiar strings, called {\it numerals}, that name numbers within
positional number systems.\index{numbers vs.~numerals}
\begin{quote}
Numbers and numerals embody what is certainly the most familiar
instance of a very important dichotomy that pervades our intellectual
lives: the distinction between objects and their names:

{\em Numbers are objects.  Numerals are the names we use to refer to
  and manipulate numbers.}

This is a crucially important distinction!  You can ``touch'' a
numeral: break it into pieces, combine two (or more) numerals via a
large range of operations.  Numbers are intangible abstractions: you
cannot compute with them.
\end{quote}
In our daily commerce, we typically deal with numerals formed within a
{\it base-$b$ positional number system,}\index{numerals in a base-$b$
  positional number system}
%
i.e., by strings of {\it digits}, often embellished with other
symbols, such as a {\em radix point}.  We describe such systems in
detail in Section~\ref{sec:Numerals}.A.  For now, we settle for a few
examples: $123456789$ (base $10$, or, {\it decimal}), $10111.001$
(base $2$, or, {\it binary}), or $0.1267 \times 8^{24}$ (base $8$, or,
{\it octal}).\footnote{The use of a period as the radix point is a US
  convention; in much of Europe, a comma denotes the radix point.}


\subsection{Numbers}\index{numbers}
\label{sec:numbers}

We begin our study of arithmetic notions with a short taxonomy of
numbers, the objects that arithmetic was invented to explicate and
exploit.  Although we assume that the reader is familiar with the most
common classes of numbers, we do spend some time highlighting
important features of each class, partly, at least, in the hope of
heightening the reader's interest in this most basic object of
mathematical discourse.

We present the four most common classes of numbers in what is almost
certainly the chronological order of their discovery/invention.
\begin{quote}
Did humans invent these classes of numbers to fill specific needs, or
did we just discover their pre-existing selves as needs prompted us to
search for them?  The great German mathematician Leopold Kronecker, as
cited on page 477 of \cite{Bell86}, shared his viewpoint on this
question: ``God made the integers; all else is the work of man.''
\end{quote}
A pleasing narrative can be fabricated to account for our multi-class
system of numbers.  In the beginning, the story goes, we needed to
count things (sheep, bottles of oil, weapons, \ldots), and the
positive {\it integers}\index{number!integer} were discovered.  As
accounting practices matured, we needed to augment this class with
both zero\index{number!zero ($0$)} ($0$) and the negative
integers\index{number!negative}---and the full class of integers was
born.  As society developed, we had to start sharing subdivisible
materials, so we needed to invent the {\it rational
  numbers}.\index{number!rational} Happily for the mathematically
inclined, the rational numbers could be developed in a way that
allowed one to view an integer as a special type of rational.
\begin{quote}
This quest for a single encompassing framework rather than a set of
isolated concepts is a hallmark of mathematical thinking.
\end{quote}
As the ancient Greeks (so the story goes) were inventing geometry and
its place within architecture, they encountered the uncomfortable fact
that the lengths of portions of eminently buildable structures were
not ``measurable,'' by which they meant ``not rational.''  The poster
child for this phenomenon was the hypotenuse of the isosceles right
triangle with unit-length legs.  As a response to this discomfort, the
{\it real numbers}\index{number!real} were invented.  Once again,
happily, one could develop the real numbers in a way that allowed one
to view a rational number as a special type of real number.  Time went
on, and mathematics matured.  Polynomials\index{polynomial} and their
roots\index{polynomial!root}\footnote{A number $r$ is a {\it root} of
  a polynomial $P(x)$ if $P(r) =0$.}~were discovered, together with
the next source of discomfort.  To understand this discomfort, one
must note that, since the invention of the real numbers, every
polynomial $P_m(x)\ \eqdef \ x^2 - m$, where $m$ is a nonnegative real
number, had two roots, denoted, respectively, $+\sqrt{m}$ and
$-\sqrt{m}$.  But---here is the source of discomfort---{\em the
  polynomial $P_m(x)$ had no roots when $m$ was negative, even a
  negative \underline{integer}}.  The response this time resided in
the invention of a new {\it imaginary} number,\index{number!imaginary}
called $i$,\index{imaginary number $i = \sqrt{-1}$} that was a root of
the polynomial $P_{-1}(x) = x^2 +1$; $i$ was often defined via the
equation, $i = \sqrt{-1}$; easily, $-i$ is also a root of
$P_{-1}(x)$.  By combining the imaginary number $i$ with the real
number system, the {\it complex numbers}\index{number!complex} were
born.

The reader is possibly anticipating a new ``discomfort'' leading to an
new augmentation of our number system, but, no, the story is now
complete, in the sense expressed in the {\it Fundamental Theorem of
  Algebra}\index{Fundamental Theorem of Algebra}, which asserts:

\begin{theorem}[{\bf The Fundamental Theorem of Algebra}]
\label{thm:fund-thm-algebra}
Every polynomial of degree $n$ with complex coefficients has $n$ roots
over the complex numbers.
\end{theorem}

We return to the later to the Theorem and its applications and
implications.  For now, though, it completes our historical tour, so
we can finally begin to get acquainted with our four classes of
numbers.

\subsubsection{The integers}
\label{sec:integers}

The most basic class of numbers are the {\it
  integers}\index{number!integer} (or, {\it whole
  numbers},\index{whole numbers} or, {\em counting
  numbers}).\index{counting numbers}  These are certainly the numbers
that our prehistoric ancestors employed in the earliest days of our
species.

\medskip

\addcontentsline{toc}{paragraph}{A. Integers and the number line}
\noindent{\small\sf A. Integers and the number line.}\index{number!the
  number line}
%
We survey a number of the most important properties of the set
$\Z$\index{$\Z$: the set of all integers} that comprises {\em all
  integers} (the positive and negative integers and zero ($0$)) and
the set $\N$\index{$\N$: the set of nonnegative integers} that
comprises the {\em nonnegative integers} (the positive integers and
$0$).  We estimate a property's importance from the vantage points of
both mathematics and its manifold applications.

Several essential properties of $\N$ and $\Z$ are consequences of the
sets' behavior under their natural order relations: strong ($<$) and
weak ($\leq$) and their converses.
\begin{itemize}
\item
The set $\Z$ is {\em totally ordered}\index{number!integer!total
  order}\index{integer!total order}, also termed {\em linearly
  ordered}\index{number!integer!linear ordering}\index{integer!linear
  ordering}.

\smallskip

This fact is embodied in the {\em Trichotomy Laws for
  integers}.\index{number!integer!Trichotomy Laws}
\index{integer!Trichotomy Laws}\index{Trichotomy Laws}

\medskip

{\it The Trichotomy laws for integers}. \\
%
(a)
%
{\it For each integer $a \in \Z$, precisely one of the following is true.}
\[
(1) \ \mbox{ $a$ equals $0$:} \ a=0 \ \ \ \
(2) \ \mbox{ $a$ is {\em positive}:} \ a>0 \ \ \ \
(3) \ \mbox{ $a$ is {\em negative}:} \ a<0
\]

Consequently, $\Z$ can be visualized via the ($2$-way infinite) number
line: $\ldots, -2, -1, 0, 1, 2, \ldots$.\index{number!the number line}

Analogously, $\N$ can be visualized via the ($1$-way infinite) number
line: $0, 1, 2, \ldots$.

\medskip

(b)
%
{\it For any integers $a, b \in \Z$, precisely one of the following is
  true.}
\[ (1) \ a=b \ \ \ \ \ \ \ (2) \ a<b \ \ \ \ \ \ \ (3) \ a>b \]

\item
The set $\N$ is {\it well-ordered}.\index{number!nonnegative
  integers!well-ordering}\index{nonnegative!integers!well-ordering}

\medskip

{\it The Well-ordering law for nonnegative integers}.
%
{\it Every subset of $\N$ has a smallest element (under the ordering
  $<$).}

\medskip

\item
The set $\Z$ obeys the {\it ``Between''
  Laws}\index{number!integer!''Between'' laws}\index{integer!''Between''
  laws}

\medskip

{\it The ``Between'' laws for integers}. \\
%
{\it For any integers $a, b \in \Z$, there are finitely many $c \in
  \Z$ such that $a < c < b$.}

\smallskip

Any such $c \in \Z$ is {\em between} $a$ and $b$, whence the name of
the law.
\end{itemize}

\medskip

\addcontentsline{toc}{paragraph}{B. Prime numbers}
\noindent {\small\sf B. Prime numbers}\index{number!prime numbers}
%
We single out a subclass of the positive integers whose mathematical
importance has been recognized for millennia but which have found
important applications (e.g., within the domain of computer security)
mainly within the past several decades.

Positive integer $p$ {\it divides} positive integer $n$ (or, {\it is a
  divisor of $n$})\index{number!integer!divisor} if there is a
positive integer $q$ such that $n = p \cdot q$.  Equivalently, we say
that $n$ {\it is divisible by}\index{number!integer!divisibility} $p$.
Thus, every positive integer $n$ is divisible by $p=1$ ($q=n$) and by
$p=n$ ($q=1$).

The class of integers we single out is defined by its divisibility
characteristics.

An integer $p >1$ is {\it prime}\index{number!integer!prime
  number}\index{number!integer!prime}\index{prime
  number}\index{integer!prime number}\index{integer!prime}
if its {\em only} positive integer divisors are $1$ (which divides
every integer) and itself (which is always a divisor).
\begin{quote}
We usually use the shorthand assertion, ``$p$ is a prime,'' instead of
the longer, but equivalent, ``$p$ is a prime integer.''
\end{quote}

A very important way to classify a positive integer $n$ is to list the
primes that divide it, coupling each such prime $p$ with its {\it
  multiplicity}, i.e., the number of times that $p$ divides $n$.  Let
$p_1, p_2, \ldots, p_r$ be the distinct primes that divide $n$, and
let each $p_i$ divide $n$ with multiplicity $m_i$.  The {\it prime
  factorization}\index{prime factorization}\index{integer!prime
  factorization}\index{number!integer!prime factorization}
%
of $n$ is the product $p_1^{m_1} \times p_2^{m_2} \times \cdots \times
p_r^{m_r}$; note that this product satisfies the equation
\begin{equation}
\label{eq:prime-factorization}
n \ = \ p_1^{m_1} \times p_2^{m_2} \times \cdots \times p_r^{m_r}
\end{equation}

When writing an integer $n$'s prime factorization, it is traditional
to list the primes $p_1, p_2, \ldots, p_r$ in increasing order, i.e.,
so that $p_1 < p_2 < \cdots < p_r$.

\noindent
A positive integer $n$ is totally characterized by its canonical prime
factorization, as attested to by the following classical theorem,
which we cite without proof.

\begin{theorem}\index{Fundamental Theorem of Arithmetic}
\label{thm:Fund-Thm-Arith}
The canonical prime factorization of every positive integer is unique.
\end{theorem}

This theorem has been known for millennia and has been honored with
the title {\em The Fundamental Theorem of
  Arithmetic}.\index{Fundamental Theorem of Arithmetic}

One very important application of Theorem~\ref{thm:Fund-Thm-Arith} is
as a mechanism for {\em encoding}\index{number!using the Fundamental
  Theorem of Arithmetic for encoding}\index{encoding sequences via the
  Fundamental Theorem of Arithmetic}
%
sequences of positive integers as single integers!  This works as
follows.  Consider the (infinite) ordered sequence of {\em all primes:}
\[ (q_1 = 2), (q_2 = 3), (q_3 = 5), \ldots  \]
Let
\begin{equation}
\label{eq:sequence-vec-s}
\vec{s} \ = \ \langle m_1, m_2, \ldots, m_k \rangle
\end{equation}
be an arbitrary sequence of positive integers.  Then
Theorem~\ref{thm:Fund-Thm-Arith} assures us that the (single) positive
integer
\[ 
\iota(\vec{s}) \ \eqdef \ q_1^{m_1} \times q_2^{m_2} \times \cdots
\times q_k^{m_k}
\]
is a (uniquely decodable) integer-representation of sequence $\vec{s}$.

We return to this idea of encoding-via-integers in a later chapter.

\subsubsection{The rational numbers}
\label{sec:rationals}

\addcontentsline{toc}{paragraph}{A. Inventing the rational numbers}
\noindent {\small\sf A. Inventing the rational numbers.}
%
Each augmentation of our number system throughout history has been a
response to a deficiency with the then-current system.  The deficiency
that instigated the introduction of the rational numbers was the
frequency with which a given integer $q$ does not divide another given
integer $p$.  This fact meant that the operation of multiplication
could often not be ``undone'', or ``inverted'', because division, the
operation that would accomplish the undoing/inverting, would be a {\em
  partial operation:} There are pairs of integers neither of which
divides the other.  The defining characteristic of the expanded system
is that, within it, every nonzero number divides every number.

\medskip

\addcontentsline{toc}{paragraph}{B. The rationals as ordered pairs of integers}
\noindent {\small\sf B. The rationals as ordered pairs of integers.}
%
The set $\Q$ of {\it rational} numbers\index{number!rational}
\index{$\Q$: the set of rational numbers}
consists of the number $0$, plus the ratios $p/q$ of all nonzero
integers:
\[ \Q \ \eqdef \ \{0\} \ \cup \ \left\{ p/q \ | \ p, q \in \Z
\setminus \{0\} \right\}
\]
Each element of $\Q$ is called a {\it rational} number;
\index{number!rational} 
each {\em nonzero} rational number $p/q$ is often called a {\em
  fraction},
\index{number!fraction}
especially when $q > p$.

An alternative, mathematically more advanced, way of defining the set
$\Q$ is to view it as the smallest set of numbers that contains the
integers, i.e., the set $\Z$, and is {\it closed under the operation
  of dividing any number by any nonzero number.}
\index{algebraic closure}
The word ``closed'' here means that, given any two numbers in $\Q$,
call them $r$ and $s \neq 0$, their quotient $r/s$ belongs to $\Q$.

Numerous notations have been developed for ``naming'' rational numbers
in terms of the integers they are ``built from.''  Most of these
notations continue our custom of employing the symbol ``$0$'' for the
number $0$.  For the nonzero elements of $\Q$, we traditionally employ
some notation for the operation of division and denote the
\begin{equation}
\label{eq:fraction}
 p/q \ \ \ \mbox{ or } \ \ \ {p \over q} \ \ \ \mbox{ or } \ \ \ p
 \div q
\end{equation}
The integer $p$ in any of the expressions in (\ref{eq:fraction}) is
the {\it numerator}
\index{number!rational!numerator}
\index{number!fraction!numerator}
of the fraction; the integer $q$ is the {\it denominator}.
\index{number!rational!denominator}
\index{number!fractions!denominator}

\medskip

\addcontentsline{toc}{paragraph}{C. Comparing the rationals with the integers}
\noindent {\small\sf C. Comparing the rationals with the integers.}
%
There are many ways to compare the sets $\Z$ and $\Q$ via properties
that enhance our understanding of the two sets.

\medskip

{\it i. Every integer is a rational.}
%
Obviously, every integer $n \in \Z$ can be viewed as a rational
\index{integers as rationals}
number, namely, the rational $p/q$ whose numerator is $p = n$ and
whose denominator is $q = 1$; i.e., $n = n/1$.  Easily, this encoding
preserves the special character of the numbers $0$ and $1$, because
$0/1 = 0$ and $1/1 = 1$.

\medskip

{\it ii. The rational and integer number lines.}\index{the rational number line}
%
The rational numbers share some, but not all, of the number-line laws
enumerated in Section~\ref{sec:integers}.A.  We mirror for $\Q$ that
section's discussion of $\Z$.
\begin{itemize}
\item
{\it The Trichotomy laws for rational numbers.} \\
\index{Trichotomy laws for rationals}
(a)
%
{\it For each rational $a \in \Q$, precisely one of the following is true.}
\[
(1) \ \mbox{ $a$ equals $0$:} \ a=0 \ \ \ \
(2) \ \mbox{ $a$ is {\em positive}:} \ a>0 \ \ \ \
(3) \ \mbox{ $a$ is {\em negative}:} \ a<0
\]

Consequently, $\Q$ can be visualized via the ($2$-way infinite) number
line.

\medskip

(b)
%
{\it For any rationals $a, b \in \Q$, precisely one of the following is
  true.}
\[ (1) \ a=b \ \ \ \ \ \ \ (2) \ a<b \ \ \ \ \ \ \ (3) \ a>b \]

\item
{\em The set $\Q$ is {\em not} well-ordered.}

For illustration:  The set
\[ S \ = \ \{ a \in \Q  \ |\ 0 < a \leq 1 \} \]
has no smallest element.  If you give me a rational $p \in S$ that you
claim is the smallest element of the set, then I shall give you $p/2$
as a smaller one.

\medskip

\item
{\em The set $\Q$ does {\em not} obey the ``Between'' laws.}

In fact, $\Q$ violates the ``Between'' laws in a very strong way:
{\it For any two unequal rationals, $a$ and $b>a$, there are
  infinitely many rationals between $a$ and $b$.}

One can specify such an infinite set for the pair $a,b$ in myriad
ways.  Here is a simple such set, call it $S_{a,b}$.
\begin{equation}
\label{eq:between-rationals}
S_{a,b} \ = \ \left\{ \frac{a+b}{k} \ \ | \ \ k \in \Z \right\}.
\end{equation}
\end{itemize}

\medskip

{\it iii. The relative ``sizes'' of $\Z$ and $\Q$: Are there more
  rationals than integers?}

\noindent Consider the following facts.
\begin{itemize}
\item
Every integer is a rational number, as attested to by the ``encoding''
\begin{equation}
\label{eq:ZintoQ}
\mbox{Encode } \ \ \ n \in \Z \ \ \ \mbox{ by } \ \ \ {n \over 1} \in \Q .
\end{equation}

\item
There are infinitely many non-integer rational numbers between every
pair of adjacent integers, as attested to by every set $S_{n,n+1}$ as
defined in (\ref{eq:between-rationals}).
\end{itemize}
Represented symbolically, we thus have $\Z \subset \Q$.

\medskip

\noindent
The preceding ``argument'' makes an intuitive case that \\
\hspace*{.35in}``there are more rational numbers than integers.'' \\ 
We put this assertion in quotes because, absent a formal notion of
``{\em more}'' for infinite sets, the assertion is meaningless!  Is
there an acceptable formal notion that would allow us to verify or
refute the assertion?  The 19th-century mathematician/logician Georg
Cantor addressed this question, and related ones, in his
groundbreaking study of the relative ``sizes'' of infinte sets
\cite{Cantor74,Cantor78}.  We adapt enough of his formulation to
suggest how such issues can be dealt with mathematically.  We shall
take a longer look at Cantor's work in Section~\ref{sec:order}.

Let us take our lead from finite sets.  Is there a notion of
``bigger'' for finite sets that can be extended to infinite sets?

We begin with a set $A$ of apples and a set $O$ of oranges, together
with challenge of determining which set is bigger.

\medskip

If sets $A$ and $O$ are both finite, then we can just {\em count} the
number of apples in $A$, call it $a$, and the number of oranges in
$O$, call it $o$, and then compare the sizes of the (nonnegative)
integers $a$ and $o$.  The Trichotomy Laws for integers
(Section~\ref{sec:integers}.A) guarantee that we shall be able to
settle the question.  {\em But} we cannot count the elements in an
infinite set, so this approach fails us when we can infinitely much
fruit!

\medskip

Here is another approach that works for finite sets and that promises
to extend to infinite sets.  Let us assume that we can ``prove''---we
shall explain the word imminently---the following.

For every apple that we extract from set $A$ {\em for the first time},
we can extract an orange from set $O$ {\em for the first time}.  It
will then follow (at least in the finite case), that \\
\hspace*{.35in}{\em There are at least as many oranges as apples!}

\noindent
This is really promising, because there is another way to describe the
fruit-matching process that readily extends to infinite sets.  \\
\hspace*{.35in}{\em There is an injection,\footnote{Recall from
    Section~\ref{s.function} that ``{\em injection}'' is synonymous
    with ``{\em one-to-one function}''.}~call it $f$, from $O$ to $A$.} \\
In more formal terms: {\em Every time you pull an apple $\alpha$ from set
  $A$, I pull the orange $f^{-1}(\alpha)$ from $O$.}

\medskip

Inspired by this formulation using injections---and by the work of
Cantor---we craft the following definition.

\noindent
{\em
Given sets $A$ and $O$ (finite or infinite), we say \\
\hspace*{.35in}{\em Set $O$ is at least as big as set $A$} \\
precisely when there is an injection from $O$ to $A$.}

\medskip

Finally, back to numbers!

\addcontentsline{toc}{paragraph}{-- A fun result: There are as many
  integers as rationals}

\begin{prop}
\label{thm:|Q|=|Z|}
{\rm (a)} There exists an injection from $\Z$ to $\Q$.  Therefore,
$\Q$ is at least as big as $\Z$.

\noindent {\rm (b)} There exists an injection from $\Q$ to $\Z$.
Therefore, $\Z$ is at least as big as $\Q$.
\end{prop}

\noindent {\it Verification}. \\
%
(a) We have already established part (a), via the injection from $\Z$
into $\Q$ implicit in (\ref{eq:ZintoQ}).

\medskip

\noindent (b)
We proceed in two steps.
\begin{enumerate}
\item
We define an injection $f_1$ that maps $\Q$ one-to-one into the set
$\Q^+$ of {\em nonegative} rationals.  We specify $f_1$ as follows.
\[ f_1(p/q) \ = \ \left\{
\begin{array}{cl}
2p/q & \mbox{ if } \ p/q \geq 0 \\
(2p+1)/q & \mbox{ if } \ p/q < 0. \\
\end{array}
\right.
\]
We leave to the reader the easy proof that $f_1$ is an injection from
$\Q$ into $\Q^+$.

\item
We define an injection from $\Q^+$ into $\Z$.

Toward this end, we represent each rational number $r \in \Q^+$ by a
pair of integers $p > 0$ and $q >0$ such that
\begin{itemize}
\item
$r = p/q$

\item
$p$ and $q$ are {\em relatively prime} \index{relatively prime
  integers} in the sense that no integer $n > 1$ divides both $p$ and
  $q$.  If such an $n$ existed, then we could divide both $p$ and $q$
  by it, with the assurance that
\[ \frac{p/n}{q/n} \ = \ \frac{p}{q}. \]
\end{itemize}
Inspired by Theorem~\ref{thm:Fund-Thm-Arith}, we now consider the
function
\[ f_2(p/q) \ \eqdef \ 2^p 3^q. \]
Thus defined, $f_2$ is a function from $\Q^+$ into $\Z$.  Moreover,
Theorem~\ref{thm:Fund-Thm-Arith} guarantees that $f_2$ maps $\Q^+$
{\em one-to-one} into $\Z$.
\end{enumerate}
Since the composition of two injections is also an injection (see
Proposition~\ref{thm:fn-composition}), the result is proved.  \qed

\subsubsection{The real numbers}
\label{sec:reals}

\addcontentsline{toc}{paragraph}{A. Inventing the real numbers}
\noindent {\small\sf A. Invemting the real numbers.}
%
Each subsequent augmentation of our system of numbers inevitably gets
more complicated than the last: one solves the easy problems first.
The deficiency in the system of real numbers harkens back to
historical time, roughly $2 {1 \over 2}$ millennia ago.  The ancient
Egyptians were prodigious builders who mastered truly sophisticated
mathematics in order to engineer their temples and pyramids.  The
ancient Greeks perpetuated this engineering tradition, but they added
to it the ``soul'' of mathematics.

Numbers were (literally) sacred objects to the Greeks, and they
invented quite ``modern'' (to our perspective) ways of thinking about
mathematical phenomena in order to understand {\em why} certain facts
were true, in addition to knowing {\em that} they were true.  One
intellectual project in this spirit had to do with the way they
designed contructions.  They were attracted to geometric contructions
that could be accomplished using only {\em straight-edges and
  compasses}.  And---most relevant to our story---they preferred that
the relative lengths of linear sections of their structures be {\em
  commensurable},
\index{number!integer!commensurable pairs of integers}
in the following sense.  {\em Integers $x,y \in \N$ are {\em
    commensurable} if there exist $a, b \in \N$ such that}
\[ ax \ = \ by \ \ \ \ \mbox{ or, equivalently, } \ \ \ \ x \ = \ {a
  \over b} y.
\]
The desire to employ only commensurable pairs of integers, at least in
moderately simple constructions, was shown to be impossible when one
considered {\it the diagonal of the square with unit-length sides} or,
equivalently, {\it the hypotenuse of the isosceles right triangle with
  unit-length legs}.  In both situations, one encountered the unit
lengths of the sides or legs of the structures, together with the {\em
  noncommensurable} length of the diagonal or hypotenuse, which, in
current terminology, is $\sqrt{2}$.  The Greek mathematicians, as
reported by the renowned mathematician Euclid,\footnote{who wrote
  extensively on this and related subjects.}~proved, using current
terminology, that $\sqrt{2}$ is not rational.  (We rephrase the proof
imminently, in Proposition~\ref{thm:sqrt(2)}.)  The conclusion from
this proof is that the number system based on the rational numbers was
inadequate.  In response, they augmented this system by introducing
{\it surds} \index{number!surd} or, as we more commonly term them,
{\it radicals}, \index{number!radical} beginning a trajectory that
culminated in the real number system.  Since our intention has been to
justify the journey along that trajectory, we leave our historical
digression and turn to our real focus, the set $\R$ of {\it real
  numbers}.\index{number!real}

\medskip

\addcontentsline{toc}{paragraph}{B. The real numbers via their numerals}
\noindent {\small\sf B. The real numbers via their numerals.}
%
For any integer $b > 1$, the real numbers are the numbers that can be
named by {em infinite} strings built out of the digits $\{0, 1,
\ldots, b-1\}$;\footnote{This is not the traditional way that a
  mathematician would define the class of real numbers, but it is
  correct and adequate for thinking about the class.}~the resulting
strings are called {\em $b$-ary numerals}.  There are a couple of ways
to form $b$-ary numerals; we shall discuss some of the most common
ones in Section~\ref{sec:Numerals}.  For now, we define real numbers
as those that can be represented by a base-$b$ numeral, for some
integer $b >1$.  Such a numeral has the form
\begin{equation}
\label{eq:real-numeral}
\alpha_n \alpha_{n-1} \cdots \alpha_1 \alpha_0                  
. \beta_0 \beta_1 \beta_2 \cdots
\end{equation}
\index{positional number system!numerical value of numeral}
and represents the (real) number
\[
\underline{\alpha_n \alpha_{n-1} \cdots \alpha_1 \alpha_0                  
. \beta_0 \beta_1 \beta_2 \cdots}
\ \ \eqdef \ \
\sum_{i=0}^n \alpha_i \cdot b^i
\ + \ \sum_{j\geq 0} \beta_j \cdot b^{-j}.
\]
By prepending a ``negative sign'' (or, ``minus sign'') $-$ to a
numeral or a number, one renders the thus-embellished entity as
negative.

The fact that every rational number (hence, also, every integer) is
also a real number is manifest in the fact that integers and rationals
can also be written as $b$-ary numerals as in (\ref{eq:real-numeral}).
But with rational and integers, we are able to insist that their
numerals have special forms.  We cite without proof the following
classical results from the theory of arithmetic.  The result for
integers is easily phrased.

\begin{theorem}{Integers as real numbers}
\label{thm:integer-real}
A real number is an integer if, and only if, it can be represented by
a {\em finite-length} numeral.\index{number!integer: real with a finite numeral}
\end{theorem}

The result for rationals needs an introductory definition.  An {\em
  infinite} sequence of numbers $\Sigma$ is {\em ultimately
  periodic}\index{ultimately periodic sequence} if there exist two
{\em finite} sequences of numbers, $\Gamma$ and $\Delta$, such that
$\Sigma$ can be written in the following form (spaces added to enhance
legibility):
\[ \Sigma \ = \ \Gamma \ \Delta \ \Delta \ \Delta \ \ldots 
\ \Delta \ \ldots
\]
The intention here is that the finite sequence $\Delta$ is repeated
{\it ad infinitum}.


\begin{theorem}{Rationals as real numbers}
\label{thm:rational-real}
\noindent
A real number is rational if, and only if, it can be represented by a
numeral that is {\em ultimately periodic}.\index{number!rational:
  real with an ultimately periodic numeral}
\end{theorem}

\noindent {\em A clarification.}
%
Note that two types of sequences of $0$s do not affect the value of
the number represented by a numeral: (1) an {\em initial} sequence
of $0$s to the {\em left} of the radix point and of all non-$0$
digits; (2) a {\em terminal} sequence of $0$s to the {\em right} of
the radix point and of all non-$0$ digits.

One consequence of this fact is that we lose no generality by
insisting that every numeral have the following form:

\smallskip

\hspace*{.15in}
\begin{tabular}{l}
a finite sequence of digits, followed by a radix point, followed by an
infinite \\
sequence of digits
\end{tabular}

Integers can then be singled out via numerals that have only $0$s to
the right of the radix point.

Theorems~\ref{thm:integer-real} and~\ref{thm:rational-real} show us
that the three sets of numbers we have defined are a nested
progression of successively more inclusive sets, in the sense that
{\em every integer is a rational number} and {\em every rational
  number is a real number}.  Those interested in the (philosophical)
foundations of mathematics might quibble about the verb ``is'' in the
highlighted sentences, but for all practical purposes, we can accept
the sentences as written.

\medskip

\addcontentsline{toc}{paragraph}{C. Not all real numbers are rational}
\noindent {\small\sf C. Not all real numbers are rational.}
%
We close this section by verifying the earlier-mentioned assertion
about the non-commensurability of the length of the diagonal of a
square with the (common) length of its sides---or, equivalently, the
leg-length of an isosceles right triangle with the length of its
hypotenuse.
\index{The non-commensurabiliy of $\sqrt{2}$}
%
The following result is a simple application of
Theorem~\ref{thm:Fund-Thm-Arith}.  Its proof only suggests the range
of the theorem's myriad applications.

\addcontentsline{toc}{paragraph}{-- A fun result: $\sqrt{2}$ is not rational}

\begin{prop}
\label{thm:sqrt(2)}
The real number $\sqrt{2} = 2^{1/2}$ is not rational.
\end{prop}

\noindent {\it Verification}.
%
We prove the result by contradiction,\index{proof by contradition}
a proof technique described in Chapter~\ref{sec:practical-logic}.

Let us assume, for contradiction, that $\sqrt{2}$ is rational.  By
definition, then $\sqrt{2}$ can be written as a fraction
\[ \sqrt{2} \ = \ {a \over b} \]
for positive integers $a$ and $b$.  In fact, we can also insist that
$a$ and $b$ {\em share no common prime factor}.  For, if $a$ and $b$
shared the prime factor $p$, then we would have $a = p \times c$ and
$b = p \times d$.  In this case, though, we would have
\[ \sqrt{2} \ = \ {a \over b} \ = \ \frac{p \times c}{p \times d}
\ = \ {c \over d}.
\]
by cancellation of the common factor $p$.  We can eliminate further
common prime factors if necessary until, finally, we find a fraction
for $\sqrt{2}$ whose numerator and denominator share no common prime
factor.  This must occur eventually because each elimination of a
common factor leaves us with smaller integers, so the iterative
elimination of common factors must terminate.

Let us say that, finally,
\begin{equation}
\label{eq:sqrt2-1}
\sqrt{2} \ = \ {k \over \ell}
\end{equation}
where $k$ and $\ell$ share no common prime factor.  Let us square both
expressions in (\ref{eq:sqrt2-1}) and multiply both sides of the
resulting equation by $\ell^2$.  We thereby discover that
\begin{equation}
\label{eq:sqrt2-2}
2 \ell^2 \ = \ k^2.
\end{equation}
This rewriting exposes the fact that $k^2$ is {\em even},\index{integer!even}
i.e., {\em divisible by $2$}.  But, Theorem~\ref{thm:Fund-Thm-Arith}
tells us that {\em if $k^2$ is divisible by $2$, then so also is $k$}!
This means that $k = 2m$ for some positive integer $m$, whoch allows
us to rewrite (\ref{eq:sqrt2-2}) in the form
\begin{equation}
\label{eq:sqrt2-3}
2 \ell^2 \ = \ k^2 \ = \ (2m)^2 \ = \ 4m^2.
\end{equation}
Hence, we can divide the first and last quantities in
(\ref{eq:sqrt2-3}) by $2$, to discover that
\[ \ell^2 \ = \ 2m^2. \]
Repeating the invocation of Theorem~\ref{thm:Fund-Thm-Arith} now tells
us that the integer $\ell$ must be even.

We now see that {\em both $k$ and $\ell$ are even, i.e., divisible by
  $2$}.  This contradicts our assumption that $k$ and $\ell$ share no
common prime divisor!

Since every step of our argument is ironclad---except for our
assumption that $\sqrt{2}$ is rational, we conclude that that
assumption is false!  The Proposition is verified!
\qed

\begin{quote}\index{Proof by contradiction}
The proof of Proposition~\ref{thm:sqrt(2)} is a classical (and early)
example of {\em proof by contradiction}, as discussed in
Section~\ref{sec:practical-logic}.
\end{quote}



\subsubsection{The complex numbers}
\label{sec:complexes}

$\C$ denotes the complex numbers


\subsection{Order within the Number System}
\label{sec:order}

One of one's biggest friends when reasoning about numbers resides in
the concept of {\em order}.\index{number!ordering of numbers}




**HERE-ORDER





\subsection{Numerals}\index{numerals}
\label{sec:Numerals}

We can identify distinct families of {\em operational}
numerals,\index{numerals!operational} i.e., numerals that allow one to
do things such as perform arithmetic (add, multiply, etc.).

**THOUGHTS*******
series expansions, strings created by a positional number system, and
hybrids built on positional systems such as ``scientific notation''.
*************


Of course, we are all familar with certain numbers that have {\em
  non-operational} names that we use all the time.  Notable among
these are (using terminology that anticipates future sections and
chapters):
\begin{itemize}
\item
$\pi$: the ratio of the circumference of a circle to its diameter;
  $\pi \approx 3.141592653 \ldots$
\item
$e$: Euler's constant; the base of ``natural'' logarithms: $e \approx
  2.718281828 \ldots$
\item
$i$: the name of the number whose square is $-1$; $i$ is appended to
  the real numbers in order to ``complete'' them to the complex
  numbers, within which system every polynomial of degree $n$ has $n$
  roots.
\end{itemize}

\medskip

\addcontentsline{toc}{paragraph}{A. Positional number systems}
\noindent {\small\sf A. Positional number systems.}\index{positional
  number system}
%
The most common way of forming numerals is via strings over a {\it
  number base}.\index{positional number system!base of the system}  
%
We begin with an integer $b>1$ that will serve as our base, and we
define the set $B_b = \{ 0, 1, \ldots, b-1\}$ of {\it digits in base
  $b$}.\index{positional number system!digits in base $b$}
%
To aid legibility, {\em within the context of base-$b$ positional
  numerals}, we denote the digit $b-1$ as a single character,
$\bar{b}$.\index{$\bar{b}$: the digit $b-1$ in base $b$}
%
We then form base-$b$ numerals in the following way.\index{positional
  number system!base-$b$ numerals}
%
This formation builds on {\em geometric sums}, a mathematical
structure that we shall learn to manipulate, evaluate, and compute
with in Section~\ref{sec:sums-series}.

A base-$b$ numeral is a string having three sections.
\begin{enumerate}
\item
The numeral begins with its {\em integral part},\index{positional
  number system!integral part of a numeral}
%
which is a finite string of digits from $B_b$: $\alpha_n \alpha_{n-1}
\cdots \alpha_1 \alpha_0$.

The base-$b$ number represented by the numeral's integral
part\index{positional number system!numerical value of integral part}
is\footnote{Our underlined notation for the numerical value of a
  numeral is not common, but we find it convenient.}
\[
\underline{\alpha_n \alpha_{n-1}\cdots \alpha_1 \alpha_0}
\ \ \eqdef \ \
\sum_{i=0}^n \alpha_i \cdot b^i
\]

\item
The numeral continues with a single occurrence of the {\it
  radix point}\index{positional number system!radix point ``$.$''}
``$.$''
\item
The numeral ends with its {\em fractional part},\index{positional
  number system!fractional part of a numeral}
%
which is a string---{\em finite or infinite}--- of digits from $B_b$:
$\beta_0 \beta_1 \beta_2 \cdots$.

The base-$b$ number represented by the numeral's fractional part
is\index{positional number system!numerical value of fractional part}
\[
\underline{. \beta_0 \beta_1 \beta_2 \cdots}
\ \ \eqdef \ \
\sum_{j\geq 0} \beta_j \cdot b^{-j}
\]
\end{enumerate}
In summary, then, the base-$b$ number represented by the numeral
$\alpha_n \alpha_{n-1} \cdots \alpha_1 \alpha_0                  
. \beta_0 \beta_1 \beta_2 \cdots$ 
is\index{positional number system!numerical value of numeral}
\[
\underline{\alpha_n \alpha_{n-1} \cdots \alpha_1 \alpha_0                  
. \beta_0 \beta_1 \beta_2 \cdots}
\ \ \eqdef \ \
\sum_{i=0}^n \alpha_i \cdot b^i
\ + \ \sum_{j\geq 0} \beta_j \cdot b^{-j}.
\]
By prepending a ``negative sign'' (or, ``minus sign'') $-$ to a
numeral or a number, one renders the thus-embellished entity as
negative.

\bigskip

\addcontentsline{toc}{paragraph}{B. Scientific notation}
\noindent {\small\sf B. Scientific notation.}\index{Scientific notation}
%
The finite numerals in subsection A are all ``exact'' in the sense
that changing any digit changes the value of the named number.  We
turn now to a class of numerals that abjure this ``exactness'' for the
sake of expedience.  There are a few reasons that one might be willing
to do this.

\begin{itemize}
\item
WHY SCIENTIFIC NOTATION
\end{itemize}

\addcontentsline{toc}{paragraph}{C. Exact arithmetic}





\section{Arithmetic and Its Laws}\index{laws of arithmetic}
\label{sec:Arithmetic-Tools+Laws}

Numbers are {\it adjectives}\index{number!as adjective}---you have
five apples and three oranges---but in contrast to adjectives that are
purely descriptive---the red ball, the big dog---numbers can be {\em
  manipulated},\index{number!as {\em manipulable} adjective} using the
tools of {\it arithmetic}.

\subsection{The Tools of Arithmetic}
\label{sec:arithmetic-tools}

The basic tools of arithmetic reside in a small set of operations,
together with two special integers that play important roles with
respect to the operations.  Since these entities are so tightly
intertwined, we discuss them simultaneously.

\smallskip

\noindent {\small\sf Two special integers}
%
The integers zero ($0$)\index{number!zero ($0$)} and one
($1$),\index{number!one ($1$)} play special roles within all four of
the classes of numbers we have described.

\smallskip

\noindent {\small\sf  The operations of arithmetic}\index{arithmetic!basic operations}
%
Arithmetic on the four classes of numbers that we have described is
built upon the following operations.  When we say that an operation
produces a number ``of the same sort'', we mean that it produces
\begin{itemize}
\item
an integer result from integer arguments;
\item
a rational (number) result from rational (number) arguments;
\item
a real (number) result from real (number) arguments;
\item
a complex (number) result from complex (number) arguments;
\end{itemize}
The fundamental operations on numbers are, of course, familiar to the
reader.  Our goal in discussing them is to stress the laws that govern
the operations.

\subsubsection{Unary (single-argument) operations}

\addcontentsline{toc}{paragraph}{A. Negating numbers}
\noindent {\small\sf A. Negating numbers.}
\index{arithmetic!basic operations!negation}
%
The operation of {\it negation:}
\index{arithmetic!basic operations!negating}
\begin{itemize}
\item
is a {\em total function} on the sets $\Z, \Q, \R, \C$.  It replaces
a number $a$ by its {\em negative},
\index{number!negative}
a number of the same sort, denoted $-a$.
\item
is a {\em partial function} on the nonnegative subsets
of $\Z, \Q, \R, \C$.  It replaces a number $a$ by its negative, $-a$,
whenever both $a$ and $-a$ belong to the nonnegative subset being
operated on.
\end{itemize}
Zero ($0$) is the unique {\it fixed point}\index{function!fixed
  point}\index{aritmetic!megation!fixed point} of the operation,
meaning that $0$ is the unique number $a$ such that $a = -a$.

\medskip

\addcontentsline{toc}{paragraph}{B. Reciprocating numbers}
\noindent {\small\sf B. Reciprocating numbers}.
\index{arithmetic!basic operations!reciprocal}
%
The operation of {\it reciprocating}
\begin{itemize}
\item
\index{arithmetic!basic operations!reciprocating}
is a {\em total function} on the sets $\Q, \R, \C$, which replaces each
number $a$ by its {\em reciprocal}, 
\index{number!reciprocal}
a number of the same sort, denoted $1/a$ or $\displaystyle {1 \over
  a}$.  We shall employ whichever notation enhances legibility.

\item
is {\em undefined} on every integer $a$ except for $1$.
\end{itemize}

\medskip

\addcontentsline{toc}{paragraph}{C. Floors and ceilings}
\noindent {\small\sf C. Floors and ceilings.}
\index{arithmetic!basic operations!floors and ceiling}
%
The operations of {\it taking floors and ceilings} are total
operations on the sets $\N, \Z, \Q, \R$:
\begin{itemize}
\item
The {\it floor} of a number $a$, also called {\it the integer part}
\index{arithmetic!basic operations!integer part of a number}
\index{arithmetic!basic operations!floor of a number}
of $a$, denoted $\lfloor a \rfloor$, is the largest integer that does
not exceed $a$; i.e.,:
\[
\lfloor a \rfloor \ \eqdef \ \max_{b \in {\mathbb{N}}} \Big[ b \ \leq a \Big]
\]
\item
The {\it ceiling} of a number $a$
\index{arithmetic!basic operations!ceiling of a number}
of $a$, denoted $\lceil a \rceil$, is the smallest integer that is 
not smaller than $a$:
\[
\lceil a \rceil \ \eqdef \ \min_{b \in {\mathbb{N}}} \Big[ b \ \geq a \Big]
\]
\end{itemize}
Thus, the operations of taking floors and ceilings are two ways to
{\em round} rationals and reals to their ``closest''
integers.\index{arithmetic!basic operations!rounding to ``closest'' integer}

\medskip

\addcontentsline{toc}{paragraph}{D. Absolute values, magnitudes}
\noindent {\small\sf D. Absolute values, magnitudes}
\index{arithmetic!basic operations!absolute value, magnitude}
%
Let $a$ be a real number.  The {\it absolute value}, or, {\it
  magnitude}, of $a$, denoted $|a|$ equals either $a$ or $-a$,
whichever is positive.  For a complex number $a$, the definition of
$|a|$ is more complicated: it is a measure of $a$'s ``distance'' from
the ``origin'' complex number $0 + 0 \cdot i$.  In detail:
\[
|a| \ = \ \left\{
\begin{array}{cl}
a & \mbox{ if } \ [a \in \R] \ \ \mbox{ and } [a \geq 0] \\
-a & \mbox{ if } \ [a \in \R] \ \ \mbox{ and } [a < 0] \\
\sqrt{b^2 + c^2} &  \mbox{ if } \ [a \in \C]  \ \ \mbox{ and } [a = (b+ci)]
\end{array}
\right.
\]


\subsubsection{Binary (two-argument) operations}
\label{sec:binary-operators}

\addcontentsline{toc}{paragraph}{A. Addition and Subtraction}
\noindent {\small\sf A. Addition and Subtraction.}
\index{arithmetic!basic operations!addition}
\index{arithmetic!basic operations!subtraction}
%
The operation of {\it addition}\index{arithmetic!addition} is a {\em
  total function} that replaces any two numbers $a$ and $b$ by a
number of the same sort.  The resulting number is the {\em sum of $a$
  and $b$}\index{arithmetic!addition!sum} and is denoted $a+b$.

\noindent
The operation of {\it subtraction}\index{arithmetic!subtraction} is a
{\em total function} on the sets $\Z, \Q, \R, \C$, which replaces any
two numbers $a$ and $b$ by a number of the same sort.  The resulting
number is the {\em difference of $a$ and $b$}
\index{arithmetic!subtraction!difference} and is denoted $a-b$.  On
the nonnegative subsets of the sets $\Z, \Q, \R, \C$---such as $\N$,
which is the largest nonnegative subset of $\Z$---subtraction is a
{\em partial function}, which is defined only when $a \geq b$.

Subtraction can also be defined as follows.  For any two numbers $a$
and $b$, {\em the difference of $a$ and $b$ is the sum of $a$ and the
  negation of $b$}; i.e.,
\[ a-b \ = \ a + (-b) \]

{\em The special role of $0$ under addition and subtraction.}
%
The number $0$ is the {\it identity} under addition and
  subtraction.\index{number!additive identity}\index{number!identity
  under addition}\index{identity!additive}
%
This means that, for all numbers $a$,
\[ a+0 \ = \ a-0 \ = \ a. \]

{\em The special role of $1$ under addition and subtraction.}
%
For any integer $a$, there is no integer between $a$ and $a+1$ or
between $a-1$ and $a$.  For this reason, on the sets $\Z$ and $\N$,
one often singles out the following special cases of addition and
subtraction, especially in reasoning about situations that are indexed
by integers.  Strangely, these operations have no universally accepted
notations.
\begin{itemize}
\item
The {\it successor} operation\index{arithmetic!integers!successor} is
a {\em total function} on both $\N$ and $\Z$, which replaces an
integer $a$ by the integer $a+1$.
\item
The {\it predecessor} operation\index{arithmetic!integers!predecessor}
is a {\em total function} on $\Z$, which replaces an integer $a$ by
the integer $a-1$.  It is a {\em partial function} on $\N$, which is
defined only when the argument $a$ is positive (so that $a-1 \in \N$).
\end{itemize}

The operations of addition and subtraction are said to be {\em
  inverse operations}\index{arithmetic!integers!additive inverse}
\index{arithmetic!integers!addition and subtraction are mutually
  inverse} of each other because each can be used to ``undo'' the
other:
\[
a \ = \ (a+b) -b \ = \ (a-b) +b
\]

\medskip

\addcontentsline{toc}{paragraph}{B. Multiplication and Division}
\noindent {\small\sf B. Multiplication and Division}.
\index{arithmetic!basic operations!multiplication}
\index{arithmetic!basic operations!division}
%
The operation of {\it multiplication}\index{arithmetic!multiplication}
is a {\em total function} that replaces any two numbers $a$ and $b$ by
a number of the same sort.  The resulting number is the {\em product
  of $a$ and $b$}\index{arithmetic!multiplication!product} and is
denoted either $a \cdot b$ \index{arithmetic!multiplication!$a \cdot  b$}
or $a \times b$.\index{arithmetic!multiplication!$a \times b$}
We shall usually favor the former notation, except when the latter
enhances legibility.

The operation of {\it division}\index{arithmetic!division} is a {\em
  partial function} on all of our sets of numbers.  Given two numbers
$a$ and $b$, the result of dividing $a$ by $b$---{\em when that result
  is defined}---is the {\it quotient of $a$ by $b$}
\index{arithmetic!division!When is $a/b$ defined?}
\index{arithmetic!division!quotient}
\index{arithmetic!division!quotient!$a/b$}
\index{arithmetic!division!quotient!$a \div b$}
\index{arithmetic!division!quotient!${a \over b}$}
and is denoted by one of the following three notations: $a/b$, $a \div
b$, $\displaystyle{a \over b}$.  The {\it quotient of $a$ by $b$} is
defined precisely when {\em both}

\noindent
\hspace*{.35in}(1) $b \neq 0$: one can never divide by $0$ \\
\hspace*{.35in}{\em and} \\
\hspace*{.35in}(2) there exists a number $c$ such that $a = b \cdot c$.

\noindent
Assuming that condition (1) holds, {\em condition (2) always holds
  when $a$ and $b$ belong to $\Q$ or $\R$ or $\C$}.

Division can also be defined as follows.  For any two numbers $a$
and $b$, {\em the quotient of $a$ and $b$ is the product of $a$ and the
reciprocal of $b$} (assuming that the latter exists); i.e.,
\[ a/b \ = \ a \cdot (1/b). \]
Computing reciprocals of nonzero numbers in $\Q$ and $\R$ is standard
high-school level fare; computing reciprocals of nonzero numbers in
$\C$ requires a bit of calculational algebra which we do not cover.
For completeness, we note that the reciprocal of the {\em nonzero}
complex number $a + bi \in \C$ is the complex number $c+di$ where
\[ c \ = \ \frac{a}{a^2 + b^2} \ \ \ \ \
\mbox{ and } \ \ \ \ \
d \ = \ \frac{-b}{a^2 + b^2}.
\]

{\em The special role of $1$ under multiplication and division.}
%
The number $1$ is the {\it identity} under the operations of
multiplication and division.\index{number!multiplicative
  identity}\index{number!identity under
  multiplication}\index{identity!multiplicative}
%
This means that, for all numbers $a$,
\[ a \cdot 1 \ = \ a \cdot (1/1) \ = \ a. \]

{\em The special role of $0$ under multiplication and division.}
%
The number $0$ is the {\it annihilator} under
multiplication.\index{multiplicative annihilator} This means that, for
all numbers $a$
\[ a \cdot 0 \ = \ 0. \]

The operations of multiplication and division are said to be {\em
  inverse operations}\index{arithmetic!integers!multiplicative
  inverse} \index{arithmetic!integers!multiplication and division are
  mutually inverse} because, when both operations can be applied, each
can be used to ``undo'' the other:
\[ a = (a \cdot b) \div b \ = \ (a \div b) \cdot b.  \]


\subsection{The Laws of Arithmetic}\index{arithmetic!basic laws}
\label{sec:Arithmetic-Laws}

The student should understand the following laws of arithmetic on the
reals, rationals, and reals---and be able to employ them cogently in
rigorous argumentation.
\begin{itemize}
\item
{\it The commutative law}.\index{commutative law!arithmetic}
\index{commutative law!addition}
\index{commutative law!multiplication}
\index{arithmetic!commutative law}
%
For all numbers $x$ and $y$:
\[
\begin{array}{llc}
\mbox{\it for addition:}
  & & x+y \ = \ y+x \\
\mbox{\it for multiplication:}
  & & x \cdot y \ = \ y \cdot x
\end{array}
\]

\item
{\it The associative law}.\index{associative law for
  arithmetic}\index{arithmetic!associative law}
%
For all numbers $x$, $y$, and $z$,
\[ (x+y)+z \ = \ x+(y+z) \ \ \ \mbox{\bf and } \ \ 
x\cdot (y\cdot z) 
(x \cdot y) \cdot z \ = \ x\cdot (y\cdot z). \] 
This allows one, for instance, to write strings of additions or of
multiplications without using parentheses for grouping.

\item
{\it The distributive law}.\index{distributive law for
  arithmetic}\index{arithmetic!distributive law}
%
For all numbers $x$, $y$, and $z$,
\begin{equation}
\label{eq:distr-law}
x \cdot (y + z) \ = \ (x \cdot y) + (x \cdot z).
\end{equation}
One commonly articulates this law as, ``{\em Multiplication
  distributes over addition.}''
\end{itemize}
One of the most common uses of the distributive law reads equation
(\ref{eq:distr-law}) ``backwards,'' thereby deriving a formula for
{\em factoring} \index{arithmetic!factoring} complex expressions that
use both addition and multiplication.

Easily, addition does {\em not} distribute over multiplication; i.e.,
in general, $x + y \cdot z \ \neq \ (x+y) \cdot (x+z)$.  Hence, when
we see ``$x + y \cdot z$'', we know that the multiplication is
performed before the addition.  In other words, {\em Multiplication
  takes priority over addition.}  \index{arithmetic!priority of
  multiplication over addition} This priority permits us to write the
righthand side of (\ref{eq:distr-law}) without parentheses, as in
\[ x \cdot (y + z) \ = \ x \cdot y + x \cdot z. \]

Via multiple invocations of the preceding laws, we can derive a recipe
for multiplying complicated expressions.  We illustrate this via the
``simplest'' complicated expression, $(a+b) \cdot (c+d)$.

\begin{prop}
\label{prop:(a+b)(c+d)}
For all numbers $a, b, c, d$:
\begin{equation}
\label{eq:(a+b)(c+d)}
(a+b) \cdot (c+d) \ = \ a \cdot c + a \cdot d + b \cdot c + b \cdot d
\end{equation}
\end{prop}

\noindent {\it Verification}.
%
Note first that because multiplication takes priority over addition,
the absence of parentheses in expressions such as
(\ref{prop:(a+b)(c+d)}) does not jeopardize unambiguity.  Our proof of
the proposition invokes the laws we have just enunciated multiple
times.
\[
\begin{array}{lclll}
(a+b) \cdot (c+d) & = & (a+b) \cdot c \ + \ (a+b) \cdot d
& & \mbox{distributive law} \\ 
  & = & c \cdot (a+b) \ + \ d \cdot (a+b)
& & \mbox{commutativity of multiplication} \ (2 \times) \\
  & = & c \cdot a + c \cdot b + d \cdot a + d \cdot b 
& & \mbox{distributive law} \ (2 \times) \\
  & = & a \cdot c + b \cdot c + a \cdot d + b \cdot d
& & \mbox{commutativity of multiplication} \ (4 \times) \\
  & = &  a \cdot c + a \cdot d + b \cdot c + b \cdot d
& & \mbox{commutativity of addition}
\end{array}
\]
\qed

We close our short survey of the laws of arithmetic with the following
important two-part law.
\begin{itemize}
\item
{\it The law of inverses}.\index{inverse laws for
  arithmetic}\index{laws of arithmetic!inverse laws}
%
  \begin{itemize}
  \item
Every number $x$ has an {\em additive inverse},\index{additive inverse}
i.e., a number $y$ such that $x+y =0$.  This inverse is $x$'s {\it
  negative} $-x$.\index{additive inverse!negative as additive inverse}
  \item
Every {\em nonzero} number $x \neq 0$ has a {\em multiplicative
  inverse},\index{multiplicative inverse} i.e., a number $y$ such that
$x \cdot y = 1$.  This inverse is $x$'s {\it reciprocal},
$1/x$.\index{multiplicative inverse!reciprocal as multiplicative inverse}
  \end{itemize}
\end{itemize}

We close this section with another of our ``fun'' propositions.

\addcontentsline{toc}{paragraph}{-- A fun result: A ``trick'' for
  squaring some integers}

\begin{prop}
\label{thm:75x65=4925}
Let $n$ be any number that has a $2$-digit decimal of the form $\delta
5$, where $\delta \in \{ 0,1,2,3,4,5,6,7,8,9 \}$,
so that
\[ n \ = \ 10 \cdot \delta + 5
\]
Then 
\[ n^2 \ = \ 100 \cdot \delta \cdot (\delta+1) + 25. \]
In other words, one obtains a base-$10$ numeral for $n^2$ by
multiplying $\delta$ by $\delta +1$ and appending $25$ to the product.
\end{prop}

\noindent
Examples of Proposition ~\ref{thm:75x65=4925} include
$25^2 = 625$ (because $2 \cdot 3 = 6$) and $75^2 = 5625$ (because $7
\cdot 8 = 56$).

\noindent {\it Verification} (for general $\delta$).
%
We invoke Proposition~\ref{prop:(a+b)(c+d)} and the distributive law.
\[
\begin{array}{lclll}
n^2 & = & (10 \cdot \delta + 5)^2 & & \mbox{Given} \\
    & = & 100 \cdot \delta^2 \ + \ 100 \cdot delta \ + \ 25
              & & \mbox{the proposition} \\
    & = & 100 \cdot (\delta^2 \ + \ \delta) \ + \ 25
              & & \mbox{factoring: distributive law} \\
    & = & 100 \cdot \delta \cdot (\delta + 1) \ + \ 25
              & & \mbox{factoring: distributive law} \\
\end{array}
\]
\qed

\subsection{Rational Arithmetic: A Worthwhile Exercise}
\label{sec:Rational-arithmetic}
\index{number!rational!arithmetic}

In Section~\ref{sec:rationals} we defined the rational numbers and
reviewed why they were needed to compensate for the general lack of
multiplicative inverses in the integers.  But we did not review how to
perform arithmetic on the elements of the set $\Q$.  We correct this
shortcoming now.  Of course, the reader will have encountered rational
arithmetic long ago---but we are now reviewing the topic in order to
provide the reader with a set of worthwhile exercise to reinforce the
mathematical thinking whose presentation is our main goal.

\medskip

The rational numbers build their rules for arithmetic upon the
corresponding rules for integers.  For all $p/q$ and $r/s$ in $\Q$:
\[
\begin{array}{|llcl|}
\hline
\mbox{\small\sf Addition:} & 
{\displaystyle
{p \over q} + {r \over s} }
  & = &
{\displaystyle
 \frac{p \cdot s + r \cdot q}{q \cdot s} }  \\
 & & & \\
\mbox{\small\sf Subtraction:} &
{\displaystyle
{p \over q} + {r \over s} }
  & = & 
{\displaystyle
{p \over q} + {(-r) \over s} } \\
 & & & \\
\mbox{\small\sf Multiplication:} &
{\displaystyle
{p \over q} \cdot {r \over s} }
  & = & 
{\displaystyle
\frac{p \cdot r}{r \cdot s} } \\
  & & & \\
\mbox{\small\sf Division:} &
{\displaystyle
{p \over q} \div {r \over s} }
  & = &
{\displaystyle
{p \over q} \cdot {s \over r} } \\
\hline
\end{array}
\]

It is worth verifying that rational arithmetic as thus defined behaves
in the required manner; in particular that rational arithmetic:
\begin{itemize}
\item
works correctly when the argument rational numbers are, in fact,
integers, i.e., when $q = s = 1$ in the preceding table.
\item
treats the number $0$ appropriately, i.e., as an additive identity and
a multiplicative annihilator; cf., Sections~\ref{sec:arithmetic-tools}
and~\ref{sec:Arithmetic-Laws}.
\item
obeys the required laws; cf., Section~\ref{sec:Arithmetic-Laws}.

Verifying the distributivity of rational multiplication over rational
addition will be a particularly valuable exercise because of the
required amount of manipulation.
\end{itemize}

\section{Basic Algebraic Concepts and Their Manipulations}

\subsection{Powers and polynomials}

A conceptually powerful notational construct is the operation of {\it
  raising a number to a power:}\index{raising a number to a power}
%
For real numbers $a$ and $b$, the {\it $b$th power} of $a$, denoted
$a^b$ is defined by the system of equations
\begin{equation}
\label{eq:power-def}
\begin{array}{llll}
\mbox{for all numbers $a>0$} & & & a^0 = 1 \\
 & & & \\
\mbox{for all numbers $a, b, c$} & & & a^b \cdot a^c = a^{b+c}.
\end{array}
\end{equation}
This deceptively simple definition has myriad consequences which we
often take for granted.
\begin{itemize}
\item
For all numbers $a>0$, the number $a^0 = 1$.

This follows (via cancellation) from (\ref{eq:power-def}) via the fact
that
\[ a^b \cdot a^0 \ = \ a^{b+0} \ = \ a^b \ = \ a^b \cdot 1.  \]

\item
For all numbers $a >0$, the number $a^{1/2}$\index{$a^{1/2}$: the
  square root of number $a$}
is the {\it square root} of $a$,\index{square root}
i.e., $a^{1/2}$ is the (unique, via cancellation) number $b$ such that
$b^2 = a$.  Another common notation for The number $a^{1/2}$ is
$\sqrt{a}$.\index{$\sqrt{a}$: the square root of number $a$}

This follows from (\ref{eq:power-def}) via the fact that
\[ a \ = \ a^1 \ = \ a^{(1/2) + (1/2)} \ = \ a^{1/2} \cdot a^{1/2} \ = \
\left(a^{1/2}\right)^2. \]

\item
For all numbers $a>0$ and $b$, the number $a^{-b}$ is the {\it
  multiplicative inverse}\index{multiplicative inverse}
of $a^b$, meaning that $a^b \cdot a^{-b} = 1$

This follows from (\ref{eq:power-def}) via the fact that
\[ a^b \cdot a^{-b} \ = \ a^{(b + (-b))} \ = \ a^0 \ = \  1 \]
\end{itemize}
When the power $b$ is a positive integer, then definition
(\ref{eq:power-def}) can be cast in the following attractive inductive
form:
\begin{equation}
\label{eq:power-def-integer}
\begin{array}{llll}
\mbox{for all numbers $a>0$} & & & a^0 = 1 \\
 & & & \\
\mbox{for all numbers $a$ and integers $b$} & & & a^{b+1} = a \cdot
a^b.
\end{array}
\end{equation}
Summing up, we now know about powers that are integral or fractional,
positive, zero, or negative

We want the student to master the notions of polynomials and their
associated notions, such as degrees and coefficients, and computations
therewith, including polynomial summation and multiplication.  While
polynomial multiplication is often considered ``non-elementary'', it
must be mastered in order to fully understand positional number
systems; it is also essential, e.g., when discussing a range of topics
relating to, say, fault tolerance and encryption).

\subsection{Exponentials and Logarithms}

This section introduces the fundamentals of two extremely important
classes of functions which are functional inverses of each other, in
the following sense.  Functions $f$ and $g$ are {\it functional
  inverses}\index{functional inverse} of each other if for all
arguments $x$
\begin{equation}
\label{eq:functional-inverse}
f(g(x)) \ = \ x.
\end{equation}

\subsubsection{Basic definitions}

\addcontentsline{toc}{paragraph}{A. Exponential functions} 
\noindent {\small\sf A. Exponential functions}.\index{Exponential functions}
%
A function $f$ is {\it exponential} if there is a positive number $b$
such that, for all $x$,
\begin{equation}
\label{eq:exponential-defn}
f(x) \ = \ b^x.
\end{equation}
The number $b$ is the {\it base}\index{base of exponential}
%
of $f(x)$.  The basic arithmetic properties of exponential functions
are derivable from (\ref{eq:power-def}), so we leave these details to
the reader and turn immediately to the functional inverses of
exponential functions..

\addcontentsline{toc}{paragraph}{B. Logarithmic functions}
\noindent {\small\sf B. Logarithmic functions}.\index{Logarithmic
  functions}
%
Given an integer $b >1$ (mnemonic for ``base''), the {\em base-$b$
  logarithm}\index{base-$b$ logarithm}
%
of a real number $a > 0$ is denoted $\log_b a$ and defined by the
equation\index{$\log_b a$: the base-$b$ logarithm of number $a$}
\begin{equation}
\label{eq:logarithm-defn}
a \ = \ b^{\log_b a}.
\end{equation}
Logarithms are partial functions: $\log_b a$ is not defined for
non-positive arguments.

The base $b = 2$ is so prominent in the contexts of computation theory
and information theory that we commonly invoke one of two special
notations for $\log_2 a$: (1) we often elide the base-$2$ subscript
and write $\log a$;\index{$\log(a)$: base-$2$ logarithm of number $a$}
(2) we employ the specialized notation $\ln a$\index{$\ln(a)$:
  base-$2$ logarithm of number $a$}.  Notationally:
\[ \log_2 a \ \eqdef \ \log a \ \eqdef \ \ln a \]

We leave to the reader the easy verification, from
(\ref{eq:logarithm-defn}), that the {\it base-$b$ logarithmic
  function}, defined by
\begin{equation}
\label{eq:log-function-defn}
f(x) \ = \ \log_b x
\end{equation}
is the functional inverse of the base-$b$ exponential function.

\subsubsection{Fun facts about exponentials and logarithms}

Definition (\ref{eq:logarithm-defn}) exposes and---even more
importantly---explains myriad facts about logarithms that we often
take for granted.

\begin{prop}
For any base $b >1$, for all numbers $x >0$, $y>0$,
\[ \log_b (x \cdot y) \ = \ \log_b x \ + \ \log_b y \]
\end{prop}

\noindent {\it Verification}.
%
Definition (\ref{eq:logarithm-defn}) tells us that $x = b^{\log_b x}$
and $y = b^{\log_b y}$.  Therefore,
\[ x \cdot y \ = \ b^{\log_b x} \cdot b^{\log_b y} \ = \
b^{\log_b x \ + \ \log_b y}, \]
by the laws of powers.  Taking base-$b$ logarithms of the first and
last terms in the chain yields the claimed equation.
\qed

Many students believe that the following result is a {\em convention}
rather than a consequence of the basic definitions.  {\em The logarith
  of $1$ to any base is $0$.}

\begin{prop}
For any base $b >1$,
\[ \log_b 1 \ = \ 0 \]
\end{prop}

\noindent {\it Verification}.
%
We note the following chain of equalities.
\[  b^{\log_b x} \ = \ b^{\log_b (x \cdot 1)} 
\ = \ b^{(\log_b x) + (\log_b 1)} 
\ = \ b^{\log_b x} \cdot b^{\log_b 1}
\]
Hence, $b^{\log_b 1} \ = \ 1$.  If $\log_b 1$ did not equal $0$, then
$b^{\log_b 1}$ would exceed $1$.
\qed

\begin{prop}
For all bases $b > 1$ and all numbers $x, y$,
\[ x^{\log_b y} \ = \ y^{\log_b x} \]
\end{prop}

\noindent {\it Verification}.
%
We invoke (\ref{eq:logarithm-defn}) twice to remark that
\[ \left[x^{\log_b y} \ = \ b^{(\log_b x) \cdot (\log_b y)}\right]
\ \ \mbox{ and } \ \ 
\left[y^{\log_b x}\ = \ b^{(\log_b y) \cdot (\log_b x)}\right] \]
The commutativity of addition completes the verification.
\qed

\begin{prop}
For any base $b >1$,
\[ \log_b (1/x) \ = \ - \log_b x \]
\end{prop}

\noindent {\it Verification}.
%
This follows from the fact that $\log_b 1 =0$, coupled with the
product law for logarithms.
\[ \log_b x + \log_b (1/x) \ = \ \log_b (x \cdot (1/x))
\  = \ \log_b 1 \ = \ 0 
\]
\qed

\begin{prop}
For any bases $a, b >1$,
\begin{equation}
\label{eq:log-exp-0}
\log_b x \ = \ \left(\log_b a \right) \cdot \left( \log_a x \right).
\end{equation}
\end{prop}

\noindent {\it Verification}.
%
We begin by noting that, by definition,
Note that
\begin{equation}
\label{eq:log-exp-1}
 x \ = \ b^{\log_b x} \ = \ a^{\log_a x} .
\end{equation}
Let us take the base-$b$ logarithm of the second and third expressions
in (\ref{eq:log-exp-1}) and then invoke the product law for logarithms.
From the second expression in (\ref{eq:log-exp-1}), we find that
\begin{equation}
\label{eq:log-exp-2}
 \log_b \left(b^{\log_b x} \right) \ = \ \log_b x .
\end{equation}
From the third expression in (\ref{eq:log-exp-1}), we find that
\begin{equation}
\label{eq:log-exp-3}
 \log_b \left( a^{\log_a x} \right) \ = \
\left(\log_b a \right) \cdot \left( \log_a x \right).
\end{equation}
We know from (\ref{eq:log-exp-1}) that the righthand expressions in
(\ref{eq:log-exp-2}) and (\ref{eq:log-exp-3}) are equal, whence
(\ref{eq:log-exp-0}) \qed

If we set $x = b$ in (\ref{eq:log-exp-0}), then we find the following
marvelous equation.

\begin{prop}
For any integers $a, b >1$,
\begin{equation}
\left(\log_b a \right) \cdot \left( \log_a b \right) \ = \ 1 \ \ \ \ \
\mbox{ or, equivalently, } \ \ \ \ \
\log_b a \ = \ \frac{1}{\log_a b} .
\end{equation}
\end{prop}



\subsubsection{Exponentials and logarithms within information theory}
\label{sec:count-strings}

The student should recognize and be able to reason about the following
facts.  If one has an alphabet of $a$ letters/symbols and must provide
distinct string-label ``names'' for $n$ items, then at least one
string-name must have length no shorter than $\lceil \log_a n \rceil$.

\begin{prop}
\label{thm:bound-stringnames-lgth-k}
Say that one must assign distinct labels to $n$ items, via strings
over an alphabet of $a$ letters.  Then at least one string-label must
have length no shorter than $\lceil \log_a n \rceil$.
\end{prop}

\noindent {\em Verification.}
%
Let $Sigma$ be an alphabet of $a$ letters/symbols.  For each integer
$k \geq 0$ (i.e., for each $k \in \N$), let $\Sigma^{(k)}$ denote the
set of all length-$k$ strings over $\Sigma$.  The bound of
Proposition~\ref{thm:bound-stringnames-lgth-k} follows by counting the
number of strings of various lengths over $\Sigma$, because each such
string can label at most one item.  Let us, therefore, inductively
evaluate the cardinality $|\Sigma^{(k)}|$ of each set $\Sigma^{(k)}$.
\begin{itemize}
\item
$|\Sigma^{(0)}| =1$

This is because the null-string $\varepsilon$ \index{$\varepsilon$:
  the null string, of length $0$}
\index{null string $\varepsilon$}
is the unique string in $\Sigma^{(0)}$, i.e., $\Sigma^{(0)} = \{
\varepsilon \}$.

\item
$|\Sigma^{(k+1)}| = |Sigma| \cdot |\Sigma^{(k)}|$.

This reckoning follows from the following recipe for creating all
strings over $\Sigma$ of length $k+1$ from all strings of length $k$.
\[
\Sigma^{(k+1)} \ = \ \{ \sigma x \ | \ \sigma \in \Sigma, x \in
\Sigma^{(k)} \}
\]
This recipe is correct because
  \begin{itemize}
  \item
Each string in $\Sigma^{(k+1)}$, as constructed, has length $k+1$.

This is because the recipe adds a single symbol to a length-$k$
string.
  \item
For each string $x \in \Sigma^{(k)}$, there are $|\Sigma|$ distinct
strings in $\Sigma^{(k+1)}$, as constructed.

This is because each string in $\Sigma^{(k+1)}$ begins with a distinct
symbol from $\Sigma$.

  \item
$\Sigma^{(k+1)}$, as constructed, contains all strings of length $k+1$
over $\Sigma$.

This is because for each $\sigma \in \Sigma$ and each $x \in
\Sigma^{(k)}$, the string $\sigma x$ is in $\Sigma^{(k+1)}$, as
constructed.
  \end{itemize}
\end{itemize}
We thus have the following recurrence.
\begin{eqnarray*}
|\Sigma^{(0)}| & = & 1 \\
|\Sigma^{(k+1)}| & = & |\Sigma| \cdot |\Sigma^{(k)}| \ \ \ \ 
\mbox{ for } \ k \geq 0
\end{eqnarray*}
Using the Master Theorem, we thus find explicitly that

\noindent
For each $\ell \in \N$,
\[ |\Sigma^{(\ell)}| \ = \ \frac{|\Sigma|^{\ell+1} \ - \ |\Sigma|}
{|\Sigma| -1} \ \leq \ c \cdot |\Sigma|^{\ell}
\]
for some constant $c$.  In order for this quantity to reach $n \in
\N$, we must have
\[ \ell \ > \ d \cdot \log_{|\Sigma|} n   \]
for some small constant $d$.
\qed

**HERE

Focus on 
Say, inductively, that there are $\ell_k$


\begin{prop}
\label{thm:Num-strings-lgth-k}
The number of distinct strings of length $k$ over an alphabet of $a$
letters is $a^k$.
\end{prop}




\subsection{Arithmetic and geometric sequences and series}
\label{sec:sums-series}

The ability to sum -- and perhaps approximate -- simple series,
including, {\em at least}, finite arithmetic series and both finite
and infinite geometric series.

\addcontentsline{toc}{paragraph}{A. Arithmetic sequences and series}
\noindent {\small\sf A. Arithmetic sequences and series.}
%
We define arithmetic sequences and learn how to calculate their sums.

\begin{equation}
\label{eq:arith-seq}
\begin{array}{l}
\mbox{An $n$-term arithmetic sequence:} \\
\hspace*{.25in}a, \ a+b, \ a+2b, \ a+3b, \ \ldots, a+(n-1)b \\
\\
\mbox{The corresponding arithmetic series:} \\
\hspace*{.25in}a + (a+b) + (a+2b) + (a+3b) + \cdots + (a+(n-1)b) \\
\hspace*{.5in} = \
an + b \cdot (1 + 2 + \cdots + n-1)
\end{array}
\end{equation}
We can, thus, sum the arithmetic series in (\ref{eq:arith-seq}) by
determining the sum of the first $m$ positive integers; $m = n-1$ in
(\ref{eq:arith-seq}).  We accomplish this via a device known to Karl
Friedrich Gauss as a pre-teen.  Let $S_m$ denote the desired sum.
\begin{equation}
\label{eq:arith-series}
\begin{array}{lccccccccc}
\multicolumn{10}{l}{\mbox{Write the sum $S_m$ ``forward'':}} \\
\hspace*{.5in}S_m = & 1 & + & 2   & + & \cdots & + & (m-1) & + & m \\
\multicolumn{10}{l}{\mbox{Write $S_m$ in reverse:}} \\
\hspace*{.5in}S_m = & m & + & (m-1) & + & \cdots & + & 2     & + & 1
\end{array}
\end{equation}
Now add the two versions of $S_m$ in (\ref{eq:arith-series}) {\em
  columnwise}.  Because each column-sum equals $m+1$, we find that
$2S_m = m(m+1)$, so that
\begin{equation}
\label{eq:arith-sum}
S_m  \ = \
1 + 2 + \cdots + (m-1) + m \ = \ {1 \over 2} m (m+1) \ 
\eqdef \ {{m+1}  \choose 2}.
\end{equation}
It follows that our original series in (\ref{eq:arith-seq}) sums as
follows.
\[
a + (a+b) + (a+2b) + (a+3b) + \cdots + (a+(n-1)b) \ = \
an + b \cdot {n \choose 2}. 
\]

\addcontentsline{toc}{paragraph}{B. Geometric sequences and series}
\noindent {\small\sf B. Geometric sequences and series.}
%
We define geometric sequences and learn how to calculate their sums.

\begin{equation}
\label{eq:geom-seq}
\begin{array}{l}
\mbox{An $n$-term geometric sequence:} \\
\hspace*{.25in}a, \ ab, \ ab^2, \ \ldots, ab^{n-1} \\
\\
\mbox{The corresponding geometric series:} \\
\hspace*{.25in}a + ab + ab^2 + \cdots + ab^{n-1} \ = \
 a (1+ b + b^2 + \cdots + b^{n-1})
\end{array}
\end{equation}
Easily, we can sum the series in (\ref{eq:geom-seq}) by summing just
the sub-series
\begin{equation}
\label{eq:geom-series}
S_{b}(n) \ \eqdef \
1+ b + b^2 + \cdots + b^{n-1}.
\end{equation}
We proceed as follows.  Write $S_{b}(n)$ so that its terms are in {\em
  decreasing} order.  We thereby isolate two cases.
\begin{enumerate}
\item
Say first that $b > 1$.  In this case, we write the series in the form
\[ S^{b>1}_{b}(n) \ = \ b^{n-1} + b^{n-2} + \cdots + b^2 + b + 1, \]
and we note that
\[ S^{b>1}_{b}(n) \ = \
b^{n-1} \ + \ {1 \over b} \cdot S^{b>1}_{b}(n) \ - \ {1 \over b}. \]
In other words, we have
\[ \left( 1 - {1 \over b} \right)  S^{b>1}_{b}(n) \ = \ b^{n-1} - {1
  \over b}, \]
or equivalently,
\begin{equation}
\label{eq:geom-sum:b>1}
S^{b>1}_{b}(n) \ = \ \frac{b^{n}- 1}{b - 1}.
\end{equation}

\item
Alternatively, if $b < 1$, then we write the series in the form
\[ S^{b<1}_{b}(n) \ = \ 1+ b + b^2 + b^3 + \cdots + b^{n-1}. \]
and we note that
\[ S^{b<1}_{b}(n) \ = \
1 \ + \ b \cdot S^{b<1}_{b}(n) \ - \ b^n. \] 
In other words,
\[ (1-b) S^{b<1}_{b}(n) \ = \ 1 \ - \ b^n \]
or equivalently,
\begin{equation}
\label{eq:geom-sum:b<1}
S^{b<1}_{b}(n) \ = \ \frac{1 - b^n}{1-b}.
\end{equation}
\end{enumerate}

Note that $S^{b>1}_{b}(n)$ and $S^{b<1}_{b}(n)$ actually have the same
form.  We have chosen to write them differently to stress their {\em
  approximate} values, which are useful in ``back-of-the-envelope''
calculations:  For very large values of $n$, we have
\begin{equation}
\label{eq:geom-sum:approx}
S^{b>1}_{b}(n) \ \approx \ \frac{b^n}{b-1} \ \ \
\mbox{while} \ \ \
S^{b<1}_{b}(n) \ \approx \ \frac{1}{1-b} .
\end{equation}

\bigskip

\addcontentsline{toc}{paragraph}{C. An amusing result about numerals}
\noindent {\small\sf C. An amusing result about numerals.}
%
We illustrate a somewhat surprising nontrivial fact about integers
that are ``encoded'' in their positional numerals.  We hope that this
``fun'' result will inspire the reader to seek kindred numeral-encoded
properties of numbers.

\begin{prop}
\label{thm:div-by-b-bar}
An integer $n$ is divisible by an integer $m$ if, and only if, $m$
divides the sum of the digits in the base-$(m+1)$ numeral for $n$.
\end{prop}

The most familiar instance of this result is phrased in terms of our
traditional use of base-$10$ (decimal) numerals. \\
{\it An integer $n$ is divisible by $9$ if, and only if, the sum of
  the digits of $n$'s base-$10$ numeral is divisible by $9$.}

\smallskip

\noindent {\it Verification} (for general base $b$).
%
Of course, we lose no generality by focusing on numerals without
leading $0$'s, for adding leading $0$'s does not alter a numeral's sum
of digits.

To enhance legibility, let $b = m+1$, so that we are looking at the
base-$b$ numeral for $n$.  Say that
\[ n \ = \ \delta_k \cdot b^k + \delta_{k-1} \cdot b_{k-1} + \cdots +
\delta_1 \cdot b + \delta_0, \]
so that the sum of the digits of $n$'s base-$b$ numeral is
\[ s_b(n) \ \eqdef \ \delta_k + \delta_{k-1} + \cdots + \delta_1 + \delta_0. \]
We next calculate the difference $n - s_b(n)$.  We proceed as
follows, digit by digit.
\begin{equation}
\label{eq:sum-of-digits}
\begin{array}{ccccccccccc}
n & = &
\delta_k \cdot b^k & + & \delta_{k-1} \cdot b^{k-1} & + & \cdots
  & + & \delta_1 \cdot b & + & \delta_0 \\
s_b(n) & = &
\delta_k & + & \delta_{k-1} & + & \cdots & + & \delta_1 & + & \delta_0 \\
\hline
n - s_b(n) & = &
\delta_k \cdot (b^k -1) & + &
\delta_{k-1} \cdot (b^{k-1} -1) & + &
\cdots & + &
\delta_1 \cdot (b-1) & & 
\end{array}
\end{equation}

We now revisit summation (\ref{eq:geom-sum:b>1}).  Because $b$ is a
positive integer, so that $1 + b + \cdots + b^{a-2} + b^{a-1}$ is also
a positive integer, we adduce from the summation that {\em the integer
  $b^a -1$ is divisible by $b-1$.}

We are almost home.  Look at the equation for $n - s_b(n)$ in the
system (\ref{eq:sum-of-digits}).  As we have just seen, every term on
the righthand side of that equation is divisible by $b-1$.  It follows
therefore, that the lefthand expression, $n - s_b(n)$, is also
divisible by $b-1$.  An easy calculation, which we leave to the
reader, now shows that this final fact means that $n$ is divisible by
$b-1$ if, and only if, $s_b(n)$ is.  \qed


\subsection{Linear recurrences}
\label{sec:linear-recurrences}

By the time the reader has reached this paragraph, she has the
mathematical tools necessary to prove and apply what is called {\it
  The Master Theorem for Linear Recurrences} \cite{CLRS}.  This level
of mathematical preparation should be adequate for most
early-undergrad courses on data structures and algorithms, as well for
for analyzing a large fraction of the algorithms that she is likely to
encounter in daily activities.

\begin{theorem}[The Master Theorem for Linear Recurrences]
\label{thm:master-thm}
\index{The Master Theorem for Linear Recurrences}
Let the function $F$ be specified by the following linear recurrence.
\begin{equation}
\label{eq:Lin-Recur:start}
F(n) \ = \ \left\{
\begin{array}{cl}
a F(n/b) + c & \mbox{for } n \geq b \\
c & \mbox{for } n < b
\end{array}
\right.
\end{equation}
Then the value of $F$ on any argument $n$ is given by
\begin{equation}
\label{eq:Lin-Recur:solve}
\begin{array}{lclll}
F(n) & = & (1 + \log_b n)c &  & \mbox{if } a=1 \\
     &   &                 &  & \\
     & = &
  {\displaystyle
  \frac{1-a^{\log_b n}}{1-a} \ \approx \ \frac{1}{1-a}
  }
 &  & \mbox{if } a<1 \\
    &   &                  & & \\
    & = &
  {\displaystyle
\frac{a^{\log_b n} -1}{a-1}
  }
 & & \mbox{if } a>1
\end{array}
\end{equation}
\end{theorem}

\begin{proof}
In order to discern the recurring pattern in
(\ref{eq:Lin-Recur:start}), let us begin to ``expand'' the specified
computation by replacing occurrences of $F(\bullet)$ as mandated in
(\ref{eq:Lin-Recur:start}).
\begin{equation}
\label{eq:Lin-Recur:expand}
\begin{array}{lcccc}
F(n) & = & a F(n/b) + c & & \\
     & = & a \left( a F(n/b^2) + c \right) + c
             & = & a^2 F(n/b^2) + (1 + a)c \\
     & = & a^2 \left( a F(n/b^3) + c \right) + (1+a)c
             & = & a^3 F(n/b^3) + (1 + a + a^2)c \\
     &   & \vdots & & \vdots \\
     & = & 
{\displaystyle
\left( 1 + a + a^2 + \cdots + a^{\log_b n} \right) c
} & &
\end{array}
\end{equation}
The segment of (\ref{eq:Lin-Recur:expand}) ``hidden'' by the vertical
dots betokens an induction that is left to the reader.  Equations
(\ref{eq:geom-sum:b>1}) and (\ref{eq:geom-sum:b<1}) now enable us to
demonstrate that (\ref{eq:Lin-Recur:solve}) is the case-structured
solution to (\ref{eq:Lin-Recur:start}).
\qed
\end{proof}



\section{Elementary counting}

\subsection{Binary Strings and Power Sets}

\begin{prop}
\label{thm:b-ary strings}
For every integer $b > 1$, there are $b^n$ $b$-ary strings of length
$n$.
\end{prop}

\noindent {\it Verification.}
%
The asserted numeration follows most simply by noting that there are
always $b$ times as many $b$-ary strings of length $n$ as there are of
length $n-1$.  This is because we can form the set of $b$-ary strings
of length $n$ as follows.  Take the set $A_{n-1}$ of $b$-ary strings
of length $n-1$, and make $b$ copies of it, call them $A^{(0)}_{n-1},
A^{(1)}_{n-1}, \ldots, A^{(b-1)}_{n-1}$.  Now, append $0$ to every
string in $A^{(0)}_{n-1}$, append $1$ to every string in
$A^{(1)}_{n-1}$, \ldots, append $\bar{b} = b-1$ to every string in
$A^{(b-1)}_{n-1}$.  The thus-amended sets $A^{(i)}_{n-1}$ are mutually
disjoint (because of the terminal letters of their respective
strings), and they collectively contain all $b$-ary strings of length
$n$.  \qed

\begin{prop}
\label{thm:power-sets}
The power set $\p(S)$ of a finite set $S$ contain $2^{|S|}$ elements.
\end{prop}

\noindent {\it Verification.}
%
Let us begin by taking an arbitrary finite set $S$---say of $n$
elements---and laying its elements out in a line.  We thereby
establish a correspondence between $S$'s elements and positive
integers: there is the first element, which we associate with the
integer $1$, the second element, which we associate with the integer
$2$, and so on, until the last element along the line gets associated
with the integer $n$.

Next, let's note that we can specify any subset $S'$ of $S$ by
specifying a length-$n$ {\em binary (i.e., base-$2$) string}, i.e., a
string of $0$'s and $1$'s.  The translation is as follows.  If an
element $s$ of $S$ appears in the subset $S'$, then we look at the
integer we have associated with $s$ (via our linearization of $S$),
and we set the corresponding bit-position of our binary string to $1$;
otherwise, we set this bit-position to $0$.  In this way, we get a
distinct subset of $S$ for each distinct binary string, and a distinct
binary string for each distinct subset of $S$.

Let us pause to illustrate our correspondence between sets and strings
by focussing on the set $S = \{a,b,c\}$.  Just to make life more
interesting, let us lay $S$'s elements out in the order $b,a,c$, so
that $b$ has associated integer $1$, $a$ has associated integer $2$,
and $c$ has associated integer $3$.  We depict the elements of $\p(S)$
and the corresponding binary strings in the following table.
\begin{center}
\fbox{
\begin{tabular}{c|c|c}
Binary string & Set of integers & Subset of $S$ \\
\hline
$000$ & $\emptyset$ & $\emptyset$ \\
$001$ & $\{3\}$     & $\{c\}$ \\
$010$ & $\{2\}$     & $\{a\}$ \\
$011$ & $\{2,3\}$   & $\{a,c\}$ \\
$100$ & $\{1\}$     & $\{b\}$ \\
$101$ & $\{1,3\}$   & $\{b,c\}$ \\
$110$ & $\{1,2\}$   & $\{a,b\}$ \\
$111$ & $\{1,2,3\}$ & $\{a,b,c\} =S$
\end{tabular}
}
\end{center}

Back to the Proposition: We have verified the following: {\em The
  number of length-$n$ binary strings is the same as the number of
  elements in the power set of $S$!}  The desired numeration thus
follows by the ($b=2$) instance of Proposition~\ref{thm:b-ary
  strings}.  \qed

\begin{quote}
The binary string that we have constructed to represent each set of
integers $N \subseteq \{0, 1, \ldots, n-1\}$ is called the {\it
(length-$n$) characteristic vector}\index{characteristic vector}
{\it of the set} $N$.  Of course, the finite set $N$ has
characteristic vectors of all finite lengths.  Generalizing this idea,
{\em every} set of integers $N \subseteq \N$, whether finite or
infinite, has an {\em infinite} characteristic vector, which is formed
in precisely the same way as are finite characteristic vectors, but
now using the set $\N$ as the base set.
\end{quote}






 within discrete frameworks, including introducing
discrete probability/likelihood as a ratio:
\[ 
\frac{\mbox{number of targeted events}}{\mbox{number of possible events}}
\]


\section{Congruences and Modular Arithmetic}



\section{Numbers and Numerals}

\subsection{Number vs.~Numeral: Object vs.~Name}


\subsection{Geometric series and positional number systems}

The relation between simple geometric series and numeration within
positional number systems -- including changing bases in such systems.


\section{Useful Nonalgebraic Notions}
\label{sec:extra-functions}

\subsection{Nonalgebraic Notions Involving Numbers}

\ignore{****************
\addcontentsline{toc}{paragraph}{Floors and Ceilings}
{\small\sf Floors and ceilings}.
%
Given any real number $x$, we denote by $\lfloor x \rfloor$ the {\em
  floor}\index{$\lfloor x \rfloor$: the floor of real number $x$}
(or {\em integer part})\index{$\lfloor x \rfloor$: the integer part
  of real number $x$}
%
of $x$, which is the largest integer that that does not exceed $x$.
Symmetrically, we denote by $\lceil x \rceil$ the 
{\em ceiling}\index{$\lceil x \rceil$: the ceiling of real number $x$}
of $x$, which is the smallest integer that is at least as large as
$x$.  For any nonnegative integer $n$,
\[ \lfloor n \rfloor  \ = \ \lceil n \rceil \ = \ n;  \]
for any positive rational number $n + p/q$, where $n$, $p$, and $q$
are positive integers and $p < q$,
\[ \lfloor n + p/q \rfloor  \ = \ n, \ \mbox{ and } \
\lceil n+ p/q \rceil \ = \ n+1.  \]

\addcontentsline{toc}{paragraph}{Absolute values/Magnitudes}
{\small\sf Absolute values, or, magnitudes}
%
Given any real number $x$, positive or negative, we denote by $|x|$
the {\it absolute value}\index{$|x|$: the absolute value of real number $x$}
or {\it magnitude}\index{$|x|$: the magnitude of real number $x$}
of $x$.  If $x \geq 0$, then $|x| = x$; if $x < 0$, then $|x| = -x$.

**********}


If the intended curriculum will approach more sophisticated
application areas such as robotics or data science or information
retrieval or data mining (of course, at levels consistent with the
students' preparation), then one would do well to insist on
familiarity with notions such as:



\section{Advanced Topics}

\subsection{Measures of distance in tuple-spaces}

including the following
norms/metrics: $L_1$ (Manhattan, or, rook's-move distance), $L_2$
(Euclidean distance); $L_\infty$ (King's-move distance).


\subsection{Edit-distance: a measure of closeness in {\em string spaces}}

