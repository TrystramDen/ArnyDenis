%version of 04-26-19

\chapter{INTRODUCTION}
\label{ch:intro}

\section{Why Is This Book {\em Needed?}}
\label{sec:bookneeded}

How much mathematics does an aspiring computing professional
need---and at what level of expertise?  We beleve that the answer to
this pedagogically fundamental question is time dependent.

\medskip

\noindent {\it The early generation}.
In the early days of computing, all aspects of the field were
considered the domain of the ``techies''---the engineers and
scientists and mathematicians who designed the early computers and
figured out how to use them to solve a range of (mostly
compute-intensive) problems.  Back then, one expected every computing
professional to have a mastery of many mathematical topics.

\medskip

\noindent {\it Children of the early generation}.
Times---and the field of computing---changed.  The ``techies'' were
able to craft a variety of sophisticated tools that opened up the
world of computing to the general population.  Even people at the
lower levels of the educational edifice were able to use imagination
and ingenuity, rather than theorems and formulas, to produce
impressive software artifacts of considerable utility.

\medskip

\noindent {\it The modern generation}.
Pendulums were made to swing.  As we note in our {\it Manifesto}, we
now encounter almost daily problems that arise from unanticipated
concomitants of the often unstructured ingenuity that produced various
software artifacts.  Many would say that we need a renewed commitment
to technical discipline that will endow artifacts with
\begin{itemize}
\item
{\em understandable structure}, so that we can determine {\em what} went
wrong when something {\em does} go wrong.
\item
{\em sustainability}, so that changes, which are inevitable in complex
artifacts, will not create new problems
\item
{\em controllability}, so that ``smart'' artifacts do not become
modern instances of Dr.~Frankenstein's monster.
\end{itemize}
While we certainly need not return to the era of the ``techies'', it
is unquestionable that we do need a larger contingent of computing
professionals who have bona fide expertise that enables the needed
technical discipline.  This text is devoted to the mathematics that
underlies the needed science and engineering.

\bigskip

Of course, there are many introductory texts on discrete mathematics.
What separates this text from the others is the stratagem we have
implemented to accommodate our intended audience of aspiring computing
professionals.  The skeleton of this stratagem appears in the {\it
  Preface}.  We expand on it in the next section.



\section{Why Is {\em This} Book Needed?}
\label{sec:thisbookneed}

We live in a world of increasing professional and social diversity.

\medskip

Historically, computing curricula began as highly technically-oriented
studies leading to a degree such as a BS, within a school of science
or engineering.
\bigskip

\noindent \fbox{
\begin{minipage}{0.95\textwidth}
{\it Terminology:}
\begin{itemize}
\item
The {\it BS} (``Bachelor of Science'') degree is the technical version
of the first university degree in the US.  BS programs may specialize
in focus, self-identifying as, say, Computer Science or Computer
Engineering or Computing Science. 
\item
The {\it BA} (``Bachelor of Arts'') degree is the nontechnical version
of the first university degree in the US.
\item
Other University-level computing-oriented programs in the US include
{\it IT} (``Information Technology'') and {\it CA} (``Computing
Arts'').  These are often computing curricula that focus on the use of
``packages'', rather than on general programming.
\end{itemize}
\end{minipage}
}
\bigskip

\noindent
The evolution of the computing field has occasioned changes in this
curricular worldview.  The traditional curricula have been joined by
curricula that have deemphasized the traditional science and
engineering and mathematics courses, in order to emphasize the
humanities or social studies (leading, say, to a BA degree), or to
emphasize the {\em practice} of computing, either in a general setting
(leading, say, to a degree in IT) or in a focused applied field
(leading, say, to a degree in business or finance or law or \ldots).
Within this new world, many aspiring computer professionals arguably
need only specialized knowledge of mathematics---but the amount and
direction of the specialization can vary greatly based on an academic
program's focus.

\medskip

Add to the preceding reality the phenomenal broadening of the audience
for technical curricula, hence for some level of mathematics
education.  Many barriers to such education have now either
disappeared or at least diminished in impact.

\medskip

The number of nontraditional students seeking technical education is
growing, and the ways in which their nontraditional goals and
backgrounds manifest themselves is increasing.  This phenomenon is
important to computing educators, who must serve a large population of
students, with quite diverse needs and aspirations.

We have striven to keep this diversity in mind as we have written this
text.  We have included a broad range of material, in both subject and
level.  As noted in the {\it Preface}, we have tried to accommodate
the different backgrounds of our readership, while leading them all to
a level of mathematical maturity that will enable them to understand
mathematics and to {\em do} mathematics.  As we
discuss a range of mathematical topics for possible inclusion as
prerequisites for the undergraduate study of computing, we try to
indicate why the selected topics are needed.  Readers can then
evaluate which topics are needed for their needs.

\section{The Structure of This Book}
\label{sec:thisbook}

\subsection{Our Main Intellectual Targets}
\label{sec:book-overwiew}

This book is devoted to covering the discrete-mathematics
underpinnings of the endeavor of computing: from the design and
implementation of devices that perform the actions necessary to
compute to the design of the processes that control the
devices---including whatever communications are needed among processes
and among (sub)devices.  We have identified several intellectual
targets that guide our exposition.
\begin{enumerate}
\item
{\it Fundamental concepts}

\medskip

{\small\sf Examples:}
\begin{itemize}
\item%
sets---and their embellishments: tuples, arrays, tables, etc.---as
embodiments of {\it object}
\item
numbers---and their operational manifestations, numerals---as
embodiments of {\it quantity}
\item
graphs---in their many, varied, forms---as embodiments of {\it
  connectivity} and {\it relationship}
\item
algebras and functions---adding operations to sets, numbers, and
graphs---as embodiments of {\it structured dynamism} and {\it
  computing} and {\it process}
\end{itemize}
We thereby expand the scope of what can be thought about
``mathematically''.

\medskip

\item
{\it Fundamental representations}

\medskip

{\small\sf Examples:}
\begin{itemize}
\item
representing and thinking about numbers via many metaphors: slices of
pie, tokens arranged in stylized ways, characteristics of rectangles
of varying dimensionalities, textual objects
\item
using grouping and/or replication to represent relationships among
objects
\item
viewing interrelated objects via many structures: tables, tuples,
graphs, geometric drawings
\end{itemize}
We thereby expand the universe of conceptual pardigms one can use
while thinking ``mathematically''.

\medskip

\item
{\it Fundamental tools/techniques}

\medskip

{\small\sf Examples:}
\begin{itemize}
\item
using induction to extrapolate from simple examples to complex ones
\item
``hopping'' between the discrete and continuous mathematical worlds,
e.g., using integration to approximate summation
\item
using the conceptual tools of asymptotics to argue qualitatively about
quantitative phenomena.
\item
``hopping'' between the mathematical reasoning used in the ``real''
  world, vs.~the formal logics that enable such reasoning
\end{itemize}
We thereby expand the conceptual tools that one has access to when
{\em doing} mathematics.

\medskip

\item
{\it Beyond the fundamentals}

Our goal in writing this text is not just to expound on the basic
notions that enable one to study the phenomenon of computation and its
accompanying artifacts.  We strive additionally to inspire each reader
to dig deeper into the lore of at least one of these notions. To this
end, we have presented many basic notions via multiple explanations
and numerous exemplars.

\medskip

{\small\sf Examples:}
\begin{itemize}
\item
The ability to ``encode'' tuples of objects via the underlying objects
themselves lie at the base of some of the pillars of computing.  We
expound on two approaches to such encodings, one based on {\em prime
  numbers} and one on {\em pairing fucntions}.  We then show how this
basic ability can be applied to ``real'' computing problems such as
{\em efficiently storing arrays whose dimensions can change
  dynamically} and {\em tracking the computing output of each agent in
  a collaborating team}.
\item
The use of recursion as a control structure in computing has a purely
mathematical analogue, recurrences, which are extremely useful in
analyzing the correctness and efficiency of these computations.  Every
student gains at least some familiarity with the most basic versions
of recurrences.  We introduce variations on the theme of these
familiar recurrences, which offer approaches to more complicated
computational situations.  We illustrate the use of recurrences in
crafting a quite approachable analysis of an apparently complex {\em
  token game}.
\item
Many phenomena that appear to arise from arcane inherently
computational sources are actually rather easily understood
applications of purely mathematical phenomena.  We spend considerable
time expounding on such phenomena and their applications.  {\em
  Satisfiability problems} are one phenomenon of this sort; certain
{\em specialized number systems} are another.
\end{itemize}
\end{enumerate}


\subsection{Allocating our Targets to Chapters}
\label{sec:the chapters}

\subsubsection{Chapter~\ref{ch:doingmath}: ``Doing'' Mathematics}

In order to acclimate the reader to mathematics as a living discipline
and an integral part of the world of computing, we begin the book with
a chapter entitled {\it Doing Mathematics}.  This chapter focuses
mainly on the practicalities of mathematical reasoning, most
importantly by expounding on the idea of a mathematical proof.  We
survey the most commonly encountered techniques for crafting such
proofs and illustrate each with several examples.

Part of understanding proofs is being aware of what makes the endeavor
of proving things difficult.

\medskip

\noindent {\em Reasoning \underline{qualitatively} about
  \underline{quantitative} phenomena.}
%
An often-underappreciated aspect of human reasoning is our ability to
abstract by blurring descriptions.  We notice, for instance, the ways
in which all phenomena that experience {\em linear} growth differ from
phenomena that experience {\em quadratic} growth, and both classes
differ from phenomena that experience {\em cubic} growth---and so on,
through {\em quintic}, \ldots, {\em exponential}, and beyond.  The
extremely important topic of {\it asymptotics} abstracts from these
enumerated abstractions and allows us to reason qualitatively about
arbitrary inherently quantitative phenomena.

\medskip

\noindent {\em Coping with infinity}.
One recurring challenge when ``doing'' mathematics is dealing with
{\em infinity}.  Infinite objects, such as sets, behave rather
differently from the more familiar finite objects that we encounter in
our daily lives.  As but one example, there are ``equally many'' odd
integers as all integers, as measured by the ability to match the two
sets element by element---even though the first of these sets is
obtained from the second by discarding half of its elements.

A more subtle challenge manifests itself in the world of the infinite
resides in the myriad {\em paradoxes} that one encounters in this
world.  Does an arrow ever reach its target---given that it begins its
journey by traversing half then distance, then traverses half of the
remaining half, then half of the remaining quarter, \ldots?

\subsubsection{Chapter~\ref{ch:sets-BA-logic}: Sets and Their Algebras}

{\em Sets} are the stem cells of mathematics.  They begin as the most
primitive mathematical objects, but as soon as one endows them with
operations---for creating more inclusive (``bigger'') sets, for
selecting more exclusive (``smaller'') subsets, for enabling the
structure of tupling---they quickly afford one a powerful substrate
for doing almost all of mathematics.

\medskip

\noindent {\em The power of tupling}.
%
One can exploit tupling to isolate entities such as {\em relations},
which are the foundations of imposing and identifying {\em order} in,
and among, sets.  One can isolate {\em functions}, which enable one to
{\em encode} various sets as other, seemingly unrelated,
ones---example: encoding computer programs as positive integers.

\medskip

\noindent {\em Algebras: Operations and the laws that govern them}.
%
The operations within any collection of operations on sets inevitably
obey certain ``laws'' as they interact; for instance, operation
$\circ$ may (or may not) be {\em commutative}, i.e., obey the relation
\[ x \circ y \ = \ y \circ x \]
or it may (or may not) be {\em associative}, i.e.,obey the relation
\[ x \circ (y \circ z) \ = \ (x \circ y) \circ z \]
The combination of a set $S$ (of ``objects'') and a set of operations
on set $S$, together with the laws that govern the operations, is an
{\em algebra}---and there are myriad algebras that play important
roles in our lives.
\begin{itemize}
\item
{\em Boolean algebras} focus on sets and operations on sets.
\item
A special class of Boolean algebras is the class of {\em Propositional
  logics}.  These algebras underlie computing-related topics that
range from {\em digital logic}, the basis of all computer design, to the
{\em satisfaction problems} that play a major role in aspects of
{\em artificial intelligence} ({\em AI}).
\item
{\em Numerical algebras} govern our daily lives, by enabling us to
perform crucial basic functions such as
counting and performing arithmetic.
\end{itemize}


\subsubsection{Chapters~\ref{ch:numbers-numerals},
~\ref{ch:numbers-advanced}, ~\ref{ch:numerals}: Numbers and Numerals}

The first mathematical concepts that children learn about usually
involve numbers.  We spend much of our early lives expanding our
number-based knowedge base.  We proceed from counting to manipulating
numbers by adding and subtracting and multiplying and dividing them.
We progress from using fingers and toes as ``names'' of numbers to
using a variety of numeral-forming schemes.  For many of us,
``mathematics'' {\em means} ``numbers and arithmetic'': We never get
to explore the powerful mathematical concepts and tools that have
enabled many of the great advances of science and engineering.  Even
fewer of us get to explore the conceptual extremities of mathematics
which led to the field's being dubbed ``{\em the queen of the
  sciences}'' by the great $19$th-century German mathematician Karl
Friedrich Gauss. \index{Gauss, Karl Friedrich}

While mathematics is assuredly much more than ``just'' numbers and
arithmetic, one could spend one's life fruitfully while exploring
nothing beyond these topics.  The three chapters described in this
section strive to introduce the reader to numbers and
arithmetic ``in layers''
\begin{itemize}
\item
Chapter~\ref{ch:numbers-numerals} introduces the subject by means of a
``biography'' of our number system as it has evolved over the
millennia---in response to the need for greater explanatory and
manipulatory power over the phenomenally expanding knowledge base
exposed by science and technology.
\item
Chapter~\ref{ch:numbers-advanced} is devoted to looking both inward
and outward at the most easily intuited component of our number
system, the {\em integers} (or, ``counting numbers'').  Looking
inward, we discover the {\em prime numbers} (familiarly, ``the
primes'').  A theorem from antiquity exposes the primes as the
``building blocks'' of the entire set of integers.  Looking outward,
we expound on the use of the primes as the basis of a variety of {\em
  coding schemes} for a broad range of structures.  Both the strengths
and the weaknesses of prime-based encoding schemes inspires the
development of many other important encoding schemes.  These encoding
schemes lead to numerous nonobvious applications, ranging from storage
schemes for dynamic data structures to provably secure encodings of
various structures.

\item
In Chapter~\ref{ch:numerals}, we shift our attention from {\em
  numbers}, the objects we count with, to {\em numerals}, the {\em
  names} we use to represent numbers.  The importance of using an
appropriate numeral system for a particular application can be
illustrated by the design of digital adders.  The digital adder that
mimics the ``carry-ripple'' scheme that we all learned in elementary
school takes roughly $n$ steps to add a pair of $n$-digit numbers.  By
using a nonstandard {\em signed-digit} representation scheme, we could
reduce the addition time to a fixed constant, independent of the
lengths of the numbers beng added.  (Of course, signed-digit schemes
entail costs that standard schemes do not, which is why they are not
used universally.)
\end{itemize}
We spread our coverage of numbers and numerals over a {\em
  noncontiguous} sequence of three chapters because we need additional
material to progress from one number-oriented chapter to the next.
Specifically, we employ concepts and tools from
Chapters~\ref{ch:arithmetic} and~\ref{ch:Summation} (which cover
arithmetic and summations, respectively) in essential ways within
Chapter~\ref{ch:numbers-numerals}, and we add to this corpus material
from Chapter~\ref{ch:Recurrences} (which covers recurrences) as we
develop Chapter~\ref{ch:numerals}.


\subsubsection{Chapters~\ref{ch:arithmetic} and~\ref{ch:Summation}:
Arithmetic and Summation}

**HERE

\subsubsection{Chapter~\ref{ch:Recurrences}: Recurrences}


\subsubsection{Chapter~\ref{ch:combinatorics}: Combinatorics,
  Probability, and Statistics}


\subsubsection{Chapter~\ref{ch:Graphs-Trees}: Graphs and Their Relatives}




**HERE



\section{How to Use This Text}
\label{sec:how-to-use}

\begin{description}
\item[{\bf Digital logic and Computer architecture}.]
\index{digital logic} \index{computer architecture}
This topic would arise in computer engineering programs and in the
early portions of a course on computer architecture.
\begin{itemize}
\item
{\bf Digital logic}. \index{digital logic}
\item
{\bf Computer arithmetic}.  \index{computer arithmetic}
\end{itemize}

\item[{\bf Big data}.]

\item[{\bf Artificial intelligence}.]
SAT

\item[{\bf Social networks}.]

\end{description}



%\subsection{Sample Curricula Based on This Text}
%\label{sec:sample-curricula}


