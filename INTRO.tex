%version of 02-11-20

\chapter{Introduction}
\label{ch:intro}

\begin{quote}
``I have only made this letter longer because I have not had the time to make it shorter."  \\
\hspace*{1.5in}Blaise Pascal, {\it The Provincial Letters} (Letter 16, 1657)
\end{quote}


\section{Why Is This Book {\em Needed?}}
\label{sec:bookneeded}

How much mathematics does an aspiring computing professional need---and at what level of expertise?  We believe that the answer to this pedagogically fundamental question is time dependent.

\medskip

\noindent {\it The early generation}.
In the early days of computing, all aspects of the field were considered the domain of the ``techies''---the engineers and scientists and mathematicians who designed the early computers and figured out how to use them to solve a range of (mostly compute-intensive) problems.  Back then, one expected every computing professional to have a mastery of many mathematical topics.

\medskip

\noindent {\it Children of the early generation}.
Times---and the field of computing---changed.  The ``techies'' were able to craft a variety of sophisticated tools that opened up the world of computing to the general population.  Even people at the lower levels of the educational edifice were able to use imagination and ingenuity, rather than theorems and formulas, to produce impressive software artifacts of considerable utility.

\medskip

\noindent {\it The modern generation}.
Pendulums were made to swing.  As we note in our {\it Manifesto}, we now encounter almost daily problems that arise from unanticipated concomitants of the often unstructured ingenuity that produced various software artifacts.  Many would say that we need a renewed commitment to technical discipline that will endow artifacts with:
\begin{itemize}
\item
{\em understandable structure}, so that we can determine {\em what} went wrong when something {\em does} go wrong.
\item
{\em sustainability}, so that changes, which are inevitable in complex artifacts, will not create new problems (especially ones we do not know how to solve)
\item
{\em controllability}, so that ``smart'' artifacts do not become modern instances of Dr.~Frankenstein's monster.\footnote{See Mary Shelley (English author), {\it Frankenstein}, or, {\it The Modern Prometheus}.}
\end{itemize}
While we certainly need not return to the era of the ``techies'', it is indisputable that we do need a larger contingent of computing professionals who have {\em bona fide} expertise that enables the needed technical discipline.  This text is devoted to the mathematics that underlies the needed computing and science and engineering.


\section{Why Is {\em This} Book Needed?}
\label{sec:thisbookneed}

There are many introductory texts on discrete mathematics.  What separates this text from its siblings is the stratagem we have implemented to accommodate our intended audience of aspiring computing professionals.

\smallskip

We begin by recognizing that we live in a world of ever-increasing professional and social diversity.

\medskip

Historically, computing curricula began as predominantly technically-oriented studies within a school of science or engineering.  The evolution of the computing field has led us to move beyond that historical curricular worldview.  Modern computing-oriented curricula offer a variety of educational trajectories, including:
\begin{itemize}
\item
the traditional path, which emphasizes science and engineering and mathematics,
\item
paths which emphasize subfields of the humanities or social studies,
\item
paths which emphasize the {\em practice} of computing, either in a general setting or within a focused applied field such as business or finance or law or \ldots.
\end{itemize}
The preceding reality has led to a phenomenal broadening of the audience for computing-related curricula, hence for some level of mathematics education.  It has also engendered two far-reaching changes in the way we think about the mathematical component of computing-related education.

\medskip

\noindent {\bf 1}.
Many of today's aspiring computer professionals---particularly those targeting the newer subareas of computing---arguably need only specialized knowledge of mathematics.  Educators in computing-related fields must, consequently, serve a large population of students in a manner that accommodates the students' quite diverse needs and aspirations.

\medskip

\index{nondeterminism}
\noindent {\bf 2}.
The fields of {\it information} \index{information in computing} and {\em computing} have developed an unprecedented level of overlap.  As recently as the mid-1900s, these fields were viewed as largely being separate concerns: computing was concerned with manipulating discrete
objects; information was concerned with transmitting sequences of bits.  As parallel computing machines were developed, beginning in the 1960s, computing practitioners had to start paying more attention to the specific ways that information flowed among the processors of a parallel machine, and between these processors and the devices in which data was stored.  The development of the Internet, around the 1990s, focused yet more attention on information flow.  The marriage between the fields of computing and information dissemination was completed by the end of the 20th century.  Orchestrating the way that information flowed and spread became a {\it bona fide} specialty within the areas of mathematics that hitherto had focused only on
computing.  The introduction of information into computing had revolutionary aspects.  Information could be replicated at a speed and to a volume that was unmatched by the objects studied in any physical science.\footnote{Of course, there are areas within the  physical sciences that generate ``astronomical" amounts of data---{\it astronomy} is such an area.  But the data accumulated in, say, astronomical studies could safely be warehoused for more relaxed later processing.  The data flying around social media have to be dealt with in real time.}  Flowing information enabled the construction and operation of the kind of virtual universe hinted at by the computation-theoretic notion of {\it nondeterminism}  (see, e.g., \cite{Rosenberg09}), which had hitherto been viewed as a ``pure'' mathematical idea.  The mathematics that students learn had to adapt so that students learned how to think about information, most especially within the context of computing.

\bigskip

We have sought to keep the increasing diversity of students and of approaches to computing in mind as we have written this text.  We have included a broad range of material, in both subject and level.  As noted in the {\it Preface}, we have tried to accommodate the different backgrounds of our expected readership, while leading them all to a level of mathematical sophistication that will enable them to {\em understand}---and {\em do}---mathematics.

\bigskip

In the remainder of this chapter we describe our approach to the text: both our strategy for selecting and developing mathematical topics and our tactical organization of topics into chapters, sections, etc.  We describe also the various types of problems that we have included
as {\em exercises} at the end of each chapter, ranging from
``practice"-oriented problems to rather challenging ones, whose
anticipated level of difficulty is signaled by the symbols
``$\oplus$'' (challenging) and ``$\oplus \oplus$'' (really
challenging).  Hints are provided for the single-$\oplus$ problems,
and solutions are provided for the double-$\oplus$ problems, in Chapter~\ref{ch:Exercises}: {\it Solutions to Selected Exercises}.  Finally, we describe the more advanced {\em enrichment}-oriented material which we have included for the ambitious reader, in several guises:
\begin{itemize}
\item
as sections and subsections in the core text: Enrichment sections are signaled by the prependage of the symbols ``$\oplus$" (less specialized or advanced) or ``$\oplus \oplus$" (more specialized or advanced) to the title.  These sections are typically a bit off the beaten curricular track, either because of specialized subject matter or more advanced argumentation.
\item
as appendix chapters: The chapters will also be signaled by ``$\oplus$" or ``$\oplus \oplus$" in their titles.  But, be aware that these chapters are being ``double-signaled" by their non-inclusion in the core text.  They are likely too specialized or advanced for the typical introductory mathematics course.
\end{itemize}


\section{The Structure of This Book}
\label{sec:thisbook}

In our quest to endow each reader with an operational understanding of how to ``{\em do}'' mathematics, we want the reader
\begin{itemize}
\item
to recognize situations---especially relating to computing---wherein mathematical reasoning and analysis can make a positive difference;
\item
to identify the mathematical tools that are appropriate for these situations.
\end{itemize}


\subsection{Our Main Intellectual Targets}
\label{sec:book-overwiew}

This book is devoted to covering the discrete-mathematics underpinnings of the endeavor of computing: from the design and implementation of devices that perform the actions necessary to
compute to the design of the processes that control the devices---including whatever communications are needed among processes and among (sub)devices.  Our exposition aims at a number of intellectual targets.
\begin{enumerate}
\item
{\it Fundamental concepts}

\medskip

{\small\sf Examples:}
\begin{itemize}
\item%
sets---and their embellishments: tuples, arrays, tables, etc.---as embodiments of {\it object}
\item
numbers---and their operational manifestations, numerals---as embodiments of {\it quantity}
\item
graphs---in their many, varied, forms---as embodiments of {\it connectivity} and {\it relationship}
\item
algebras and functions---adding operations to sets, numbers, and graphs---as embodiments of {\it structured dynamism} and {\it computing} and {\it process}
\end{itemize}
We thereby expand the scope of what can be thought about ``mathematically''.

\medskip

\item
{\it Fundamental representations}

\medskip

{\small\sf Examples:}
\begin{itemize}
\item
representing and thinking about numbers via many metaphors: slices of pie, tokens arranged in stylized ways, characteristics of geometrical figures (e.g., rectangles or circles), textual objects
\item
understanding the strengths and weaknesses of various positional number representations (e.g., what can you tell about a number by studying its representation?)
\item
using grouping and/or replication to represent relationships among objects
\item
viewing interrelated objects via many structures: tables, tuples, graphs, pictures, geometric drawings
\end{itemize}
We thereby expand the universe of conceptual paradigms that one can use while thinking ``mathematically''.

\medskip

\item
{\it Fundamental tools/techniques}

\medskip

{\small\sf Examples:}
\begin{itemize}
\item
using induction to extrapolate from simple examples to complex ones
\item
``hopping'' between the discrete and continuous mathematical worlds, e.g., using integration to approximate summation
\item
using the conceptual tools of asymptotics to argue {\em qualitatively} about {\em quantitative} phenomena
\item
``hopping'' between the mathematical reasoning used in the {\em ``real'' world}, vs.~the formal logics that embody {\em idealized reasoning}
\end{itemize}
We thereby expand the conceptual tools that one has access to when {\em doing} mathematics.

\medskip

\item
{\it Beyond the fundamentals}

Our goal in writing this text is not just to expound on the basic notions that enable one to study the phenomenon of computation and its accompanying artifacts.  We strive additionally to inspire each reader to dig deeper into the lore of at least one of these notions.  To this end, we have presented many basic notions via multiple explanations and numerous exemplars.

\medskip

{\small\sf Examples:}
\begin{itemize}
\item
The ability to ``encode'' tuples of objects via the underlying objects themselves lie at the base of some of the pillars of computing.  We expound on two approaches to such encodings, one based on {\em prime numbers} and one on {\em pairing functions}.  We provide pointers to the literature which show how this basic ability can be applied to ``real'' computing problems such as {\em efficiently storing arrays whose dimensions can change dynamically} and {\em tracking the  computing output of each agent in a collaborating team}.

\item
The use of recursion as a control structure in programming/computing has a purely mathematical analogue, namely, recurrences.  Recurrences are extremely useful in analyzing the correctness and efficiency of recursively specified computations.  Every student gains at least some familiarity with the most basic versions of recurrences.  We introduce variations on the theme of these familiar recurrences, which offer approaches to more complicated computational situations.  We illustrate the use of recurrences in crafting a quite approachable analysis of an apparently dauntingly complex {\em token game}.
\item
Many phenomena that appear to arise from arcane, inherently computational sources are actually rather easily understood applications of purely mathematical phenomena.  We spend considerable time expounding on such phenomena and their applications.   {\em Satisfiability problems} are one phenomenon of this sort; certain {\em specialized number systems} are another.
\end{itemize}
\end{enumerate}


\subsection{Allocating our Targets to Chapters}
\label{sec:the chapters}

\subsubsection{Chapter~\ref{ch:doingmath}: ``Doing'' Mathematics}

This chapter is a toolkit for mathematical reasoning.  We strive to acclimate the reader to mathematics as a living discipline and an integral part of the world of computing, by beginning the book with a chapter entitled {\it Doing Mathematics}.  This chapter focuses mainly on the practicalities of mathematical reasoning, most importantly by expounding on the idea of a mathematical proof.  We survey the most commonly encountered techniques for crafting such
proofs and illustrate each with several examples.

\smallskip

Part of understanding proofs is being aware of what makes the endeavor of proving things difficult.  We provide both mathematical and historical background relating to important intellectually challenging topics such as how to reason about objects that are infinite and objects that are finite but very large (and growing).

%{\Denis I moved the asymptotic and infinity in the appropriate place later on}

\ignore{************
\medskip

\noindent {\em Reasoning qualitatively about quantitative phenomena.}
%
An often-underappreciated aspect of human reasoning is our ability to
abstract by blurring descriptions.  We notice, for instance, the ways
in which all phenomena that experience {\em linear} growth differ from
phenomena that experience {\em quadratic} growth, and both classes
differ from phenomena that experience {\em cubic} growth---and so on,
\ldots, {\em exponential}, and beyond.  
The extremely important topic of {\it asymptotics} abstracts from these
enumerated abstractions and allows us to reason qualitatively about
arbitrary inherently quantitative phenomena.

\medskip

\noindent {\em Coping with infinity}.
One recurring challenge when ``doing'' mathematics is dealing with
{\em infinity}.  Infinite objects, such as sets, behave rather
differently from the more familiar finite objects that we encounter in
our daily lives.  As but one example, there are ``equally many'' odd
integers as all integers, as measured by the ability to match the two
sets element by element---even though the first of these sets is
obtained from the second by discarding half of its elements.

A more subtle challenge manifests itself in the world of the infinite
resides in the myriad {\em paradoxes} that one encounters in this
world.  Does an arrow ever reach its target---given that it begins its
journey by traversing half then distance, then traverses half of the
remaining half, then half of the remaining quarter, \ldots?
**********}

\subsubsection{Chapter~\ref{ch:sets-BA-logic}: Sets and Their Algebras}

{\em Sets} are the stem cells of mathematics.  They begin as the most primitive mathematical objects, but as soon as one endows them with operations---for creating more inclusive (``bigger'') sets, for selecting more exclusive (``smaller'') subsets, for enabling the structure of tupling---they quickly afford one a powerful substrate for doing almost all of mathematics.

\medskip

\noindent {\em The power of tupling}.
One can exploit tupling to isolate entities such as {\em relations}, which are the foundations of imposing and identifying {\em order} in, and among, sets.  One can isolate {\em functions}, which enable one to {\em encode} various sets as other, seemingly unrelated, ones---example: encoding computer programs as positive integers.

\medskip

\noindent {\em Algebras: Operations and the laws that govern them}.
The operations within any collection of operations on sets inevitably obey certain ``laws'' as they interact; for instance, operation $\circ$ may (or may not) be {\em commutative}, i.e., obey the relation
\[ x \circ y \ = \ y \circ x \]
or it may (or may not) be {\em associative}, i.e., obey the relation
\[ x \circ (y \circ z) \ = \ (x \circ y) \circ z \]
The combination of a set $S$ (of ``objects'') and a set of operations on set $S$, together with the laws that govern the operations, is an {\em algebra}---and there are myriad algebras that play important roles in our lives.
\begin{itemize}
\item
{\em Boolean algebras} focus on sets and operations on sets.
\item
A special class of Boolean algebras is the class of {\em Propositional logics}.  These algebras underlie computing-related topics that range from {\em digital logic}, the basis of all computer design, to the {\em satisfaction problems} that play a major role in aspects of  {\em complexity theory} and {\em artificial intelligence} ({\em AI}).
\item
{\em Numerical algebras} govern our daily lives, by enabling us to perform crucial basic functions such as counting and performing arithmetic.
\end{itemize}


%{\Denis I add below th 3 subtitles, keep free to remove or change...}
\subsubsection{Chapters~\ref{ch:numbers-numerals},
~\ref{ch:numbers-advanced}, ~\ref{ch:numerals}: Numbers and Numerals}

The first mathematical concepts that children learn about usually involve numbers.  We spend much of our early lives expanding our number-based knowledge base.  We proceed from counting to manipulating numbers by adding and subtracting and multiplying and dividing them.
We progress from using fingers and toes as ``names'' of numbers to using a variety of numeral-forming schemes.  For many of us, ``mathematics'' {\em means} ``numbers and arithmetic'': We never get to explore the powerful mathematical concepts and tools that have enabled many of the great advances of science and engineering.  Even fewer of us get to explore the conceptual extremities of mathematics which led to the field's being dubbed ``{\em the queen of the  sciences}'' by the great 19th-century German mathematician Karl Friedrich Gauss. 

\index{Gauss, Karl Friedrich}

While mathematics is assuredly much more than ``just'' numbers and arithmetic, one could spend one's life fruitfully exploring nothing beyond these topics.  The three chapters described in this
section strive to introduce the reader to numbers and arithmetic ``in layers''.
\begin{itemize}
\item
{\em The basic objects and properties of our number system.}

\smallskip

Chapter~\ref{ch:numbers-numerals} introduces the subject by means of a ``biography'' of our number system as it has evolved over the millennia---in response to the need for greater explanatory and manipulatory power over the phenomenally expanding knowledge base exposed by science and technology.
\item
{\em Building the integers and building with the integers.}

\smallskip

Chapter~\ref{ch:numbers-advanced} is devoted to looking both inward and outward at the most easily intuited component of our number system, the {\em integers} (or, ``counting numbers'').  Looking inward, we discover the {\em prime numbers} (familiarly, ``the primes'').  A theorem from antiquity exposes the primes as the ``building blocks'' of the entire set of integers.  Looking outward, we expound on the use of the primes as the basis of a variety of {\em coding schemes} for a broad range of structures.  Both the strengths and the weaknesses of prime-based encoding schemes have inspired the development of many other important encoding schemes.  These encoding schemes lead to numerous nonobvious applications, ranging from storage schemes for dynamic data structures to provably secure encodings of various structures.

\item
{\em Operational number representation and their consequences.}

\smallskip

In Chapter~\ref{ch:numerals} we shift our focus from {\em numbers}, the objects that we count with, to {\em numerals}, the {\em names} that we use to represent and manipulate---i.e., compute with---numbers.  The importance of using an appropriate numeral system for a particular application can be illustrated by the design of digital adders.  The digital adder that mimics the ``carry-ripple'' scheme which we all learned in elementary school takes roughly $n$ steps to add a pair of $n$-digit numbers.  By using a nonstandard {\em signed-digit} representation scheme, we can reduce the addition time to a fixed constant, independent of the lengths of the numbers being added.  (Of course, signed-digit schemes entail costs that standard schemes do not, which
is why they are not in common use.)
\end{itemize}
We spread our coverage of numbers and numerals over a {\em noncontiguous} sequence of three chapters because we need additional material to progress from one number-oriented chapter to the next.  Specifically, we employ concepts and tools from Chapters~\ref{ch:arithmetic} and~\ref{ch:Summation} (which cover arithmetic and summations, respectively) in essential ways within Chapter~\ref{ch:numbers-numerals}, and we add to this corpus of material from Chapter~\ref{ch:Recurrences} (which covers recurrences), as we develop Chapter~\ref{ch:numerals}.


\subsubsection{Chapters~\ref{ch:arithmetic} and~\ref{ch:Summation}:
Arithmetic and Summation}

Numbers are important in our daily lives only when we {\em use} and {\em manipulate} them.

Chapter~\ref{ch:arithmetic} discusses {\it arithmetic}, the basic operations---addition, multiplication, etc.---that we use to manipulate numbers and the laws that these operations obey.  The chapter then moves on to complex operations on numbers---polynomials, exponentials, and logarithms.  It closes with pointers to topics for advanced study---including topics that are central to modern computer applications such as big data.  The chapter's treatment of polynomials is far reaching:
\begin{itemize}
\item
The chapter begins with basic facts about these special functions.
\item
The chapter then introduces the centuries-old study of solving---or being unable to solve---single-variable polynomials using radicals (the latter topic being discussed only informally because of its
advanced nature).  The well-known {\it quadratic formula} is the simplest instance of solving polynomials by radicals: The two solutions to the polynomial equation
\[ ax^2 \ + \ bx \ + \ c \ = \ 0 \]
are
\[ x \ = \ \frac{-b + \sqrt{b^2 - 4ac}}{2a}
 \ \ \ \mbox{ and } \ \ \
   x \ = \ \frac{-b - \sqrt{b^2 - 4ac}}{2a}
\]
\item
The chapter's treatment of polynomials then digresses to present an extremely important result about two-variable polynomials, Newton's famous {\it Binomial Theorem}.  This result enables one to go easily between polynomials in a certain family and their roots.  The Theorem's ``smallest'' instances provide the following equations.
\begin{eqnarray*}
(x + y)^2 & = & x^2 \ + \ 2xy \ + \ y^2 \\
(x + y)^3 & = & x^3 \ + \ 3x^2y \ + \ 3 xy^2 \ + \ y^3
\end{eqnarray*}
\item
The chapter's treatment of polynomials closes with a short, informal, discussion of {\it Hilbert's Tenth Problem}, an advanced topic of immense mathematical import.  In one line: The work on this Problem demonstrates that the single topic of discovering integer roots of arbitrary polynomials with integer coefficients---or proving that no such roots exist---captures (read: {\em encodes}) the full complexity of performing arbitrary computations!
\end{itemize}
The chapter finally introduces the mutually inverse operations of taking exponentials and logarithms, in the sense of the equations
\[  x \ = \ \log_b(b^x) \ = \ b^{\log_b(x)} \]
A fundamental insight here is that the arithmetical system based on these functions is {\em almost} identical to our conventional arithmetic---but with multiplication replacing addition and division replacing subtraction.  ({\em This is the basic insight underlying the {\em slide rules} that were techies' pocket calculators for many decades.}). The qualifier ``almost'' hints at the adjustments needed to accommodate the impossibility of dividing by $0$.  This pair of operations play a fundamental role in the field of information theory, whose importance to computing cannot be overstated.

\bigskip

Chapter~\ref{ch:Summation} is basically a {\em tools} chapter.  It studies how to break a complex operation into simple constituents. It is devoted to analyzing a broad variety of families of summations, providing exact solution-sums for many and approximate sums for others.  All of the finite summations we study have the general form
\[ S \ = \ s_1 \ + \ s_2 \ + \cdots + \ s_n \]
In situations where we allow summations to have infinitely many terms---for instance, with the famous summation
\[ 1 \ + \ {1 \over 2} \ + \ {1 \over 4} \ + \ {1 \over 8} \ + \cdots
+ \ {1 \over {2^k}} \ + \cdots
\]
(whose sum is $2$), we allow this pattern to continue without end.  We include here {\it arithmetic summations}, in which all adjacent summands, $s_{i+1}$ and $s_i$, have a common {\em difference}, and {\it geometric summations}, in which all adjacent summands, $s_{i+1}$ and $s_i$, have a common {\em ratio}.  We discuss also powerful techniques for estimating the solution-sums of summations of consecutive terms of a ``smooth'' function.

The mathematics we exploit to derive exact and/or approximate sums for many of the classes of summations we study provides us with the opportunity of looking at numbers and their summations in many quite distinct ways---from textual to pictorial to geometric, and beyond.  For this reason, this chapter is one of the most important as the reader gains traction in the endeavor of ``doing mathematics''.

\subsubsection{Chapter~\ref{ch:infinity}: The Vertigo of Infinity}

This chapter deals with mathematical objects whose very size---ranging from the finite but very large to the infinite---is difficult to reason about, for one of two reasons.
\begin{enumerate}
\item
We have developed the important ability to reason abstractly by blurring descriptions.  We notice, e.g., the ways in which all phenomena that experience {\em linear} growth differ from phenomena
that experience {\em quadratic} growth, and both classes differ from phenomena that experience {\em cubic} growth---and so on.

Additionally, we often encounter finite objects whose sizes can change dynamically or can be known only approximately.  Consider, e.g., social networks: they expand and contract in unpredictable manners, so their population statistics cannot be known exactly.

We require a language for talking---and rigorously reasoning---about blurry distinctions and approximately-known objects.  And, we need a formal analogue of arithmetic for ``calculating'' the statistics of such objects.

The topic of {\it asymptotics} (Section~\ref{sec:asymptotics}) fills both needs, for a large variety of dynamic mathematical phenomena.  Asymptotics thereby enables {\em reasoning qualitatively about inherently quantitative phenomena}.

\item
We often have to reason about objects that are actually infinite.  We encounter such situations, e.g., in areas that combine, in some way,
  \begin{itemize}
  \item
mathematics---e.g., classes of numbers or of functions
  \item
logic---e.g., expressions with special properties, such as theorems or proofs
  \item
linguistics---e.g., sentences that share (syntactic or semantic) characteristics
  \end{itemize}
Since antiquity, we have been confronted by infinite objects that behave very differently than the finite objects of daily discourse.  As just two examples:
  \begin{itemize}
  \item
Why does a shot arrow reaches its target.  The following seemingly cogent argument shows that it does not.

The argument states---correctly---that after the arrow traverses one-half the distance to the target, it still has one-half the distance to go; after traversing half of the remaining distance, it still has one-quarter of the original distance to go.  Continuing, the arrow will have a never-ending shrinking distance that it has yet to traverse: one-eighth, then one-sixteenth, then one-thirty-second, and so on.  How can the arrow ever reach the target?

  \item
We have had to adapt to the fact that ``provable'' and ``true'' are distinct concepts within most realistic systems of logic, despite their intuitive coincidence.
  \end{itemize}

In a slightly more sophisticated vein, we all know that there are ``equally many'' odd integers as all integers---even though the former set is obtained by discarding half of the latter set's elements.

Section~\ref{sec:coping-infinity} provides the background necessary to cope with these puzzles and navigate the hazard-laden unfamiliar world of the infinite.
\end{enumerate}

\subsubsection{Chapter~\ref{ch:Recurrences}: Recurrences}

The functions discussed in Chapter~\ref{ch:arithmetic} are described by static ``closed-form'' expressions.  In contrast, Chapter~\ref{ch:Recurrences} is devoted to a family of {\em computational procedures}, {\it recurrences}---which calculate the value of a function $f$ on an argument $n$ in terms of the values of $f$ on smaller arguments.  The patterns that underlie recurrent computations have been known for millennia to occur throughout nature---in the growth patterns of many plants and in the demographics of many animals, among other places.  The mathematics that describes such computations is as elegant and aesthetic as it is useful.  Among
the myriad recurrent patterns that one could identify, we select three main ones for their computational importance.
\begin{enumerate}
\item
{\it Linear recurrences}.  Recurrences of the form
\begin{equation}
\label{eq:general-linear-recurrence}
f(n) \ = \ a f(bn+c) \ + \ dn \ + e
\end{equation}
(where $a, b, c, d, e$ are constants) are a welcome friend when one analyzes the costs of a broad range of algorithms, using a broad range of cost measures (e.g., time, memory usage, power requirements, amount of communication).

We introduce two simple, yet nontrivial uses of recurrence (\ref{eq:general-linear-recurrence}) in the analysis of algorithms. We refer the reader to an algorithms text such as \cite{CLRS} for
details.
%{\Denis remark: both next examples are dealing with algorithms, may be too much?}
  \begin{itemize}
  \item
The {\em binary search algorithm}, which determines whether an item $x$ occurs within a given ordered list of $n$ items, begins by partitioning the list in two (so $b = 1/2$ and $c = 0$ in
(\ref{eq:general-linear-recurrence})).  It then compares $x$ with the list's middle item(s)---there is one middle item when $n$ is odd, two when $n$ is even---(so $d = 0$, and $e$ is the fixed constant cost of the comparison(s)).  Based on the outcome of the comparison, the algorithm recurses with a binary search on the half of the list that could contain $x$ (so $a = 1$).

  \item
The {\em merge-sort algorithm} builds on a natural $n$-step algorithm for merging two $n$-item lists of ``order-comparable'' items (i.e., for every pair, $x,y$, of list items, either $x < y$ or $y < x$).  The algorithm first merges adjacent odd-even pairs of items, to end up with sorted $2$-item lists.  It then merges adjacent $2$-item lists, to end up with sorted $4$-item lists.  It continues thus with $8$-item lists, then $16$-item lists, \ldots, until it finally merges the ``top-level'' two $n/2$-item lists to achieve the goal of a sorted $n$-item list.  The performance of this algorithm, as measured by operation-count, is given by recurrence (\ref{eq:general-linear-recurrence}) with $a=2$, $b = 1/2$, $c = 0$, $d=1$, and $e =$ (the fixed cost of a $2$-item comparison).
  \end{itemize}

\smallskip

The centerpiece of our discussion of linear recurrences is the so-called {\em Master Theorem}, which uses geometric summations to generate explicit---rather than recurrent---expressions for the values of a function $f$ on an arbitrary argument $n$.

\item
We highlight two {\it bilinear} recurrences\footnote{The expression at the heart of a {\it bilinear} recurrence involves a linear combination of two variables.}~of especial computational importance. \index{bilinear recurrence}
  \begin{itemize} 
  \item
The {\it binomial coefficient} articulated as ``$n$ choose $k$'', and commonly denoted $\displaystyle {n \choose k}$ or $\Delta_{n,k}$ or $C(n,k)$, plays a central role in a broad range of mathematical domains.  Two rather distinct examples:
%{\Denis I found both examples too much detailed. I suggest to shorten}
       \begin{itemize}
       \item
$C(n,k)$ is the number of ways to select $k$ items out of a set of $n$ items.  For instance, $C(4,2) =6$, because the $2$-element subsets of $\{a, b, c, d\}$ are: $\{a, b\}$, $\{a, c\}$, $\{a, d\}$,  $\{b, c\}$, $\{b, d\}$,  and $\{c,d\}$.  Analyzing $C(n, 5)$ will reveal, for example, why ``three of a kind'' beats ``two pair'' in poker.
       \item
These numbers play a prominent role in evaluating arithmetic summations; e.g., we derive the following equation (in several ways) in later chapters:
\[ 1 \ + \ 2 \ + \ 3 \ + \cdots + \ n \ \ = \ \ C(n+1, 2) \]
       \end{itemize}
The {\em family} of binomial coefficients is defined by the bilinear recurrence
\begin{eqnarray*}
C(n+1, k+1) & = & C(n, k) \ + \ C(n, k+1) \ \ \mbox{for any } \ \ k \geq 0 \\
C(n, 0) & = & 1 \\
C(n, 1) & = & n 
\end{eqnarray*}

  \item
The sequence of {\it Fibonacci numbers}, named for the Pisano mathematician known by the nickname ``Fibonnaci'', are defined by the recurrence
\begin{eqnarray*}
F(n+1) & = & F(n) \ + \ F(n-1) \\
F(0) & = & F(1) = 1
\end{eqnarray*}
Fibonacci invented his eponymous sequence as he observed the populations of consecutive generations of progeny produced by a pair of rabbits; the sequence also arises in the structure of many plants; it further is said to have a semi-religious aspect in the architecture
of ancient Greek temples.  Aside from its important descriptive role, the sequence plays a
significant role in the analysis of algorithms and as the basis for a nonstandard system of numerals.
  \end{itemize}
\end{enumerate}
Variations on the preceding three families of recurrences provide supplemental material in this chapter.


\subsubsection{Chapter~\ref{ch:combinatorics}: 
Combinatorics, Probability, Statistics}

\index{Leibniz (Leibnitz), Gottfried Wilhelm}
Certain subfields of mathematics have been known for centuries as ``the art of counting''.\footnote{Gottfried Leibniz used the phrase as the title of his 1666 doctoral thesis.}  This chapter introduces three related areas of discrete mathematics that are based on that art.  Indeed, one can argue that the following chain of applications, while grossly oversimplified, is not misleading.

\smallskip

\begin{tabular}{lcl}
{\it Statistics} & can be viewed as applied & {\it Probability} \\
{\it Probability} & can be viewed as applied & {\it Combinatorics} \\
{\it Combinatorics} & can be viewed as applied & {\it Counting}
\end{tabular}

\smallskip

\noindent
These are important connections.  Elements of probability and statistics infuse every area of endeavor---from science to finances to informatics, and beyond---where computing is involved.  In order to function successfully in today's world, one needs statistical and probabilistic literacy---a command of the foundations and the operational rules of ``the art of counting''.

\begin{itemize}
\item
This chapter begins by developing the basic {\em rules of counting:}  How many strings of length $n$ can one form using $c$ characters?  How many subsets does a set of $k$ elements have?  We discover that many seemingly distinct problems of this type are actually {\em encodings} of one another!

\item
The chapter moves on to concepts related to {\em grouping and arrangement} using mechanisms such as permutations, combinations, and derangements.  Variations of these themes allow us to develop {\em selection}-based concepts, such as: In how many deals of five playing cards do all cards have the same suit?

\item
We develop the elements of {\em discrete} (or, {\em combinatorial}) {\em probability}.  The probability (or, {\it likelihood}) of an event is defined as the ratio of the number of ways that the event occurs, divided by the total possible number of outcomes.  For instance, the probability of achieving the result ``$7$'' when rolling two dice is the ratio of the number of rolls that produce ``$7$'', divided by the total number of rolls.

\smallskip

We illustrate the elements of probability by simple, fun, examples such as assessing the relative values of various deals in the card game {\it poker} and of various rolls of a pair of dice in the game {\it craps}.

\smallskip

The most provocative portion of the chapter discusses the value of probability-related concepts in two situations where one might not expect to encounter them.  Both situations are described via puzzles which, while lacking in intrinsic importance, provide good platforms for developing significant aspects of probabilistic thinking.  The first puzzle, in Section~\ref{sec:monty-hall}, arose from a television game show that was popular a few decades ago.  We demonstrate in this
instance how a knowledge of probabilities can enhance one's likelihood of making good guesses in the present of partial information.  The second puzzle, in Section~\ref{sec:birthday-puzzle}, asks how many people one must assemble in order to expect (at least) two to have the same birthday. Not surprisingly, this second puzzle is widely known as {\it the birthday puzzle}.

\item
Finally, we illustrate the {\em statistical} way of thinking by carefully analyzing two ways for deriving the likelihood of achieving a specific sum---such as $6$---when rolling {\em three} dice.  This example will enable us to generalize from the probabilities of specific events to (statistical) {\it distributions} of these probabilities.

\smallskip

Even if one intends to ``do'' statistics mainly with the aid of preprogrammed packages (or apps), it is valuable to understand what the numbers produced by an app mean---and what the numbers {\em do not} mean!   Anyone who aspires to designing and/or executing and/or analyzing experiments {\em must} understand crucial notions such as {\em randomness} and should be conversant with the most common statistical distributions.  {\em Lives can depend on such knowledge!}
\end{itemize}

\subsubsection{Chapters~\ref{ch:Graphs1} and~\ref{ch:Graphs2}: Graphs and Related Topics}

Graphs are arguably the most important representational concept in all of mathematics---most certainly in the subfields related to computation.  In their most basic form, graphs represent any binary relation; a brief sampler:
\begin{itemize}
\item
the structure of a family, as exposed by the parent-child relation;
\item
the structure of an electronic or a communication circuit, where certain pairs of entities have the right to intercommunicate---perhaps only directionally).
\end{itemize}
The structure represented by a graph can expose not only the fact that certain pairs of entities can intercommunicate, but also the number of inter-entity ``links'' that must be traversed to achieve communication.

Even with this rudimentary discussion, one can intuit how myriad real-life problems can be modeled using graphs.  Many such problems use a graph to represent entities that can ``talk'' to one another, in some sense.  A variety of associated questions could be of the form, ``Who knew what when?''

Graphs are immensely important in countless scheduling applications, or the underlying relation can be {\em directional}---exposing dependencies.  A common relation studied in computer applications exposes that task $A$ in a program depends on input from task $B$, so that $B$ must be performed {\em before $A$}.  The notion of {\it graph coloring} is exceedingly important in scheduling and related applications: In its conceptually simplest form, one colors the task of a dependency graph in such a way that like-colored tasks can be executed concurrently---they are computationally independent.  The challenge in this scenario is to color a give task-graph with as few colors as possible.  This is a computationally difficult task in general, but we expose some of the sophisticated mathematics that has been developed in order to study the graph-coloring problem.  {\em Path problems} in graphs provide another entry to myriad scheduling applications.  Of particular interest are problems that require some object (e.g., a datum) to be passed around within a graph in a manner that achieves a ``coverage'' goal, e.g., so that the object encounters
all of the graph's entities or traverses all of the graph's inter-entity links.

\medskip

If {\em binary} relations are not adequate for a person's modeling needs, the expanded notion of {\em hypergraphs} can be used to model relations beyond binary, even those in which the ``arity'' of the relation varies from one related group of items to the next; a brief sampler:
\begin{itemize}
\item
the structure of a family, as exposed by two relations: the parent-child relation and the sibling relation;
\item
the structure of a bus-connected communication setup: entities on a single bus can all ``hear'' one another;
\item
social networks in which aggregations of ``friends'' have special intercommunicating privileges
\end{itemize}


\section{Using the Text in Courses}
\label{sec:how-to-use}

We have striven to design this book to meet the mathematical needs of a very broad range of readers, in terms of both preparation and goals.   We hope that the reader will use the book as the first step in a lifelong journey.  Toward this end, we have designed the book to fill multiple needs for both students and instructors:
  \begin{itemize}
  \item
as a \underline{textbook} for a beginning university discrete math course

  \item
as a \underline{source of mathematical preliminaries} for a broad range of beginning and intermediary university courses that benefit from background in mathematical topics

  \item
as a \underline{tutorial} on how mathematicians model and think about the worlds of mathematics, of science, of nature, and of people

  \item
as a \underline{launchpad} into further study of the historical, social, and technical aspects of mathematics
  \end{itemize}

\subsection{Resources}

\subsubsection{References}

We have provided several types of references, which are clearly marked, either by labels or by their positioning within the text.  
\begin{description}
\item[{\sc Source material for all covered topics}]
We provide pointers to references for all topics that we cover in this text---both concepts and results.  By following these pointers, the reader will have access to the original presentations of covered topics.  (Of course, we have worked toward finding a single level of discourse and a single ``voice" for the cited material.  But it is often beneficial to observe the multiple voices in the cited references.)
\end{description}

\noindent
We have sought to make this book more valuable to the reader by supplementing the standard types of citations with two others genres.

\begin{description}
\item[{\sc Historical and/or cultural sources}]
We provide a large number of references to original work on many important topics. 
  \begin{itemize}
  \item
By perusing historical sources such as (naming\footnote{Complete citations appear in {\it References}.} just a diverse few): 
     \begin{itemize}
     \item
J.~Arbuthnot: An argument for Divine Providence taken from the constant regularity observed in the births of both sexes
     \item
J.~Backus et al.: The FORTRAN automatic coding system
      \item
T.~Bayes, R.~Price: An essay towards solving a problem in the doctrine of chance
      \item
G.~Boole: {\it An Investigation of the Laws of Thought}   
     \item
A.~de Moivre: {\it The Doctrine of Chances}
     \item
Euclid: {\it Elements} 
    \item
J.~Nash: Equilibrium points in $n$-person games
    \item
I.~Newton: {\it Philosophia Naturalis Principia Mathematica}
(Referred to popularly as {\it Principia Mathematica})
    \item
A.M.~Turing (1936): On computable numbers, with an application to the Entscheidungsproblem.
    \item
J.~von Neumann, O.~Morgenstern: {\it Theory of Games and Economic Behavior.}
    \end{itemize}
the reader can gain a sense of ``where it all began".  Some of these sources are challenging to read---for example, Euclid's verbal exposition in {\it Elements} surely differs from that in a modern text (although his drawings will just as surely have some familiarity)---but one cannot escape the feeling that one is observing greatness!

\item
Most of the preceding references are the original treatises on various subjects, but we must never forget our indebtedness to the---usually anonymous---workers in the vineyards of mathematics who 
made ``minor" advances {\em and} to the greats of ages past who made sure that the myriad contributions of these workers would be available to successive generations.  A stalwart among these transmitters of knowledge was the great calculator Al-Khwarizmi ({\em eponym of the word ``algorithm"}) whose monumental {\it Liber Abaci} transmitted much mathematical wisdom from Asia to the then-desolate lands of Europe.

\item
Some of the historical/cultural references read like an adventure story---albeit of a very intellectual sort:
   \begin{itemize}
   \item
There is no better way to absorb the drama and excitement of the mathematical journey than to read of Hilbert's celebrated 23-problem agenda for 20th-century mathematics and then to read of the blockbuster studies that exploded some of the goals of that agenda:

   \item
The epochal results of K.~G\"{o}del in the 1930s which showed that mathematics could {\em provably} never capture {\em all} truths.

   \item
the blockbuster solution of D.~Hilbert's famous {\em Tenth Problem} in the 1970s.  In fact, you can  read the firsthand account by M.~Davis, Y.~Matijasevic, and J.~Robinson of the hills and valleys of their lengthy journey toward this solution.

  \item
Highlighting a different 20th-century adventure:  You can read the firsthand account by K.~Appel and W.~Haken of the journey that led to their proof of the celebrated {\em Four Color Theorem}.  This result answers a question that originated in the 19th century: How many distinct inks suffice to color the countries on {\em any} map of the world in such a way that bordering countries received distinct colors?  Easily (we shall see in Section~\ref{sec:graph-color}), the answer can be no smaller than four; Appel and Haken show that four colors suffice!  Adding to the drama of Appel and Haken's accomplishment:  {\em This was the first proof of a major mathematical theorem in which a digital computer played a {\em mathematical} role rather than just a {\em calculational} role.}
   \end{itemize}
\end{itemize}

\item[{\sc Advanced/specialized texts and reference books}]
We provide pointers to a long list of excellent texts that ``pick up" where this introductory text ``leaves off".  These texts cover a broad range of mathematical topics, including: Algorithms, Coding Theory, Combinatorics, Computation Theory, Computer Arithmetic, Graph Theory, Mathematical Logic, Number Theory, Probability Theory, and Statistics.
\end{description}



%\subsubsection{Indexes}

\subsubsection{Cultural asides}

Mathematics is invented by humans.  Motivated by superficial curiosity, humans exploit their innate powers to detect patterns.  Motivated by a deeper form of curiosity, humans develop conceptual tools in order to explicate the observed patterns.

\smallskip

It is our continuing goal to lend the reader a historical perspective on the way that progress has been made in major areas in mathematics.  By exposing the human side of what have turned out to be fruitful pursuits of curiosity, we strive to motivate readers to launch their own journeys of discovery.  Illustrative examples of fruitful curiosity:

Archimedes's quest to count the number of grains of sand on beaches; Leonardo Pisano's (Fibonacci) curiosity about the demographics of rabbits; John Arbuthnot's curiosity about the relative frequencies of male vs.~female births in humans. 

\smallskip

We strive to help the reader learn how to {\em do mathematics}s by describing the entire lifespans of several mathematical ideas and results:  We exhort the readers to:
\begin{enumerate}
\item
cultivate the ability to observe possible patterns in the world
\item
develop facility with mathematical concepts so as to describe observed patterns with precision and rigor
\item
hone their analytical skills so that they can pursue the implications of the patterns that they have either verified or refuted.
\end{enumerate}

\medskip

The stories within our text are intended to foster the first step in this program.  Our inclusion of  multiple proofs for many results, developed from quite distinct viewpoints, is intended to foster both the second and third steps in the program. 

\subsubsection{Exercises}

Continuing with the pedagogical philosophy that motivates all other aspects of this book, we have included a spectrum of exercises to supplement the text.
\begin{itemize}
\item
At the lowest rungs in the ladder---a physical metaphor seems consistent with the noun ``exercise"---are the practice problems that reinforce the reader's command over the concepts and tools we develop.

These exercises will expose possible places that need further study, and they will hopefully suggest digressions that the reader will find motivating and captivating.

\item
In the intermediate range of the ladder are the exercises that require the reader to ``do" some original mathematics---but none that the text to that point has not adequately prepared the reader for.

\item
As the reader reaches the upper rungs of the ladder, she will encounter exercises that really require mathematical insights.  Each of these exercises is annotated with
  \begin{itemize}
  \item
an explanation of why the exercise was included
  \item
a hint at how to get started
  \item
possibly---for the most challenging problems---a sketch of a solution.
  \end{itemize}  
\end{itemize}

\subsubsection{Appendices}

Mathematics is done by humans!  The authors are human mathematicians.  We each have our own favorite tidbits that we would like to share with you.  Some of these tidbits come from our own research; others come from studies that are kindred to those we have engaged in ourselves; yet others just strike our personal fancies.  We have prepared a buffet of such tidbits for the interested readers' amusement, including each tidbit in an appendix at the end of the text.  None of the tidbits requires mathematics that reaches far beyond the topics covered in the text.  Our hope is that just skimming an appendix that deals with a topic of interest might kindle a flame that will motivate some readers to make mathematics a part of their lives.


\subsection{Paths through the text for selected courses}

The material in this Introductory chapter and in the early parts of Chapters 2--13 is designed to help the reader decide which topics are appropriate for her desired course of study.  The development within each chapter will help the reader fine tune a path through a topic, deciding whether the chapter's content should become the reader's
\begin{itemize}
\item
{\em casual acquaintance}

\smallskip

A reader who is unlikely to need detailed information about graphs might, for instance, want only passing familiarity with the basic concepts of graph theory (the first half of Chapter~\ref{ch:Graphs1}).

\item
{\em good friend}

\smallskip

A reader who is interested in the architecture of communication networks and/or parallel architectures---as a user---might supplement the first half of Chapter~\ref{ch:Graphs1} with the second half of that chapter, which gives some detail about graphs that have proven useful in such specialties.

\item
{\em intimate friend}

\smallskip

A reader who expects to use graph-theoretic modeling in sophisticated endeavors such as task scheduling or designing compilers or analyzing complex data might add Chapter~\ref{ch:Graphs2} to her study list---or perhaps even use this material as a springboard to more advanced material such as appears in enrichment sections in the text.
\end{itemize}

\bigskip

To assist the reader in selecting appropriate paths through the text, we close this chapter with the following course-chapter table which focuses on eleven courses that appear in many computation-oriented university curricula.  To help the reader focus on the table, we provide the following capsule summary of the course-content in the (sub-)chapters that head the columns of the table.

 \bigskip

%{\small

\noindent
\begin{tabular}{|l|l|}
\hline
 {\bf Chapter} & {\bf Topic} \\
\hline
\hline
Ch.~2    & Intro to mathematical reasoning \\
Ch.~3a  & Sets \\
Ch.~3b  & Boolean algebra and logic \\
Ch.~4  & Numbers 1: Intro to our number system: \\
            & \hspace*{.67in}the integers, the rationals, the reals, the complex numbers \\
Ch.~5  &Intro to arithmetic (operations and their laws) \\
Ch.~6  & Summation: Exact and approximate evaluation of finite and infinite series   \\
Ch.~7a & Calculating with infinite objects: asymptotics \\
Ch.~7b  & Reasoning about infinity objects and systems: paradoxes \\
Ch.~8  & Numbers 2: Prime numbers; Pairing functions; Finite number systems; \\
            & \hspace*{.67in}Countability; Number-based Encoding  \\
Ch.~9  & Recurrences: Solving recurrences; Applications of recurrences \\
Ch.~10 & Numbers 3: Numerals: Representations of numbers: \\
             & \hspace*{.67in}Uncountability; Recognizing integers, rationals via numerals \\
Ch.~11  & Combinatorics; Discrete Probability; Statistics \\
Ch.~12a & Graphs 1: Basic properties and definitions; Graph-based models \\
Ch.~12b & Graphs 1: ``Named" classes of graphs and their areas of application \\
Ch.~13 & Graphs 2: Vertex- and Edge-coloring, with applications; \\
             & \hspace*{.67in}Path- and Cycle-detection and selection \\
\hline
\end{tabular}
%}

\bigskip

%{\small

\noindent
\begin{tabular}{|l||c|c|c|c|c|c|c|c|c|c|c|c|c|c|c|}
\hline
{\bf Course}/{\bf Chapter}
   & 2 & 3a & 3b & 4 & 5 & 6 & 7a & 7b & 8 & 9 & 10 & 11 & 12a & 12b & 13 \\
\hline
\hline
{\em Everyone} 
   & $\bullet$ & $\bullet$ & $\bullet$ & $\bullet$ & $\bullet$ &  & $\bullet$ & $\bullet$ &  &  &  &  & $\bullet$ & &  \\
\hline
%Algorithms  &  &  &  &  &  &  &  &  &  &  & &  &  & & \\
Algorithm Design and Analysis
%\hspace*{.1in}Design/Analysis
   & $\bullet$ & $\bullet$ & $\bullet$ & $\bullet$ & $\bullet$ & $\bullet$ & $\bullet$ & $\bullet$ & $\bullet$ & $\bullet$ & $\bullet$ & $\bullet$ & $\bullet$ & $\bullet$ & $\bullet$ \\
   \hline
Artificial Intelligence
   & $\bullet$ & $\bullet$ & & $\bullet$ &  &  & $\bullet$ & $\bullet$ &  &  &  & $\bullet$ & $\bullet$ & &  \\
   \hline
Computational Science
   & $\bullet$ & $\bullet$ & & $\bullet$ & $\bullet$ &  & $\bullet$ & $\bullet$ &  &  & $\bullet$ & $\bullet$ & $\bullet$ &  &  \\
\hline
Computer Architecture 
   & $\bullet$ & $\bullet$ & $\bullet$ & $\bullet$ &  & $\bullet$ & $\bullet$ & $\bullet$ &  &  & $\bullet$ & $\bullet$ & $\bullet$ & $\bullet$ & $\bullet$ \\
\hline
Cryptography and Cryptology 
   & $\bullet$ & $\bullet$ & $\bullet$ & $\bullet$ & $\bullet$ & $\bullet$ & $\bullet$ & $\bullet$ & $\bullet$ & $\bullet$ & $\bullet$ & $\bullet$ & $\bullet$ & &  \\
\hline
Data \& Information Retrieval
   & $\bullet$ & $\bullet$ & $\bullet$ & & $\bullet$ &  & $\bullet$ & $\bullet$ &  &  &  &  & $\bullet$ & & $\bullet$ \\
%\hspace*{.1in}Retrieval    &  &  &  &  &  &  &  &  &  &  & &  &  & & \\
\hline
Digital Logic 
   & $\bullet$ & $\bullet$ & $\bullet$ & $\bullet$ &  &  & $\bullet$ & $\bullet$ &  & $\bullet$ & $\bullet$ & $\bullet$ & $\bullet$ & $\bullet$ & $\bullet$ \\
\hline
Networks
   & $\bullet$ & $\bullet$ & & $\bullet$ &  & $\bullet$ & $\bullet$ & $\bullet$ &  &  &  & $\bullet$ & $\bullet$ & $\bullet$ & $\bullet$ \\
\hline
Social Media 
   & $\bullet$ & $\bullet$ & $\bullet$ &  &  &  & $\bullet$ & $\bullet$ &  &  &  &  & $\bullet$ & $\bullet$ &  \\
\hline
Systems \& Program Models
  & $\bullet$ & $\bullet$ & $\bullet$ &  & $\bullet$ &  & $\bullet$ & $\bullet$ &  &  &  &  & $\bullet$ & & $\bullet$ \\
%\hspace*{.1in}Program Models &  & &  &  &  &  & &  &  &  &  &  &  & & \\ 
\hline
Theory: Complexity, Computation
   & $\bullet$ & $\bullet$ & $\bullet$ & $\bullet$ & $\bullet$ & $\bullet$ & $\bullet$ & $\bullet$ & $\bullet$ & $\bullet$ & $\bullet$ & $\bullet$ & $\bullet$ & & $\bullet$ \\
%\hspace*{.1in}\& Computation    &  &  & &  &  &  & &  &  &  &  &  &  & & \\
\hline
\end{tabular}
%}

\bigskip

Of course, the {\em extent} and {\em depth} with which a reader studies a relevant chapter will depend on the reader's detailed goals.






\ignore{*****************
The material concerning graph theory requires only a linear path as one progresses from ``casual acquaintance" to ``intimate".  The material concerning numbers and their kin has considerably more branching as one increases degree of mastery.
\begin{itemize}
\item
{\em casual acquaintance}

\smallskip

Every reader should be familiar with our number system and its history.

\item
{\em good friend}

\smallskip

\item
{\em intimate}

\smallskip

\end{itemize}


{\Arny This section is being deferred until the end, so we have a
more complete picture of the various topics covered---which ones and
to what level.}




index (I) and citations (C) partitioned into
historical (I, C)
reference (C)
related to material in text (I, C)
applications (C)


special programs and special sections


\begin{description}
\item[{\bf Digital logic and Computer architecture}.]
\index{digital logic} \index{computer architecture}
This topic would arise in computer engineering programs and in the
early portions of a course on computer architecture.
\begin{itemize}
\item
{\bf Digital logic}. \index{digital logic}
\item
{\bf Computer arithmetic}.  \index{computer arithmetic}
\end{itemize}

\item[{\bf Cryptography and Computer security}.]


\item[{\bf Big data}.]


\item[{\bf Artificial intelligence}.]


\item[{\bf Social networks}.]

\end{description}

********************}



