%version of 06-04-19

\chapter{INTRODUCTION}
\label{ch:intro}

\begin{quote}
{\em I have only made this letter longer because I have not had the
  time to make it shorter.}  \\
\hspace*{1.5in}Blaise Pascal, {\it The Provincial Letters}
(Letter 16, 1657)
\end{quote}


\section{Why Is This Book {\em Needed?}}
\label{sec:bookneeded}

How much mathematics does an aspiring computing professional
need---and at what level of expertise?  We believe that the answer to
this pedagogically fundamental question is time dependent.

\medskip

\noindent {\it The early generation}.
In the early days of computing, all aspects of the field were
considered the domain of the ``techies''---the engineers and
scientists and mathematicians who designed the early computers and
figured out how to use them to solve a range of (mostly
compute-intensive) problems.  Back then, one expected every computing
professional to have a mastery of many mathematical topics.

\medskip

\noindent {\it Children of the early generation}.
Times---and the field of computing---changed.  The ``techies'' were
able to craft a variety of sophisticated tools that opened up the
world of computing to the general population.  Even people at the
lower levels of the educational edifice were able to use imagination
and ingenuity, rather than theorems and formulas, to produce
impressive software artifacts of considerable utility.

\medskip

\noindent {\it The modern generation}.
Pendulums were made to swing.  As we note in our {\it Manifesto}, we
now encounter almost daily problems that arise from unanticipated
concomitants of the often unstructured ingenuity that produced various
software artifacts.  Many would say that we need a renewed commitment
to technical discipline that will endow artifacts with
\begin{itemize}
\item
{\em understandable structure}, so that we can determine {\em what} went
wrong when something {\em does} go wrong.
\item
{\em sustainability}, so that changes, which are inevitable in complex
artifacts, will not create new problems
\item
{\em controllability}, so that ``smart'' artifacts do not become
modern instances of Dr.~Frankenstein's monster.\footnote{See Mary
  Shelley (English author), {\it Frankenstein}, or, {\it The Modern
    Prometheus}.}
\end{itemize}
While we certainly need not return to the era of the ``techies'', it
is unquestionable that we do need a larger contingent of computing
professionals who have bona fide expertise that enables the needed
technical discipline.  This text is devoted to the mathematics that
underlies the needed science and engineering.


\section{Why Is {\em This} Book Needed?}
\label{sec:thisbookneed}

There are many introductory texts on discrete mathematics.  What
separates this text from its siblings is the stratagem we have
implemented to accommodate our intended audience of aspiring computing
professionals.

We begin by recognizing that we live in a world of ever-increasing
professional and social diversity.

\medskip

Historically, computing curricula began as predominantly
technically-oriented studies within a school of science or
engineering.  The evolution of the computing field has led us to move
beyond that historical curricular worldview.  Modern
computing-oriented curricula offer a variety of educational
trajectories, including:
\begin{itemize}
\item
the traditional path, which emphasizes science and engineering and
mathematics,
\item
paths which emphasize subfields of the humanities or social studies,
\item
paths which emphasize the {\em practice} of computing, either in a
general setting or within a focused applied field such as business or
finance or law or \ldots.
\end{itemize}
The preceding reality has led to a phenomenal broadening of the
audience for computing-related curricula, hence for some level of
mathematics education.  It has also engendered two far-reaching
changes in the way we think about the mathematical component of
computing-related education.

\noindent {\bf 1}.
Many of today's aspiring computer professionals---particularly those
targeting the newer subareas of computing---arguably need only
specialized knowledge of mathematics.  Educators in computing-related
fields must, consequently, serve a large population of students, in a
manner that accommodates the students' quite diverse needs and
aspirations.

\noindent {\bf 2}.
The fields of {\it information} \index{information in computing} and
{\em computing} have developed an unprecedented level of overlap.  As
recently as the mid-1900s, these fields were viewed as largely being
separate concerns: computing was concerned with manipulating discrete
objects; information was concerned with transmitting sequences of
bits.  As parallel computing machines were developed, beginning in the
1960s, computing practitioners had to start paying more attention to
the specific ways that information flowed among the processors of a
parallel machine, and between these processors and the devices in
which data was stored.  The development of the Internet, around the
1990s, focused yet more attention on information flow.  The marriage
between the fields of computing and information dissemination was
completed by the end of the 20th century.  Orchestrating the way that
information flowed and spread became a {\it bona fide} specialty
within the areas of mathematics that hitherto had focused only on
computing.  The introduction of information into computing had
revolutionary aspects.  Information could be replicated at a speed and
to a volume that was unmatched by the objects studied in any physical
science.  Flowing information enabled the construction and operation
of the kind of virtual universe hinted at by the computation-theoretic
notion of {\it nondeterminism} \index{nondeterminism} (see, e.g.,
\cite{Rosenberg09}), whch had hitherto been viewed as a ``pure''
mathematical idea.  The mathematics that students learn had to adapt
so that students learned how to think about information, most
especially within the context of computing.


\bigskip

We have striven to keep the increasing diversity of students and of
approaches to computing in mind as we have written this text.  We have
included a broad range of material, in both subject and level.  As
noted in the {\it Preface}, we have tried to accommodate the different
backgrounds of our readership, while leading them all to a level of
mathematical sophistication that will enable them to {\em
  understand}---and {\em do}---mathematics.

\medskip

In the remainder of this chapter we describe our approach to the text:
both our strategy for selecting and developing mathematical topics and
our tactical organization of topics into chapters, sections, etc.  We
describe also the various types of problems that we have included in
the chapter devoted to Exercises
(Chapter~\ref{ch:Exercises})---ranging from problems that allow every
student to practice with ideas from the text to problems that challege
the dedicated student to develop material that we had no space to
develop the text.  Finally, we describe the advanced material that we
have included in appendices as enrichment material for the ambitious
reader.  These discussions will allow the reader to evaluate which
topics are most appropriate for their particular educational needs.



\section{The Structure of This Book}
\label{sec:thisbook}

In our quest to endow each reader with an operational understanding of
how to ``{\em do}'' mathematics, we want the reader
\begin{itemize}
\item
to recognize situations---especially relating to computing---wherein
mathematical reasoning and analysis can make a positive difference
\item
to identify the mathematical tools that are approprite for these
situations.
\end{itemize}


\subsection{Our Main Intellectual Targets}
\label{sec:book-overwiew}

This book is devoted to covering the discrete-mathematics
underpinnings of the endeavor of computing: from the design and
implementation of devices that perform the actions necessary to
compute to the design of the processes that control the
devices---including whatever communications are needed among processes
and among (sub)devices.  We have identified several intellectual
targets that guide our exposition.
\begin{enumerate}
\item
{\it Fundamental concepts}

\medskip

{\small\sf Examples:}
\begin{itemize}
\item%
sets---and their embellishments: tuples, arrays, tables, etc.---as
embodiments of {\it object}
\item
numbers---and their operational manifestations, numerals---as
embodiments of {\it quantity}
\item
graphs---in their many, varied, forms---as embodiments of {\it
  connectivity} and {\it relationship}
\item
algebras and functions---adding operations to sets, numbers, and
graphs---as embodiments of {\it structured dynamism} and {\it
  computing} and {\it process}
\end{itemize}
We thereby expand the scope of what can be thought about
``mathematically''.

\medskip

\item
{\it Fundamental representations}

\medskip

{\small\sf Examples:}
\begin{itemize}
\item
representing and thinking about numbers via many metaphors: slices of
pie, tokens arranged in stylized ways, characteristics of geometrical
figures (e.g., rectangles or circles), textual objects
\item
understanding the strengths and weaknesses of various positional
number representations (e.g., what can you tell about a number from
its representation?)
\item
using grouping and/or replication to represent relationships among
objects
\item
viewing interrelated objects via many structures: tables, tuples,
graphs, geometric drawings
\end{itemize}
We thereby expand the universe of conceptual paradigms that one can
use while thinking ``mathematically''.

\medskip

\item
{\it Fundamental tools/techniques}

\medskip

{\small\sf Examples:}
\begin{itemize}
\item
using induction to extrapolate from simple examples to complex ones
\item
``hopping'' between the discrete and continuous mathematical worlds,
e.g., using integration to approximate summation
\item
using the conceptual tools of asymptotics to argue qualitatively about
quantitative phenomena
\item
``hopping'' between the mathematical reasoning used in the ``real''
  world, vs.~the formal logics that enable such reasoning
\end{itemize}
We thereby expand the conceptual tools that one has access to when
{\em doing} mathematics.

\medskip

\item
{\it Beyond the fundamentals}

Our goal in writing this text is not just to expound on the basic
notions that enable one to study the phenomenon of computation and its
accompanying artifacts.  We strive additionally to inspire each reader
to dig deeper into the lore of at least one of these notions. To this
end, we have presented many basic notions via multiple explanations
and numerous exemplars.

\medskip

{\small\sf Examples:}
\begin{itemize}
\item
The ability to ``encode'' tuples of objects via the underlying objects
themselves lie at the base of some of the pillars of computing.  We
expound on two approaches to such encodings, one based on {\em prime
  numbers} and one on {\em pairing functions}.  We provide pointers to
the literature which show how this basic ability can be applied to
``real'' computing problems such as {\em efficiently storing arrays
  whose dimensions can change dynamically} and {\em tracking the
  computing output of each agent in a collaborating team}.

\item
The use of recursion as a control structure in computing has a purely
mathematical analogue, recurrences, which are extremely useful in
analyzing the correctness and efficiency of these computations.  Every
student gains at least some familiarity with the most basic versions
of recurrences.  We introduce variations on the theme of these
familiar recurrences, which offer approaches to more complicated
computational situations.  We illustrate the use of recurrences in
crafting a quite approachable analysis of an apparently complex {\em
  token game}.
\item
Many phenomena that appear to arise from arcane inherently
computational sources are actually rather easily understood
applications of purely mathematical phenomena.  We spend considerable
time expounding on such phenomena and their applications.  {\em
  Satisfiability problems} are one phenomenon of this sort; certain
{\em specialized number systems} are another.
\end{itemize}
\end{enumerate}


\subsection{Allocating our Targets to Chapters}
\label{sec:the chapters}

\subsubsection{Chapter~\ref{ch:doingmath}: ``Doing'' Mathematics}

A toolkit for mathematics reasoning.
In order to acclimate the reader to mathematics as a living discipline
and an integral part of the world of computing, we begin the book with
a chapter entitled {\it Doing Mathematics}.  This chapter focuses
mainly on the practicalities of mathematical reasoning, most
importantly by expounding on the idea of a mathematical proof.  We
survey the most commonly encountered techniques for crafting such
proofs and illustrate each with several examples.

Part of understanding proofs is being aware of what makes the endeavor
of proving things difficult.  We provide both mathematical and
historical background relating to important intellectually challenging
topics such as how to reason about objects that are infinite and
objects that are finite but very large (and growing).


%{\Denis I moved the asymptotic and infinity in the appropriate place later on}

\ignore{************
\medskip

\noindent {\em Reasoning qualitatively about quantitative phenomena.}
%
An often-underappreciated aspect of human reasoning is our ability to
abstract by blurring descriptions.  We notice, for instance, the ways
in which all phenomena that experience {\em linear} growth differ from
phenomena that experience {\em quadratic} growth, and both classes
differ from phenomena that experience {\em cubic} growth---and so on,
\ldots, {\em exponential}, and beyond.  
The extremely important topic of {\it asymptotics} abstracts from these
enumerated abstractions and allows us to reason qualitatively about
arbitrary inherently quantitative phenomena.

\medskip

\noindent {\em Coping with infinity}.
One recurring challenge when ``doing'' mathematics is dealing with
{\em infinity}.  Infinite objects, such as sets, behave rather
differently from the more familiar finite objects that we encounter in
our daily lives.  As but one example, there are ``equally many'' odd
integers as all integers, as measured by the ability to match the two
sets element by element---even though the first of these sets is
obtained from the second by discarding half of its elements.

A more subtle challenge manifests itself in the world of the infinite
resides in the myriad {\em paradoxes} that one encounters in this
world.  Does an arrow ever reach its target---given that it begins its
journey by traversing half then distance, then traverses half of the
remaining half, then half of the remaining quarter, \ldots?
**********}

\subsubsection{Chapter~\ref{ch:sets-BA-logic}: Sets and Their Algebras}

{\em Sets} are the stem cells of mathematics.  They begin as the most
primitive mathematical objects, but as soon as one endows them with
operations---for creating more inclusive (``bigger'') sets, for
selecting more exclusive (``smaller'') subsets, for enabling the
structure of tupling---they quickly afford one a powerful substrate
for doing almost all of mathematics.

\medskip

\noindent {\em The power of tupling}.
%
One can exploit tupling to isolate entities such as {\em relations},
which are the foundations of imposing and identifying {\em order} in,
and among, sets.  One can isolate {\em functions}, which enable one to
{\em encode} various sets as other, seemingly unrelated,
ones---example: encoding computer programs as positive integers.

\medskip

\noindent {\em Algebras: Operations and the laws that govern them}.
%
The operations within any collection of operations on sets inevitably
obey certain ``laws'' as they interact; for instance, operation
$\circ$ may (or may not) be {\em commutative}, i.e., obey the relation
\[ x \circ y \ = \ y \circ x \]
or it may (or may not) be {\em associative}, i.e., obey the relation
\[ x \circ (y \circ z) \ = \ (x \circ y) \circ z \]
The combination of a set $S$ (of ``objects'') and a set of operations
on set $S$, together with the laws that govern the operations, is an
{\em algebra}---and there are myriad algebras that play important
roles in our lives.
\begin{itemize}
\item
{\em Boolean algebras} focus on sets and operations on sets.
\item
A special class of Boolean algebras is the class of {\em Propositional
  logics}.  These algebras underlie computing-related topics that
range from {\em digital logic}, the basis of all computer design, to the
{\em satisfaction problems} that play a major role in aspects of 
{\em complexity theory} and {\em artificial intelligence} ({\em AI}).
\item
{\em Numerical algebras} govern our daily lives, by enabling us to
perform crucial basic functions such as
counting and performing arithmetic.
\end{itemize}


%{\Denis I add below th 3 subtitles, keep free to remove or change...}
\subsubsection{Chapters~\ref{ch:numbers-numerals},
~\ref{ch:numbers-advanced}, ~\ref{ch:numerals}: Numbers and Numerals}


The first mathematical concepts that children learn about usually
involve numbers.  We spend much of our early lives expanding our
number-based knowledge base.  We proceed from counting to manipulating
numbers by adding and subtracting and multiplying and dividing them.
We progress from using fingers and toes as ``names'' of numbers to
using a variety of numeral-forming schemes.  For many of us,
``mathematics'' {\em means} ``numbers and arithmetic'': We never get
to explore the powerful mathematical concepts and tools that have
enabled many of the great advances of science and engineering.  Even
fewer of us get to explore the conceptual extremities of mathematics
which led to the field's being dubbed ``{\em the queen of the
  sciences}'' by the great $19$th-century German mathematician Karl
Friedrich Gauss. \index{Gauss, Karl Friedrich}

While mathematics is assuredly much more than ``just'' numbers and
arithmetic, one could spend one's life fruitfully while exploring
nothing beyond these topics.  The three chapters described in this
section strive to introduce the reader to numbers and
arithmetic ``in layers''
\begin{itemize}
\item
{\em The basic objects and properties of our number system.}

Chapter~\ref{ch:numbers-numerals} introduces the subject by means of a
``biography'' of our number system as it has evolved over the
millennia---in response to the need for greater explanatory and
manipulatory power over the phenomenally expanding knowledge base
exposed by science and technology.
\item
{\em Building the integers and building with the integers.}

Chapter~\ref{ch:numbers-advanced} is devoted to looking both inward
and outward at the most easily intuited component of our number
system, the {\em integers} (or, ``counting numbers'').  Looking
inward, we discover the {\em prime numbers} (familiarly, ``the
primes'').  A theorem from antiquity exposes the primes as the
``building blocks'' of the entire set of integers.  Looking outward,
we expound on the use of the primes as the basis of a variety of {\em
  coding schemes} for a broad range of structures.  Both the strengths
and the weaknesses of prime-based encoding schemes have inspired the
development of many other important encoding schemes.  These encoding
schemes lead to numerous nonobvious applications, ranging from storage
schemes for dynamic data structures to provably secure encodings of
various structures.

\item
{\em Operational number representation and their consequences.}

In Chapter~\ref{ch:numerals} we shift our focus from {\em numbers},
the objects that we count with, to {\em numerals}, the {\em names}
that we use to represent and manipulate numbers.  The importance of
using an appropriate numeral system for a particular application can
be illustrated by the design of digital adders.  The digital adder
that mimics the ``carry-ripple'' scheme that we all learned in
elementary school takes roughly $n$ steps to add a pair of $n$-digit
numbers.  By using a nonstandard {\em signed-digit} representation
scheme, we can reduce the addition time to a fixed constant,
independent of the lengths of the numbers beng added.  (Of course,
signed-digit schemes entail costs that standard schemes do not, which
is why they are not in common use.)
\end{itemize}
We spread our coverage of numbers and numerals over a {\em
  noncontiguous} sequence of three chapters because we need additional
material to progress from one number-oriented chapter to the next.
Specifically, we employ concepts and tools from
Chapters~\ref{ch:arithmetic} and~\ref{ch:Summation} (which cover
arithmetic and summations, respectively) in essential ways within
Chapter~\ref{ch:numbers-numerals}, and we add to this corpus material
from Chapter~\ref{ch:Recurrences} (which covers recurrences) as we
develop Chapter~\ref{ch:numerals}.


\subsubsection{Chapters~\ref{ch:arithmetic} and~\ref{ch:Summation}:
Arithmetic and Summation}

Numbers are important in our daily lives only when we {\em use} and
{\em manipulate} them.

Chapter~\ref{ch:arithmetic} discusses {\it arithmetic}, the basic
operations---addition, multiplication, etc.---that we use to
manipulate numbers and the laws that these operations obey.  The
chapter then moves on to complex operations on numbers---polynomials,
exponentials, and logarithms.  It closes with pointers to topics for
advanced study---including topics that are central to modern computer
applications such as big data.  The chapter's treatment of polynomials
is far reaching:
\begin{itemize}
\item
The chapter begins with basic facts about these special functions.
\item
The chapter then introduces the centuries-old study of solving---or
being unable to solve---single-variable polynomials using radicals
(the latter topic being discussed only informally because of its
advanced nature).  The well-known {\it quadratic formula} is the
simplest instance of solving polynomials by radicals: The two
solutions to the polynomial equation
\[ ax^2 \ + \ bx \ + \ c \ = \ 0 \]
are
\[ x \ = \ \frac{-b + \sqrt{b^2 - 4ac}}{2a}
 \ \ \ \mbox{ and } \ \ \
   x \ = \ \frac{-b - \sqrt{b^2 - 4ac}}{2a}
\]
\item
The chapter's treatment of polynomials then digresses to present an
extremely important result about two-variable polynomials, Newton's
famous {\it Binomial Theorem}.  This result enables one to go easily
between polynomials in a certain family and their roots.  The
Theorem's ``smallest'' instances provide the following equations.
\begin{eqnarray*}
(x + y)^2 & = & x^2 \ + \ 2xy \ + \ y^2 \\
(x + y)^3 & = & x^3 \ + \ 3x^2y \ + \ 3 xy^2 \ + \ y^3
\end{eqnarray*}
\item
The chapter's treatment of polynomials closes with a short, informal,
discussion of {\it Hibert's Tenth Problem}, an advanced topic of
immense mathematical import.  In one line: The work on this Problem
demonstrates that the single topic of discovering integer roots of
arbitrary polynomials with integer coefficients---or proving that no
such roots exist---captures (read: {\em encodes}) the full complexity
of performing arbitrary computations!
\end{itemize}
The chapter finally introduces the mutually inverse operations of
taking exponentials and logarithms, in the sense of the equations
\[  x \ = \ \log_b(b^x) \ = \ b^{\log_b(x)} \]
A fundamental insight here is that the arithmetical system based on
these functions is {\em almost} identical to our conventional
arithmetic---but with multiplication replacing addition and division
replacing subtraction.  ({\em This is the basic insight underlying the
  {\em slide rules} that were techies' pocket calculators for many
  decades.})  The qualifier ``almost'' hints at the adjustments needed
to accommodate the impossibility of dividing by $0$.  This pair of
operations play a fundamental role in the field of information theory,
whose importance to computing cannot be overstated.

\bigskip

Chapter~\ref{ch:Summation} is basically a {\em tools} chapter. 
It studies how to break a complex operation into simple constituents.
It is devoted to analyzing a broad variety of families of summations,
providing exact solution-sums for many and approximate sums for
others.  All of the finite summations we study have the general form
\[ S \ = \ s_1 \ + \ s_2 \ + \cdots + \ s_n \]
In situations where we allow summations to have infinitely many
terms---for instance, with the famous summation
\[ 1 \ + \ {1 \over 2} \ + \ {1 \over 4} \ + \ {1 \over 8} \ + \cdots
+ \ {1 \over {2^k}} \ + \cdots
\]
(whose sum is $2$), we allow this pattern to continue without end.  We
include here {\it arithmetic summations}, in which all adjacent
summands, $s_{i+1}$ and $s_i$, have a common difference, and {\it
  geometric summations}, in which all adjacent summands, $s_{i+1}$ and
$s_i$, have a common ratio.  We discuss also powerful techniques for
estimating the solution-sums of summations of consecutive terms of a
``smooth'' function.

The mathematics we exploit to derive exact and/or approximate sums for
many of the classes of summations we study provides us with the
opportunity of looking at numbers and their summations in many quite
distinct ways---from textual to pictorial to geometric, and beyond.
For this reason, this chapter is one of the most important as the
reader gains traction in the endeavor of ``doing mathematics''.

\subsubsection{Chapter~\ref{ch:infinity}: The Vertigo of Infinity}

This chapter deals with mathematical objects whose very size---ranging
from the finite but very large to the infinite---is difficult to
reason about, for one of two reasons.
\begin{enumerate}
\item
We have developed the important ability to reason abstractly by
blurring descriptions.  We notice, e.g., the ways in which all
phenomena that experience {\em linear} growth differ from phenomena
that experience {\em quadratic} growth, and both classes differ from
phenomena that experience {\em cubic} growth---and so on.

Additionally, we often encounter finite objects whose sizes can change
dynamically or can be known only approximately.  Consider, e.g.,
social networks: they expand and contract in unpredictable manners, so
their population statistics cannot be known exactly.

We require a language for talking---and rigorously reasoning---about
blurry distinctions and approximately-known objects.  And, we need a
formal analogue of arithmetic for ``calculating'' the statistics of
such objects.

The topic of {\it asymptotics} (Section~\ref{sec:asymptotics}) fills
both needs, for a large variety of dynamic mathematical phenomena.
Asymptotics thereby enables {\em reasoning qualitatively about
  inherently quantitative phenomena}.

\item
We often have to reason about objects that are actually infinite.  We
encounter such situations, e.g., in areas that somehow combine
  \begin{itemize}
  \item
mathematics---e.g., classes of numbers or of functions
  \item
logic---e.g., expressions with special properties, such as theorems or
proofs
  \item
linguistics---e.g., sentences that share some syntactic or semantic
characteristics
  \end{itemize}
Since antiquity, we have been confronted by infinite objects that
behave very differently than the finite objects of daily discourse.
As just two examples:
  \begin{itemize}
  \item
Why does a shot arrow reaches its target.  The following seemingly
cogent argument shows that it does not.

The argument states---correctly---that after the arrow traverses
one-half the distance to the target, it still has one-half the
distance to go; after traversing half of the remaining distance, it
still has one-quarter of the original distance to go.  Continuing, the
arrow will have a never-ending shrinking distance that it has yet to
traverse: one-eighth, then one-sixteenth, then one-thirty-second, and
so on.  How can the arrow ever reach the target?

  \item
We have had to adapt to the fact that ``provable'' and ``true'' are
distinct concepts within most realistic systems of logic, despite
their intuitive coincidence.
  \end{itemize}

In a slightly more sophisticated vein, we all know that there are
``equally many'' odd integers as all integers---even though the former
set is obtained by discarding half of the latter set's elements.

\ignore{******
Even more puzzling: we know that there are infinitely many fractions
between any two integers.  Yet, it is easy to show (see
Section~\ref{sec:Q-Z-cardinality}) that the set of all these
fractions---this infinite collection of infinite sets---is ``no
larger'' than the set comprising just the integers!
***********}

Section~\ref{sec:coping-infinity} provides the background necessary to
cope with these puzzles and navigate the unfamiliar world of the
infinite.
\end{enumerate}

\ignore{*********
\noindent {\em Coping with infinity}.
One recurring challenge when ``doing'' mathematics is dealing with
{\em infinity}.  Infinite objects, such as sets, behave rather
differently from the more familiar finite objects that we encounter in
our daily lives.  As but one example, there are ``equally many'' odd
integers as all integers, as measured by the ability to match the two
sets element by element---even though the first of these sets is
obtained from the second by discarding half of its elements.

A more subtle challenge manifests itself in the world of the infinite
resides in the myriad {\em paradoxes} that one encounters in this
world.  Does an arrow ever reach its target---
************}

\subsubsection{Chapter~\ref{ch:Recurrences}: Recurrences}

Imposing manageable structures on constructs and computations.

The functions discussed in Chapter~\ref{ch:arithmetic} are described
by static ``closed-form'' expressions.  In contrast,
Chapter~\ref{ch:Recurrences} is devoted to a family of {\em
  computational procedures}, {\it recurrences}---which calculate the
value of a function $f$ on an argument $n$ in terms of the values of
$f$ on smaller arguments.  The patterns that underlie recurrent
computations have been known for millennia to occur throughout
nature---in the growth patterns of many plants and in the demographics
of many animals, among other places.  The mathematics that describes
such computations is as elegant and aesthetic as it is useful.  Among
the myriad recurrent patterns that one could identify, we select three
main ones for their computational importance.
\begin{enumerate}
\item
{\it Linear recurrences}.  Recurrences of the form
\begin{equation}
\label{eq:general-linear-recurrence}
f(n) \ = \ a f(bn+c) \ + \ dn \ + e
\end{equation}
(where $a, b, c, d, e$ are constants) are a welcome friend when one
analyzes the cost of myriad basic algorithms.  Perhaps the simplest
nontrivial use of recurrence (\ref{eq:general-linear-recurrence}) is
in the analysis of 
%{\Denis remark: both next examples are dealing with algorithms, may be too much?}
%  \begin{itemize}
the {\em binary search algorithm}, which determines whether an item
$x$ occurs within a given ordered list of $n$ items.  The algorithm
begins by partitioning the list into two (so $b = 1/2$ and $c = 0$ in
(\ref{eq:general-linear-recurrence})).  It then compares $x$ with the
list's middle item(s)---there is one middle item when $n$ is odd, two
when $n$ is even---(so $d = 0$, and $e$ is the fixed constant cost of
the comparison(s)).  Based on the outcome of the comparison, the
algorithm recurses with a binary search on the half of the list that
could contain $x$ (so $a = 1$).

\ignore{**********
  \item
In somewhat simplified terms, the {\em merge-sort algorithm} builds on
a natural $n$-step algorithm for merging two $n$-item lists of
``order-comparable'' items (i.e., for every pair, $x,y$, of list
items, either $x < y$ or $y < x$).  The algorithm first merges
adjacent odd-even pairs of items, to end up with sorted $2$-item
lists.  Then it merges adjacent $2$-item lists, to end up with sorted
$4$-item lists.  It continues thus with $8$-item lists, then $16$-item
lists, and so on, until it finally merges the ``top-level'' two
$n/2$-item lists to achieve the goal of a sorted $n$-item list.  We
hope that the reader can see intuitively that the performance of this
algorithm is given by a recurrence of the form
(\ref{eq:general-linear-recurrence}) for which $a=2$, $b = 1/2$ and $c
= 0$, $d=1$, and $e$ is the fixed cost of the $2$-item comparisons that
orchestrate each merge.
  \end{itemize}
*******}

The centerpiece of our discussion of linear recurrences is the
so-called {\em Master Theorem}, which uses geometric summations to
generate explicit---rather than recurrent---expressions for the values
of a function $f$ on an arbitrary argument $n$.

\item
We highlight two {\it bilinear} recurrences of especial computational
importance.
  \begin{itemize}
  \item
The {\it binomial coefficient} articulated as ``$n$ choose $k$'', and
commonly denoted $\displaystyle {n \choose k}$ or $\Delta_{n,k}$ or
$C(n,k)$, plays a central role in a broad range of mathematical
domains.  Two rather distinct examples:

%{\Denis I found both examples too much detailed. I suggest to shorten}
       \begin{itemize}
       \item
$C(n;k)$ is the number of ways to select $k$ items out of a set of $n$
items.  For instance, $C(4;2) =6$, because the $2$-element subsets of
$\{A, B, C, D\}$ are: $\{A, B\}$, $\{A, C\}$, $\{A, D\}$,  $\{B, C\}$,
$\{B, D\}$,  and $\{C, D\}$.  Analyzing $C(n, 5)$ will reveal, for
example, why ``three of a kind'' beats ``two pair'' in poker.
       \item
These numbers play a prominent role in evaluating arithmetic
summations; e.g., we derive the following equation (in several ways)
in later chapters:
\[ 1 \ + \ 2 \ + \ 3 \ + \cdots + \ n \ \ = \ \ C(n+1, 2) \]
       \end{itemize}
The {\em family} of binomial coefficients is defined by the bilinear
recurrence
\begin{eqnarray*}
C(n, 0) & = & 1 \\
C(n, 1) & = & n \\
C(n+1, k+1) & = & C(n, k) \ + \ C(n, k+1)
\end{eqnarray*}

  \item
The sequence of {\it Fibonacci numbers}, named for the Pisano
mathematician known by the nickname ``Fibonnaci'', are defined by the
recurrence
\begin{eqnarray*}
F(0) & = & F(1) \ = \ 1 \\
F(n+1) & = & F(n) \ + \ F(n-1)
\end{eqnarray*}
Fibonacci invented his eponymous sequence as he observed the
populations of consecutive generations of progeny produced by a pair
of rabbits; the sequence also arises in the structure of many plants;
it further is said to have a semi-religious aspect in the architecture
of ancient Greek temples.
%  The story of the sequence is told in a bit more detail in Section~\ref{sec:Fibonacci-story}.
Aside from its important descriptive role, the sequence plays a
significant role in the analysis of algorithms and as the basis for a
nonstandard system of numerals.
  \end{itemize}
\end{enumerate}
Variations on the preceding three families of recurrences provide
supplemental material in this chapter.

\subsubsection{Chapter~\ref{ch:combinatorics}: 
The art of counting, with applications on
Combinatorics,
  Probability, and Statistics}

******TO BE DONE******

\subsubsection{Chapter~\ref{ch:Graphs-Trees}: An Introduction on Graphs and Their Relatives}

Graphs are perhaps the most important representational concept in all
of mathematics.  In their most basic form, graphs represent any binary
relation; a brief sampler:
\begin{itemize}
\item
the structure of a family, as exposed by the parent-child relation;
\item
the structure of an electronic or a communication circuit, where
certain pairs of entities have the right to intercommunicate---perhaps
only directionally).
\end{itemize}
The structure represented by a graph can not only expose the fact that
certain pairs of entities can intercommunicate; it can also expose the
number of ``links'' that must be traversed to achieve communication.

Even with this rudimentary discussion, one can intuit how myriad
real-life problems can be modeled using graphs.  Many such problems
use a graph to represent entities that can ``talk'' to one another, in
some sense.  A variety of associated questions could be of the form,
``Who know what when?''

Graphs are immensely important in countless scheduling applications,
for the underlying relation can be {\em directional}---exposing
dependencies.  A common relation studied in computer applications
exposes that task $A$ in a program depends on input from task $B$, so
that $B$ must be performed {\em before $A$}.  The notion of {\it graph
  coloring} is exceedingly important in scheduling and related
applications: In its conceptually simplest form, one colors the task
of a dependency graph in such a way that like-colored tasks can be
executed concurrently---they are computationally independent.  The
challenge in this scenario is to color a give task-graph with as few
colors as possible.  This is a computationally difficult task in
general, but we expose some of the sophisticated mathematics that has
been developed in order to study the graph-coloring problem.
{\em Path problems} in graphs provide another entry to myriad
scheduling applications.  Of particular interest are problems that
require some object (e.g., a datum) to passed within a graph so that
it encounters all of the graph's entities.

%{\Denis add a sentence on paths, hamiltonian and eulerian, which are
%the way to establish connections between the entities.}

\medskip

If {\em binary} relations are not adequate for a person's modeling
needs, the expanded notion of {\em hypergraphs} can be used to model
relations beyond binary, even those in which the ``arity'' of the
relation varies from one related group of items to the next; a brief
sampler:
\begin{itemize}
\item
the structure of a family, as exposed by two relations: the
parent-child relation and the sibling relation;
\item
the structure of a bus-connected communication setup: entities on a
single bus can all ``hear'' one another;
\item
social networks in which aggregations of ``friends'' have special
intercommunicating privileges
\end{itemize}


\section{How to Use This Text}
\label{sec:how-to-use}

{\Arny This section is being deferred until the end, so we have a
more complete picture of the various topics covered---which ones and
to what level.}

\begin{description}
\item[{\bf Digital logic and Computer architecture}.]
\index{digital logic} \index{computer architecture}
This topic would arise in computer engineering programs and in the
early portions of a course on computer architecture.
\begin{itemize}
\item
{\bf Digital logic}. \index{digital logic}
\item
{\bf Computer arithmetic}.  \index{computer arithmetic}
\end{itemize}

\item[{\bf Cryptography and Computer security}.]


\item[{\bf Big data}.]


\item[{\bf Artificial intelligence}.]


\item[{\bf Social networks}.]

\end{description}



%\subsection{Sample Curricula Based on This Text}
%\label{sec:sample-curricula}


