%version of 04-25-19

\chapter{INTRODUCTION}
\label{ch:intro}

\section{Why Is This Book {\em Needed?}}
\label{sec:bookneeded}

How much mathematics does an aspiring computing professional
need---and at what level of expertise?  We beleve that the answer to
this pedagogically fundamental question is time dependent.

\medskip

\noindent {\it The early generation}.
In the early days of computing, all aspects of the field were
considered the domain of the ``techies''---the engineers and
scientists and mathematicians who designed the early computers and
figured out how to use them to solve a range of (mostly
compute-intensive) problems.  Back then, one expected every computing
professional to have a mastery of many mathematical topics.

\medskip

\noindent {\it Children of the early generation}.
Times---and the field of computing---changed.  The ``techies'' were
able to craft a variety of sophisticated tools that opened up the
world of computing to the general population.  Even people at the
lower levels of the educational edifice were able to use imagination
and ingenuity, rather than theorems and formulas, to produce
impressive software artifacts of considerable utility.

\medskip

\noindent {\it The modern generation}.
Pendulums were made to swing.  As we note in our {\it Manifesto}, we
now encounter almost daily problems that arise from unanticipated
concomitants of the often unstructured ingenuity that produced various
software artifacts.  Many would say that we need a renewed commitment
to technical discipline that will endow artifacts with
\begin{itemize}
\item
{\em understandable structure}, so that we can determine {\em what} went
wrong when something {\em does} go wrong.
\item
{\em sustainability}, so that changes, which are inevitable in complex
artifacts, will not create new problems
\item
{\em controllability}, so that ``smart'' artifacts do not become
modern instances of Dr.~Frankenstein's monster.
\end{itemize}
While we certainly need not return to the era of the ``techies'', it
is unquestionable that we do need a larger contingent of computing
professionals who have bona fide expertise that enables the needed
technical discipline.  This text is devoted to the mathematics that
underlies the needed science and engineering.

\bigskip

Of course, there are many introductory texts on discrete mathematics.
What separates this text from the others is the stratagem we have
implemented to accommodate our intended audience of aspiring computing
professionals.  The skeleton of this stratagem appears in the {\it
  Preface}.  We expand on it in the next section.



\section{Why Is {\em This} Book Needed?}
\label{sec:thisbookneed}

We live in a world of increasing professional and social diversity.

\medskip

Historically, computing curricula began as highly technically-oriented
studies leading to a degree such as a BS, within a school of science
or engineering.
\bigskip

\noindent \fbox{
\begin{minipage}{0.95\textwidth}
{\it Terminology:}
\begin{itemize}
\item
The {\it BS} (``Bachelor of Science'') degree is the technical version
of the first university degree in the US.  BS programs may specialize
in focus, self-identifying as, say, Computer Science or Computer
Engineering or Computing Science. 
\item
The {\it BA} (``Bachelor of Arts'') degree is the nontechnical version
of the first university degree in the US.
\item
Other University-level computing-oriented programs in the US include
{\it IT} (``Information Technology'') and {\it CA} (``Computing
Arts'').  These are often computing curricula that focus on the use of
``packages'', rather than on general programming.
\end{itemize}
\end{minipage}
}
\bigskip

\noindent
The evolution of the computing field has occasioned changes in this
curricular worldview.  The traditional curricula have been joined by
curricula that have deemphasized the traditional science and
engineering and mathematics courses, in order to emphasize the
humanities or social studies (leading, say, to a BA degree), or to
emphasize the {\em practice} of computing, either in a general setting
(leading, say, to a degree in IT) or in a focused applied field
(leading, say, to a degree in business or finance or law or \ldots).
Within this new world, many aspiring computer professionals arguably
need only specialized knowledge of mathematics---but the amount and
direction of the specialization can vary greatly based on an academic
program's focus.

\medskip

Add to the preceding reality the phenomenal broadening of the audience
for technical curricula, hence for some level of mathematics
education.  Many barriers to such education have now either
disappeared or at least diminished in impact.

\medskip

The number of nontraditional students seeking technical education is
growing, and the ways in which their nontraditional goals and
backgrounds manifest themselves is increasing.  This phenomenon is
important to computing educators, who must serve a large population of
students, with quite diverse needs and aspirations.

We have striven to keep this diversity in mind as we have written this
text.  We have included a broad range of material, in both subject and
level.  As noted in the {\it Preface}, we have tried to accommodate
the different backgrounds of our readership, while leading them all to
a level of mathematical maturity that will enable them to understand
mathematics and to {\em do} mathematics.  As we
discuss a range of mathematical topics for possible inclusion as
prerequisites for the undergraduate study of computing, we try to
indicate why the selected topics are needed.  Readers can then
evaluate which topics are needed for their needs.

\section{The Structure of This Book}
\label{sec:thisbook}

\subsection{Our Main Intellectual Targets}
\label{sec:book-overwiew}

This book is devoted to covering the discrete-mathematics
underpinnings of the endeavor of computing: from the design and
implementation of devices that perform the actions necessary to
compute to the design of the processes that control the
devices---including whatever communications are needed among processes
and among (sub)devices.  We have identified several pillars of
computing that merit extensive study.
\begin{enumerate}
\item
{\it Fundamental concepts}

\medskip

{\small\sf Examples:}
\begin{itemize}
\item%
sets---and their embellishments: tuples, arrays, tables, etc.---as
embodiments of {\it object}
\item
numbers---and their operational manifestations, numerals---as
embodiments of {\it quantity}
\item
graphs---in their many, varied, forms---as embodiments of {\it
  connectivity} and {\it relationship}
\item
algebras and functions---adding operations to sets, numbers, and
graphs---as embodiments of {\it structured dynamism} and {\it
  computing} and {\it process}
\end{itemize}
We thereby expand the scope of what can be thought about
``mathematically''.

\medskip

\item
{\it Fundamental representations}

\medskip

{\small\sf Examples:}
\begin{itemize}
\item
representing and thinking about numbers via many metaphors: slices of
pie, tokens arranged in stylized ways, characteristics of rectangles
of varying dimensionalities, textual objects
\item
using grouping and/or replication to represent relationships among
objects
\item
viewing interrelated objects via many structures: tables, tuples,
graphs, geometric drawings
\end{itemize}
We thereby expand the universe of conceptual pardigms one can use
while thinking ``mathematically''.

\medskip

\item
{\it Fundamental tools/techniques}

\medskip

{\small\sf Examples:}
\begin{itemize}
\item
using induction to extrapolate from simple examples to complex ones
\item
``hopping'' between the discrete and continuous mathematical worlds,
e.g., using integration to approximate summation
\item
using the conceptual tools of asymptotics to argue qualitatively about
quantitative phenomena.
\item
``hopping'' between the mathematical reasoning used in the ``real''
  world, vs.~the formal logics that enable such reasoning
\end{itemize}
We thereby expand the conceptual tools that one has access to when
{\em doing} mathematics.

\medskip

\item
{\it Beyond the fundamentals}

**HERE -- explanatory sentence

\medskip

{\small\sf Examples:}
\begin{itemize}
\item
TBD
\end{itemize}

\end{enumerate}


\subsection{Allocating our Targets to Chapters}
\label{sec:the chapters}

**HERE



\section{How to Use This Text}
\label{sec:how-to-use}


%\subsection{Sample Curricula Based on This Text}
%\label{sec:sample-curricula}


