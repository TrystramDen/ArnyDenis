%version of 03-19-18

\chapter{INTRODUCTION}


\begin{center}
{\it Entia non sunt multiplicanda praeter necessitatem.} \\
\hspace*{3in}{\footnotesize {\bf Occam's Razor}}
\end{center}

\noindent
This famous admonition by William of Occam (14th cent.) to strive for
simplicity is worth heeding when seeking mathematical models of
computational phenomena.

\section{Overview}

In the early days of computing, all aspects of the field were
considered the domain of the ``techies''---the engineers and
scientists and mathematicians who designed the early computers and
figured out how to use them to solve a range of problems that is
expanding even to this day.  Back then, one expected every
computer-oriented professional to have a mastery of many mathematical
topics.  Times---and the field of computing---have changed: classical
computing curricula, which have traditional led to a BS degree within
a school of science or engineering have been joined by curricula that
lead to a BA degree or a degree in IT or in business or \ldots.
Within this new world, many aspiring computer professionals arguably
need little knowledge of mathematics.  However, many argue that the
computing field suffers for this lack of mathematics, which has
weakened the ability of many practitioners to design and perform
experiments, to analyze their results, and to formulate well-reasoned
conclusions based on these results.  This situation has denied
computer science the popular confidence enjoyed by other empirical
disciplines; indeed, many would question the math-deprived
practitioners' ability to reason rigorously about basic computational
phenomena.  Yet others, however, argue that such ``scientific'' acumen
is not needed by practitioners in many of the newer segments of the
computing field.

The preceding discussion is important to computing educators because
we serve a large population of students, with quite diverse needs and
aspirations.  We strive to keep this diversity in mind within this
essay.  As we discuss a range of mathematical topics for possible
inclusion as prerequisites for the undergraduate study of computing,
we try to indicate why the selected topics are needed.  Readers can
then evaluate which topics are needed for students enrolled in their
computing program.

\section{The Elements of Rigorous Reasoning}


\subsection{Basic Reasoning}

{\em Distinguishing name from object}.
%
A fundamental stumbling block in the road to cogent reasoning arises
from the inability to distinguish names from the objects they denote.
A prime example within the world of computing resides in the inability
to distinguish a function (which can be viewed as an infinite set of
argument-value pairs) from a program, which can be viewed as a name
for the function.  {\bf Note} that the often-used view of a function
as a {\em rule} for assigning values to arguments should be avoided,
because it suggests -- \underline{\em erroneously} -- that an
implementable such rule always exists!

{\em Quantitative reasoning}.
%
Students should understand the foundational distinction between
``growing without bound'' and being innite.  Within this theme, they
should appreciate situations such as the following.  Every integer,
and every polynomial with integer coefficients, is finite, but there
are infinitely many integers and infinitely many polynomials.
Students should be able to verify (cogently but not necessarily via
any particular formalism) assertions such as the following.
\begin{itemize}
\item
Let us be given polynomials $p(x)$ of degree $a$ and $q(x)$ of degree
$b > a$, where $a, b$ need not be integers.  There must exist a
constant $X_{p,q}$ (i.e., a constant that depends on the properties of
polynomials $p$ and $q$) such that for all $x > X_{p,q}$, $p(x) <
q(x)$.

Thus, polynomials having bigger degrees eventually {\em
  majorize}---i.e., have larger values than---polynomials having
smaller degrees.

\item
Continuing with polynomial $q$ of degree $b$: For any real number $c >
1$, there exists a constant $Y_{c;q}$ (i.e., a constant that depends
on the properties of polynomial $q$ and constant $c$) such that for
all $x > Y_{c;q}$, $c^x > q(x)$.

Thus, exponential functions eventually {\em majorize} polynomials.
\end{itemize}

\subsubsection{The Elements of Formal Reasoning}

induction

proof by contradiction

\subsubsection{The Elements of Empirical Reasoning}

Empirical reasoning does not convey the certitude that formal
reasoning does.  



\section{Our Approach to Mathematical Preliminaries}

\hfill{\small\em
``If your only tool is a hammer \ldots''
}

\vspace*{.25in}

\noindent
We now review a broad range of mathematical concepts that are central
to the study and practice of CS/CE.  As we develop these concepts, we
shall repeatedly observe instances of the following ``self-evident
truth'' (which is what ``axiom'' means).

\begin{quote}
{\bf The conceptual axiom}.
\index{conceptual axiom}
{\em
One's ability to think deeply about a complicated concept is always
enhanced by having more than one way to think about the concept.}
\end{quote}
