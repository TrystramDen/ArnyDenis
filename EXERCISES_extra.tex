\documentclass{article}[12pt]

\usepackage{mathptmx}
\usepackage{amssymb}
\usepackage{graphicx}

\begin{document}



%%%%%%%%%%%%%%%%%%%%%%%%%%%%%%%%
\section{$\oplus$ A fun result dealing with divisibility}

THIS NOW LIVES IN CH. 8 \\
***********************

Chapter 4 or alternatively chapter 10.

This exercice was a favorite question by the famous mathematician Paul Erdos, 
he often used it to test the ability of young students in mathematics...

The problem is described as follows: Let consider the 2n first integers.

Take any $n+1$ integers in this set and prove that there exists a pair $(p,q)$ such that $p$ divides $q$.
\smallskip

Hint : Write the $2n$ numbers by decomposing the sequence into multiples of powers of 2. 
When there are multiple ways, we take the one with the largest power of 2 in order to make the decomposition unique. 

Example for n = 7:
$(1, 3, 5, 7, 9, 11, 13)$ (multiples of $2^0$), 
$(2, 6, 10, 14)$ (multiples of $2^1$),
$(4, 12)$ (multiples of $2^2$) and $(8)$ (multiples of $2^3$).

Thus, according to such a decomposition, all the integers of the sequence are written as: 

$2k \times m$ where $m$ is odd.
Use the pigeon hole principle to exhibit $p$ and $q$.

\subsection{Solution: To be completed}


%%%%%%%%%%%%%%%%%%%%%%%%%%%%%%%%
\section{The Horner-Estrin scheme for polynomial evaluation}

*********************
THIS IS NOW IN NUMBERS3 (where Horner lives)
*********************


Chapter 5

Notice that the analysis leads asymptotically to the same complexity as the classical Horner's scheme but in a different way.
\bigskip

Let consider a polynomial of degree $d$:
\[
P(x) \ \ = \ \ a_0 \ + \ a_1 x \ + \ a_2 x^2 \ + \cdots + \ a_{d-1} x^{d-1} \ + \ a_d x^d
\]

Below is an example for $d=7$.

We rewrite $P(x)$ as:
\[
P(x) \ \ = \ \ a_0 \ + \ a_1 x \ + \ x^2 \ ( a_2  \ + a_3 x \ ) \ + 
\ x^4 \ ( \ a_4 \ + \ a_5 x \ + \ x^2 \ ( a_6  \ + a_7 x \ ) \ )
\]

%\noindent General degree $d$:
%
%$P(x) \ \ = \ \ a_0 \ + \ x \cdot (a_1 \ + \ x \cdot (a_2  \ +  \cdots
%+ x \cdot (a_{d-2} \ + \ x \cdot (a_{d-1} \ + \ a_d x)) \cdots ))$  

We introduce 

$C_i^{(0)} = a_i + x a_{i+1}$ 

$C_i^{(n)} = C_i^{(n-1)} + x^{2n} C_{i+2^n}^{(n-1)}$ 
\bigskip

Write $P(d)$ using the $C_i$ (consider $d$ is a power of $2$ to simplify).

How many additions and multiplications are required?

\bigskip

\subsection{Solution}
\[
P(x) \ \ = \ \ C_{0}^{(0)} \ + \ x^2 \ C_2^{(0)} \ + 
\ x^4 \ ( \ C_4^{(0)} \ + \ x^2 \ C_6^{(0} \ )
\ = \ C_0^{(1)} \ + \ x^4 \ C_4^{(1)} 
\]

\begin{itemize}
\item
For $d=7$, we have $2$ multiplications for computing $x^2$ and $x^4$
\item
Then, 4 multiplications for the products $a_7 x$,  $a_5 x$, $a_3 x$ and $a_1 x$

followed by 4 additions
\item
2 multiplications for the $x^2 C_2^{(0)}$ and  $x^2 C_6^{(0)}$

followed by 2 additions
\item
Finally, 1 multiplication for computing $x^4 C_4^{(1)}$

followed by 1 addition
\end{itemize}

Total of $2 + 7$ multiplications and $7$ additions.

Generalization: 
$d + log_2(d)$ multiplications and $d$ additions.
\bigskip

This has to be compared to the Horner's scheme: $d$ multiplications and $d$ additions.


%%%%%%%%%%%%%%%%%%%%%%%%%%%%%%%%
\section{Product of $4$ consecutive integers}

Chapter 5

************************
THIS NOW RESIDES IN NUMBERS2
***********************


Show that the product of $4$ consecutive integers $P(k) = k.(k+1).(k+2).(k+3)$
is a multiple of $4!$


\subsection{Solution}

The result directly follows from the following facts:

In $4$ consecutive integers, at least two are even, one of both even is a multiple of $4$.

In $3$ consecutive integers, at least one in a multiple of $3$.


%%%%%%%%%%%%%%%%%%%%%%%%%%%%%%%%
\section{A geometric proof for the product of $4$ consecutive integers}

Chapter 2 or chapter 6

************************
THIS LOOKS TOO HARD
**********************


Develop a geometric proof for the product of $4$ consecutive integers $P(k) = k.(k+1).(k+2).(k+3)$
is a multiple of $4!$

Hint:
The geometric construction is tricky and comes from a representation of the product $P(k)$ as a rectangle whose sides are the product of the two extreme factors 
$k.(k+3)$ of the product by the two internal factors $(k+1)(k+2)$.

\subsection{Solution}
Let representant the product $P(k)$ as a rectangle whose sides are the product of the two extreme factors 
$k.(k+3)$ by the two internal factors $(k+1)(k+2)$.

$k.(k+3) = k^2 + 3k$ and $(k+1)(k+2) = k^2 + 3k +2$, thus, $P(k)$ is almost a square (a square minus 1)
if the last column is put as the last rows as shown in the figure.

We can even go further since the elementary pattern is a triangular number which is divisible by $3$:

The side of this square $k^2+3k+1$ is odd. 

$\Delta_{3k-1} = \frac{3k.(3k-1)}{2}$
\begin{figure}[h]
\begin{center}
        \includegraphics[scale=0.5]{FiguresArithmetic/Product4consecutivePhase1} 
        \caption{The rectangle $k.(k+3)$ by $(k+1)(k+2)$. Moving the last column in grey (left) 
        in place of an extra row gives almost a perfect square (right).}
\end{center}
\end{figure}
\begin{figure}[h]
\begin{center}
        \includegraphics[scale=0.5]{FiguresArithmetic/Product4consecutivePhase2} 
        \caption{Dividing the internal area of a odd square into $8$ equal pieces.}
\end{center}
\end{figure}

%%%%%%%%%%%%%%%%%%%%%%%%%%%%%%%%
\section{Sum of finite geometric progression}

Chapter 6

***************
This is now in SUMMATION
***************

Extend the geometrical proof for computing the sum of a finite geometric progression 

$u_{n+1} = b.u_{n−1}$ and $u_0 = 1$.

\subsection{Solution} 

Use Thales from the reserve side...
\begin{figure}[h]
\begin{center}
        \includegraphics[scale=0.4]{FiguresArithmetic/ThalesGeometricSumFinite} 
        \caption{Thales.}
\end{center}
\end{figure}



%%%%%%%%%%%%%%%%%%%%%%%%%%%%%%%%
\section{Counting triangles}

***********************
FOR NOW, I AM PUTTING THIS IN Ch 11
***********************

Chapter 9.

Let consider a suite of triangles constructed by assembling smaller size equilateral $(1,1,1)$ triangles. 
The objective is to enumerate all the triangles contained at each rank of the progression
(see Fig.~\ref{fig:countingTriangles}). Let denote 
The goal is to compute the elements of the progression.
\begin{figure}[h]
\begin{center}
        \includegraphics[scale=0.4]{FiguresArithmetic/CountingTriangles} 
        \caption{The first three triangles of the progression.}
        \label{fig:countingTriangles}
\end{center}
\end{figure}

Check the first terms of the progression are $1$, $5$ and $13$.

Draw the fourth term and compute its number $N_4$.

Give a recurrence expression for determining $N_k$.

Solve this progression. 

Hint: Distinguish the case of even and odd $k$.

\subsection{Solution}

$N_4 = 27$.

\begin{figure}[h]
\begin{center}
        \includegraphics[scale=0.3]{FiguresArithmetic/CountingTriangles2} 
        \caption{7 triangles at level 2 triangles (4 + 3).}
        \label{fig:countingTriangles2}
\end{center}
\end{figure}
\begin{figure}[h]
\begin{center}
        \includegraphics[scale=0.3]{FiguresArithmetic/CountingTriangles3} 
        \caption{3 triangles at level 3.}
        \label{fig:countingTriangles3}
\end{center}
\end{figure}
\bigskip

The solution is:

$N_k = \frac{k.(k+2).(2k+1)}{8}$ for $k$ even

$N_k = \frac{k.(k+2).(2k+1)-1}{8}$ for $k$ odd
\bigskip

The expressions can be derived by the following trick:
Make the difference between two consecutive numbers and again and again, then, the pattern is obvious
(alternance of $2$ and $1$, ...). See Fig.~\ref{fig:countingTrianglesproof}.
\begin{figure}[h]
\begin{center}
        \includegraphics[scale=0.5]{FiguresArithmetic/CountingTrianglesProof} 
        \caption{Computing the differences of consecutive $N_k$.}
        \label{fig:countingTrianglesProof}
\end{center}
\end{figure}


%%%%%%%%%%%%%%%%%%%%%%%%%%%%%%%%
\section{A special type of permutation: the Monge's shuffle}


IT IS NOW IN Ch. 11


Chapter 11?

A card trick: the Shuffle of Monge.

We are given 2n cards, which are divided evenly into two decks. 
The shuffle corresponds to put them such that the cards in the final deck come alternatively from left to right as depicted 
in Fig~\ref{fig:suffleMonge}.
The order $(1,2,3,4,5,6,7,8)$ becomes $(5,1,6,2,7,3,8,4)$.
\begin{figure}[h]
\begin{center}
        \includegraphics[scale=0.5]{FiguresArithmetic/suffleMonge} 
        \caption{Shuffle for $8$ cards ($n=4$).}
        \label{fig:suffleMonge}
\end{center}
\end{figure}

Let consider a prime $p$ $53$ and an integer $n$ such that $2n+1=p$.
Here $n=26$, $2n=52$ is a usual deck of Poker cards. 
We apply the same shuffle several times.

Show that the number of steps of Monge shuffle to come back to the initial deck divides $p-1$. 



%%%%%%%%%%%%%%%%%%%%%%%%%%%%%%%%
\section{An expression of $\Delta_n^2$}

Chapter 2.

Let consider the sum of the $n$ first integers $\Delta_n$.

Show the following expression:
for any positive integer $n$,
\[ \Delta_{n}^2 \ = \ n^3 + \Delta_{n-1}^2 \]

\subsection{solution}

easy by decomposing $\Delta_{n} = n + \Delta_{n-1}$
and then, developing the square:

 $\Delta_{n}^2 = n^2 + 2.\Delta_{n-1} + \Delta_{n-1}^2$



%%%%%%%%%%%%%%%%%%%%%%%%%%%%%%%%
\section{Fast multiplication of polynomials}


\subsection{Case study for degree 1}

Let consider $P(x) = a_1 + a_2 x $ and $P'(x) = a'_1 + a'_2 x $

$(P \times P')(x) = (a_1.a'_1) + (a_1.a'_2 + a_2.a'_1) x + (a_2.a'_2) x^2$

The coefficients of this new polynomial require $4$ multiplications and $1$ addition, 
namely $a_1.a'_1$, $a_1.a'_2$, $a_2.a'_1$ and $a_2.a'_2$.
\bigskip

Show how to reduce the number of multiplications using the Karatsuba's trick 
introduced in Exercise~\ref{} that is, 
to use the following arithmetic identity $ad+bc = ac + bd + (b-a)(c-d)$.

\subsection{$\oplus$Extension}

The goal now is to extend the previous analysis to polynomial of higher degrees.
Let consider that the degree is a power of $2$.
\bigskip

Solution:
Let detail what happens for polynomials of degree $2$:

Let consider $P(x) = a_1 + a_2 x + a_3 x^2$ and $P'(x) = a'_1 + a'_2 x + a'_3 x^2$

$(P \times P')(x) = (a_1.a'_1) + (a_1.a'_2 + a_2.a'_1) x + (a_1.a'_3 + a_2.a'_2 + a_3.a'_1) x^2 + (a_2.a'_3 + a_3.a'_2) x^3 +  (a_3.a'_3) x^4$
\bigskip

Now, write $P(x) = A_1 + A_2 x^2$ 

where $A_1 = a_1 + a_2 x$ and $A_2 = a_3$ and similarly for $P'$

$(P \times P')(x) = (A_1.A'_1) + (A_1.A'_2 + A_2.A'_1) x^2 + (A_2.A'_2) x^4$ 

and apply the same method as before using again the Karatsuba's trick on the $A_i$ and $A'_i$.



%%%%%%%%%%%%%%%%%%%%%%%%%%%%%%%%
\section{A global property on degrees}

***************
THIS NOW LIVES IN GRAPHS1
***************

Chapter 12.

Let consider a graph $G=(V,E)$

Show that the number of vertices of odd degree is even.

Hint:
First, let remark that the sum of all the degrees is even
(more precisely, $\Sigma_{x \in V} \delta(x) = 2 |E|$ since each edge counts exactly for two vertices -- its extremities). 

%
%%%%%%%%%%%%%%%%%%%%%%%%%%%%%%%%%
%\section{Tournament}
%
%Chapter 13.
%
%let consider a Directed Acyclic Graph $G$.
%Show that every Tournament is hamiltonian.
%
%Definition of Tournament.
%



\end{document}


