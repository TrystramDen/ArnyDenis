%version of 08-26-19
\chapter{$\oplus \oplus$ The Diverse Delights of de Bruijn Networks}
\label{ch:de-Bruijn-delights}



\section{Cycles in de Bruijn Networks}
\label{Appendix:deBruijn-Pancyclic}

This section is devoted to the following marvelous extension of
Proposition~\ref{thm:deBruijn-Hamiltonian}.

\index{self-loop}

\begin{prop}
\label{thm:deBruijn-pancyclic}
Every de Bruijn network $\d_n$ is directed-pancyclic.
In detail:  $\d_n$ has directed cycles of every length from $1$ to $2^n$.\footnote{The cycles of
length $1$ arise from  the {\it self-loops} on vertices $0 \cdots 0$ and $1 \cdots 1$ of every
de Bruijn network.}
\end{prop}

\begin{proof}
We argue by induction on the index $n$ of de Bruijn network $\d_n$.

Our base case is provided by $\d_2$ and $\d_3$.  A brief perusal of Figs.~\ref{fig:dB2by2} 
and~\ref{fig:dB2by3} will verify that both networks are directed-pancyclic.

Let us assume inductively that every de Bruijn network $\d_m$ with $m \leq n$ is 
directed-pancyclic.  Our progress from this assumption to a proof that $\d_{n+1}$ is 
directed-pancyclic relies heavily on two results from Section~\ref{sec:path-cycle-problems}, 
namely:
\begin{itemize}
\item
Corollary~\ref{corol:eulerian-named-graphs}: {\em Each $\d_n$ admits a directed Eulerian cycle.}
\item
Lemma~\ref{thm:deBruin-linegraph}: {\em The line-graph of $\d_n$ is (isomorphic to) $\d_{n+1}$.}
\end{itemize}
Our extension of the inductive hypothesis focuses separately on ``small"  and ``large" cycles.

\medskip

\noindent {\bf Case 1.}  {\em ``Small" cycles in $\d_{n+1}$: lengths $\ell \in \{1, \ldots, 2^n\}$.} \\
We know by our inductive hypothesis that $\d_n$ contains directed cycles of every length
$\ell \in \{1, \ldots, 2^n\}$.  Focus on one such cycle, say the length-$k$ cycle $\cc$.  Clearly,
the line-digraph of $\cc$ is another length-$k$ cycle, which is isomorphic to $\cc$.  The proof of
Lemma~\ref{thm:deBruin-linegraph} therefore tells us that the line-digraph of $\cc$ is
isomorphic to a length-$k$ cycle in $\d_{n+1}$.

It follows that $\d_{n+1}$ contains directed cycles of all lengths $\ell \in \{1, \ldots, 2^n\}$.

\medskip

\noindent {\bf Case 2.} {\em ``Large" cycles in $\d_{n+1}$: lengths $\ell \in \{2^n+ 1, \ldots, 2^{n+1}\}$.} \\
Focus on an arbitrary integer $m = 2^n +k$, where $k \in \{1, \ldots, 2^n\}$.  We verify that $\d_{n+1}$
contains a directed cycle of length $m$.

By Case 1 and our inductive hypothesis, we know that $\d_{n+1}$ contains a directed cycle, call
it $\cc$, of length $M = 2^{n+1} - m \ = \ 2^n -k$.  By Lemma~\ref{thm:deBruin-linegraph}, the
existence  of cycle $\cc$ implies that $\d_n$ has a connected Eulerian subgraph
containing $M$ arcs.  We claim that $\d_n$ {\em also} has a connected Eulerian subgraph
containing $m$ arcs.  Once we verify this claim, an invocation of
Lemma~\ref{thm:deBruin-linegraph} will establish the existence in $\d_{n+1}$ of a cycle that
contains $m = 2^n +k$ vertices.  Therefore, Case 2 will follow from a proof of the following lemma.

\begin{lemma}
\label{lem:compl-cycle}
For any $p \in \{0, \ldots, 2^{n+1}\}$, if $\d_n$ has a connected Eulerian sub-digraph $\g$ that
contains $p$ arcs, then it also has such a sub-digraph that contains $2^{n+1} - p$ arcs.
\end{lemma}

\begin{proof}{Lemma~\ref{lem:compl-cycle}}
Fix an arbitrary $p$ for which $\d_n$ has a connected $p$-arc Eulerian sub-digraph $\g$.  Because
both $\g$ and $\d_n$ are Eulerian, each vertex of each of these digraphs has equal indegree and
outdegree.  Therefore, if we remove from $\d_n$ the $p$ arcs of $\g$, then we are left with a
(not-necessarily connected) Eulerian sub-digraph $\h$ of $\d_n$ that contains $2^{n+1} - p$ arcs.
Let $\f_1, \ldots, \f_r$ be all of the maximal connected components of $\h$ that are {\em nontrivial}
in the sense of containing at least one arc apiece.  Of course, each $\f_i$ is connected and
Eulerian.

If $r=1$, then $\h$ is the sub-digraph $\g$ of $\d_n$ guaranteed by the Lemma.
If $r>1$, then we need to do some work to create $\g$ from the $\f_i$.

Because $\d_n$ is connected, and all of its arcs reside in either $\g$ or $\h$, $\g$ must contain
some arc $(u \rightarrow v)$ where $u$ is a vertex of some $\f_i$ and $v$ is a vertex of some
different $\f_j$ (i.e., $i \neq j$).  The existence of this arc implies that vertex $u$ has outdegree
$\leq 1$ in $\h$, even though $u$ has outdegree $2$ in $\d_n$.  Because $\h$ is Eulerian, we know
that $u$ has equal equal indegree and outdegree in $\h$.  Moreover, because $\h$ is connected 
and contains at least one arc, we know that $u$ cannot be an isolated vertex in $\h$.  Therefore:
{\em vertex $u$ has outdegree {\em exactly} $=1$ in $\h$}.  It follows that there must be an arc
$(u \rightarrow w)$ in $\f_i$ for some vertex $w$ of $\f_i$.  By symmetric reasoning (using 
indegrees instead of outdegrees), there must be an arc  $(t \rightarrow v)$ in $\f_j$ for some 
vertex $t$ of $\f_j$.  Because $(u \rightarrow v)$, $(u \rightarrow w)$, and $(t \rightarrow v)$ are
all arcs of $\d_n$, there must exist a length-$(n-1)$ bit-string $x$ and bits
$\beta, \gamma, \delta, \varepsilon \in \{0,1\}$ such that
\begin{eqnarray*}
t & = & \beta x    \\
u & = & \gamma x   \\
v & = & x \delta   \\
w & = & x \varepsilon
\end{eqnarray*}
It follows that $\g$ contains an arc $(t \rightarrow w)$.  This arc resides in $\d_n$ by definition;
it cannot reside in either $\f_i$ or $\f_j$ because it would connect these two components which
are disconnected in $\h$.

We now transform sub-digraph $\h$ in the following way. We remove from $\h$ two arcs: 
$(u \rightarrow w)$, which belongs to $\f_i$, and $(t \rightarrow v)$, which belongs to $\f_j$.  We
add in place of these arcs the arcs $(t \rightarrow w)$ and $(u \rightarrow v)$.  The resulting
new version of $\h$, call it $\h'$:
\begin{itemize}
\item
{\em contains the same number of arcs as $\h$ does};
\item
{\em is Eulerian}, because we just exchanged one arc that enters each of vertices $u$ and $w$
for another, and made a similar exchange for arcs that leave vertices $t$ and $u$;
\item
{\em is connected}, because each of $\f_i$ and $\f_j$, being directed-Eulerian, admit directed walks
that cross each arc precisely once---and our exchanged arcs connect these directed walks into
a composite directed walk through the new component.
\item
{\em has one fewer nontrivial maximal connected component than $\h$ does.}
\end{itemize}

If we now iterate the just-described transformation, each iteration yields an Eulerian sub-digraph
of $\d_n$ which has $p$ arcs (as desired) and has one fewer nontrivial maximal connected 
component than its predecessor.  After $r-1$ iterations, we therefore achieve the desired connected
Eulerian sub-digraph of $\d_n$.
\qed-Lemma~\ref{lem:compl-cycle}
\end{proof}

The $m$-arc connected Eulerian sub-digraph of $\d_n$ guaranteed by
Lemma~\ref{lem:compl-cycle} implies the existence of an $m$-vertex cycle in $\d_{n+1}$.  Since
the number $k$, hence the number $m$, was arbitrary, this completes the proof.  \qed
\end{proof}


\section{de Bruijn Networks as ``Escherian" Trees} 
\label{Appendix:tree-DB}

\index{de Bruijn networks!and binary trees}
\index{Escher, Maurits Cornelis (M. C.)}
This section is devoted to exposing a mathematically charming connection between a genre of
directed rooted tree and the family of de Bruijn networks.  The section title acknowledges the 
``spiritual" relationship between our mathematical connection and the well-known piece 
``Drawing Hands" (1948) of the Dutch artist Maurits Cornelis (M. C.) Escher.  (The shared
nationality of the artist Escher and the mathematician de Bruijn is an amusing coincidence.)

\index{digraph!algebraically generated arc-labeled digraphs}
The root of the connection lies in the following algebraic way of representing certain 
arc-labeled directed graphs.  Let us be given a set $V$ (which may be finite or infinite),  
together with functions $F_1$,  $F_2$, \ldots, $F_k$, each $F_i$ being a function from 
$V$ to $V$.  In our examples, the $F_i$ will be total injections from $V$ to $V$, but neither 
of these qualifiers (``total" or ``injection") is necessary for the concept we describe.  One 
can generate an arc-labeled digraph $\g = \g(V; F_1, \ldots, F_k)$ as follows.
\begin{itemize}
\item
The set $V$ comprises the vertices of $\g$.
\item
For each vertex $v \in V$ and each function $F_i$, $\g$ will have an arc with label $F_i$:
\[ (v \ \rightarrow F_i(v) \]
\end{itemize}
We provide three examples, the second two providing the correspondence that motivates this section.
{\Arny We will need a few figures here.  We should discuss exactly what.}

\medskip

\index{the directed infinite binary tree represented algebraically}
\noindent {\it 1. The directed infinite binary tree.}
Let the set $V$ be the set $\N^+$ of positive integers.  Define the arc-generating functions
\[
F_0(v) \ = \ 2v \ \ \ \ \mbox{ and } \ \ \ \ F_1(v) \ = \ 2v+1 \]
The rationale for the subscripts of $F_0$ and $F_1$ becomes clearer when we point out
the effect of each of these functions on the binary representations of the integer vertices of
$\g(\N^+; F_0, F_1)$: if $x$ is the binary representation of vertex $v$, then $x0$ is the 
binary representation of $F_0(v)$, and $x1$ is the binary representation of $F_1(v)$.
We indicate in Fig.~\ref{fig:one-node-tree} how the described system can be viewed
as specifying a graph-theoretic structure.  The figure depicts just one tree-vertex and its
children.  In the system as described, every vertex is a positive integer, and the arcs are
generated by the functions $F_0$ and $F_1$; when recast into ``string-mode", every vertex
is a binary string, and the arcs are generated by the functions ``append a $0$" and ``append a $1$".
\begin{figure}[hbt]
\[  \begin{array}{ccc}
\begin{array}{|ccccc|}
\hline
     &                & v &                &           \\
     & \swarrow &    & \searrow &           \\
2v &                 &   &                & 2v+1 \\
\hline
\end{array}
  & \hspace*{.5in} &
\begin{array}{|ccccc|}
\hline
     &                & x  &               &       \\ 
     & \swarrow &    & \searrow &       \\
x0 &                 &   &                & x1 \\
\hline
\end{array}
\end{array}
\]
\caption{The graph-theoretic action of one application of the functions $F_0$ and $F_1$:
(left) when the vertices of $\g(\N^+; F_0, F_1)$ are viewed as integers; (right) when the vertices
are viewed as bit-strings.}
\label{fig:one-node-tree}
\end{figure}

{\Arny Will this figure suffice, or do we want a fancy one?}

\medskip

\noindent {\it 2. The root-looped directed infinite binary tree.}
For reasons that will become clear in the upcoming paragraph 3, we amend the just-described
system so that it generates a cousin of the directed infinite binary tree.  This cousin is the tree
with a new vertex, call it $r$; vertex $r$ has two emerging arcs (as do all vertices): One of the 
new arcs is a {\em self-loop} on $r$; the other points to the root of the
directed infinite binary tree of paragraph 1.  One generates the cousin of interest by:
\begin{itemize}
\item
using the set $\N$ of {\em nonnegative} integers as vertices, instead of the earlier-used
set $\N^+$ of {\em positive} integers;
\item
using the same functions $F_0$ and $F_1$ to generate the arcs of the new graph.
\end{itemize}
In the new digraph, vertex $0$ is the ``new" vertex $r$ we just described.  The {\em self-loop} on
vertex $r=0$ occurs because $F_0(0) = 2 \times 0 = 0$.  The remainder of this new digraph is the
directed infinite binary tree of paragraph 1.

{\Arny Perhaps a small picture to suggest what the loop looks like graph-theoretically?}

\medskip

\index{the de Bruijn network represented algebraically}
\noindent {\it 3. The de Bruijn network as the Escherian root-looped directed infinite binary tree.}
We derive the desired connection between the tree-like digraph of paragraph 2 and the order-$n$
de Bruijn network by presenting the latter digraph algebraically, specifically, as the system 

\[ \g(\{0, 1, \ldots, 2^n-1\}; F^{(n)}_0, F^{(n)}_1) \]
where
\[ F^{(n)}_0(v) \ = \ 2v \bmod 2^n \ \ \ \ \mbox{ and } \ \ \ \ F^{(n)}_1(v) \ = \ 2v +1 \bmod 2^n \]
The system $\g(\{0, 1, \ldots, 2^n-1\}; F^{(n)}_0, F^{(n)}_1)$ thus differs from the infinite system 
$\g(\N; F_0, F_1)$---which generates the root-looped directed infinite binary tree---by truncating
both the set of vertices (by keeping only the first $2^n$ nonnegative integers) and the arc-generators
(by reducing all integers modulo $2^n$).

To see that the system $\g(\{0, 1, \ldots, 2^n-1\}; F^{(n)}_0, F^{(n)}_1)$ is, in fact, the order-$n$
de Bruijn network, let us observe what the functions $F^{(n)}_0$ and $F^{(n)}_1$ do to the
argument integers.
\begin{itemize}
\item
The {\em number-related} specification
\[ v \ \longrightarrow \ 2v \bmod 2^n \]
corresponds to the {\em numeral-related} specification
\[ \beta x \ \longrightarrow \ x0 \]
where $\beta \in \{0,1\}$ and $x$ is a length-$(n-1)$ bit-string.

\item
The {\em number-related} specification
\[ v \ \longrightarrow \ 2v+1  \bmod 2^n \]
corresponds to the {\em numeral-related} specification
\[ \beta x \ \longrightarrow \ x1 \]
where $\beta \in \{0,1\}$ and $x$ is a length-$(n-1)$ bit-string.
\end{itemize}
The system $\g(\{0, 1, \ldots, 2^n-1\}; F^{(n)}_0, F^{(n)}_1)$ thus specifies the local
graph-theoretic structure depicted in Fig.~\ref{fig:one-DB-node}.
\begin{figure}[hbt]
\[  \begin{array}{ccc}
\begin{array}{|ccccc|}
\hline
     &                & 0x &               &      \\
     & \swarrow &      & \searrow &      \\
x0 &                 &     &                & x1 \\
\hline
\end{array}
  & \hspace*{.5in} &
\begin{array}{|ccccc|}
\hline
     &                & 1x  &               &       \\ 
     & \swarrow &      & \searrow &       \\
x0 &                 &     &                & x1 \\
\hline
\end{array}
\end{array}
\]
\caption{The graph-theoretic action of one application of the arc-generating functions 
$F^{(n)}_0$ and  $F^{(n)}_1$ to vertices of the forms $x0$ (left) and $x1$ (right).}
\label{fig:one-DB-node}
\end{figure}
A look back at Section~\ref{sec:deBruijn} will verify that the system
$\g(\{0, 1, \ldots, 2^n-1\}; F^{(n)}_0, F^{(n)}_1)$ is, in fact, (isomorphic to) $\d_n$.

\ignore{***********
\[
\begin{array}{|ccccccc|}
\hline
\multicolumn{3}{c}{\mbox{Arc-labels: Tree}} & \hspace*{.2in}& \multicolumn{3}{c}{\mbox{Arc-labels: De Bruijn network}} \\
\hline 
F_0(x) & = & 2x      &  &  F^{(n)}_0(x) & = & 2x \bmod 2^n \\
F_1(x) & = & 2x+1  &  &  F^{(n)}_1(x) & = & 2x +1 \bmod 2^n \\ 
\hline
\end{array}
\]

\[
\begin{array}{|c|ccccc|}
\hline
\mbox{vertex} & \multicolumn{2}{c}{\mbox{Tree-successor}} & \hspace*{.2in} & \multicolumn{2}{c}{\mbox{de Bruijn-successor}} \\
\hline
v & F_0(v) & F_1(v) & & F^{(n)}_0(v))& F^{(n)}_1(v) \\
\hline
0 & 0 & 1 & & 0 & 1 \\ 
1 & 2 & 3 & & 2 & 3 \\
2 & 4 & 5 & & 4 & 5 \\
3 & 6 & 7 & & 6 & 7 \\
4 & \cdots & \cdots  & & 0 & 1 \\
5 & \cdots & \cdots  & & 2 & 3 \\
6 & \cdots & \cdots  & & 4 & 5 \\
7 & \cdots & \cdots  & & 6 & 7 \\
\hline
\end{array}
\]
***************}
