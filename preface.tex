%version of 04-09-19 

\chapter*{PREFACE AND MANIFESTO}


\begin{center}
{\it Entia non sunt multiplicanda praeter necessitatem.} \\
\hspace*{3in}{\footnotesize {\bf Occam's Razor}}
\end{center}

\noindent
This famous admonition by William of Occam (14th cent.) to strive for
simplicity is worth heeding when seeking mathematical models of
computational phenomena.

\section{The ``Manifesto'' Underlying This Text}
\label{sec:Manifesto}

**ADD SHORT PARAGRAPH FOR EACH POINT

\begin{itemize}
\item
We aim for UNDERSTANDING rather than just knowledge
\item
We want to help students/readers THINK rather than just learn —
which explains multiple proofs
\item
We want to expose the reader to some culture (as an aid to thinking
and understanding) — which explains some digressions and stories
\item
Part of the culture comes via the $\oplus$ sections … even if they are
rather short and informal.
\end{itemize}




As the technology that enabled both the hardware and software systems
of modern computers has advanced, the ability to design and utilize
such systems ``by the seat of one's pants'' has commensurately
decreased.  A vast array of formal aids for the activities of
designing, analyzing, utilizing, and verifying computer systemshas
developed, and mathematical tools have always been at or near the
forefront of such aids.

The fundamental goal of this book---and, therefore, of each of its
chapters and sections---is to endow the reader with an operational
level of conceptual and methodological understanding of the discrete
mathematics that is used to study and understand the activity of
computing and the systems that enable that activity.  We construe an
``operational'' level of understanding to be one that enables the
reader to ``do'' the relevant mathematics.

Somewhat surprising to the non-mathematician, a large portion of
``doing'' mathematics, the often-touted ``queen of the sciences'', is
pattern-matching---albeit of a monumentally sophisticated variety.
Mathematicians are trained to understand pieces of reality to a depth
that allows them to understand how apparently unrelated concepts $A$
and $B$ can be conceptualized via the same abstract representation,
and to analyze (computational, in our bailiwick) advantages to
exploiting such representations.

Toward the end of guiding the reader through this forest of
abstractions, we categorize our targets in three ways
\begin{enumerate}
\item
{\it Fundamental concepts}

{\sf Examples:}
\begin{itemize}
\item
sets---and their embellishments: tuples, arrays, tables, etc.---as the
embodiment of {\it object}
\item
numbers---and their operational manifestations, numerals---as the
embodiment of {\it quantity}
\item
graphs---in their many, varied, forms---as the embodiment of {\it
  connectivity} and {\it relationship}
\item
algebras---adding operations to sets, numbers, and graphs---as the
embodiment of {\it structured dynamism} and {\it computing}
\end{itemize}

\item
{\it Fundamental representations}

{\sf Examples:}
\begin{itemize}
\item
viewing numbers via the metaphors of: slices of pie, tokens arranged in
stylized ways, rectangles of varying dimensions
\item
using grouping and/or replication to represent relationships among
objects
\item
viewing interrelated objects via varying structures: tables, tuples,
graphs, geometric drawing
\end{itemize}

\item
{\it Fundamental tools/techniques}

{\sf Examples:}
\begin{itemize}
\item
using induction to expand from simple examples to complex ones
\item
using integration to approximate summation, and vice versa, to
(mentally) ``hop'' between the discrete and continuous worlds
\item
using the conceptual tools of asymptotics to argue qualitatively about
quantitative phenomena.
\end{itemize}
\end{enumerate}

