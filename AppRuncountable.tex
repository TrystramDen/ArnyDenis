\chapter{$\R$ IS UNCOUNTABLE, HENCE, ``BIGGER THAN'' $\Z$ AND $\Q$}
\label{ch:Q-Z-R-cardinality}
\label{ch:Reals-uncountable}

This chapter is a companion to Section~\ref{sec:FNS-uncountable}.  The
main result of this section establishes the uncountability of the set
$\R$.  We thereby have an argument that the infinitude of the set of
real numbers is {\em of a higher order} than the infinitude of the set
of integers.

We achieve our goal by providing a proof of
Proposition~\ref{thm:R-uncountable}:

\noindent
{\em The set $\R$ of real numbers is not countable.
In particular, there is no injection $g: \R \rightarrow \N$.}

We simplify our exposition by making extensive use of the proof in
Section~\ref{sec:FNS-uncountable} that the set $\F$ of functions from
$\N$ into $\{0,1\}$ is not countable
(Proposition~\ref{thm:F-uncountable}).

Our proof of Proposition~\ref{thm:R-uncountable} begins with the
following lemmas.

\begin{lemma}
\label{lem:N-leq-R}
There exists an injection $f: \N \rightarrow \R$; i.e., $|\N| \leq
|\R|$.
\end{lemma}

\noindent {\it Verification}.
For each nonnegative integer $n \in \N$, let $\mbox{\sc name}_b(n)$
denote the shortest base-$b$ numeral for $n$, i.e., a numeral with no
leading $0$s.  The mapping that associates each $n \in \N$ with the
infinite string
\[ \mbox{\sc name}_b(n) \ . \ 00 \cdots \]
is an injection from $\N$ into $\R$.  To wit, when given a real
numeral that has only $0$s to the right of its radix point, one
produces the integer $n$ by stripping the numeral of its radix point
and all $0$s to the right of the point, and then evaluating the
remaining string of digits, which is $\mbox{\sc name}_b(n)$, according
to Section~\ref{sec:define-Reals}'s rules for evaluating integer
numerals.  \qed

\begin{lemma}
\label{lem:N-=-R}
If there exists an injection $g: \R \rightarrow \N$, then there exists
a bijection $h: \N \leftrightarrow \R$.  In other words, if $|\R| \leq
|\N|$, then $|\N| = |\R|$.
\end{lemma}

\noindent {\it Verification}.
This is an immediate consequence of the Schr\"{o}der-Bernstein Theorem
(Theorem~\ref{thm.S-B}).  \qed

\medskip

We make a few technical assumptions that simplify our proof without
weakening the result.

\noindent
{\em We henceforth focus on the proper subset of $\R$ that consists of
  the set $\R_{(0,1)}$ of real numbers between $0$ and $1$.}

This simplifies our argument because every real number in the set
$\R_{(0,1)}$ has an infinite decimal numeral of the form
\[ 0 \ . \ \delta_0 \delta_1 \cdots \]
where each $\delta_i$ is a decimal digit: $\delta_i \in \{0, 1, 2, 3,
4, 5, 6, 7, 8, 9\}$.\footnote{We choose {\em decimal} numerals for
  convenience: converting the argument to other number
  bases---especially to the base $b=2$---slightly complicates clerical
  details.}

\smallskip

Of course, if we prove that the proper subset $\R_{(0,1)} \subset \R$
is uncountable, then it will follow that $\R$ is uncountable.  (In
informal terms which can be made formal, any putative injection $f: \R
\rightarrow \N$ ``contains'' an injection $f_{(0,1)}: \R_{(0,1)}
\rightarrow \N$.)

\bigskip

Now we begin to seek a bijection $h: \N \leftrightarrow \R_{(0,1)}$.

We assume, for contradiction, that the targeted bijection $g$ exists.
As part of this two-way mapping, there exists an {\em injection}
\[ 
h: \N \ \rightarrow \ \R_{(0,1)},
\]
which we view as an {\em enumeration} of the elements of $\R_{(0,1)}$.
Specifically, for each integer $k \in \N$, we can think of $h(k)$ as
the ``$k$th number in the set $\R_{(0,1)}$.''  We thereby view $h$ as
producing an ``infinite-by-infinite'' matrix $\Delta$ of decimal
digits, whose $k$th row is the infinite string of decimal digits
$\mbox{\sc name}_{10}(h(k))$.  Let us visualize $\Delta$:
\[ \Delta \ = \
\begin{array}{ccccccc}
\delta_0 = &
\delta_{0,0} & \delta_{0,1} & \delta_{0,2} & \delta_{0,3} &
	\delta_{0,4} & \cdots \\
\delta_1 = &
\delta_{1,0} & \delta_{1,1} & \delta_{1,2} & \delta_{1,3} &
	\delta_{1,4} & \cdots \\
\delta_2 = &
\delta_{2,0} & \delta_{2,1} & \delta_{2,2} & \delta_{2,3} &
	\delta_{2,4} & \cdots \\
\delta_3 = &
\delta_{3,0} & \delta_{3,1} & \delta_{3,2} & \delta_{3,3} &
	\delta_{3,4} & \cdots \\ 
\delta_4 = &
\delta_{4,0} & \delta_{4,1} & \delta_{4,2} & \delta_{4,3} &
	\delta_{4,4} & \cdots \\ 
\vdots &
\vdots  & \vdots  & \vdots  & \vdots  & \vdots  & \ddots
\end{array}
\]

\noindent We summarize, for emphasis:
\begin{itemize}
\item
Each row of $\Delta$ consists of the decimal numeral for a number in
the set $\R_{(0,1)}$.

\item
Each number in the set $\R_{(0,1)}$ contributes at least one numeral
to the rows of $\Delta$.

A number may contribute more than one numeral because of an artifact
of positional number systems, which is exemplified by equations such
as
\[ 0.25 \ = \ 0.24999\cdots \]
\end{itemize}
We can, thus, view the successive rows of $\Delta$: $h(0)$, $h(1)$,
\ldots, as an ordered listing (with possible repetitions) of all of
the real numbers in the set $\R_{(0,1)}$.

\medskip

We show now that every bijection $h: \N \leftrightarrow \R$ ``misses''
some real $x \in \R_{(0,1)}$.

We are finally poised to find the contradiction to our assumption that
$|\R_{(0,1)}| \leq |\N|$.  Specifically, we define from $\Delta$ an
infinite decimal numeral
\[ \Psi \ = \ \psi_0 \ \psi_1 \ \psi_2 \ \psi_3 \ \psi_4 \cdots, \]
that {\em does not} appear in $\Delta$, even though $\mbox{\sc
  val}_{10}(\Psi) \in \R_{(0,1)}$.  For each index $i \in \N$, we
define the $i$th digit $\psi_i$ of $\Psi$ from the $i$th {\em diagonal
  digit}\footnote{Our use of $\Delta$'s ``diagonal digits'' in this
  definition is the origin of the term ``{\em diagonal argument}'' to
  describe this proof and its intellectual kin.}~$\delta_{i,i}$ of
$\Delta$ in the following manner.
\[ \psi_i \ \eqdef \
%= \ \overline{\delta}_{i,i} \ \eqdef \
\left\{
\begin{array}{cc}
0 & \mbox{ if } \ \delta_{i,i} \ > \ 5 \\
9 & \mbox{ if } \ \delta_{i,i} \ \leq \ 5 \\
\end{array}
\right.
\]
The important feature of the definition is the following.

\begin{lemma}
\label{lem:PSI-notin-DELTA}
The string $\Psi$ does not occur as a row of $\Delta$.
\end{lemma}

\noindent {\it Verification}.
Focus on an arbitrary row of $\Delta$, say row $k$, and on the numeral,
$\delta_k$, in that row.

\begin{tabular}{lclc}
If & $\delta_{k,k} \ > \ 5$ & then & \ \
$\mbox{\sc name}_{10}(\delta_k) \ - \ \mbox{\sc name}_{10}(\Psi) \ > \ 4
\cdot 10^{-k}$ \\
If & $\delta_{k,k} \ \leq \ 5$ & then & \ \
$\mbox{\sc name}_{10}(\Psi) \ - \ \mbox{\sc name}_{10}(\delta_k) \ > \ 4
\cdot 10^{-k}$
\end{tabular}

\noindent
In either case, we have $\mbox{\sc name}_{10}(\Psi) \ \neq \ \mbox{\sc
  name}_{10}(\delta_k)$ so that $\Psi$ does not appear as row $k$ of
$\Delta$.  Since $k \in \N$ is an arbitrary row-index of $\Delta$, we
conclude that $\Psi$ does not occur as any row of $\Delta$.  \qed

\subsection{The denouement: There is no bijection  $h: \N \leftrightarrow \R$}

Because the infinite decimal string $\Psi$ differs from every row of
$\Delta$, even though $\mbox{\sc name}_{10}(\Psi) \in \R_{(0,1)}$, we
have shown that $\Delta$ {\em does not} contain as a row {\em every}
infinite decimal numeral of a number in $\R_{(0,1)}$.  But this
contradicts $\Delta$'s assumed defining characteristic!

Where could we have gone wrong?  Every step of our argument, save one,
is backed up by a proof---so the one step that is not so bolstered
must be the link that has broken.  This one unsubstantiated step is
our assumption that the set $\R_{(0,1)}$ is countable.  Since this
assumption has led us to a contradiction, we must conclude that the
set $\R_{(0,1)}$, and hence, the set $\R$, is {\em not} countable!
\qed

